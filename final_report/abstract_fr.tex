\begin{otherlanguage}{french}
\begin{abstract}
La diffusion d’appareil mobiles intelligents et le développement de l’Internet des Objets a amené la puissance de calcul dans les poches de tout le monde, permettant aux personnes d’effectuer des tâches simples grâce à leurs smartphones. Il existe cependant certaines tâches qui demandent plus de puissance de calcul et ne pouvant donc pas être effectuées sur des appareils mobiles (e.g., une flotte de drones capturant des images multicouches devant être traitées avec des opérations de machine learning telles que la reconnaissance de plaques d’immatriculation or de visages); pour ces applications en temps réel, la délocalisation de calculs représente une solution viable.

Le processus de délocalisation de calculs consiste en la délégation de tâches complexes à des serveurs situés à la périphérie du réseau. Il est utile pour minimiser les temps de réponse ou la consommation d’énergie, des contraintes critiques pour les appareils mobiles et les objets connectés. Délocaliser de telles tâches au cloud est inefficace étant donné que l’infrastructure des serveurs de cloud sont trop distantes des objets connectés. Le Edge Computing (ou calcul de périphérie) déplace le calcul du cœur à la périphérie du réseau, améliorant l’expérience de l’utilisateur en réduisant les temps de latence perçus. Un des mécanismes fondamentaux pour réduire les temps de latence grâce au edge computing est de choisir un chemin adéquat jusqu’à la destination; les algorithmes de recherche du plus court chemin communément utilisés sont agnostiques face à la performance (et donc aux temps de latence): ils ne prennent pas en considération les conditions du réseau. Une des hypothèses que nous validons dans cette publication est que les méthodes basées sur le routage coopératif peuvent diriger (i.e. acheminer ou transmettre) le trafic de périphérie avec des délais bout-en-bout (statistiquement) plus courts que des méthodes basées sur la réaction, telles que les répartiteurs de charge~\cite{facebook_LB}.

Pour prédire les métriques de chemin, nous utilisons des techniques de machine learning, qui ont beaucoup impacté la manière dont nous développons du logiciel ces dernières années~\cite{karpathy}. En particulier, dans ce projet, nous présentons une architecture pour l’orchestration de la délocalisation périphérique: le design de l’architecture est modulaire de manière à ce que les politiques de délocalisation puissent être facilement adjointes au système. Une politique efficace de prédiction de chemin pour les mécanismes de délocalisation que nous avons implémentée est le Long Short Term Memory (LSTM), une approche de Deep Learning. Notre évaluation de performances montre que notre méthode fonctionne mieux que les politiques traditionnelles de routage en termes de surcharge de débit de données. 

\end{abstract}
\end{otherlanguage}