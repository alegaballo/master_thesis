\chapter{Introduction}

\section{Overview}
Data-intensive computing requires seamless processing power which is often not available at the network-edge but rather hosted in the cloud platforms. The huge amount of mobile and IOT devices that has become available in the past few years, is able to produce a massive quantity of data, which contributes to the very famous world of big data. The majority of these devices do not have or can not handle the computational requirements to process the data they capture and so leave to the cloud the responsability to perform the computations. This process of transferring computation tasks to another platform is called computation offloading and it is crucial to the edge devices because it results in lower processing
time and energy consumption. In critical scenarios, such as natural disasters, not only computation offloading becomes necessary, but latency requirements become more strict, making paths management essential to satisfy these requisites. In this work we develop an architecture to be deployed at the edge of the network to assist the offloading process and make use of machine learning techniques to perform path management. The goal is to outperform the classic routing policies in terms of latency and througput by learning implicit patterns in network traces.

\section{Structure}
\paragraph{Chapter 2} contains a brief background about machine learning general notions, and the main techniques used in this project.
\paragraph{Chapter 3} describes the personal contribution to this project, including the detail about the architecture and the implementation of the path predictor.
\paragraph{Chapter 4} shows considerations and results of the implemented system.
\paragraph{Chapter 5} presents comments about the outcome of the project and possible future implementations.

