\chapter{Related Work}
\label{ch:related_work}
In this chapter we describe the works related to this project, the relevant topics are cyber-foraging or task offloading, and machine learning applied to networking.\\\\
Cyber-foraging is a highly complex problem since it requires to take into consideration multiple issues. Lewis and Lago~\cite{catalog} dive into those issues and present several tactics to tackle them. Tactics are divided in \textit{functional} and \textit{non-functional}, with the former identifying the elements that are necessary to meet Cyber-foraging requirements and the latter the ones that are architecture specific. Functional tactics cover computation offload, data staging, surrogate provisioning and discovery; non-functional tactics deal with resource optimization, fault tolerance, scalability and security. They describe a simple architecture which include an \textit{Offload Client} running on the edge device and an \textit{Offload Server} running on the surrogate (cloud or local servers).

Wang et. al~\cite{edge_cloud_offloading_undersubmission:} report the state-of-the-art efforts in mobile offloading. More than ten architectures are described, each one in a different possible offloading situation. In particular the reported works face the single/multiple servers as offloading destination scenario, the online/offline methods for server load balancing, the devices mobility support, the static/dynamic offloading partitioning and the partitioning granularity.

In the last few years machine learning is being used to solve various challenges but it is not being widely adopted in networking problems; however, many people are trying to change this tendency. Malmos~\cite{malmos} is a mobile offloading scheduler that uses machine learning techniques to decide whether mobile computations should be offloaded to external resources or executed locally. A machine learning classifier is used to mark tasks for local or remote execution based on application and network information. To handle the dynamics of the network an online training mechanism is used so that the system is able to adapt to the network conditions. Malmos has proven to have higher scheduling performances than static policies under various network conditions.

In \cite{crutcher2017hyperprofile} machine learning is used for computation offloading in mobile edge networks. Regression is used to predict the energy consumption during the offloading process as well as the time to for the access point to receive the payload. Available servers are represented in a feature space according to a hyper-profile, then K-NN (K-nearest neighbor) is used to determine the closest server based on metrics related to the hyper-profile. By using K-NN, if an application needs to partition a task into multiple parts onto multiple servers, one should simply vary the value of K. 

Kato et. al~\cite{deep_learning_heterogeneus} show the use of deep learning techniques for network traffic control. A DNN (deep neural network) is used for the prediction of a router next hop. The decision is based on the number of inbound packets in a router at a given time and OSPF paths are used for training. By combining the next hop decision for each router the system is able to predict the whole path from source to destination. Results show that the system is able to improve performances in terms of signaling overhead, throughput and average per hop delay with respect to the classic OSPF algorithm.

Another use of machine learning in networking is described in~\cite{end-to-end}. Bui, Zhu, Pescapé \& Botta designed a system for a long horizon end-to-end delay forecast. The idea is to use measured samples of end-to-end delays to create a model for long horizon forecast. Considering the set of samples as a discrete-time signal, wavelet transform is applied which results in two groups of coefficient. A NN (neural network) and a K-NN classifier are then used to predict the coefficients. Once again ML techniques seem to provide good results when applied to networking.