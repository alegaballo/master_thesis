\chapter{Conclusions and future work}\label{ch:conclusion}
In this work, we presented a system meant to support mobile devices task offloading. We described an architecture designed to provide the guidelines for protocol implementations and prototyped it in a small test bed. We also developed an offloading policy to be adopted in our architecture; this policy uses deep learning to predict the best path towards a destination, increasing throughput and reducing perceived latency. The preliminary results showed that even with some limitations, machine learning could be a valid alternative to traditional routing algorithms and can be leveraged to improve network performance. Our results showed that cooperative routing can steer traffic with better performance than traditional methods, suggesting that applying machine learning in this context is worthy research area. 

There are nonetheless, some limitations that need to be addressed: the decreasing performance of the prediction system with the increasing loss rate described in the previous section, are very likely due to the poor quality of the dataset. Even though we put our effort in trying to create a complete dataset, the result is probably far from completeness; having a large and representative dataset is crucial for the outcome of the model, therefore more effort should be put by the research community into making available public dataset. Another problem that needs to be solved is the scalability of this approach: training numerous deep learning model requires a lot of computational power; being able to reduce the number of models to train would save a lot of time and make this system more scalable.

As a future work, we intend to expand the dataset and improve its quality and measure how the quality of the model improves. Additionally, we plan to perform a deeper performance analysis by deploying a network which completely replaces OSPF with our system and collect measurements such as throughput and latency. Finally, we want to explore new machine learning techniques, especially, reinforcement learning, which appears to be able to solve policy-based problem.