% !TEX encoding = IsoLatin



%%%%%%%%%%%%%%%%%%%%%%%%%%%%%%%%%%%%%%%%%%%%%%%%%%%% 1.o Esempio con la classe toptesi
%\documentclass[b5paper,10pt,twoside,cucitura]{toptesi}
%\usepackage{lipsum}
%\documentclass[twoside,cucitura,pdfa]{toptesi}
%%%%%%%%%%%%%%%%%%%%%%%%%%%%%%%%%%%%%%%%%%%%%%%%%%%% 2.o Esempio con la classe toptesi
\documentclass[11pt,twoside,oldstyle,autoretitolo,classica,greek,evenboxes]{toptesi}
\usepackage[or]{teubner}
%%%%%%%%%%%%%%%%%%%%%%%%%%%%%%%%%%%%%%%%%%%%%%%%%%%%
% Commentare la riga seguente se si � specificata l'opzione "pdfa"
\usepackage{hyperref}

\hypersetup{%
    pdfpagemode={UseOutlines},
    bookmarksopen,
    pdfstartview={FitH},
    colorlinks,
    linkcolor={blue},
    citecolor={red},
    urlcolor={blue}
  }
%
% Esempio di composizione di tesi di laurea con PDFLATEX <---------------- !
%
% Questo esempio e' stato preparato inizialmente il 13-marzo-1989
% e poi e' stato modificato via via che TOPtesi andava
% arricchendosi di altre possibilit�.
%
% Per comporre con XeLaTeX, invece che con pdfLaTeX, vedere il file
% toptesi-example-xetex.tex
% A parte i font, bisogna specificare alcune cose dopo la fine del preambolo.
%
% Nel seguito laurea "quinquennale" sta anche per "specialistica" o "magistrale".
%
% Cambiare encoding a piacere; oppure non caricare nessun encoding se si usano
% solo caratteri a 7 bit (ASCII) nei file d'entrata. La codifica utf8 sarebbe
% quella maggiormente desiderabile; richiede per� un editor che sappia gestirla.
%
\usepackage[latin1]{inputenc}% per macchine Linux/Mac/UNIX/Windows; meglio utf8
\usepackage[T1]{fontenc}\usepackage{lmodern}
%
\ateneo{Politecnico di Torino}
%\nomeateneo{Castello di Miramare}
%\FacoltaDi{}
%\facolta[III]{Matematica, Fisica\\e Scienze Naturali}
%\Materia{Remote sensing}
%\monografia{La pressione barometrica di Giove}% per la laurea triennale
\titolo{An Architecture for Task and Traffic Offloading in Edge Computing\\~via Deep Learning}% per la laurea quinquennale e il dottorato
%\sottotitolo{Metodo dei satelliti medicei}% per la laurea quinquennale e il dottorato
\corsodilaurea{Ingegneria Informatica}% per la laurea
%\corsodidottorato{Meccanica}% per il dottorato

%%%%%%% Trucco per inserire la matricola sotto il nome di ogni candidato
\candidato{\tabular{@{}l@{}}Alessandro \textsc{Gaballo}\\matricola: 231587\endtabular}% per tutti i percorsi
%\secondocandidato{\tabular{@{}l@{}}Evangelista \textsc{Torricelli}\\matricola: 123457\endtabular}% per la laurea magistrale solamente

%%%%%%% 
%\direttore{prof. Albert Einstein}% per il dottorato
%\coordinatore{prof. Albert Einstein}% per il dottorato
\relatore{prof.\ Flavio Esposito}% per la laurea e/o il dottorato
\secondorelatore{prof.\ Guido Marchetto}% per la laurea magistrale
%%%%%%%%%% Trucco per mettere il correlatore senza usare l'opzione classica
%\relatore{\tabular{@{}l@{}}
%prof.\ Albert Enstein\\[1.5ex]
%\textbf{Correlatore:}\\
%dipl.~ing.~Werner von Braun
%\endtabular}
%%%%%%%%%%
%
%%%%%%%%%% Trucco per far scrivere Laureando/a/i/e al posto di "Candidato/a/i/e"
%\def\Candidato{Laureando:}
%\def\Candidati{Laureandi:}
%\def\Candidata{Laureanda:}
%\def\Candidate{Laureande:}
%%%%%%%%%%
%\terzorelatore{{\tabular{@{}l}dott.\ Neil Armstrong\\prof. Maria Rossi\endtabular}}% per la laurea magistrale
%\tutore{ing.~Karl Von Braun}% per il dottorato
%\tutoreaziendale{dott.\ ing.\ Giovanni Giacosa}
%\NomeTutoreAziendale{Supervisore aziendale\\Centro Ricerche FIAT}
%\sedutadilaurea{Agosto 1615}% per la laurea quinquennale; oppure:
\sedutadilaurea{\textsc{Anno~accademico} 2017-2018}% per la laurea magistrale
%\esamedidottorato{Novembre 1610}% per il dottorato
%\annoaccademico{1615-1616}% solo con l'opzione classica
%\annoaccademico{2006-2007}% idem
\ciclodidottorato{XV}% per il dottorato
%\logosede{logotrieste}% questo e' ovviamente facoltativo, ma e' richiesto per
% il dottorato al PoliTO; in questo calso si usa il "logopolito"
\logosede[35mm]{img/politologo} % auxiliary ShareLaTeX logo
%
%\chapterbib %solo per vedere che cosa succede; e' preferibile comporre una sola bibliografia
%\AdvisorName{Supervisors}
\newtheorem{osservazione}{Osservazione}% Standard LaTeX

%\setbindingcorrection{3mm}

\begin{document}\errorcontextlines=9% debugging

\english%  di default vale \italiano

% Comment the following lines if you don't care about the title page in English
% Change the strings if you want a title page and a copyright page in another language
% Comment just the \iflanguage statement and the closing line of the language test
%	 if you want to make a global change instead of a conditional one.
%%%%%%%%%%%%%%%%%%%%%%%%%%%%%%%%%%%%%%%%%
	\iflanguage{english}{%
	\retrofrontespizio{This work is subject to the Creative Commons Licence}
	\DottoratoIn{PhD Course in\space}
	\CorsoDiLaureaIn{Master degree course in\space}
	\NomeMonografia{Bachelor Degree Thesis}
	\TesiDiLaurea{Master Degree Thesis}
	\NomeDissertazione{PhD Dissertation}
	\InName{in}
	\CandidateName{Candidate}% or Candidate
	\AdvisorName{Supervisors}% or Supervisor
	\TutorName{Tutor}
	\NomeTutoreAziendale{Internship Tutor}
	\CycleName{cycle}
	\NomePrimoTomo{First volume}
	\NomeSecondoTomo{Second Volume}
	\NomeTerzoTomo{Third Volume}
	\NomeQuartoTomo{Fourth Volume}
	\logosede{img/politologo}% or comma separated list of logos
	}{}
%%%%%%%%%%%%%%%%%%%%%%%%%%%%%%%%%%%%%%%%%

\expandafter\ifx\csname StileTrieste\endcsname\relax
    \frontespizio
\else
    \paginavuota
    \begin{dedica}
        A mio padre

        \textdagger\ A mio nonno Pino
    \end{dedica}
    \tomo
\fi


\sommario

 La pressione barometrica di Giove viene misurata
mediante un metodo originale  messo a punto dai candidati, che si basa
sul rilevamento telescopico della pressione.

\ringraziamenti

I candidati ringraziano vivamente il Granduca di Toscana per i mezzi
messi loro a disposizione, ed il signor Von Braun, assistente del
prof.~Albert Einstein, per le informazioni riservate che egli ha
gentilmente fornito loro, e per le utili discussioni che hanno permesso
ai candidati di evitare di riscoprire l'acqua calda.

\tablespagetrue\figurespagetrue % normalmente questa riga non serve ed e' commentata
\indici

\expandafter\ifx\csname StileTrieste\endcsname\relax
\else
    \begin{citazioni}
        \textit{testo testo testo\\testo testo testo}

        [\textsc{G.\ Leopardi}, Operette Morali]

        \textgreek{>all'a p'anta <o k'eraunos d'' >oiak'izei}

        [Eraclito, fr.\ D-K 134]
    \end{citazioni}

\fi

\mainmatter


\begin{thebibliography}{9}
\bibitem{gal} G.~Galilei, {\em Nuovi studii sugli astri medicei}, Manunzio,
        Venetia, 1612.
\bibitem{tor1} E.~Torricelli, in ``La pressione barometrica'', {\em Strumenti
        Moderni}, Il Porcellino, Firenze, 1606.
\bibitem{tor2} E.~Torricelli e A.~Vasari, in ``Delle misure'', {\em Atti Nuovo
        Cimento}, vol.~III, n.~2 (feb. 1607), p.~27--31.
\end{thebibliography}




\end{document}
