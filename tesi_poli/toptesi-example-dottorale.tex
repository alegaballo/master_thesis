% !TEX encoding = UTF-8 Unicode
% !TEX TS-program = pdflatex

%%%%%%% La riga soprastante serve per configurare gli editor
%%%%%%% TeXShop, TeXworks e TeXstudio per gestire questo file
%%%%%%% con la codifica UFF-8.
%%%%%%% Se si vuole usare un'altra codifica si veda sotto.
%%%%%%%

%%%%%%%  Esempio con molte opzioni
%%%%%%% Le opzioni nella forma "chiave=valore" sono definite
%%%%%%% perché la classe dalla versione 6.1.00 usa il pacchetto
%%%%%%% xkeyval. Vedere sulla documentazione in inglese o
%%%%%%% in italiano quali chiavi accettano valori.

%%%%%%% L'opzione per il corpo accetta qualsiasi valore, anche fratto
%%%%%%% (per esempio: corpo=11.5pt) e va sempre scritto con una
%%%%%%% unità di misura. L'utente è pregato di non esagerare con
%%%%%%% corpi normali minori di 9.5pt o maggiori di 13pt.
%%%%%%%
%%%%%%% Le opzioni per inputenc e fontenc vanno per prime.
%%%%%%% Vengono ignorate se NON si compone con pdfLaTeX. Ma
%%%%%%% questo è un esempio per pdfLaTeX.
%%%%%%%

 \documentclass[%
    corpo=11pt,
    twoside,
%    stile=classica,
    oldstyle,
    autoretitolo,
    tipotesi=dottorale,
    greek,
    evenboxes,
]{toptesi}
%%%%%%%%%%%%%%%%%%%%%%%%%%%%%%%%%%%%%%%%%%%%%%%%%%%%
%%%%%% Per la codifica d'entratasi può scegliere quella che si vuole,
%%%%%% ma si consiglia di preferire utf8; in ogni caso non scegliere
%%%%%% codifiche specifiche del sistema operativo.

\usepackage[utf8]{inputenc}% codifica d'entrata
\usepackage[T1]{fontenc}%    codifica dei font
\usepackage{lmodern}%        scelta dei font
\usepackage{pict2e}%         estensione dell'ambiente picture
% Vedere la documentazione toptesi-it.pdf per le
% attenzioni che bisogna usare al fine di ottenere un file
% veramente conforme alle norme per l'archiviabilità.


\usepackage{hyperref}

\hypersetup{%
    pdfpagemode={UseOutlines},
    bookmarksopen,
    pdfstartview={FitH},
    colorlinks,
    linkcolor={blue},
    citecolor={blue},
    urlcolor={blue}
  }
%
%%%%%%% Esempio di composizione di tesi di laurea con PDFLATEX 
%
%
% Per scrivere testo fasullo in "latinorum"
\usepackage{lipsum}
%

%%%%%%% Definizioni locali
\newtheorem{osservazione}{Osservazione}% Standard LaTeX


\begin{document}\errorcontextlines=9
\DeclareSlantedCapitalGreekLetters % per avere in matematica le
                  % maiuscole greche inclinate. Per esempio
                  % l'angolo solido \Omega è diverso dall'ohm...
%%%%%%% Questi comandi è meglio metterli dentro l'ambiente
%%%%%%% frontespizio o frontespizio*, oppure in un file di
%%%%%%% configurazione personale. Si veda la documentazione
%%%%%%% inglese o italiana.
%%%%%%% Comunque i presenti comandi servono per comporre la
%%%%%%% tesi con i moduli di estensione standard del pacchetto
%%%%%%% TOPtesi.
\begin{ThesisTitlePage}
\ateneo{Università degli Studi di Marconia}
%
% Non tutte le università hanno un nome proprio
\nomeateneo{Sede di Torre Elettra}
%
\scuoladidottorato{Università di Marconia}
\ciclodidottorato{XVIII}
%\Materia{Remote sensing}
\titolo{La pressione barometrica di~Giove}% per la laurea quinquennale e il dottorato
\sottotitolo{Metodo dei satelliti medicei}% per la laurea quinquennale e il dottorato
%
%%%%%%% Corso degli studi
\corsodidottorato{Astronomia Applicata}% per la laurea

%%%%%%% L'eventuale numero di matricola va fra parentesi quadre
\candidato{Galileo \textsc{Galilei}}[123456] 

%%%%%%% Relatori o supervisori
%
\tutore{prof.~Albert Einstein}
% 
%%%%%%% Per mettere altri relatori consultare toptesi-it.pdf

%%%%%%% Tutore aziendale
\tutoreaziendale{dott.\ ing.\ Giovanni Giacosa}
\NomeTutoreAziendale{Supervisore aziendale\\Centro Ricerche FIAT}

%%%%%%% Seduta dell'esame
\esamedidottorato{\textsc{Anno~accademico} 1615-1616}% 

%%%%%%% Logo della sede
\logosede{logodue}% 
\end{ThesisTitlePage}


%%%%%%% Per cambiare l'offset per la rilegatura; meno offset
%%%%%%% c'e', meglio e'
%\setbindingcorrection{3mm}




%%%%%%% Questo test è usato appunto per collaudare diversi stili,
%%%%%%% non per comporre una vera tesi.
%%%%%%% Non usarlo mai, solo perché qui è usato!
\ifclassica%
{\begin{dedica}
    A mio padre

    \textdagger\ A mio nonno Pino
\end{dedica}\fi
%%%%%%% Fine esperimento

\sommario

La pressione barometrica di Giove viene misurata
mediante un metodo originale  messo a punto dai candidati, che si basa
sul rilevamento telescopico della pressione.

% \paginavuota % funziona anche senza specificare l'opzione classica

\ringraziamenti

I candidati ringraziano vivamente il Granduca di Toscana per i mezzi
messi loro a disposizione, ed il signor Von Braun, assistente del
prof.~Albert Einstein, per le informazioni riservate che egli ha
gentilmente fornito loro, e per le utili discussioni che hanno permesso
ai candidati di evitare di riscoprire l'acqua calda.

\tablespagetrue\figurespagetrue % normalmente questa riga non serve ed e' commentata
\indici

%%%%%%%% Altro esperimento con l'opzione classica
%%%%%%%% Non usare mai anche se qui lo si è fatto!
%%%%%%%% Oltretutto funziona solo se si è specificata la lingua greca fra le opzioni.
%%%%%%%% Commentare fra \ifclassica fino a \fi compresi. 
\ifclassica   
\begin{citazioni}
    \textit{testo testo testo\\testo testo testo}

    [\textsc{G.\ Leopardi}, Operette Morali]\vspace{1em}

    \textgreek{>all'a p'anta <o k'eraunos d'' >oiak'izei}

    [\textsc{Eraclito}, fr.\ D-K 134]
\end{citazioni}

\fi
%%%%%%%% fine esperimento
\mainmatter

\part{Prima Parte}
\chapter{Introduzione generale}

\section{Principi generali}
Il problema della determinazione della pressione barometrica dell'atmosfera di
Giove non ha ricevuto finora una soluzione soddisfacente, per l'elementare
motivo che il pianeta suddetto si trova ad una distanza tale che i mezzi attuali
non consentono di eseguire una misura diretta.

Conoscendo però con grande precisione le orbite dei satelliti principali di
Giove, e segnatamente le orbite dei satelliti medicei, è possibile eseguire
delle misure indirette, che fanno ricorso alla nota formula \cite{gal}:
\[
\Phi = K\frac{\Xi^2 +\Psi\ped{max}}{1+\gei\Omega}
\]
dove le varie grandezze hanno i seguenti significati:
\begin{enumerate}
\item
$\Phi$ angolo di rivoluzione del satellite in radianti se $K=1$, in gradi se
$K=180/\pi$;
\item
$\Xi$ eccentricità dell'orbita del satellite; questa è una grandezza priva
di dimensioni;
\item
$\Psi\ped{max}$ rapporto fra il semiasse maggiore ed il semiasse minore
dell'orbita del satellite, nelle condizioni di massima eccentricità;
poiché le dimensioni di ciascun semiasse sono $[l]=\unit{km}$, la grandezza
$\Psi\ped{max}$ {è} adimensionata;
\item
$\Omega$ velocità istantanea di rotazione; si ricorda che è $[\Omega]=%
\unit{rad}\unit{s}^{-1}$;
\item bisogna ancora ricordarsi che $10^{-6}\unit{m}$ equivalgono a 1\unit{\micro m}.
\end{enumerate}
%

Le grandezze in gioco sono evidenziate nella figura \ref{fig:orbita}.
\begin{figure}[ht]\centering
\setlength{\unitlength}{0.01\textwidth}
\begin{picture}(40,30)(30,0)
\put(50,15){\scalebox{1}[0.7]{\circle{25}}}
\put(47,15){\circle*{1}}
\put(30,0){\line(0,1){30}}
\put(30,30){\line(1,0){40}}
\put(70,30){\line(0,-1){30}}
\put(70,0){\line(-1,0){40}}
\end{picture}
\caption{Orbita del generico satellite; si noti l'eccentricità dell'orbita rispetto al pianeta.}\label{fig:orbita}
\end{figure}

Per misurare le grandezze che compaiono in questa formula è necessario
ricorrere ad un pirometro con una resistenza di 120\unit{M\ohm}, altrimenti gli
errori di misura sono troppo grandi, ed i risultati completamente falsati.

\section{I satelliti medicei}
I satelliti medicei, come noto, sono quattro ed hanno dei periodi di rivoluzione
attorno al pianeta Giove che vanno dai sette giorni alle tre settimane.

Essi furono per la prima volta osservati da uno dei candidati mentre
sperimentava l'efficacia del tubo occhiale che aveva appena inventato
rielaborando una idea sentita di seconda mano da un viaggiatore appena arrivato
dai Paesi Bassi.

%\blankpagestyle{headings}

%\lipsum[1-2]



\chapter{Il barometro}
\section{Generalità}
\begin{interlinea}{0.87} Il barometro, come dice il nome, serve per
misurare la pesantezza; più precisamente la pesantezza dell'aria
riferita all'unità di superficie.
\end{interlinea}

\begin{interlinea}{2} Studiando il fenomeno fisico si può concludere
che in un dato punto grava il peso della colonna d'aria che lo
sovrasta, e che tale colonna è tanto più grave quanto maggiore
è la superficie della sua base; il rapporto fra il peso e la base
della colonna si chiama pressione e si misura in once toscane al cubito
quadrato, \cite{tor1}; nel Ducato di Savoia la misura in once al piede
quadrato è quasi uguale, perché colà usano un piede molto
grande, che è simile al nostro cubito.
\end{interlinea}

\subsection{Forma del barometro}
Il barometro consta di un tubo di vetro chiuso ad una estremità e
ripieno di mercurio, capovolto su di un vaso anch'esso ripieno di
mercurio; mediante un'asta graduata si può misurare la distanza fra
il menisco del mercurio dentro il tubo e la superficie del mercurio
dentro il vaso; tale distanza è normalmente di 10 pollici toscani,
\cite{tor1,tor2}, ma la misura può variare se si usano dei pollici
diversi; è noto infatti che gl'huomini sogliono avere mani di
diverse grandezze, talché anche li pollici non sono egualmente
lunghi.
\section{Del mercurio}
Il mercurio è un a sostanza che si presenta come un liquido, ma ha il colore
del metallo. Esso è pesantissimo, tanto che un bicchiere, che se fosse pieno
d'acqua, sarebbe assai leggiero, quando invece fosse ripieno di mercurio,
sarebbe tanto pesante che con entrambe le mani esso necessiterebbe di essere
levato in suso.

Esso mercurio non trovasi in natura nello stato nel quale è d'uopo che sia
per la costruzione dei barometri, almeno non trovasi così abbondante come
sarebbe necessario.

\setcounter{footnote}{25}

Il Monte Amiata, che è locato nel territorio del Ducato%
\footnote{Naturalmente stiamo parlando del Granducato di Toscana.%
\ifclassica\NoteWhiteLine\fi
} del nostro Eccellentissimo et Illustrissimo Signore Granduca di Toscana\footnote{Cosimo IV de' Medici.}, è uno dei
luoghi della terra dove può rinvenirsi in gran copia un sale rosso, che
nomasi \emph{cinabro}, dal quale con artifizi alchemici, si estrae il mercurio
nella forma e nella consistenza che occorre per la costruzione del barometro
terrestre%
\ifclassica
\nota{Nota senza numero\dots

\dots e che va a capo.
}\fi.


La densità del mercurio è molto alta e varia con la temperatura come
può desumersi dalla tabella \ref{t:1}.


Il mercurio gode della sorprendente qualità et proprietà, cioè che esso
diventa tanto solido da potersene fare una testa di martello et infiggere
chiodi aguzzi nel legname.
\begin{table}[htp]              % crea un floating body col nome Tabella nella
                                % didascalia
\centering                      % comando necessario per centrare la tabella
\begin{tabular}%                % inizio vero e proprio della tabella
{rrrrrr}                        % parametri di incolonnamento
\hline\hline                    % filetti orizzontali sopra la tabella
                                % intestazione della tabella
\multicolumn{3}{c}{\rule{0pt}{2.5ex}Temperatura} % \rule serve per lasciare
& \multicolumn{3}{c}{Densità} \\               % un po' di spazio sopra le parole
    &\unit{\gradi C} & & & $\unit{t/m^3}$ &  \\
\hline%                         % Filetto orizzontale per separare l'intestazione
\hspace*{1.3em}& 0  &  & & 13,8 &  \\   % I numeri sono incolonnati % 
              & 10  &  & & 13,6 &  \\   % a destra; le colonne vuote
              & 50  &  & & 13,5 &  \\   % servono per centrare le colonne
              &100  &  & & 13,3 &  \\   % numeriche sotto le intestazioni
\hline \hline                           % Filetti di fine tabella
\end{tabular}
\caption[Densità del mercurio]{Densità del mercurio. Si può fare molto meglio usando il pacchetto \textsf{booktabs}.} \label{t:1}  % didascalia con label
\end{table}

%\selectlanguage{italian}

\begin{osservazione}\normalfont
Questa proprietà si manifesta quando esso è estremamente freddo, come
quando lo si immerge nella salamoia di sale e ghiaccio che usano li maestri
siciliani per confetionare li sorbetti, dei quali sono insuperabili artisti.
\end{osservazione}

Per nostra fortuna, questo grande freddo, che necessita per la confetione de
li sorbetti, molto raramente, se non mai, viene a formarsi nelle terre del
Granduca Eccellentissimo, sicché non vi ha tema che il barometro di mercurio
possa essere ruinato dal grande gelo e non indichi la pressione giusta, come
invece deve sempre fare uno strumento di misura, quale è quello che è
descritto costì.\cite{duane1964}

\chapter{Il listato del pacchetto \texttt{topcoman.sty}}
\listing{topcoman.sty}


\begin{thebibliography}{9}
\bibitem{gal} G.~Galilei, {\em Nuovi studii sugli astri medicei}, Manuzio,
        Venetia, 1612.
\bibitem{tor1} E.~Torricelli, in ``La pressione barometrica'', {\em Strumenti
        Moderni}, Il Porcellino, Firenze, 1606.
\bibitem{tor2} E.~Torricelli e A.~Vasari, in ``Delle misure'', {\em Atti Nuovo
        Cimento}, vol.~III, n.~2 (feb. 1607), p.~27--31.
\bibitem{duane1964} Duane J.T., \emph{Learning Curve Approach To Reliability 
		Monitoring}, IEEE Transactions on Aerospace, Vol. 2, pp. 563-566, 1964
\end{thebibliography}




\end{document}

% altri riferimenti da usare come esempi.

\bibitem{chiesa2008} Chiesa S., \emph{Affidabilità, sicurezza e manutenzione 
		nel progetto dei sistemi}, CLUT, gennaio 2008
\bibitem{chiesa2}Chiesa S., Fioriti M., Fusaro R., \emph{On Board System 
		Technological  Level Improvement Effect on UAV MALE}
\bibitem{bigliano2010} Bigliano M., \emph{Sicurezza nell'installazione di un velivolo 
		senza pilota MALE; applicazione di metodologia di Zonal Safety 
		Analysis al velivolo del Progetto SAvE}, Politecnico di Torino, 
		maggio 2010
\bibitem{astrid2012} Chiesa S., Di Meo G.A., Fioriti M., Medici G., Viola N.,
		\emph{ASTRID - Aircraft on board Systems sizing and TRade-off 
		analysis in Initial Design}, Research Bulletin, Warsaw University 
		of Technology, Institute of Aeronautics and Applied Mechanics, 
		p. 1-28, 17-19, ottobre 2012
