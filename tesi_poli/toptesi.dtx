% \iffalse meta-comment
%<*internal>
\begingroup
\input docstrip.tex
\keepsilent


\preamble
  ______________________________________________________
  The TOPtesi bundle 
  Copyright (C) 2015-2017 Claudio Beccari 
  All rights reserved

  License information appended

\endpreamble
\postamble

Copyright 2005-2017 Claudio Beccari

Distributable under the LaTeX Project Public License,
version 1.3c or higher (your choice). The latest version of
this license is at: http://www.latex-project.org/lppl.txt

This work is "author-maintained"

This work consists of this file toptesi.dtx, a README file
the manifest.txt file, and the derived files: 
    toptesi.cls, 
    toptesi.sty, 
    topfront.sty, 
    topcommand.sty,
    toptesi-scudo.sty, 
    toptesi.cfg, 
    toptesi-monografia.sty,
    toptesi-sss.sty, 
    toptesi-magistrale.sty, 
    toptesi-dottorale.sty, 
and the English documentation toptesi.pdf.
 
Furthermore the bundle contains the documentation source
file toptesi-it.tex and the derived file 
toptesi-it.pdf.

The toptesi-example.tex source file is just an example 
that shows how to use the various commands; by commenting 
or uncommenting certain source lines it is possible to 
typeset different kind of theses and their front pages.

The topfront-example.tex source file is an example of how 
to produce just the title page with TOPtesi and the external
package frontespizio.

The toptesi-scudo-example.tex source file is an example 
that shows the particular features available with TOPtesi, 
when a doctoral thesis is produced for the Scuola di 
Dottorato (ScuDo) of Politecnico di Torino.

The toptesi.cfg file is a sample of a local configuration 
file that can be copied to another file and its copy freely 
edited and customised.

The other toptesi-*.sty files are extension modules for 
typesetting the specific thesis kinds described by the 
suffix that replaces the asterisk.

By running pdflatex on toptesi.dtx the user gets the class, 
sty and cfg files and the English documentation file in
PDF format.

The source file of the Italian documentation file 
toptesi-it.pdf is another example of how to use TOPtesi. 
This file toptesi-it.tex may be typeset with pdfLaTeX, 
XeLaTeX, and LuaLateX; see the first comment lines of 
the file for how-to information. In spite of being mainly
written in Italian, it contains several sections in English
for the benefit of foreign students attending the Doctoral 
School of Turin.
\endpostamble

\askforoverwritefalse

\generate{\file{toptesi.cls}{\from{toptesi.dtx}{class}}
          \file{toptesi.sty}{\from{toptesi.dtx}{topsty}}
          \file{topcoman.sty}{\from{toptesi.dtx}{topcmn}}
          \file{topfront.sty}{\from{toptesi.dtx}{topfrn}}
          \file{toptesi.cfg}{\from{toptesi.dtx}{topcfg}}
          \file{toptesi-scudo.sty}{\from{toptesi.dtx}{topscu}}
          \file{toptesi-monografia.sty}{\from{toptesi.dtx}{topmon}}
          \file{toptesi-sss.sty}{\from{toptesi.dtx}{topsss}}
          \file{toptesi-magistrale.sty}{\from{toptesi.dtx}{topmag}}
          \file{toptesi-dottorale.sty}{\from{toptesi.dtx}{topdoc}}
         }

\def\tmpa{plain}
\ifx\tmpa\fmtname\endgroup\expandafter\bye\fi
\endgroup
%</internal>
%
% Copyright 1989--2017 Claudio Beccari
%
% This file is part of the TOPtesi bundle
% -------------------------------------------
%
% It may be distributed and/or modified under the
% conditions of the LaTeX Project Public License, either version 1.3
% of this license or (at your option) any later version.
% The latest version of this license is in
%    http://www.latex-project.org/lppl.txt
% and version 1.3 or later is part of all distributions of LaTeX
% version 2003/12/01 or later.
%
% This file has the LPPL maintenance status "author-maintained".
%
% The list of all files belonging to the TOPtesi bundle is
% given in the file `manifest.txt'.
%
% The set of derived (unpacked) files belonging to the distribution
% and covered by LPPL is created by the self unpacking file toptesi.dtx
% which is the principal part of the distribution.
%
% \fi
%
% \iffalse
%^^A The following trick uses the same date for every file header.
%<class,topsty,topcmn,topfrn,topscu,topmon,topsss,topmag>\NeedsTeXFormat{LaTeX2e}[2016/01/01]
%<*driver>
\ProvidesFile{toptesi.dtx}[%
%</driver>
%<class>\ProvidesClass{toptesi}[%
%<topsty>\ProvidesPackage{toptesi}[%
%<topcmn>\ProvidesPackage{topcoman}[%
%<topfrn>\ProvidesPackage{topfront}[%
%<topscu>\ProvidesPackage{toptesi-scudo}[%
%<topmon>\ProvidesPackage{toptesi-monografia}[%
%<topsss>\ProvidesPackage{toptesi-sss}[%
%<topmag>\ProvidesPackage{toptesi-magistrale}[%
%<topdoc>\ProvidesPackage{toptesi-dottorale}[%
%<*driver,class,topsty,topcmn,topfrn,topscu,topmon,topsss,topmag,topdoc>
2018/01/12 v.~6.1.12 
%<class>Class for typesetting university theses]
%<topsty>Extension for toptesi.cls]
%<topcmn>Additional commands for the TOPtesi bundle]
%<topfrn>Title page for TOPtesi and other classes]
%<topscu>Extension for ScuDo theses]
%<topmon>Bachelor degree title page]
%<topsss>High school title page]
%<topmag>Master degree title page]
%<topdoc>Ph.D. degree title page]
%</driver,class,topsty,topcmn,topfrn,topscu,topmon,topsss,topmag,topdoc>
%<*driver>
]
%</driver>
%<*driver>
\documentclass{ltxdoc}
\GetFileInfo{toptesi.dtx}
\title{The TOPtesi bundle}
\date{\fileversion\space--- \filedate}
 \author{Claudio Beccari\thanks{E-mail: \texttt{claudio dot beccari at gmail dot com}}}
 
\usepackage{metalogo,multicol,enumitem,xspace,
   xparse,curve2e,etoolbox,textcomp}
\let\originalmeta\meta
\renewcommand\meta[1]{{\normalfont\originalmeta{#1}}}
\def\prog#1{\textsf{#1}}
\def\pack#1{\textsf{\slshape#1}}
\def\env#1{\textit{#1}}\let\amb\env
\def\file#1{\texttt{#1}}
\def\marg#1{{\normalfont\texttt{\{}\meta{#1}\texttt{\}}}}
\def\Marg#1{{\texttt{\{#1\}}}}
\def\opz#1{{\normalfont\textsl{#1}}}
\def\oarg#1{{\normalfont\texttt{[\meta{#1}]}}}
\def\Oarg#1{{\normalfont\texttt{[#1]}}}
\def\parg#1{{\normalfont\texttt{(\meta)}}}
\def\Parg#1{{\normalfont\texttt{(#1)}}}
\def\TeXLive{TeXLive\xspace}
\def\pdfLaTeX{pdf\/\LaTeX\unskip\xspace}
\newcommand*\Bambiente[1]{\texttt{\char92begin}\Marg{#1}}
\newcommand*\Eambiente[1]{\texttt{\char92end}\Marg{#1}}
\DeclareRobustCommand\sigla[1]{\lowercase{\textsc{#1}}}
\newenvironment{ttsintassi}{\flushleft\ttfamily\obeylines}{\endflushleft}
%
%+++++++++++++++++++++++++++++++++++++++++++++++++++++++++++++
% Some macros to typeset a framed box of natural or specified
% width and height within the picture environment.
% Needs packages curve2e, etoolbox, and xparse.
% Syntax:
% \Zbox(<position>)(<dimensions>)[<alignment>]{<box contents>}
% The first and fourth arguments are mandatory; the second
% and third arguments are optional; their default values are
% (0,0) and [bl]. Furthermore the <position> and <dimension>
% values are a comma separated pair of numbers representing
% measures in \unitlength units, as well as it is done within the
% picture environment. No provision is coded to take the measure
% of the <box contents>. This contents should be some text, and when
% <dimensions> are specified larger than zero, a \parbox is used
% to typeset it. The <alignment> value(s) define the reference point
% that is \put at the <position> coordinates.
%
\def\TOPsplitArgs(#1,#2)#3#4{\edef#3{#1}\edef#4{#2}}

\NewDocumentCommand\Zbox{R(){0,0} D(){0,0} O{bl} m}{%
\TOPsplitArgs(#2)\ZboxX\ZboxY % separa la x e la y della scatola
\fboxsep=2\unitlength
\ifnum\ZboxX=\csuse{z@}
  \def\ZTesto{\fbox{#4}}%
\else
  \ifnum\ZboxY=\csuse{z@}
    \def\ZTesto{\fbox{\parbox{\ZboxX\unitlength}{#4}}}%
  \else
    \def\ZTesto{%
    \setbox2560=\hbox{\fbox{%
         \parbox[c][\ZboxY\unitlength][c]{\ZboxX\unitlength}{#4}}}%
      \dimen2560=\dimexpr(\ht2560 +\dp2560)/2\relax
      \ht2560=\dimen2560\relax 
      \dp2560=\dimen2560\relax
      \box2560%
      }%
  \fi
\fi
\put(#1){\makebox(0,0)[#3]{\ZTesto}}}

\NewDocumentCommand\fileicon{R(){0,0} m}{%
\put(#1){\moveto(0,-10)
\curveto(7.5,-10)(7.5,-7.5)(15,-7.5)
\lineto(15,10)
\curveto(7.5,10)(7.5,7.5)(0,7.5)
\lineto(0,-10)
\strokepath}\put(#1){\put(7.5,0){\makebox(0,0){\ttfamily#2}}}
}

\def\ifTOPnextchar{\csuse{@ifnextchar}}

\def\polyvector(#1){\def\TOPlastpoint{#1}\TOPpolyvector}

\def\TOPpolyvector(#1){\ifTOPnextchar({\segment(\TOPlastpoint)(#1)%
\def\TOPlastpoint{#1}\TOPpolyvector}{\VECTOR(\TOPlastpoint)(#1)}} % )


%+++++++++++++++++++++++++++++++++++++++++++++++++++++++++++++

\begin{document}\errorcontextlines=9
\maketitle
\begin{multicols}{2}
\tableofcontents
\end{multicols}
 \setlength\hfuzz{20pt}
 \DocInput{toptesi.dtx}
\end{document}
%</driver>
% \fi
%
% \CheckSum{7526}
%
% \begin{abstract}
% This file describes the TOPtesi bundle; it is a set of files
% designed to typeset  a university final report that in Italian
% is generally called ``tesi''; it was originally developed at
% the Technical University of Turin (Politecnico di Torino) but
% it was adapted for typesetting theses in any Italian university.
% Well\dots\ since the Erasmus student mobility is very extended
% and many Italian students participate in the so called
% \textit{double degree} programs, their theses, or whatever they
% are called in other countries, may be typeset so as to comply
% also with the host university rules; therefore this set of
% files has the ambition to be suited for typesetting theses in
% any university in the world\dots\ This ambition can't be
% fulfilled, though, because of the complexity of the title
% page (and possibly of the legal page) requirements. This
% version is compliant with the \XeLaTeX\ and the \LuaLaTeX\ 
% programs. Up to now the few conflicts that have
% been spotted have been solved with suitable corrections or
% additions. The most important feature with \XeLaTeX\ is that
% the option \opz{pdfa} cannot be used any more; this is due to the
% fact that the typesetting engine \XeTeX\ does not directly
% produce any \sigla{PDF} output but a modified, extended DVI one,
% that is immediately converted into a \sigla{PDF} file through
% |xdvipdfmx|, a special version of the conversion program. Another
% minor \XeLaTeX\ feature is that it cannot fully exploit the
% typesetting facilities of the |microtype| package, but the wide
% choice of OpenType font properties replaces almost completely
% such missing |microtype| features.
%
% With version~6.xx, the bundle has been modularised in a (hopefully)
% more efficient way, although it is backwards compatible, so that
% version~5.xx functionalities are still available.
%
% A totally new module is the one that allows to typeset the Italian
% high school ``tesina'', a document that is examined by the
% graduation exam committee at the end of secondary studies.
%
% The user should not be scared by the number of pages of this
% documentation; s/he should read part of the first 31 pages,
% those that concern his/her thesis. The rest is just the code
% and its documentation; it is useful mostly for me, as the
% author, and to other programmers and/or curious users.
% The ``normal'' user should not spend his/her time to read
% unnecessary information.
% \end{abstract}
%
% \section{History}
% The TOPtesi bundle at its very beginning in the early nineties
% was just one |.sty| file to ben used with the|report| class.
% At that time there was only \LaTeX{\,209} and extension
% packages were called ``style files''; The |.sty| extension
% is still used to day for backwards compatibility, but |.sty|
% files are not any more used to change the style of the document.
%
% With version 2.x, suitable for \LaTeXe, the bundle was
% modified in order to have a class of its own,
% |toptesi.cls|, which provided to collecting the necessary
% options and to pass them to the |report| class and
% loaded the |toptesi.sty| just afterwords.
% The latter file was not monolithic any more; it was split
% in three parts: |toptesi.sty| contained most settings and
% the macros to be used in the body of the document;
% |topfront.sty| provided the commands for entering all the
% necessary and optional data for the frontispiece (or title
% page) and actually typeset the title page in different
% styles; |topcoman.sty| contained the definitions of several
% useful commands; actually this module could be used without
% reference to the TOPtesi bundle, being suitable as an
% extension to any class.
% The general scheme may be represented by the flow diagram in figure~\ref{fig:simple-flow-diagram}.
%\begin{figure}\centering
%\unitlength=0.00925\textwidth
%\begin{picture}(106,20)
%\fileicon(0,10){.tex}\VECTOR(15,10)(20,10)
%\Zbox(20,10)(18,0)[l]{\ttfamily toptesi.cls\\toptesi.sty}\VECTOR(41.5,10)(47,10)
%\Zbox(47,10)[l]{\ttfamily topfront}\VECTOR(63.5,10)(70,10)
%\Zbox(70,10)[l]{\ttfamily topcoman}\VECTOR(86.5,10)(91,10)
%\fileicon(91,10){.pdf}
%\end{picture}
%\caption{Quasi original TOPtesi bundle flow diagram}
%\label{fig:simple-flow-diagram}
%\end{figure}
%
% Version 3.x was modified, on request of several users, in
% order to avoid the |toptesi.cls| class file, but to allow
% the other files, from |toptesi.sty| on, to be used with
% any standard or custom class.
%
% On request of other users, after other upgrades, the bundle was
% upgraded to version 5.85; the purpose was to introduce the
% possibility of typesetting the thesis title page with a package
% totally independent from the TOPtesi bundle. In particular the
% |frontespizio| package by Enrico Gregorio was considered
% a fancy choice for most Italian user, but the approach
% was open to any other package. The choice was made by
% setting suitable switches (boolean variables). The solution
% opened the bundle to other external contributions, although
% the mechanism was sort of tricky. The bundle was made
% compatible with other packages, for example |caption|
% and/or |geometry|, while before such packages produced
% options clashes and or other errors, so that their use
% was explicitly forbidden. Meanwhile the |topfront| module
% was configured to be loaded just after the preamble end,
% so as to be able to check if the user loaded packages
% required special special attention in order to avoid conflicts;
% see figure~\ref{fig:toptesi+frontspizio}.
%
%\begin{figure}\centering\unitlength=0.009\textwidth
%\begin{picture}(115,45)(0,-3)
%\fileicon(0,10){.tex}\VECTOR(15,10)(20,10)
%\Zbox(20,10)(18,0)[l]{\ttfamily toptesi.cls\\toptesi.sty}
%\Zbox(20,30)(18,0)[l]{\centering\ttfamily user loaded\\packages}
%\VECTOR(42,10)(50,10)
%\VECTOR(31,23)(31,15)
%\polygon(50,10)(55,5)(60,10)(55,15)
%\polyvector(55,5)(55,0)(60,0)
%\Zbox(60,0)[l]{\ttfamily topfront}
%\polyvector(55,15)(55,20)(60,20)
%\Zbox(60,20)[l]{\ttfamily frontespizio}
%\VECTOR(42,30)(47,30)\put(49,30){\circle{4}}\VECTOR(51,30)(60,30)
%\polyvector(42,12)(49,12)(49,28)
%\Zbox(60,30)[l]{\ttfamily topcoman}
%\put(87,10){\circle{4}}\VECTOR(89,10)(95,10)
%\polyvector(84,20)(87,20)(87,12)
%\polyvector(77,0)(87,0)(87,8)
%\polyvector(77,30)(90,30)(90,15)(95,15)
%\fileicon(95,10){.pdf}
%\end{picture}
%\caption{Advanced TOPtesi bundle flow diagram}
%\label{fig:toptesi+frontspizio}
%\end{figure}
%
% Version 5.x introduced a new feature; the specific settings
% for the doctoral theses written at Politecnico di Torino
% (Technical Institute od Turin); after all the bundle was
% born in that university. The switching mechanism to select
% the various thesis title page settings still worked with
% suitable boolean variables, but it was becoming a little
% too clumsy.
%
% In order to put some order in this situation I decided to
% upgrade to version 6.x; the class options now use the
% \emph{key\,$=$\,value} syntax (of course the value
% |true| is the default setting for boolean variables).
% This allows to select the thesis type with a
% \emph{key\,$=$\,value} option, therefore the selection and
% configuration of the title page should be simpler than before.
%
% The new situation flow diagram is represented in
% figure~\ref{fig:key-value-option}; the diagram is
% complicated but, once selected  a specific option
% for the thesis type, the diagram reduce to a linear one.
%
%\begin{figure}[p]\centering
%\unitlength=0.008\textwidth
%\begin{picture}(112.5,140)
%\fileicon(0,33){.tex}
%\polyvector(15,33)(20,33)(20,48)(0,48)(0,58)(5,58)
%\Zbox(5,58)(18,0)[l]{\small\ttfamily toptesi.cls\\toptesi.sty}
%\VECTOR(27,58)(32,58)
%\put(36.5,57){\rotatebox[origin=Bl]{90}{%
%  \makebox(0,0){\fboxsep=2\unitlength
%    \framebox(116,9){\parbox[c][5\unitlength][c]{112\unitlength}%
%    {\centering Options\\\texttt{tipotesi\,=\,}\emph{value}}}}}}
%\VECTOR(41,6)(45,6)\Zbox(45,6)(25,0)[l]{%
%\centering\texttt{topfront} (\emph{default})}
%\VECTOR(41,22)(45,22)\Zbox(45,22)(31,0)[l]{\centering\texttt{toptesi-triennale}}
%\VECTOR(41,37)(45,37)\Zbox(45,37)(33,0)[l]{\centering\texttt{toptesi-magistrale}}
%\VECTOR(41,52)(45,52)\Zbox(45,52)(31,0)[l]{\centering\texttt{toptesi-dottorale}}
%\VECTOR(41,67)(45,67)\Zbox(45,67)(31,0)[l]{\centering\texttt{toptesi-scudo}}
%\VECTOR(41,81)(45,81)\Zbox(45,81)(32,0)[l]{\centering\texttt{toptesi-sss}}
%\VECTOR(41,96)(45,96)\Zbox(45,96)(32,0)[l]{\centering\texttt{frontespizio}}
%\VECTOR(41,111)(45,111)\Zbox(45,111)(32,0)[l]{\centering\texttt{custom}}
%\Zbox(5,129.5)(18,0)[l]{\centering\ttfamily user loaded\\packages}
%\VECTOR(16,122)(16,63.5)
%\VECTOR(27,129,5)(35,129.5)\put(37,129.5){\circle{4}}
%\VECTOR(39,129.5)(45,129.5)
%\Zbox(45,129.5)(16,0)[l]{\centering\ttfamily topcoman}
%\polyvector(27,61)(29,61)(29,120)(37,120)(37,127.5)
%\put(90,67){\circle{4}}\VECTOR(92,67)(97.5,67)
%\polyvector(65,129.5)(93,129.5)(93,70)(97.5,70)^^A topcomand
%\polyvector(81,111)(90,111)(90,69)^^A custom
%\polyvector(81,96)(88,96)(88,72)(89,68.4)^^A frontespizio
%\polyvector(81,81)(86,81)(86,70)(88.4,68.2)^^A secondaria
%\VECTOR(80,67)(88,67)^^A scudo
%\polyvector(80,52)(86,52)(86,63)(88.4,65.6)^^A dottorale
%\polyvector(82,37)(88,37)(88,60)(89,65.2)^^A magistrale
%\polyvector(80,22)(90,22)(90,65)^^A triennale
%\polyvector(74,6)(92,6)(92,62)(91,65.1)^^A topfront
%\fileicon(97.5,68.5){.pdf}
%\end{picture}
%\caption{Version 6.x TOPtesi bundle flow diagram}
%\label{fig:key-value-option}
%\end{figure}
%
% Special care was paid to maintain backwards compatibility;
% the various (|.sty| module files with descriptive names)
% should be usable also with other classes, if desired.
% It is possible to specify a |custom| value for a thesis type; 
% in this case the user is required to explicitly load his/her
% personal package, or to avoid loading anything and use
% either a different typesetting program to produce a title
% page (in \sigla{pdf} format to be included by means of
% the |pdfpages| package), or even create his/her personal
% title page by means of the standard |titlepage| environment.
%
% All this implied the rewriting of both the English and
% Italian documentation files; the user can find useful
% information in the English file |toptesi.pdf| included
% in the bundle; this file is pretty lengthy because it
% contains also the documented code. The Italian documentation
% in |toptesi-it.pdf|, is also part of this bundle; it is
% partly written in English for the benefit of the many foreign
% students that study in the doctoral school of Politecnico di
% Torino); it contains many examples enriched with graphic
% ``demonstrations''. These two documentation files complement
% one another.
%
% \section{Introduction}
% The TOPtesi bundle contains a certain number of files.
% \begin{enumerate}[noitemsep]
% \item
%   A class file \texttt{toptesi.cls} to be used as the main
%   document class; the main document may be any of a certain
%   number of reports that in Italy are called with various names:
%   ``monografia'', ``monografia di laurea'', ``tesi di laurea'',
%   ``tesi di laurea triennale'', ``tesi di laurea specialistica'',
%   ``tesi di laurea magistrale'', ``tesi di dottorato'',
%   ``dissertazione di dottorato'', and so on.
%   All these documents have in common the fact that they all
%   conclude a period of university education. Moreover since they
%   may conclude a double degree university program, they may receive
%   foreign names such as, for example, ``Projet de fin d'\'etudes'',
%   ``Master thesis'', and the like.
% \item
%   An extension package \texttt{toptesi.sty} that contains most
%   of the code for the real typesetting; it might be used as an
%   extension to other classes, although this use is seldom necessary.
% \item
%   A second extension package \texttt{topfront.sty} that contains
%   the commands and the typesetting macros for the title page; this
%   file may be used as an independent extension package to be added
%   to, say, the report or the book class files for typesetting just
%   the title page; this file might be used as a template for setting
%   up the title page fixed information in languages different from
%   Italian. This package is part of the bundle, but it is
%   automatically loaded only if the user did not specify the
%   |tipotesi=|\meta{thesis type} option to the class; if the
%   |toptesi=frontespizio| option is specified, the external package
%   |frontespizio| gets loaded; if the |toptesi=custom| option is
%   specified the user should everything by him/herself.
%   The |tipotesi=custom| is very tolerant; in the sense that the user
%   is allowed to avoid loading any custom package, and simply create
%   the title page by means of the standard |titlepage| environment.
%  \begin{itemize}[noitemsep]
%    \item \opz{tipotesi=triennale} (or \opz{monografia} for
%          backwards compatibility) for the bachelor degree final work;
%    \item \opz{tipotesi=magistrale} for the master thesis;
%    \item \opz{tipotesi=dottorale} for the Ph.D.\ thesis;
%    \item \opz{tipotesi=scudo} for the special version of the Ph.D.\
%          thesis at Politecnico di Torino;
%    \item \opz{tipotesi=secondaria} for the high school ``tesina'';
%    \item \opz{tipotesi=custom} for using another title page package,
%    	 possibly  a customised one the user may create in
%    	 accordance with a certain template that implements his/her
%    	 university requirements.
%     \item \opz{tipotesi=frontespizio} for typesetting a
%    	 frontispiece according to the style of the |frontespizio|
%    	 package.
%  \end{itemize}
%   Without using a custom module, we suggest to use the package
%   |frontespizio| and we address the user to its documentation.
%  
%   It is important to recall that if the |topfront| package has
%   to be loaded, it is only at the execution of the |\begin{document}|
%   command. Therefore no title page commands of any kind (those
%   defined by |topfront| or the other modules) can be used in the
%   preamble.
% \item
%   A third extension package \texttt{topcoman.sty} that defines
%   a certain number of user commands suitable for typesetting
%   technical matters.
% \item
%   Several  extension files are available:
%   \texttt{toptesi-\meta{module}.sty} where \meta{module} is one
%   of |monografia|, |magistrale|, |dottorale|, |scudo|, |sss|;
%   these are specifically designed to typeset particular
%   thesis kinds; |scudo| refers to the particular doctoral school
%   of Politecnico di Torino, whose official name is \emph{Scuola di
%   Dottorato}, whence the acronym ``ScuDo''. Such ScuDo theses must
%   be typeset in English only, and must have a special title page,
%   very different from the other title pages that can be typeset with
%   the various options of the TOPtesi bundle and by the
%   commands contained in the  various modules. In all cases where
%   specific modules exists, a specific module option is available for
%   the class such that \texttt{topfront.sty} does not get loaded, and
%   another set of commands is made available to set up
%   the necessary information.
% \item
%   Previous versions of this bundle contained also the logos of
%   a set of universities; these logos are not distributed anymore
%   because of legal constraints. Every user who is working on
%   his/her degree course final report may retrieve the logo of
%   his/her university, but s/he should pay attention to use it
%   according to the rules and limitations of the university.
% \item
%   A documentation file \texttt{toptesi-it.pdf} written mostly
%   in Italian, explains in detail every feature; essential information
%   is given in this English documentation. The source file of the
%   documentation \texttt{toptesi-it.tex} may be used as a sample
%   or template for typesetting one's thesis with any of \pdfLaTeX,
%   \XeLaTeX, or \LuaLaTeX.
% \end{enumerate}
%
% The above files are complemented with a configuration file
% that any user may customise at will; this customisation file
% makes it easy to configure the bundle so as to make it suitable
% for another language; in facts the \textsf{babel} and
% \textsf{polyglossia} packages contain localisations for many
% languages, but such localisations deal with the standard infix
% \LaTeX\ names and phrases and do not cope with the thesis
% title page requirements. This configuration file is usable only
% when no custom or external files are used for typesetting
% the title page; but they may be used for all standard modules
% of this bundle.
%
% TOPtesi was specifically conceived for typesetting theses
% with the \LaTeX\ mark-up, and initially it was using the
% \textsf{tex} typesetting engine; later on this engine was
% substantially substituted by the \textsf{pdftex} one, which
% was capable of direct output of \sigla{PDF} files. Since about 2006
% the typesetting engine \XeTeX\ has been available; the most important
% feature of this engine is its capability of using OpenType
% fonts, among which those that are available to the
% operating system of the specific platform where the document
% is being typeset.  Since 2017 a stable version of \LuaTeX\ is
% available that does much more than \XeTeX.
%
% For what concerns theses these programs might be of essential
% importance when they deal with specific languages that use
% different scripts (Greek, Russian, Bulgarian, Chinese, Japanese,
% Korean, Hebrew, Arabic, Farsi, Thai, and so on). Of course
% \LuaLaTeX\ has many more functionalities and can perform very
% well where the other engines are sort of limited; in particular
% \LuaLaTeX\ can produce \sigla{ISO}-compliant archivable-documents more
% easily than the other typesetting engines; but the user does not
% have to learn too many more concepts in order to typeset a 
%``normal'' thesis.
%
% This version of TOPtesi has been tested also with \XeLaTeX\ and
% \LuaLaTeX. Some conflicts have been spotted and solved; may
% be there are still hidden ones, so that feedback from users
% is particularly welcome. The only main drawback still present
% when running \XeLaTeX\ is the fact that this program cannot still
% \emph{directly} produce the output file in \sigla{PDF} format,
% although it automatically transforms its specific output file
% into that format.
% This implies that the specific \textsf{pdftex} features required
% to produce a \sigla{PDF/A} compliant output \sigla{PDF} file suitable for
% long term archiving cannot be used. But with some attention the
% \sigla{PDF} file output by \XeLaTeX\ may be converted to \sigla{PDF/A}
% by using the \pack{pdfpages} package.
%
% \section{User commands}
% The \texttt{toptesi.cls} is basically an extension of the
% standard class \texttt{report.cls}; it redefines the page
% typesetting grid, the headers and the footers, and the title
% page layout and commands. \texttt{toptesi.cls} does not set
% such crazy settings as ``double spaced'' text; it is intended
% to typeset the thesis with the quality of a \LaTeX\ well typeset
% document, not as a typewriter written one.
%
% Theses very often are full of specialised material: formulas,
% diagrams and pictures, texts written in non Latin alphabets,
% special symbols for philological mark-up, and the like; a
% common typewriter would not be suitable and the quality of
% the contents requires professional typesetting; this is why
% I strongly believe that instructions on typesetting styles that
% refer to the ``good old gone-by times'' of mechanical typewriters
% should be banned.
%
% At the same time the class allows to typeset theses on any paper
% format; nowadays, in facts, Universities are requesting theses
% in smaller formats than A4 or letter paper. With smaller paper
% sizes the layout changes automatically, but the title page might
% require some more attention. The default paper size is A4, but
% the user can set any paper size among those accepted by the
% |report| standard class. If the \opz{tipotesi=frontespizio}
% option was specified so that package |frontespizio| gets
% loaded, it is necessary to remind that it can easily create
% pretty nice title pages, but by default it is preset to use A4
% paper where it can typeset the title page in two styles, one so
% called ``standard'' and the other called ``elements'', that
% mimics that of the famous book \emph{Elements of typographic
% style} by Robert Bringhurst; this is so at least with
% |frontespizio|'s version 1.4a of 2011/09/21 or earlier. It is
% possible to use |frontespizio| to use a different paper size,
% but it's necessary to use certain customisation commands
% described in its documentation.
%
% Therefore if you want to typeset your thesis while typesetting
% the title page by means of the functionalities of package
% |frontespizio|, you have to follow carefully how to use that
% package's options in the proper way as to bypass the two fixed
% styles  it can produce. See also the Italian documentation
% contained in the document |toptesi-it.pdf|.
%
% Most new commands refer themselves to the information that should
% be typeset in the title page; some class options specify special
% stylistic page details; the rest is simple and traditional
% \LaTeX\ mark-up as it is implemented in the \LaTeX\ kernel and
% in the report class. If \XeLaTeX\ or \LuaLaTeX\ has to be used,
% some essential preamble specific commands are to be used, but
% the body of the thesis has the same mark-up.
%
% For theses to be typeset for the ScuDo doctoral school,
% all necessary commands become available when the class
% option \opz{tipotesi=scudo} is specified; such commands are
% available \emph{only} with that option. See the documentation
% in the suitable section~\ref{sec:scudo}. Similarly see the suitable
% sections for the special thesis modules available with their
% specific options. 
%
% \subsection{Class options}
% The class accepts all the options accepted by the |report|
% document class plus the ones defined here:
% \begin{description}[noitemsep]\def\Item[#1]{\item[\opz{#1}]}
% \Item[chapterbib] Allows to typeset a list of references at the
% end of each chapter and the bibliography items are numbered with
% a \emph{chapter.item} indication. This implies a manual build
% up of each end-of-chapter bibliography by means off a specific
% |thebibliography| environment. This work may be avoided if the
% user relies on the various packages already available in any
% complete \TeX\ system distribution and the facilities offered
% by packages such as \pack{biblatex} and sorting and formatting
% engines such as \prog{biber}. The use of this option, kept for
% backwards compatibility, is strongly deprecated.
%
% \Item[classica] Specifies a general modification of certain
% details that are supposed to be  more adequate for theses in
% humanities; specifically this option lets old style numbers to
% be used for some numerical pieces of information; some variations
% are also introduced in the title page. This option is active also
% when using some of the modules for specific theses, in particular
% with the \textsf{dottorale} thesis option.
%
% \Item[cucitura=\meta{displacement}]  In two sided printing
% it is better to move the typesetting grid towards the outer
% edge so as to cope with the thesis binding that is generally
% not made up by sewing together a number of signatures; the
% default outer displacement is fixed to~7mm, but it can be
% customised by means of a proper value to the class option
% |cucitura=|\meta{displacement} or by using  an explicit command
% in the preamble: \cs{setbindingcorrection}\marg{displacement}.
% The \meta{displacement} specification is necessary but it
% remains inactive if the key \texttt{cucitura} is absent while
% using the \cs{setbindingcorrection} command.
%
% \Item[corpo=\meta{font size}] Extends the normal size
% choice to the chosen \meta{font size}; it is
% appreciated in certain fields of humanities, but I would
% discourage this use in a technical thesis, where 10 point
% perhaps is too small, but 11 point or 12 point typesetting
% may be adequate. In any case the \meta{font size} may be
% specified in any units the \TeX\ system accepts, and any
% value, even fractional, may be used. Please, do not
% exaggerate with small sized normal fonts, where I would
% say that 9.5pt is the absolute minimum acceptable size for
% normal text. Do not exaggerate with large sizes; 14pt is
% acceptable for certain theses where unusual signs are
% used above or below some letter strings, otherwise 14pt
% is already too much; let us say that 12.5pt or 13pt are
% the maximum normally accepted; larger sizes are used only
% for books to be used in the first years of the elementary
% school or for people with various kinds of visual impairment. 
%
% \Item[autoretitolo] This option modifies the left hand (even
% numbered) pages in two side typesetting; normally the even
% numbered page headings contain the chapter title, while the
% odd numbered ones contain the current section title.
% If the \opz{classica} option has been specified, then
% with this option (that is effectively working only with the
% \opz{classica} option in force) it is possible to have even
% numbered page headings  contain the author's name and the thesis
% title, while the odd numbered ones contain the chapter title.
% Since the thesis title might be too long to fit in the header
% together with the author's name, the |\title| macro as been
% redefined so as to accept an optional short title, similarly
% to the ordinary sectioning commands.
%
% \Item[oldstyle] Also this option works only if \opz{classica}
% had already been specified; it typesets several numerical data
% with the old style numbers.
%
% \Item[numerazioneromana] Sets the front matter folios in roman
% numerals; if this option is not used, the folios are in arabic
% numerals and do not restart from~1 when the first main matter
% chapter is typeset. The traditional roman numbering of front
% matter folios had its explanation when typesetters used metal
% type; therefore they could produce all the front matter material,
% including reference, tables of contents, and the like, only
% when the main matter was complete, corrected and stable.
% Nowadays with electronic typesetting there is no need to need
% to use this artifice, but in some disciplines it is still
% highly appreciated.
%\end{description}
%
% Since version 6.xx.yy options to the TOPtesi class have the
% syntax \emph{key${}={}$value}; depending on the kind of thesis,
% there there are some vlues that are particularly significant;
% the key for all of them is |tipotesi| (\emph{thesis kind});
% the values are indicated hereafter.
%
%\begin{description}[noitemsep]\def\Item[#1]{\item[\opz{#1}]}
%
% \Item[tipotesi-topfront] is the default and it is not necessary
% to specify it. This means that if you specify |tipotesi| without
% a value, or if you completely omit this option, the generic
% |topfront| module is used to typeset the title page.
%
% \Item[tipotesi=frontespizio] omits loading the generic module
% |topfront| and loads the external package |frontespizio|'
% without any option; this implies that the user should read its
% documentation and use it accordingly.
%
% \Item[tipotesi=scudo] sets up the suitable switches and allows
% loading the specific extensions to typeset doctoral theses for
% the ScuDo doctoral school. This implies that not only the title
% page is typeset according to this School regulations, but that
% some other packages get preloaded; the user can read further
% on what actually becomes available with this option; the most
% relevant difference is that the default language becomes English
% also for the naming of the user commands..
%
% \Item[tipotesi=triennale] and \opz{tipotesi=monografia}
% (aliases of one another)  load the specific module to
% typeset the bachelor degree final-work title-page. Some
% user commands are redefined so as to be more user friendly
% than those of the generic package |topfront|.
%
% \Item[tipotesi=magistrale] loads the specific module
% to typeset the master degree thesis title-page. This module
% is substantially equivalent to the generic module |topfront|
% except that typesets only the master degree thesis title page.
%
% \Item[tipotesi=dottorale] loads the specific
% module to typeset doctoral thesis title-pages. This module is
% very basilar and has nothing to do with the module for the
% ScuDo doctoral theses that require a particular title-page
% layout; we suggest to customise this module through a suitable
% configuration file, rather then tweaking the ScuDo module.
%
% \Item[tipotesi=secondaria] loads the module to typeset the
% secondary school (high school) final work title page. The
% default language, also for the user commands, is Italian.
% Of course the secondary school final work title page is very
% different from the one suited for university theses. In any
% case, except for few additional packages, the body of the
% TOPtesi functionality is maintained intact.
%
% \Item[tipotesi=custom]
% is a very simple option: sets the various internal switches
% so as not to load any internal module or external file.
% If the user explicitly loads one before the |\begin{document}|
% statement, this file becomes the one that the user must follow
% to create his/her own title page; otherwise if the user wants
% a custom title page s/he uses the |titlepage| environment;
% in this case the user is completely free to write what s/he
% wants, where s/he prefers, using the fonts s/he likes, and
% so on. In alternative the user prepares the title page in
% \sigla{pdf} format with some other program and includes it
% in his/her thesis by means of the functionalities of the
% |pdfpages| package. In any case the user is not forced by
% this TOPtesi bundle to use any of its title page facilities.
%
% \item[Attention!] It goes by itself that each one of the last
% options does not load the generic module |topfront|, therefore
% the commands provided by this standard module are not available
% any more. Every effort has been made in the various modules to
% maintain the same commands, but in certain thesi kinds they
% might behave differently than expected. 
%
% The user therefore is strongly urged to consult the documentation
% files, both |toptesi.pdf| and |toptesi-it.pdf|, in order to
% avoid problems with commands that are not suited for a particular
% thesis type. Please examine also the examples that accompany
% this bundle
% \end{description}
%
% \subsection{Title page commands}
% Note: Skip this section if you decide to load an external
% package, such as, for example, |frontespizio|, for typesetting the
% title~page. Notice also that if you want to load an external
% package for this task, you should load it \emph{after}
% specifying the input encoding you use for your text files.\bigskip
%
% \noindent The user must specify a certain number of commands
% in order to have the title page contain all the required
% information. It must be specified that most of these commands
% may be used in the configuration file so as to avoid repeating
% the same data for different ``final reports'': a university
% student might write a bachelor's ``monografia'', then a master
% thesis and finally a doctoral dissertation;  why should s/he
% repeat his/her name, the name of the institution, and so on?
%
% All the user commands for the title page redefine default
% values or strings; therefore if none of the required information
% is given, the default values and strings are typeset, possibly
% with hilarious results\dots
% 
% Except for the self made customised module for the title-page,
% all the modules accept the same commands as those available in
% the |TOPfront| module. Pay attention that some commands are
% meaningless in certain theses, therefore do not reverse the
% statement: for example the doctoral cycle command (|\cycle|)
% is meaningless in a bachelor degree final-work title-page.
%
% Since most users are supposed to be Italian, the user commands
% are mostly in Italian; the following description gives their
% names and meanings; every command receives one argument; only
% the command |\title| accepts an optional argument according to
% the usual \LaTeX\ syntax:
% \[
%   \makebox[0pt]{%
%  \texttt{\char92}\meta{command}\oarg{opt-arg}\marg{req-arg}\relax
%               }
% \]
% Since the only language allowed in the ScuDo doctoral thesis is
% English, all its commands are in English; this was done to help
% the high number of foreign Ph.D. students in that school.
% Therefore, for example, the general environment |frontespizio|
% becomes |ThesisTitlePage|\footnote{Actually
% \texttt{ThesisTitlePage} is a wrapper to a particular instance
% of environment \texttt{frontespizio}. Similar wrappers are
% defined in the other specific thesis modules.}. Except this
% module, and, possibly, the personal customised module, the
% other ones use mostly the same commands for the same purposes.
%
% \begin{description}[noitemsep]
% \def\cs#1{\texttt{\char92#1}}
% \def\Item[#1]{\item[\normalfont\cs{#1}]}
% \item[\textnormal{\ttfamily frontespizio}] and
% \texttt{frontespizio*} are two environments that typeset the
% title page with two opposite styles: with the university logo
% in the header vs the university logo in the lower part of the
% front page. The commands described in the following items shall
% be used either in the configuration file or within the body of
% these environments.
%
% \Item[frontespizio] {\tolerance=6000 Even if the above
% environments are strongly recommended, the command
% |\frontespizio|, defined in the previous versions of the
% TOPtesi bundle, is still usable. It typesets the title page
% according to the actual status of the boolean |\topTPTlogos|;
% the user, before issuing the |\frontespizio| command, may set
% this boolean by means of the |\topTPTlogostrue|
% or |\topTPTlogosfalse|, or with either one of the commands
% |\booltrue{topTPTlogos}| and |\boolfalse{topTPTlogos}|.\par}
%
% \Item[monografia]  sets the bachelor's report style and retrieves
% its title; the command\cs{titolo} that is not available for this 
% purpose with the standard module |TOPfront|, is available when
% the specific module has been loaded by means of the
% \opz{monografia} or \opz{triennale} options.
%
% \Item[titolo]  gets the master or PhD thesis title and an
%  optional thesis short title.
%
% \Item[sottotitolo]  gets the thesis subtitle if any.
%
% \Item[materia] gets the name of the subject the thesis deals with.
%
% \Item[Materia]  alias for \cs{materia}.
%
% \Item[direttore] gets the name of the Doctoral School Director.
%
% \Item[coordinatore]  gets the name of the Doctoral School
%                 Coordinator
% \Item[QualificaDirettore]  gets the phrase that describes the
%                 director or coordinator official position; by
%                 using the command \cs{direttore} the default phrase
%                 ``Direttore della Scuola di Dottorato'' is printed
%                 above the ``director's'' name; if \cs{coordinatore}
%                 is used the default phrase ``Coordinatore della
%                 Scuola di Dottorato'' is printed instead.
%                 If neither one is applicable or a description 
%                 in another language is required, this macro is
%                 available for specifying such position.
%
% \Item[relatore]  gets the name of the thesis principal supervisor.
%
% \Item[secondorelatore] gets the name of the second supervisor,
%                 if any.
% \Item[terzorelatore]  gets the name of the third supervisor,
%                 if any; it is assumed that the number of 
%                 supervisors never exceeds three; in case follow
%                 one of the tricks exemplified
%                 in the example file \file{toptesi-example.tex} in
%                 order to manage such situation.
% \Item[tutore]   gets the name of the doctorate tutor; there is no
%                 difference with regards to the \cs{relatore}, but
%                 the default phrase ``Tutore'' is printed above this
%                 person's name.
% \Item[TutorName]  gets the phrase that describes the tutor
%                   position, possibly in a different language.
% \Item[AdvisorName] gets the string that qualifies the
%                   supervisor(s); the default string
%                   is ``Relatore:'' or ``Relatori:'' for the plural;
%                   in another language this command is used to
%                   define the string, say, ``Supervisors:'' if the
%                   thesis has been supervised by more than one
%                   person.
% \Item[CoAdvisorName] gets the string that qualifies the 
%                   co-supervisor(s); the default string is
%                   ``Correlatore:'' or ``Correlatori:'' in the
%                   plural; this command may be used to define the
%                   string, say, ``Corapporteur:'' in a French Projet
%                   de fin d'\'etudes. This command may be used also
%                   for setting an Italian singular name when the
%                   tricks exemplified in file
%                   \file{toptesi-example.tex} are used.
% \Item[candidato]  gets the name of the male author
% \Item[candidata]  gets the name of the female author
% \Item[secondocandidato]  gets the name of the second male author
% \Item[secondacandidata]  gets the name of the second female author
% \Item[terzocandidato]  gets the name of the third male author
% \Item[terzacandidata]  gets the name of the third female author;
%                   most often the thesis author is just one person;
%                   but there are some institutions where group final
%                   works are accepted; it is assumed that the group
%                   does not contain more than three authors. The
%                   specification of the gender allows the software
%                   to determine the correct labelling phrase in
%                   the proper gender and proper number.
%                   For different languages there might be no
%                   difference in gender but there is a difference 
%                   in the plural ending.
% \Item[CandidateName]  gets the string that describes the student
%                   status in a foreign
%                   language or even in Italian; the default string
%                   is `Candidato:'' (with colons) adjusted to
%                   masculine or feminine, singular or plural; with
%                   option \opz{classica} the string becomes
%                   ``Laureando:''; in other languages it is
%                   necessary to specify this string in the proper
%                   gender and number.
% \Item[sedutadilaurea] gets the date of the final exam, or
%                   presentation, or defence of the thesis; if this
%                   date is omitted the default date is the current
%                   month and year in the current language (generally
%                   Itallian).
% \Item[esamedidottorato] an alias for \cs{sedutadilaurea} to be used
%                   for doctoral dissertations.
% \Item[ciclodidottorato]  gets the roman numeral that specifies
%                   the doctoral cycle.
% \Item[CycleName]  redefines the string that expresses the name of
%                   the doctoral cycle; by default this is ``ciclo''
%                   but this command is useful to set the name in a
%                   different language.
% \Item[corsodilaurea] gets the proper name of the degree course;
%                   the phrase that describes the degree course is
%                   specified, if necessary, with the next command;
%                   with this one you specify just, say,
%                 ``Electrical Engineering''.
% \Item[CorsoDiLaureaIn]  gets the generic name of the degree course,
%                   for example ``Bachelor Degree in''.
% \Item[TesiDiLaurea] gets the generic phrase that describes the
%                   thesis; by default it is ``Tesi di Laurea''; in
%                   English one might set it to ``Master Thesis''.
% \Item[NomeMonografia]  gets the phrase that describes the
%                   bachelor's report; by default it is ``Monografia
%                   di Laurea''. In some Italian universities
%                   iˀt might be called
%                   ``Tesi di Laurea'', so that the master thesis
%                   should be given another qualification, for
%                   example`Tesi di Laurea Magistrale''.
% \Item[NomeDissertazione]  gets the phrase that describes the
%                   doctoral thesis; by default it is ``Tesi di
%                   Dottorato''.
% \Item[InName]     infix strings often require adjusting of the
%                   prepositions; this macro gets the preposition
%                   that stands for ``in'' (the default). In German
%                   it might become ``auf''.
% \Item[NomeAnnoAccademico] defines the infix string that stands for
%                   ``Academic year''. This macro is defined only if
%                   the option \opz{classica} is in force;
%                   after all the command |\annoaccademico| is
%                   defined only with that option. In any case, even
%                   without the \opz{classica} option you can use
%                   any string in the field of \cs{sedutadilaurea}.
% \Item[logosede]   specifies the name of the file or the files that
%                   contain the university logos; no  default is
%                   defined; rather a warning message is issued if
%                   no name is given or the file is missing, but
%                   typesetting goes on without the inclusion of any
%                   logo. A list of logos can be specified, useful
%                   when a thesis is carried on in a multiple
%                   University environment such as, for example, in
%                   a double degree Erasmus program; or under the
%                   Erasmus Mundus program. The ``string'' of
%                   logos is scaled properly so that they may fit
%                   in |\textwidth|.
% \Item[setbindingcorrection] sets up the length to displace the text
%                   block to the external margin so as to have a
%                   wider internal margin to accommodate the binding
%                   correction. Its argument is not optional and is
%                   used to modify the default correction of 7\,mm.
%                   Notice that 7\,mm is already a large 
%                   displacement; most often than not the binding
%                   correction is unnecessary.
% \Item[retrofrontespizio] with its argument, made up of one or more 
%                   paragraphs, defines what should be printed on the
%                   verso of the title page, generally named
%                   ``copyright page''; if this command is specified
%                   with an empty text or if it is not used at all,
%                   no copyright page is assumed.
% \end{description}
%
% Since the infix strings are all memorised into control sequences
% and for each of them it is possible to use a defining command,
% all strings can be modified at will, so that there is no
% difficulty to localise the package in another language; this
% comes particularly handy for the Erasmus students on double
% degree programs.

% As a final remark notice that the commands for typesetting the
% title page and the copyright page are contained in the 
% package \texttt{topfront.sty}, which can be used as an autonomous 
% extension to the \texttt{report} document class. One could easily
% typeset either just the title page with a separate \TeX\ source
% so as to test the completeness of the commands and coherence of the 
% configuration file, or for just printing the isolated title and 
% copyright pages (if any). The same applies if you use the special
% modules for each thesis type.
%
% \section{ScuDo doctoral theses}\label{sec:scudo}
% Theses at the ScuDo doctoral school require special facilities
% that are described in the code description section; here we
% focus on the title page.
%
% The title page for such special theses is typeset by making
% use of a specific environment within which the following
% commands are available to set up the necessary information.
% \begin{description}[noitemsep]
% \def\cs#1{\texttt{\char92#1}}
% \def\Item[#1]{\item[\normalfont\cs{#1}]}
%
% \item[\normalfont\ttfamily ThesisTitlePage] is the environment
% within which the title page data are specified; some data are
% preset to default values; some data receive global definitions
% so that they remain available for setting up some metadata.
% In the following every command with its argument is described
% and it is specified which is the preset value, if any. If a
% preset value is specified, the user does not need to use the
% command, unless s/he wants to change the preset value. If an
% empty preset value is specified, the command and its value
% may be optional; but personal data are never preset because
% they are, obviously, mandatory.
%
% \Item[PhDschoolLogo]\marg{logo file name} (preset: Logo-Scudo) 
% selects an image
% specified by \meta{logo file name} to insert the school logo;
% a file logo \file{Logo-Scudo.jpg} is available from the ScuDo
% doctoral school; the students must ask such logo file from
% the Secretary Office, because such file is legal property of
% the School and can be used only with permission. In any case
% the logo file must be of one of the allowed formats: \sigla{PDF}, \sigla{EPS},
% \sigla{JPG}, \sigla{PNG}; vector files are generally preferred, but they
% might not be available; raster files should be of a minimum
% definition of 150 pixel per inch.
%
% \Item[Ndissertation]\marg{dissertation name} (preset: Doctoral
% Dissertation) Specifies the specific name of the document.
%
% \Item[Ndoctoralprogram]\marg{doctoral program} (preset: Doctoral
% Program in) specifies the degree course name prefix. 
%
% \Item[ProgramName]\marg{program name} (mandatory: preset empty)
% specifies the degree course name.
%
% \Item[CycleNumber]\meta{ordinal number} (optional; preset
% empty) specifies the ordinal number of the program cycle; at
% ScuDo it is a mandatory information. In other doctoral schools
% it might be useless.
%
% \Item[title]\marg{dissertation title} (preset empty) sets the
% specific title of the dissertation; it is better to keep it
% short, and use the sub title to add further titling information.
%
% \Item[subtitle]\marg{dissertation subtitle} (optional: preset
% empty) adds a possible subtitle.
%
% \Item[author]\marg{author's name} (mandatory: preset
% empty); it is recommended to pay attention to insert proper
% name(s) first and surname(s) second. 
%
% \Item[Nsupervisor]\meta{Supervisor(s) heading} (preset:
% Supervisor:) specifies the short heading labelling the list
% of supervisors; the preset value is singular and followed by
% a colon. In case of more than one supervisor, it is necessary
% to specify a plural name; the colon mark is optional.
%
% \Item[SupervisorList]\marg{list of supervisors} (mandatory:
% preset empty) specifies the list of supervisors, one per line,
% with a |\\| at the end of each line; the name(s) and surname(s)
% are preceded by their academic title, and followed by their
% role as supervisor, or co-supervisor.
%
% \Item[NExaminationCommittee]\marg{examination committee
% heading} (preset: Doctoral Examination Committee:) specifies
% the heading of the list of examiners.
%
% \Item[ExaminerList]\marg{list of examiners} (preset empty)
% specifies the list of examiners, one per line, each line
% terminated by |\\|; each line contains the academic title,
% the name(s) and surname(s), the role (referee for the
% supervisors; nothing for the other examiners), the academic
% institution they belong to -- full official institution name,
% city and country.
%
% \Item[Nlocation]\marg{examination location} (preset:
% Politecnico di Torino) specifies the institution where the
% examination takes place.
%
% \Item[ExaminationDate]\marg{examination date} (preset empty)
% specifies the date when the examination is to take place.
%
% \Item[Disclaimer]\marg{disclaimer statement} (preset: some long
% sentences agreed upon by the ScuDo doctoral school) mandatory
% in most universities; the specific statement(s) may be different
% in different universities.
%
% \Item[Signature]\marg{the author's signature} (preset: a few
% lines where the city is preset to ``Turin'', and the date is
% assumed to be equal to the examination date; also the signer
% name is assumed equal to that of the dissertation author)
% specifies a different layout of the signature field.
%
% \end{description}
% Several examples of thesis title-page are shown in the
% \texttt{toptesi-it.pdf} Italian"-English documentation.
%
% \subsection{Typesetting commands}
% The TOPtesi bundle and the |toptesi| document class accept
% all \LaTeX\ commands provided by the \LaTeX\ \texttt{kernel},
% the \texttt{report} document class, and the \texttt{graphicx}
% extension package, besides those provided by the \texttt{babel}
% package. If the source thesis file is being typeset by means
% of \XeLaTeX\ and \LuaLaTeX\ the \texttt{babel} package is not
% loaded; in its place the \texttt{polyglossia} package gets
% loaded that should implement in \XeLaTeX\ and \LuaLaTeX\ most of
% the functionality provided by \texttt{babel} in \LaTeX.
% ``Most'' means that not  all the functionality is available,
% therefore it is better to  consult the documentation of
% \texttt{polyglossia} before using its built in commands.
%
% With this respect it must be underlined that the Italian and
% English languages are specified by default, the Italian one
% being the main language. Only with the \opz{scudo} option
% in force, the main language is English. An initial specification
% of |\selectlanguage{english}| sets the English language as the
% default (unnecessary when \opz{scudo} is in force).
%
% Should a student typeset the thesis in French by means of
% \prog{pdflatex}, it would be necessary to specify the option
% \opz{french} among the \emph{class} options, and then start
% the document by specifying |\selectlanguage{french}|.
% But the user should pay attention to use \texttt{babel} in the
% proper way.
% \begin{enumerate}[noitemsep]
% \item Due to the way \LaTeX\ classes load the requested files,
% and to the fact that the \texttt{babel} package has already
% been loaded by the \texttt{toptesi} class, the user cannot
% reload it with a different list of language options; therefore
% the latter language options \emph{must be specified} as a global
% class options; so if the thesis has to be typeset, for example,
% in French, is is necessary to do the following:
%\begin{verbatim}
%\documentclass[...,french,...]{toptesi}
%...
%\begin{document}
%\selectlanguage{french}
%\end{verbatim}
% But if the thesis should be typeset in French by means of
% \XeLaTeX\ or \LuaLaTeX, then it is perfectly legal to specify the
% auxiliary language in this way:
%\begin{verbatim}
%\documentclass[...]{toptesi}
%\setotherlanguage{french}
%...
%\begin{document}
%\selectlanguage{french}
%\end{verbatim}
%
% \item With the 2010 \TeX\ system complete distributions,
% both \TeXLive and MiKTeX, all language hyphenation rules
% known (to \LaTeX) are preloaded; with other less up to date
% \TeX\ system distributions this might not be true. With both
% distributions also the language hyphenation rules known to
% \XeLaTeX\ are all preloaded. On the opposite \LuaLaTeX\ 
% loads the hyphenation rules at run time and only for the
% declared languages for the current document. 
%
% Remember that \texttt{babel} and \texttt{polyglossia} macros
% select the language typesetting rules, but hyphenation is
% activated only if the program format file has been generated
% with the pertinent language hyphenation rules. You can check
% this detail by reading the first dozen lines of your thesis
% \texttt{.log} file; there the list of all language hyphenation
% rules that are available in the format file is being shown.
%
% While (American) English is the default language and almost
% any \emph{basic} distribution of the \TeX\ system has several
% preloaded languages, it is more likely that French is preloaded
% while Italian is not. Complete distributions don't exhibit
% this~flaw.
%
% Should the required language(s) be missing, the user is forced
% to read his/her distribution instructions, so as to find out
% how to configure his/her system in order to preload the
% languages s/he wants to work with, and finally s/he must recreate
% the format files.
%
% The user is invited to carefully investigate on these fine
% points and to properly configure the system; it would be very
% upsetting to use fine software to produce a perfectly typeset
% thesis that, unfortunately, has wrong hyphenation points!
% Luckily enough, most recent distributions of the \TeX\ 
% system have all the known hyphenation rules preloaded; in any
% case, even older distributions have available command-line
% commands or graphical user interfaces that make it easy to
% perform the tasks of changing the list of preloaded hyphenation
% rules and rebuilding all the format files, moving them to the
% proper places.
% \end{enumerate}
%
% The TOPtesi bundle adds very little to the user commands;
% nevertheless the package \texttt{topcoman.sty}, that is part
% of this bundle and is automatically loaded, defines some useful
% commands for typesetting technical matters in such a way as to
% fulfil some (not all)  \sigla{ISO} regulations.
% Some of these commands are already
% defined with the \texttt{babel} Italian option, but if your
% thesis is written in different languages it may happen that such
% commands are not available any more when you select another
% language; with the presence of the definitions contained in
% \texttt{topcoman.sty} such useful commands should remain available
% with every language. The \texttt{polyglossia} package does not
% produce any useful additional command for writing in an almost \sigla{ISO}
% compliant way\footnote{Nevertheless, when typesetting with
% \XeLaTeX\ or \LuaLaTeX\ and using OpenType fonts, even for
% typesetting math, the calling of the |unicode-math| package
% with option \opz{mathstyle=\sigla{ISO}} lets you typeset in an \sigla{ISO}
% compliant way almost everything you need.}, therefore the macros
% contained in the \texttt{topcoman.sty} package may come very handy. 
%
% The following description specifies these particular commands.
% \begin{description}[noitemsep]
% \def\Item[#1]{\item[\normalfont\texttt{\char92#1}]}
%
% \Item[DeclareSlantedCapitalGreekLetters] does exactly what
% its name means: it changes the definitions of the mathematical
% capital Greek letters so that they are typeset in ``italics'';
% they are actually taken from the math italic alphabet,
% instead of the default roman one. This command is useful with
% \LaTeX, while with \XeLaTeX\ and \LuaLaTeX\ it is unnecessary
% due to the larger set of math alphabets and math font commands
% that are available with proper \sigla{UNICODE} math fonts and
% options.
% 
% \Item[ensuremath] should be already defined in the \LaTeX\ kernel;
% should one still be using an obsolete version, this command gets
% available anyhow.
% %
% \Item[ohm] typesets an upright capital omega even if the
% capital Greek letters are in italics; another good point
% is that |\ohm| can be used also  in text mode.
% %
% \Item[ped] inserts an index/subscript in upright type; the \sigla{ISO}
% regulations require the use of italics for physical or
% mathematical variables, and upright type for anything that
% is not a variable, from the names of functions (such as sin,
% cos, log, etc.), to the indices that contain information
% on something that is not variable. This means that $V_i$
% requires an italic index to imply that the object $V$ is
% the $i$-th in a set, while, say, $V_{\mathrm{max}}$
% indicates the maximum value of the variable $V$. This
% command |\ped| may be used  in both math and text mode.
% %
% \Item[ap] similarly |\ap| inserts an apex/superscript in
% upright type, in both math and text mode.
% %
% \Item[unit] sets the unit of measure close to the numerical
% measure value by inserting a non breakable thin space and
% by setting the units of measure in upright type; this works
% both in math and text mode. Of course it is necessary to
% input the |\unit| command without intervening spaces in the
% source file; it's necessary to typeset, say, |35\unit{km}|
% while avoiding a form such as \verb*+35 \unit{km}+.
% This command, as it is defined, conflicts with the definition
% of the homonymous command |\unit| as defined by the
% \texttt{unitx} package, but since this latter package is
% necessarily input in the preamble, while \texttt{topcomand.sty}
% is read at the execution of |\begin{document}|, the latter
% file has provisions to test  if |unitx| has already been loaded,
% so as not to overwrite the latter package definition and to avoid
% every possible conflict.
% %
% \Item[micro] sets the decimal prefix $\mu$ (providing, if
% possible, the sign with serifs, \textmu, taken from the Text
% Companion font or any other font that contains this variety
% of the Greek letter $\mu$) when typesetting units of measure.
% %
% \Item[gradi] sets the small circle that defines the sexagesimal
% degrees, for example $35^\circ$; it may be used also for the
% celsius degrees by writing in the source file, say,
% |35\unit{\gradi C}| in order to get $35\,^\circ\mathrm{C}$;
% notice the difference with |35\gradi|, that yields $35^\circ$
% without any space between the measure and the ``degree'' little
% circle; this is what the \sigla{ISO} regulations require; the small
% space inserted by |\unit| is required when the unit of measure
% symbol contains at least one letter.
% %
% \Item[gei] inserts the imaginary unit in upright type with the
% ``spelling'' used by the technologists: ``j''. This command may be
% redefined, of course, but this strange name is due to the fact that
% nowadays the letter ``j'' in Italian is called with the
% one-syllable English name (much shorter than the traditional
% Italian three syllable name ``i~lunga'') and the indicated spelling 
%``gei'' is the phonetic Italian rendering of the English word.
% The imaginary unit is not a variable, and the \sigla{ISO} regulations
% require it is typed with an upright serifed font, just as
% operators are.
% %
% \Item[eu] inserts the Napier number symbol ``e'' in upright
% type; since this entity is not a variable, but it is
% a mathematical constant, the \sigla{ISO} regulations require it to
% be written in upright type. The \sigla{ISO} regulations require the
% upright type for ``e'' and any other mathematical constant,
% but the electron charge $e$ is typeset in math italics because
% this is a physical ``constant'', not a mathematical one.
% \XeTeX\ and \LuaLaTeX\ allow to typeset upright math Greek
% letters, so there is the facility to typeset an upright
% ``pi'' (the number) to be distinguished  from an slanted
% ``pi'' (the angle).
% %
% \Item[goodpagebreak] inserts a conditional page break after
% the point where this command appears; it is not equivalent to
% the |\raggedbottom| declaration that does not vertically
% justify any page, but tests the amount of space remaining
% on the page, and if it is less than a given number of lines,
% it inserts a page break; the result is a ragged
% bottom single page, that sometimes is better than excessive
% spacing between paragraphs, or within items of a list, or
% before or after displayed material. It is handy also to use
% before starting a new section, because at least four
% lines are required by \LaTeX\ in order to avoid orphan lines.
%
% \Item[listing] requires for its argument the name of a file
% and typesets it in verbatim mode; this command is very useful
% for typesetting the listings of the programs that were written
% for the thesis; for best results it is recommended that the
% source program has lines not longer than 80 characters. The
% original definition of command |\listing| dates back to the
% old times of \LaTeX\,209; it has become obsolete; for
% backwards compatibility it is maintained as an alias to the
% command |\VerbatimInput| of package |fancyvrb|; its
% functionality may be configured with the same commands
% specified in the |fancyvrb| documentation for the original
% command. The |fancyvrb| package was preferred to the |listings|
% package, because it typesets verbatim material even while
% using font with multibyte encoding (therefore it works fine
% with \XeLaTeX\ and \LuaLaTeX), while the fancier package
% |listings| performs correctly only with \sigla{ascii}
% characters and requires special settings to use characters
% encoded with more than 7~bits.
% \end{description}
%
% All these commands are defined into the separate package
% \texttt{topcoman.sty} that might be used as an independent
% extension package with any document class.
%
% \section{The monografia module}
%
% The |toptesi-monografia| module gets loaded if one of the
% \opz{tipotesi=triennale} or \opz{tipotesi=monografia} options
% was specified to the |TOPtesi| class. It can be used as a
% standalone module, as well as the |topfront| general one, although
% supposedly, it is used most often with its mother class.
%
% In order to use it as a standalone module, some definitions
% must be established in order to have available at least some
% if not all of the settings and the switches that are available
% when the module is used with |TOPtesi|. Of course suitable
% tests are made in order to check if certain packages are
% already loaded and if certain switches are already defined;
% and some options are defined.
%
% The rest of the module is a subset of the |topfront| package;
% all commands that refer to information used for other kinds
% of theses are eliminated and the title page is typeset by means
% of another environment that does not rely on the environment
% |frontespizio| (with or without asterisk). The new environment
% |ThesisTitlePage| is more or less the only difference compared
% to the way a bachelor thesis title page is typeset with the
% |topfront| package. It will be described later on; but, with
% the due differences, this environment mimics the one defined
% in the |toptesi-scudo| module.
%
% The commands contained in this module are divided in there sorts.
% \begin{enumerate}[noitemsep]
% \item
%   The commands to define the infix words, so as to use them
%   explicitly within the |ThesisTitlePage|  environment or
%   within a configuration file  either to change the default
%   strings or to define them for languages different from
%   Italian; no attempt was made to establish such strings
%   for English, because is is supposed the most often than
%   not this module is used by Italian students to write their
%   bachelor degree final works in Italian universities;
%   nevertheless the possibility to change these strings is
%   always available.
% \item
%   The commands that defines the default values for every
%   command that may be used or may be skipped while setting
%   up the title page.
% \item 
%   The commands that define the values for the variable
%   information necessary for every individual thesis.
% \end{enumerate}
% As explained above the first group is used essentially to
% define or redefine the infix strings. The second group is to
% define mosts default values for information that might or
% might not be entered by the user; these values are tested
% against the |\empty| value in order to decide if using or
% neglecting them. The third group is the one that the user
% directly or through a configuration file uses for his/her
% personal data and his/her specific thesis.
%
% \subsection{The configuration file}
% This module for the bachelor thesis final work searches
% for a configuration file with the same name as the thesis
% main file and with extension |.cfg|. Internally this main
% file name is stored in the internal variable |\jobname|;
% therefore this module searches the file |\jobname.cfg|. If
% this file does not exist, the typesetting engine does
% not use any specific setup, except what has been defined
% in this module for the default settings.
%
% Such configuration file might contains the redefining commands
% for the infix strings, and/or the other commands that can be
% used within the |ThesisTitlePage| environment.
%
% If a configuration file was loaded, the commands its contains
% are overwritten by those that are specified within the
% |ThesisTitlePage| environment. So the user should pay attention
% to what the configuration file contains and to what s/he writes
% within the |ThesisTilePage| environment, but if s/he uses again
% a command already used in the configuration file, it should be
% for a good reason; simply s/he should remember that the last
% setting is the one that is actually used.
%
% \subsection{Module options}
% If the module is loaded by |toptesi|, it receives the necessary
% options by |toptesi| itself. If the module is used as a self
% standing one and called by another class, it may be called with
% specific options that are listed here. In general they behave
% as the |toptesi| options; should they perform differently the
% details are explained hereafter.
% \begin{description}[noitemsep]
%\def\Item[#1]{\item[\opz{#1}]}
%
% \Item[evenboxes] The boxes containing the supervisor name(s)
%   and containing the candidate name(s) are typeset with 
%   their first lines aligned; if this option is not specified 
%   the boxes are skewed, so that there is enough place to set 
%   the supervisor signature(s) on the right and the candidate 
%   signature on the left; if the bachelor degree final work does 
%   not imply a supervisor, the candidate name is centred and 
%   there is no question of aligning any box.
%
% \Item[cucitura]
%   Specifies a displacement of the text body  towards 
%   the outer margin. See the |toptesi| description. This option,
%   differently from the one defined in |toptesi| and |topfront|
%   is not of the type \emph{key\,$=$\,value}; therefore in
%   order to change its preset value it is necessary to use
%   the \cs{setbindingcorrection} command. This remark  applies
%   only to the case when this module is called by another class.
%
% \Item[classica]
%   Uses some \opz{classica} features; see the |toptesi| 
%   description.
%
% \Item[autoretitolo]
%   Similarly with the main stream |toptesi| bundle, this
%   questionable option is for changing the internal page
%   style; it is useless for this module, but there are some
%   tests that derive from the general |topfront| package,
%   so that the option is maintained for compatibility purposes.
%
% \Item[oldstyle] On the opposite this option is maintained,
%   at least to typeset the academic year interval with old
%   style numbers.
% \end{description}
%
% \subsection{User commands and environments}
%
% The only user environment defined by this module is
% |ThesisTitlePage|. It plays the same role as the |frontespizio|
% environment, but it is incompatible with the |\frontespizio|
% command. In other words, if this module is used the user must
% use the |ThesisTitlePage| environment only and does not have
% available the various title page environments and commands
% available with the |topfont| package.
%
% This environment accepts an optional asterisk as an argument;
% that is the syntax is the following:
%\begin{ttsintassi}
%\cs{begin}\Marg{ThesisTitlePage}\meta{*}
%\meta{thesis definition commands}
%\cs{end}\Marg{ThesisTitlePage}
%\end{ttsintassi}
% where, differently from other \LaTeX\ environments that accept
% an optional asterisk, this optional token is not within the
% opening (and closing) environment braces, but it is the very
% first entered element into the environment body. Its function
% is the following: if the asterisk is missing, the university
% logo(s) are typeset at the top of the page; if the asterisk
% is present, the logo(s) are set in the page lower half, below 
% the possible subtitle.
%
% The commands available with this module, divided into the three
% groups described above, are the following.
% \begin{description}[noitemsep]
% \def\Item[#1]{\item[\normalfont\cs{#1}]}
% \item[String settings]
%   \begin{description}[noitemsep]
%   \Item[NomeAnnoAccademico] Sets the default string for the
%      ``Academic Year''; by default it is ``Anno accademico''.
%   \Item[NomeCorso] sets the default string ``Corso di laurea
%      in'' followed by an intelligent space.
%   \Item[AnnoAccademico] sets a different string in place
%      of the default ``Anno Accademico'' one.
%   \Item[Struttura di] sets the default structure prefix;
%      for example this prefix might be ``Department of'',
%      and the structure name entered with \cs{struttura}
%      might be ``Electrical Engineering''; therefore the
%      title page is typeset with the complete string 
%    ``Department of Electrical Engineering.  The default
%      string is in Italian, but by using this command,
%      remember to enter the string you need terminated by
%      the intelligent space macro \cs{xspace}.
%   \Item[CorsoDiStudi] sets the default string for the
%      degree corse type; it might be ``Laurea triennale'',
%      or ``Bachelor degree'' or similar strings in other
%      languages. It replaces the obsolete following command:
%   \Item[CorsoDiLaureaIn] that remains available with the backwards
%      compatible standard |topfront| module.
%   \Item[NomeElaborato] Sets the name that qualifies the
%      document; by default it is ``Monografia'', but this
%      user command may be used to set it, for example, to
%      ``Tesi di laurea triennale'', or anything else in
%      Italian, in English, or the used main language.
%      The |topfront| command |\NomeMonografia| il let to
%      to |\NomeElaborato| for backwards compatibility; its
%      use is evidently discouraged.
%   \Item[CandidateNames] sets the labelling command for the
%      box that lists the candidate names; by default these names are the 
%      masculine, feminine, singular and plural Italian names 
%      starting with ``Candidato''; if the \opz{classica} option 
%      is in force they start with ``Laureando'';
%      the user can use this command which it hass a non standard syntax
%      (delimited arguments); therefore it should be avoided by the
%      end users.
%      This command sets the four strings to any desired value, 
%      if, as in English, there is no difference between masculine 
%      and feminine, repeat the same name for the feminine positions, 
%      the last ones; the syntax is in facts:
%\begin{ttsintassi}
%\cs{CandidateNames},\% 
%\qquad\marg{masculine singular},\% 
%\qquad\marg{masculine plural},\% 
%\qquad\marg{feminine singular}\% 
%\qquad\marg{feminine plural},!
%\end{ttsintassi}
%   \Item[TitoloListaCandidati] is a much simpler command the user
%      should prefer to the previous one; in facts it accepts a single
%      argument formed by a comma separated list of names; from zero
%      up to four names (in the same order as described above).
%      Therefore the following commands are all valid ones:
%\begin{ttsintassi}
%\cs{TitoloListaCandidati}\Marg{}
%\cs{TitoloListaCandidati}\Marg{\'Etudiant}
%\cs{TitoloListaCandidati}\Marg{Estudiante,Estudiantes}
%\cs{TitoloListaCandidati}\Marg{Kandidat,Kandidaten,Kandidatin,Kandidatinen}
%\end{ttsintassi}
%   \Item[CandidateName] this command used to be the default one
%      with the standard |topfront| module, and remains available
%      with this package; it sets just one name (disregarding
%      singular/plural and masculine/feminine variants) to substitute
%      the default ones. Please notice that this command is not
%      available with the other thesis type options.
%   \Item[Relatore] Sets the supervisor string that labels
%      the supervisor box; by default it is the Italian
%      name ``Relatore''; the command my be used to set
%      any different string.
%   \Item[Correlatore] set the co-supervisor string that
%      labels the list of co-superpervisors when there is
%      only one ``second'' supervisor.
%   \Item[Correlatori] sets the co-supervisor string that
%      labels the list of co-supervisors when there are
%      three supervisors, one principal and two co-supervisors.
%    \end{description}
% \item[Variable data settings]
%   \begin{description}[noitemsep]
%   \Item[titolo] sets the bachelor degree final work title.
%      The main module |topfront| uses in its place the command
%      \cs{monografia}; the user can still use this command,
%      but |\titolo| is more descriptive.
%   \Item[sottotitolo] Sets the optional subtitle.
%   \Item[corsodistudi] and \cs{corsodilaurea} specify the actual
%      name of the degree course; \cs{corsodistudi} should be
%      preferred; \cs{corsodilaurea} is deprecated, but is maintained
%      for backwards compatibility.
%   \Item[materia] and \cs{Materia} set the optional subject
%      the thesis deals with.
%    \Item[relatore] sets the name of the only or the principal
%      supervisor.
%    \Item[secondorelatore] sets the name of a second supervisor,
%      if any.
%    \Item[terzorelatore] sets the name of a third supervisor,
%      if any.
%    \Item[candidato] sets the name of the first or only male
%      candidate; in general bachelor degree final works are
%      done by just one candidate, but in case there are
%      available also the commands for a second and a third
%      candidate. Masculine and feminine commands are available
%      and the heading of the candidates box il labelled accordingly.
%    \Item[candidata] sets the name of the first or only female
%      candidate.
%    \item[\normalfont\cs{secondocandidato} or \cs{secondacanddata}]
%       set the names of a possible second male or female candidate.
%    \item[\normalfont\cs{terzocandidato} or \cs{terzacandidata}]
%       set the names of a possible male or female third candidate.
%    \Item[ID] to be used just after the candidate's name, but as
%       part of the setting command  argument, establishes the
%      candidate's ID number (or matricola).
%    \Item[sedutadilaurea] sets the date of the final examination;
%      by default the current month and year are typeset if this 
%      command is not specified; but the user can specify anything; 
%      s/he can even prefix the year range with the 
%      \cs{AnnoAccademico} command, that typesets (in the current 
%      language if localisation has been made by means of the 
%      suitable command of group~1) a string equivalent to 
%      ``Academic Year''; in any case the user can write down what
%      s/he likes best in place of the predefined value.
%    \Item[NomeAteneo] sets the proper name of a specific
%      university; it will be typeset in suitable uppercase type
%      under the generic name of the university; if this command
%      is not used, no name gets typeset in the title page.
%    \Item[struttura] sets the name of the university entity
%      that supervises the development of the degree course;
%      it may be a ``faculty'', a ``department'', a ``school'',
%      it depends from the specific university. Some universities
%      do not require this information on the title page. In any
%      case if this command is used it accepts two arguments: 
%      the first one is optional, and, if used, is an uppercase 
%      roman numeral that indicates the ordinal of the structure;
%      the second argument is the complete structure name.
%    \Item[setlogodistance] sets the distance between adjacent
%      university  logos, if more then one is used.
%    \Item[logosede] Sets the name of a graphics file that
%      contains the logo of the university; the default value of
%      this command il \cs{empty}; this macro may be used several 
%      times, in case the bachelor degree final work is developed 
%      within a multiple degree Erasmus program; this might be 
%      unusual but not impossible. In any case if no logo file 
%      name is specified, no logos are printed and no space 
%      is used in the title page; if one or more logos are used 
%      they are assembled with the same height into a horizontal 
%      box, that before being printed is measured; if the width 
%      exceeds the text width, the whole box is scaled down by 
%      preserving the aspect ratio so as to fit it inside the 
%      text width; after these tests and possible scaling down, 
%      the logo box is printed. The logo distance my be varied 
%      from the 3em default value by using \cs{setlogodistance}, but 
%      it is suggested not to vary it too much. The \cs{logosede} 
%      accepts an optional value that is the common height of the 
%      logos; this possibility does exist but it is strongly
%      discouraged.
%    \Item[tutoreaziendale] set the name of a possible company
%      tutor, in case the final work is developed in a Company
%      external to the university
%    \end{description}
% \end{description}
%
% \section{The module for a master degree thesis}
%\label{sec:topmagistrale}
%
% The module for the master degree thesis,
% |toptesi-magistrale.sty| is almost identical to the generic
% module |topfront|; the obvious differences are that it
% lacks the latter module commands that refer to the bachelor
% degree or the Ph.D.~theses. Evidently such specific commands
% are useless in a master degree thesis.
%
% \begin{sloppypar}
% Such eliminated commands include |\manografia|,
% |\sedutadidottorato|, |\ciclo|, |\Direttore|, and so forth.
% \end{sloppypar}
%
% A tricky situation arises if the thesis is typeset in several
% tomes. There are substantially two different situations with
% different implementations with the generic module |topfront|
% and this specific module |toptesi-magistrale|.
%\begin{enumerate}[noitemsep]
%\item
%   With enumerate each tome title page is realised by means 
%   of the |\tomo| command that requires a preliminary 
%   specification of all the necessary data before |\tomo| 
%   can play its role; it actually uses the |\frontespizio| 
%   command, but before the |tomo| counter is stepped up.
%
%  This allows to typeset the various tomes in a single 
%  document, so that the user can cross-reference elements 
%  belonging to different tomes. When the single multi\-tome 
%  document is completed the user can split its \sigla{PDF} file 
%  in separate files (by means of the |pdfpages| package 
%  functionalities) and print them; in printed pages the 
%  cross-references do not require hyperlinking, while 
%  the single document file may be completely hyperlinked; 
%  reading the printed tomes is correct, and reading the 
%  whole thesis in a single document on the screen allows 
%  the easy navigation from one point to another.
%
%  The draw back of this procedure is that splitting the 
%  single \sigla{PDf} file in several file require another processing 
%  job, and any editing of the source thesis files may require 
%  repeating this splitting job because the number of pages 
%  may be modified and/or  cross references do not point 
%  to the correct page number.
%  
%\item
%  With the |ThesisTitlePage| environment of this 
%  |toptesi-magistrale| module the |\tomo| command is used 
%  before or within within the environment and it just 
%  steps up the |tomo| counter while allowing typesetting 
%  the necessary tome information on the title page.
%  
%  In this way it is necessary to typeset each tome with 
%  its own source file(s) to produce different output \sigla{PDF} 
%  files; when printed the are substantially identica to 
%  the split documents obtained with the previous approach 
%  and simply avoids splitting  the single document file 
%  in several files. But if hyperlinked cross-referencing 
%  is used, each tome is correctly hyperlinked to its 
%  internal targets, but no cross-referencing can be 
%  automatically dono to items in other tomes. I do not 
%  think this is a real draw back in practical situations, 
%  but of course its produces a limitation to reading the 
%  tomes on the screen.
%\end{enumerate}
%
% I assume that multi\-tome theses are very rare; \LaTeX can
% handle the production of different documents (different
% tomes/volumes) maintaining correct cross references also to
% material external to any specific volume, but the handling 
% of the whole job becomes very complicated not to mention
% the programming difficulties. For master theses I would avoid 
% splitting one's thesis is several tomes; nevertheless this
% functionality has been asked for, so that this bundle
% realises a couple of solutions that require some compromise
% and some manual work by the author. A better solution 
% is on the TODO list, but it is not to come in a short time.
% 
% In summary this |toptesi-magistrale| module keeps all the
% commands defined on the |topfront| module; modifies the
% definition of the |\tomo| command; adds the |ThesiTitlePage|
% environment within which the configuration file is read and all
% commands for the title page data are to be inserted. The
% peculiarity of theis environment is that the possible asterisk
% necessary to switch from the the logos at  the title page
% lower half (no asterisk) to the logos at the page top
% (asterisk) must be inserted the same way as in the other
% modules, that is as the first token at the beginning of
% the environment.
%
% \section{The Ph.D.\ module for general use}
% Except for  the ScuDo theses, for which the University
% specified the necessary requirements for both contents
% and appearance of the title page, we provide a generic
% module for Ph.D.\ theses for other universities. It goes
% by itself the if the specifications cannot be customised
% by using the customising commands of the ScuDo module,
% nor by this module, it is necessari to use a custom
% module that the user must create as specified in the next
% section.
%
% This module is generic as much as the |topfront| one.
% It differs only for the elimination of the other thesis
% kinds and the only addition is the |ThesisTitlePage|
% environment, just as it has been done for the other
% bundle modules.
%
% The data insertion or modification commands are a subset
% of those described in subsection~\ref{sec:topfront},
% of course without those tha are specific of other thesis
% kinds; therefore we do not repeat them here.
%
% Special care should be dedicated to the university
% specifications; in some universities the supervisor(s)
% should not be indicated and in such a case the (unique)
% candidate  may be indicated without any legend on top of
% it. The University logo might already contain the explicit
% wording with the the university name and in this case
% repeating the name would be pleonastic. At my level I
% cannot foresee the user requirements therefore it must
% be the user that takes care of inputting the correct
% fixed word data and the specific thesis data. I hope
% this module is useful for most Ph.D.\ students, but if
% the user  comparer the title page requirements with those
% prescribed for the ScuDo theses, s/he sees how much such
% requirement might be different from one university to the other. 
%
% May be in such circumstances the user might prefer to create
% his/her custom title page and would not use this module at all. 
%
% \section{The custom module for the title-page}\label{sec:custom}
% The construction of the customised title page module is the full
% responsibility of the user.
% In facts the user, beyond specifying the |tipotesi=custom|
% to the |toptesi|  class  class, at the beginning of the
% document explicitly inputs a personal package file, say
% |mycustomfile.sty| by means of the simple
% |\usepackage{mycustomfile}| command, and this is all the user
% needs. Of course this is an understatement; in facts s/he
% should select a package form the many existing ones, s/he
% can create him/herself a new package branched on other existing
% software; s/he mai refrains form using external packages,
% and simply insert his/her title page data within a |titlepage|
% environment where s/he wants, with the fonts families, series,
% shapes, and sizes s/he prefers and so on.
%
% Needless to say that in order to create a custom title page
% module the user must perfectly know the intricacies of the
% \LaTeX\ language. It may be helpful to read the Italian
% documentation |toptesi-it.pdf| where an example of landscape
% title page is already described in full detail. Of course a
% landscape title page requires a custom layout and is no
% surprise that the example deals with this particular one.
% Nevertheless that example gives a good idea on how to create
% a custom title page.
%
% We don't go further in describing the |tipotesi=custom| option,
% except remembering the this option just inhibits loading both
% the standard default module, and any other module, unless the
% user loads it directly with or without the specific module options.
%
% \section{The high school final work module}
%
% The suffix |sss| is the acronym for the Italian name of
% ``High School'', \emph{Scuola Secondaria Superiore}.
% Elegance of bureaucratic language; the elementary school
% now is `Scuola Primaria', and junior high school is `Scuola
% Secondaria Inferiore'. As well as it was in Germany and
% in some Swiss Cantons, the total number of school years
% is 13 instead of 12, as  it is in most countries. The advantage
% is that no general education is taught in university, no
% \emph{classes préparatoires}, because it is assumed that
% the high school diploma assures a sufficient level of general
% education.
%
% Differently from most high school systems in the world,
% in Italy each school has a special ``mission'' and receives
% different names, all under the general definition of Scuola
% Secondaria Superiore. But it s possible to distinguish
% between Licei (University preparatory), Istituti tecnici
% with various specialisation (for technical education in
% different domains); Istituti professionali (for vocational
% training); all of them end with a state examination, and
% almost all of them allow entrance in University, although
% a certain level of auto\-selection is actually performed.
% Many university Schools, formerly Faculties, submit applicants
% to entrance tests intended mostly to let them perform a
% conscious choice of their future studies; sometimes the
% tests are intended also to form ranks in those schools where
% the \emph{numerus clausus} is applied.
% 
% This off topic is necessary to let the general reader
% understand the importance of the \emph{esame di maturità}
% and the important role that the high school final work
% (nicknamed \emph{thesina}, `little thesis') plays in this
% generally difficult exam; statistics say that the number of
% flunking students is very low, but the final numerical grade
% (passing grade; from 60 to 100 with honours) plays also an
% important role in the following career, therefore students
% work very hard to pass with good grades.
%
% The above off topic explains why in Italy the final high
% school state exam includes the evaluation of a personal
% written report on some subject chosen by each student;
% this choice is often performed  with
% an initial support by each student teachers, that give help in
% choosing the subject, and, maybe, offer some initial input.
% They do not supervise the student work; but this work is
% submitted to the examining committee (the members of which
% are external to the specific school) that after due reading,
% ask questions on the work and investigate the depth to which
% the student has gone into his/her chosen subject.
%
% Certainly this final work has not the importance of a
% university thesis, but it is the first time when the
% student is asked questions on a subject of his/her choice,
% where the examining committee assumes s/he has examined all
% the facets with the approach of a young adult, not that
% of an old teenager.
%
% Some high school students have discovered \LaTeX, and some
% of them do really appreciate the typesetting results all
% \LaTeX\ users love; therefore they like the idea of
% typesetting their final high school work with the powerful
% \LaTeX\ functionalities.
%
% Of course their title page must be very different from those
% suitable for university theses, therefore a special module is
% necessary, and I decided to to add such a module to this bundle.
%
% Actually high school students require less information on the
% title page; the school generic and proper names, obviously the
% title and a possible subtitle, their name, maybe the logo of
% their school, the examination date, and such similar data.
%
% They certainly do not need the advanced features needed for
% the functionalities of the |topfront| extension package;
% nevertheless the title page must have a professional look,
% not just what they may achieve by using a word processor.
%
% The contents of the \emph{tesina} may also have a professional
% look if the students do not tickle the typesetting program
% to behave as a word processor. Several guides are available
% for high school students, as well for university ones, to
% recall how a ``technical report'' should be properly written;
% unfortunately today students prefer to read their smartphones,
% rather than real printed or electronic books; but those
% students who use \LaTeX\ know very well that they need reading
% suitable guides for using the program \emph{and} for structuring
% their document.
% 
% Mirko Mondini suggested to extend the TOPtesi bundle to
% include the title page for high school final works. He
% also suggested the layout of the page. This module may be
% considered ``work in progress'' because we both have little
% experience with the requirements of such final works.
% As the maintainer of this bundle I thank him very much;
% Marco may suggest specific changes in the near future;
% his contributions are very welcome. I would add that
% commentaries and suggestions by the students to whom this
% module is dedicated are very welcome just as well.
%
% We assume the student already knows the fundamentals of \LaTeX;
% we assume the Italian student has read the companion
% documentation |toptesi-it.pdf| distributed with this bundle;
% we assume also that the student has already a valid
% installation of the \TeX\ system, preferably \TeXLive,
% that is available for all most popular operating systems,
% Windows included of course, although for Windows there is
% also the alternative installation MiKTeX. As my  personal opinion,
% I think that \TeXLive may be updated more easily than MiKTeX
% and, not negligibly, its default installation is complete,
% while MiKTeX by default has a basic installation, so that very
% often offline MiKTeX users suffer from ``missing package'' errors.
%
% Since this module is going to be used only by Italian high
% school students, all user commands are in Italian.
%
% Here we concentrate on the title page. But before entering
% this subject let us recall that this module already loads a
% certain number of packages.
% \begin{enumerate}[noitemsep]
%\item
%   Packages |amsmath|, |amssymb| (only \pdfLaTeX) and |amsthm|
%   are already loaded, but a test is made to verify if the user
%   has loaded in his/her preamble the math  packages |newtxmath|
%   or |newpsmath| (possibly together with their companions
%   |new..text|), because the latter packages already provide
%   for their functionalities.
%\item
%   If they typeset with \LuaLaTeX\ or \XeLaTeX, the suitable
%   packages are already loaded, in particular the |unicode-math| 
%   one; the user, though, should load a matching OpenType Math 
%   font among those available with the \TeXLive distribution.
%\item
%   Service packages such as |xparse|, |xspace|, and |xcolor| 
%   are already loaded for the benefit of this bundle and this 
%   module, but the user is free to use their functionalities, 
%   mostly those of |xcolor|.
%\item
%   Packages |calc| and |ifthen| are functional for the module; 
%   supposedly the end user should not have any necessity to use 
%   them, but, of course, they can if they need to.
%\item
%   Package |multirow| has been loaded, because many users like
%   to imitate word processor functionalities; this package allows 
%   to create table cells that span several rows; as a general 
%   rule this practice should be avoided.
%\item
%   Package |booktabs|, is very useful to set professional
%   tables (reading its documentation is necessary).
%\item
%   No particular package has been loaded to create a specialised
%   bibliography; either the student directly uses its own 
%   |thebibliography| environment, or provides to chose whatever 
%   style s/he likes best and, s/he decides to load package 
%   |biblatex| and specifies the desired options. In the first 
%   case one of the most useful packages is |natbib|; in the 
%   second case |biblatex| supports other bibliographic style 
%   files. Both approaches require the preparation of a 
%   bibliography database by means of programs |jabref|,
%   |bibdesk|, or similar applications: but with |natbib|
%   the |bibtex| database handler is required, while with
%   |biblatex| case the |biber| handler yields the best
%   performances. In both cases it is required the careful
%   reading of either package documentation.
%\end{enumerate}
%
% The commands the user can use for the title page follow.
% Remember that they may be used directly within the
% |FrontespizioTesina| environment; if the user prefers, s/he
% can write them in a configuration file; such file must be
% named with the same name as the document main file, and
% it must be saved in the same folder; the information
% read from the configuration file is possibly overridden
% by the same commands, with different data, specified within
% the |FrontespizioTesina| environment.
% Just to be sure to be clear, if your document main file is
% named |LaMiaBellaTesina.tex|, the configuration file should
% be named |LaMiaBellaTesina.cfg| and both files should be
% saved in the same folder.
%
% The title page commands are the following:
%\begin{description}\def\Item[#1]{\item[\normalfont\cs{#1}]}
%\Item[SSSLogo] 
%   is used to define the name of the graphic file containing 
%   the high school logo; not necessarily the school allows to 
%   use its logo; in any case if such command is not used, the 
%   logo information is ignored. The syntax is
%   \begin{ttsintassi}
%   \cs{SSSLogo}\marg{file name}
%   \end{ttsintassi}
%   Remember that the graphic file must have one of these 
%   extensions: |.pdf|, |.eps|, |.png|, or |.jpg|; the first 
%   two may contains vector images and give the best results; 
%   the second two are bitmapped, therefore their rendering 
%   depends very much on the pixel density of the file contents.
%\Item[NomeTesina] sets the name of the document; the default value 
%   is ``Tesina di maturità''. Syntax:
%   \begin{ttsintassi}
%   \cs{NomeTesina}\marg{nome dell'elaborato}
%   \end{ttsintassi}
%\Item[NomeCandidato]
%   is used to set the label to typeset over the author's name;
%    default ``Studente'' or ``Studentessa'' depending on the
%   state of the boolean switch |femminile|, but if this command
%   is used for the user tesina, s/he needs not to specify any
%   switch and uses the only name necessary for his/her document.
%   Syntax:
%   \begin{ttsintassi}
%   \cs{NomeCandidato}{examinand label}
%   \end{ttsintassi}
%   This command might be used to set the label with a different
%   name in the correct gender.
%\Item[IndirizzoMiur] 
%   sets the Ministry of Education special address. If this 
%   command is omitted, this address, considered optional, 
%   is neglected. Syntax: 
%   \begin{ttsintassi}
%   \cs{IndirizzoMiur}\marg{Ministry of Education special address}
%   \end{ttsintassi}
%\Item[OpzioneMiur]
%   sets the specification that distinguishes the  school 
%   mission, if any. Syntax:
%   \begin{ttsintassi}
%   \cs{OpzioneMiur}\marg{Ministry of Education option}
%   \end{ttsintassi}
%\Item[TipoScuola]
%   sets the specific generic name of the school; default 
%   none; therefore use of this commands is mandatory; 
%   the user might specify ``Liceo classico'', or ``Istituto 
%   tecnico industriale di stato'', or whatever applies for 
%   the students' specific school. Syntax:
%   \begin{ttsintassi}
%   \cs{TipoScuola}\marg{generic school name}
%   \end{ttsintassi}
%\Item[NomeScuola]
%   sets the specific name of the school; generally schools 
%   are dedicated to famous scientists or heroes, or presidents, 
%   or kings/queens, or writers, et cetera; default ``empty''.
%   Syntax:
%   \begin{ttsintassi}
%   \cs{NomeScuola}\marg{specific school name}
%   \end{ttsintassi}
%\Item[SedeScuola]
%   set the actual location of the school it may be just the 
%   city or a full postal address; it may contain new line 
%   commands in order to format a full address of several lines; 
%   no default. Syntax:
%   \begin{ttsintassi}
%   \cs{SedeScuola}\marg{the school basic or full address}
%   \end{ttsintassi}
%\Item[AnnoScolastico]
%   sets the school year at the end of which the state 
%   examination is undertaken; no default. Syntax:
%   \begin{ttsintassi}
%   \cs{AnnoScolastico}\marg{school year}
%   \end{ttsintassi}
%\Item[titolo]
%   sets the title of the document; please don't use long titles;
%   if a short sentence is not descriptive enough, use the 
%   |\sottotitolo| argument to add specifications to the title; 
%   no default. Syntax:
%   \begin{ttsintassi}
%   \cs{titolo}\marg{main title}
%   \end{ttsintassi}
%\Item[sottotitolo]
%   sets an optional subtitle; if this command is not used
%   the subtitle field is ignored; no default. Syntax:
%   \begin{ttsintassi}
%   \cs{sottotitolo}\marg{optional subtitle}
%   \end{ttsintassi}
%   The subtitle, if any, is typeset in a normal sized font, 
%   therefore it can be several lines long; nevertheless pay 
%   attention to use the tie command |~| to connect  with an
%   unbreakable space to the following ones such words as
%   prepositions, articles, short adverbs, and such similar
%   items.
%\Item[studente] or \cs{studentessa}, depending on the gender
%   of the student, sets the name of the author of the
%   document, i.e.\ the name of the student undergoing the
%   examination; no default. Syntax:
%   \begin{ttsintassi}
%   \cs{studente}\marg{proper name(s) followed by the surname(s)}
%   \cs{studentessa}\marg{proper name(s) followed by the surname(s)}
%   \end{ttsintassi}
%   Remember: \textbf{never ever use the surname(s) before 
%   the name(s)}.
%   If the \cs{NomeCandidato} macro has not been used, these commands
%   set the proper status to the |femminile| switch, so that the
%   proper default label is used.
%\Item[Presidente]
%   sets the name of the chairperson of th examining committee; 
%   no default. Syntax:
%   \begin{ttsintassi}
%   \cs{Presidente}\marg{examining committee chairperson full name}
%   \end{ttsintassi}
%   Again remember to write the necessary academic titles 
%   in lowercase, end the name(s) before the surname(s).
%\Item[NumeroCommissione]
%   sets the examining committee ID number; no default.Syntax:
%   \begin{ttsintassi}
%   \cs{NumeroCommissione}\marg{Examing committee ID number}
%   \end{ttsintassi}
%\end{description}
%
% The actual title page is created within the environment
% |FrontespizioTesina|. This environment first reads a the
% configuration file if one exists, then it analyses the
% commands possibly set within the environment by the author, 
% possibly superseding the information read from the 
% configuration file; the closing commands actually typeset 
% the title page; the schematic working is the following:
%\begin{flushleft}\ttfamily\obeylines
%\Bambiente{FrontespizioTesina} \% reads the config.\ file if any
%\meta{user commands with user data}
%\Eambiente{FrontespizioTesina} \% typesets the title page
%\end{flushleft}
% There are no options to specify; the title page is sort
% of rigid in the sense that every piece of information goes
% where it should  and interlaced with proper spaces so as to fill
% up the text grid in a sensible way. No doubt some users may
% complain that the title page does not allow to do this or
% that; that the spacings should be different from what they
% actually are. 
%
% There are work-arounds to these complains.
%\begin{enumerate}[noitemsep]
% \item
% Instead of using the above environment, s/he uses the |titlepage|
% environment within which s/he inserts what s/he wants and
% where s/he likes best, with the typefaces s/he prefers,
% with the vertical spacings that accomodate his/her aesthetical
% feelings. Of corse it is necessary to know what is being done;
% if not, the results may be worse than the predefined layout.
%\item
% Even better: the user may use a custom layout and set it
% following the instructions described in section~\ref{sec:custom}.
% \end{enumerate}
%
%^^A                          PDFA/A
%
% \section{\sigla{PDF/A} archivable format}
% This section in general applies also if the thesis is typeset
% by means of  \LuaLaTeX\ or \XeLaTeX, but with\XeLaTeX\  special
% attention should be paid to the details of the procedure as
% described in the documentation of the \pack{pdfx} package. Even
% better, \LuaLaTeX, whose source file is practically the same as
% that of \XeLaTeX, may produce in a simpler way a \sigla{PDF/A}
% compliant output file if the users follow attentively the
% instructions contained in the documentation of the \pack{pdfx}
% package.
%
% More details are given in the Italian documentation contained
% in the \texttt{toptesi\discretionary{-}{}{-}it} which is part
% of the TOPtesi bundle.
%
% Politecnico di Torino and several other Italian and foreign
% universities are moving towards archiving theses in electronic
% format, specifically in the \sigla{PDF} one. The problem of
% course is: ``Will it be possible to read the archived documents,
% say, fifty years from now?''
%
% This essential question has been answered by the International
% Standards Organisation (\sigla{ISO}) that in 2005 published the
% regulation \sigla{ISO}~19005-1; in 2011 and 2012 enhancements of
% this regulation have been published. This regulation defines
% a \sigla{PDF} variant suitable for archiving, named \sigla{PDF/A}, that
% has two sub-formats distinguished as \sigla{PDF/A}-1a, and \sigla{PDF/A}-1b.
% The `a' sub-format is more exacting, while the `b' one is less
% stringent. In spite of the recently published  upgrades,
% here we stick to the older ones, that are a sub set of the
% newer ones, and for which suitable software is available.
%
% The requirements for the `a' sub-format imply not only those
% requested for the `b' one, but also that the logical structure
% of the document be accessible from outside (Tagged \sigla{PDF}).
% The requirements for the `b' sub-format are that the
% document must be reproducible without modifications exactly
% as it was at the moment of archiving. Both sub-formats must
% contain \emph{metadata} that are searchable even without
% decompressing the normally compressed \sigla{PDF} file, and that
% contain information useful for archive maintenance; among
% these informations, of course, the \sigla{PDF/A} category the document
% belongs to, the document title, the authors, and few other
% optional pieces of information, such as the keywords that
% ease up library searches.
%
% Since version 1.40, the program \texttt{pdflatex} is capable
% of producing \sigla{PDF/A}-1b compliant files, provided that some
% attention is put into editing of the thesis  source file.
%
% With the distribution of the 2008 version of the \TeX\ system,
% the executable \texttt{pdflatex} has version number 1.40.9
% (or higher) and is capable of producing \sigla{PDF/A}-1b files.
%
% The particular attention needed to avoid problems with the
% \sigla{PDF/A} compliance is summarised as follows:
% \begin{enumerate}[noitemsep]
% \item
% The preliminary essential requirement is that the
% \texttt{pdflatex} engine used to typeset the thesis be
% sufficiently recent to support the \sigla{PDF/A} requirements.
% It's better to have the most recent distribution of
% the \TeX\ system installed on your PC. Do not try to
% typeset the thesis with the `old' \prog{latex}; you
% \emph{must} process the input thesis file(s) with
% \texttt{pdflatex} or with \prog{xelatex} or with \prog{lualatex};
% in the following, no specific check will be made in
% order to verify if you are actually using \texttt{pdflatex}.
% If you really need to use the `old' \LaTeX, you get a DVI
% file and you need to transform it with |dvips| into a
% PS file; at this point you have lost the possibility of
% exploiting the internal commands of |pdflatex|
% version~1.40.9 or later. You can still produce a \sigla{PDF/A}
% final document, but you have to transform it by means of the
% procedures described in the Italian documentation; such
% documentation describes several more or less effective
% methods to transform a non compliant \sigla{PDF/A} file into a
% compliant one, and how to correct the ``imperfections''
% that forbid any valid transformation.
%
% \item 
% Since 2016 the TeXLive distribution contains a new version
% of \pack{pdfx} and now \sigla{PDF} documents produced with
% \prog{lualatex} may be fully compatible with the standard
% \sigla{PDF/A}. With some special care and some post"processing,
% even \prog{xelatex} can now produce \sigla{PDF/A} compliant output
% files.
%
% \item
% Your up-to-date complete \TeX\ distribution should
% contain the package \textsf{pdfx}; if it does not,
% upgrade your \TeX\ system complete distribution. Be sure
% that the version number of the \pack{pdfx} package has
% version number~1.5.8 or higher.
%
% \item
% If your thesis main file, the one you run your typesetting
% program on, is named, say, \texttt{JohnSmithMasterThesis.tex},
% prepare in the same directory another file named
% \texttt{JohnSmithMasterThesis.xmpdata} that contains the
% \emph{metadata} relative to the thesis; pay attention
% to follow the stringent syntax described and exemplified
% in the \textsf{pdfx} documentation. A minimal set of
% \emph{metadata} example would be the following one:
% \begin{verbatim}
% \Title{Experiments in Trichotetratomy}
% \Author{John Smith}
% \end{verbatim}
% The keyword list requires a specific format that can be
% examined in the package documentation. The best way to
% be sure that this file accompanies the thesis main file,
% is to put its contents in a \env{filecontents*} environment
% before the |\documentclass| declaration. The details are
% in the documentation of the \pack{pdfx} package.
%
% \item
% Some mathematical symbol commands obtained from the standard
% \LaTeX\ set-up and the standard mathematical fonts require
% some patching that is already included in this \texttt{TOPtesi}
% bundle; but it is not excluded that with other fonts similar
% patches might be requested. The \sigla{UNICODE} math fonts used
% by \XeLaTeX\ and \LuaLaTeX\ do not require any patch.
%
% \item
% Use only \sigla{PNG} and \sigla{JPG} images with \sigla{RGB} color profiles.
% If a \sigla{PNG} image file contains some transparencies, the
% document \sigla{PDF} cannot be validated as \sigla{PDF/A} compliant.
%
% \item
% If you include \sigla{PDF} images that contain some text, be sure
% that the fonts for this text are completely embedded in
% the included files. Should the \sigla{PDF} files come from an external
% drawing program be sure to configure that program so that
% it embeds all the fonts used in each image. If you don't
% succeed, open the \sigla{PDF} file with the free software
% \texttt{inkscape} and save it back in \sigla{PDF} format; the
% missing fonts will be replaced with their outlines and
% this will not disturb the \sigla{PDF/A} conformity. Pay attention,
% because Inkscape might save your file with some
% specifications for transparency; but transparency
% specifications destroy the \sigla{PDF/A} compliance.
% 
% Should this inconvenience take place, instead of asking
% \texttt{inkscape} to save in \sigla{PDF} format, ask to save in \sigla{EPS}
% format; this one does not support transparencies, therefore
% they get abolished from the very beginning. This is useful
% also when \texttt{inkscape} is used to change the color
% profile of an image. Once used this program to convert an
% image to \sigla{EPS} format without transparencies and/or with the
% accepted \sigla{RGB} color profile, import the image as usual with
% |\includegraphics|; all the suggested typesetting programs,
% \prog{pdflatex} included, can import \sigla{EPS} files, therefore
% these problems, thanks \texttt{inkscape} may be easily
% overridden. 
%
% \item
% Verify your final \sigla{PDF} file with a suitable program and
% do not give up doing the necessary corrections or
% modifications while the verification program keeps saying
% that this or that is not compliant with the \sigla{PDF/A}
% specification. A suitable program is the Preflight module
% of Adobe Acrobat Pro~XI (or later), but this, although the most
% authoritative, is a commercial program; probably your
% university has special facilities for this task. In any
% case you can install the \prog{veraPDF} free software
% from the Internet, and use this handy verification program
% (it is still in beta stage until the official stable release that
% is due during the year 2017).
%
%\end{enumerate}
%
% Warning: if you are using Adobe Reader X (or later), this
% program will open a \sigla{PDF} file beginning with an information
% header stating that the file claims to be \sigla{PDF/A} compliant;
% maybe it's true, but do not trust this information too much,
% at least don't believe that this information is a
% ``certification'' of the \sigla{PDF/A} compliance. I have seen
% files with this comforting information that did not pass
% the Preflight test! Actually the Reader is correct in saying
% that ``\dots\ the program claims to be \sigla{PDF/A} compliant\dots'';
% the Reader just notices that the metadata contains such a claim,
% but this does not imply that the claim is true.
%
% Up to today the realisation of \sigla{PDF/A} conforming files sets
% forth several problems that are of great concern for 
% large Institutions that have thousands of documents a year
% to archive; it is not a question implied in the free nature
% of the \prog{pdflatex}, \prog{xelatex}, and \prog{lualatex}
% programs, that, on the opposite, according to my experience,
% have a very high rate of success in producing \sigla{PDF/A}
% compliant documents.  If you stick to the default \TeX\ system
% Type~1, 256-glyph fonts, or to the \sigla{UNICODE} encoded otf
% or ttf fonts and use this version of the TOPtesi bundle
% you should be able to avoid most problems.
%
% \StopEventually{}
%
% \newpage
%
% \section{Documented code}
% \subsection{The class code}
%
% Here begins the usual machinery for stating the required
% \TeX\ format and for sharing some code between the driver
% and the class part of the code, since they are supposed to
% carry the same date and version number, besides their specific
% description string.
% \iffalse
%<*class>
% \fi
%
% The class itself is very simple since it requires just the
% \texttt{report} document class and some packages with some
% default options. All options specified to the \texttt{toptesi}
% class are passed on to the \texttt{report} class; the
% |a4paper| option is made the default, but the user can invoke
% the class with any valid \LaTeX\ paper size. According to the
% \LaTeX\ and |xkeyval| machinery of option passing, the |\ExecuteOptionX|
% command assures that |a4paper| is the default size, but it
% is not used if another paper size code is specified to the class.
% The same holds true for the other class options, except for
% the encoding string because their passing machinery does
% not hold true for the \pack{inputenc} package; therefore this
% package is not loaded any more as it was in previous versions
% of the class\footnote{Thanks to Enrico Gregorio who pointed
% out this feature.}. On the opposite, while typesetting with
% |pdfLaTeX|, if another language is specified in the list of
% \pack{toptesi} class options this language is appended to the
% \pack{babel} package default options and, beware!, it would
% become the default language; with XeLaTeX and LuaLaTeX a
% language option is useless (and harmless) due to their way of
% specifying languages; with these programs other languages must
% be specified with their |\setotherlanguage| or |\setotherlanguages|
% commands which do not change the preset main language.
%    \begin{macrocode}
\usepackage{trace}
\RequirePackage{xkeyval}
\DeclareOptionX{formato}[a4paper]{\PassOptionsToClass{#1}{report}}
\DeclareOptionX*{\PassOptionsToClass{\CurrentOption}{report}}
\ExecuteOptionsX{formato}
\ProcessOptionsX*\relax
\LoadClass{report}

\RequirePackage{iftex}
\ifPDFTeX
  \usepackage[english,italian]{babel}
\else
  \usepackage{fontspec}
  \defaultfontfeatures{Ligatures=TeX}
  \usepackage{polyglossia}
  \setmainlanguage{italian}
  \setotherlanguage{english}
  \renewcommand*{\iflanguage}[1]{%
  \ifnum\the\language=\csname l@#1\endcsname
  \expandafter\@firstoftwo\else
  \expandafter\@secondoftwo\fi}
\fi
\usepackage{toptesi}
%
%    \end{macrocode}
% \iffalse
%</class>
% \fi
% \subsection{The \texttt{toptesi.sty} code}
% The greatest part of the \texttt{toptesi} class code is
% saved into a separate file partly for backward compatibility
% reasons (before version 3.x \texttt{toptesi} was just an
% extension to the \texttt{report} class) and partly because
% it might be used as a stand alone package, but, take
% notice, it might create some conflicts with with other
% classes internal code, although with the latest versions
% of |toptesi| we did not notice any conflict with the |book|
% class; it is certainly  incompatible with the \texttt{article}
% class.
%
% This package |toptesi.sty| must contain its own \TeX\ format
% declaration in view that it might be used as an extension to
% other classes 
%
% \iffalse
%<*topsty>
% \fi
%
% We start by defining the debugging macros; these trace
% commands and macros are the usual ones I used for debugging.
% I know the \texttt{trace} package is supposed to be much better,
% but sometimes I use these ones, especially when I want to
% trace math snippets, where the tracing functionalities of
% \pack{trace} are turned off.
%    \begin{macrocode}
\def\TRON{\tracingcommands \tw@ \tracingmacros \tw@}
\def\TROFF{\tracingcommands\z@ \tracingmacros \z@}
\let\TROF\TROFF
%    \end{macrocode}
%
% Now we define the specific package options: \texttt{classica} and
% \texttt{trieste} are identical; the use of \texttt{trieste} is
% deprecated, but this option is maintained for backward
% compatibility.
%
% The  option \opz{scudo} is used only to typeset doctoral
% dissertations with the special look required by the ``Scuola
% di Dottorato'' of Politecnico di Torino. The other options for
% specific types of thesis, \opz{triennale}, \opz{monografia},
% \opz{magistrale}, \opz{dottorale}, \opz{secondaria}, are
% described further on; options \opz{frontespizio}  and
% \opz{custom} are avoid loading the  default module |topfront| or any of the previous listed modules; the are used to create title pages
% different from the ones that are available with TOPtesi.

% The |corpo = value| option can receive ``any'' value; if this
% value is not lower than  14\,pt , the |\if@xixpt| switch is
% set to true; the value assigned to |corpo| specifies the normal
% size; the TOPtesi bundle uses the |\changefontsizes| macro
% provided by package |scrextend|, that is part of any \TeX\ 
% system complete and up-to-date TeXLive or MiKTeX  installation.
%
% Option \opz{chapterbib} allows to set a list of references at
% the end of each chapter. The option is maintained fore backward
% compatibility, but we discourage its use, because of the amount
% of manual work involved; it is much better to use the |biblatex|
% functionalities and other packages created for this purpose.
%
% For other options some boolean variables are
% defined and the option definitions set some of them to the value
% ``true''.
%    \begin{macrocode}
\newif\ifevenboxes \evenboxesfalse
\newif\if@utoretitolo \@utoretitolofalse
\newif\if@ldstyle \@ldstylefalse
\newif\if@xivpt \@xivptfalse
\newif\iflibro \librofalse
\newif\ifT@Pfrontespizio \T@Pfrontespiziofalse
\newif\ifTOPfront \TOPfronttrue
%
\newif\ifScuDo \ScuDofalse
\newif\ifmonografia\monografiafalse
\newif\iftriennale\monografiafalse
\newif\ifmagistrale\magistralefalse
\newif\ifdottorato\dottoratofalse
\newif\ifsecondaria\secondariafalse
\newif\ifcustom\customfalse
\newif\ifTOPnumerazioneromana\TOPnumerazioneromanafalse
%    \end{macrocode}
%
% A binding correction is established; its default value is
% parametrised to the paper dimensions, even if this correction
% should actually not depend on the paper size; the paper
% flexibility at the spine margin should be independent from
% its width; nevertheless large sizes allow for larger default
% corrections. In any case the user can override this setting by
% using the specific command |\setbindingcorrection|.
% The recommendation to the user is to not exaggerate with this
% correction. In particular the user is warned not to use this
% option in order to mode che taxt body grid towards the outer
% edge, simply because s/he does not understand why two side
% typesetting uses an inner margin smaller than the outer one.
%
% The options |usefrontespizio| and |noTOPfront| are set by the specific ``function'' executed by the |tipotesi=|~value options.
%    \begin{macrocode}
\newlength\T@Pbinding
\def\setbindingcorrection#1{\T@Pbinding=#1}
\newif\if@binding \@bindingfalse
\newif\ifchapterbibliography \chapterbibliographyfalse
\newif\ifclassica \classicafalse

\@ifpackageloaded{xkeyval}{}{\RequirePackage{xkeyval}}
\@ifpackageloaded{scrextend}{}{\RequirePackage{scrextend}}
\@ifpackageloaded{iftex}{}{\RequirePackage{iftex}}
\@ifpackageloaded{xspace}{}{\RequirePackage{xspace}}
\@ifpackageloaded{xparse}{}{\RequirePackage{xparse}}

\DeclareOptionX{cucitura}[7mm]{\setlength\T@Pbinding{#1}\@bindingtrue}
\DeclareOptionX{corpo}[10pt]{\unless\ifdim#1<\@xivpt\p@\@xivpttrue\fi
   \changefontsizes{#1}}
\DeclareOptionX{chapterbib}[true]{\chapterbibliographtrue}
\DeclareOptionX{evenboxes}[true]{\evenboxestrue}
\DeclareOptionX{libro}[true]{\librotrue}
\DeclareOptionX{stile}[standard]{\setkeys{toptesi}{stile=#1}}
\define@choicekey{toptesi}{stile}[\val\nr]{standard,classica,trieste}{%
    \ifcase\nr
       \typeout{valore=\number\nr}\classicafalse
    \or
       \typeout{valore=\number\nr}\classicatrue
    \or
       \typeout{valore=\number\nr}\classicatrue
    \else
       \typeout{valore=\number\nr}\classicafalse
    \fi}
\DeclareOptionX{autoretitolo}[true]{\ifclassica\@utoretitolotrue\fi}
\DeclareOptionX{oldstyle}[true]{\ifclassica\@ldstyletrue\fi}
\DeclareOptionX{pdfa}{\PackageWarning{toptesi}{%
    ***********************************************\MessageBreak%
    Option pdfa ignored.\MessageBreak%
    Read documentation toptesi-it-*.pdf,\MessageBreak%
    because package pdfx has changed\MessageBreak%
    ***********************************************\MessageBreak%
    }}

\DeclareOptionX{numerazioneromana}[true]{\TOPnumerazioneromanatrue}

\newcommand\NoValidTipo[1]{%
\PackageError{toptesi}{Option value `#1' not defined\MessageBreak
}{Type `X' to exit and check your input}}


\DeclareOptionX{tipotesi}[topfront]{%
  \ifstrequal{#1}{frontespizio}{\TOPfrontfalse\T@Pfrontespiziotrue
    \AtEndOfPackage{\RequirePackage{frontespizio}}%
    }{%
    \ifstrequal{#1}{topfront}{\TOPfronttrue\T@Pfrontespiziofalse
      \AtEndOfPackage{\RequirePackage{topfront}}%
    }{%
      \TOPfrontfalse\T@Pfrontespiziofalse 
        \ifstrequal{#1}{custom}{\relax}{%
          \ifstrequal{#1}{triennale}{%
             \def\TPT@tipo{toptesi-monografia}% opz:monografia
          }{\ifstrequal{#1}{secondaria}{% opz: secondaria
             \edef\TPT@tipo{toptesi-sss}%
            }{%
             \edef\TPT@tipo{toptesi-#1}% opz: triennale,
            }%                         % magistrale, dottorale
          }%
          \AtEndOfPackage{%
            \IfFileExists{\TPT@tipo.sty}%
              {\RequirePackage{\TPT@tipo}}{\NoValidTipo{#1}}%
          }%
        }% opz: custom
      }% opz: topfront
    }% opz: frontespizio
  }
\ProcessOptionsX*

%    \end{macrocode}
% The \texttt{graphicx} package is loaded by default; it is
% required for inserting the university logo(s); if the user
% forgets that this package has already been loaded nothing
% dramatic happens, because the |\usepackage| and |\RequirePackage|
% macros perform the necessary tests in order to avoid
% reloading the same packages again and again.
% On the occasion of loading package|etoolbox|, we define
% |\ifempty| that controls if its argument is a void or empty
% string; it is not sufficient to use |\ifdefempty| or |\ifdefvoid|,
% because the argument generally is not a macro; neither it is
% possible to check the string by means of |\ifcsempty| or
% |\ifcsvoid|, because sometimes it is a macro and in any case
% there might be situations when the argument string happens
% to equal an existing control sequence name; therefore first
% we expand the possibly argument into an internal macro, then we test
% such macro for emptiness.
%
%    \begin{macrocode}
\RequirePackage{graphicx}
\RequirePackage{etoolbox}
\RequirePackage{xspace}
\newcommand*\ifempty[1]{\edef\@tempA{#1}%
\ifx\@tempA\@empty\expandafter\@firstoftwo\else\expandafter\@secondoftwo\fi}
%    \end{macrocode}
%
% The |\textheight| is parametrised to the paper height and
% adjusted so as to contain an integer number of normal text
% lines. A new dimension is defined to hold the actual value
% of the inner/spine margin.
%
%    \begin{macrocode}
\newlength\interno
\normalsize% necessary to have the correct value for \baselineskip
\textheight \paperwidth
\setlength{\textheight}{%
\dimexpr\textheight*\baselineskip/\baselineskip+\topskip}
%    \end{macrocode}
% The inner margin is parametrised to the paper width, but
% a small correction is made if the extra size of 14pt is
% chosen. Also |\footskip| is parametrised to the paper height
% in a slightly different way when the large normal font size  
% is not smaller than 14\,pt; the |\if@xivpt| switch is set to true when the value is assigned to the key |corpo|.
%    \begin{macrocode}
\iflibro
  \unless\if@xivpt 
    \textwidth=\dimexpr0.70711\paperwidth\relax
    \setlength\interno{\dimexpr\paperwidth/8}
    \footskip=1.5\baselineskip
  \else
    \textwidth=\dimexpr0.79549875\paperwidth\relax
      \setlength\interno{\dimexpr\paperwidth/9}
      \footskip=2\baselineskip
  \fi
\else
  \unless\if@xivpt 
    \textwidth=\dimexpr0.70711\paperwidth\relax
    \setlength\interno{\dimexpr\paperwidth/6}
    \footskip=1.5\baselineskip
  \else
    \textwidth=\dimexpr0.79549875\paperwidth\relax
      \setlength\interno{\dimexpr\paperwidth/7}
      \footskip=2\baselineskip
  \fi
\fi
%    \end{macrocode}
% The convenience of holding the spine margin within a
% dimensional register becomes really useful now in order
% to define the other text-body-grid dimensions.
% Without binding correction the inner and outer margin
% are chosen equal, but the grid is moved outwards if the
% binding correction option is specified.
%    \begin{macrocode}
\iflibro 
  \interno=\dimexpr\paperwidth-\textwidth\relax
  \oddsidemargin=0.414214\interno\relax
  \evensidemargin=\dimexpr\paperwidth-\textwidth-\oddsidemargin\relax
\else
  \oddsidemargin=\dimexpr(\paperwidth-\textwidth)/2\relax
  \evensidemargin=\oddsidemargin
\fi
\marginparwidth=\dimexpr\evensidemargin-2.5\marginparsep
\oddsidemargin=\dimexpr\oddsidemargin-1in\relax
\evensidemargin=\dimexpr\evensidemargin-1in\relax
\AtBeginDocument{%
\if@binding
 \PackageInfo{TOPtesi}{Margin width recalculation}
 \PackageInfo{TOPtesi}{Before:\MessageBreak
 oddsidemargin\space\space \the\oddsidemargin\MessageBreak
 evensidemargin\space \the\evensidemargin}
 \advance\oddsidemargin \T@Pbinding
 \advance\evensidemargin -\T@Pbinding
 \advance\marginparwidth -\T@Pbinding
 \PackageInfo{TOPtesi}{After:\MessageBreak
 oddsidemargin\space\space \the\oddsidemargin\MessageBreak
 evensidemargin\space \the\evensidemargin}
\fi}

%    \end{macrocode}
%
% We now establish the page style. We start by setting to
% ``empty'' the tokens that keep the left and the right
% marks; we define a box so as to set the headers inside
% this box; we redefine also the plain page styles; it is
% actually a leftover from the previous versions when the
% page number was set at the foot in bold face, but we leave
% it here without the bold face specification, so that in
% future versions folios may be redefined in a common way
% with the other page styles. Notice that in all page styles
% folios are always in the footers.
%
% By defining |\lapagina| to be equivalent to |\thepage|
% we can later on redefine |\lapagina| the way we like; we
% actually do so with the option \texttt{classica}.
%    \begin{macrocode}
\def\lapagina{\thepage}
\mark{{}{}}
\newbox\@intesta
%
\def\ps@plain{\let\@mkboth\@gobbletwo
\def\@oddfoot{\null\hfill {\scshape\lapagina}\hfill \null}\def\@oddhead{}
\def\@evenhead{}\let\@evenfoot\@oddfoot}
%    \end{macrocode}
% Other page styles are defined in a different way according
% to the choice of one side or two side printing. In any case
% the header is set without the uppercasing that is done in
% all the default document classes, and it is underlined at
% a fixed distance from the base line. If the chapter or section
% heading exceed the |\textwidth| a warning is issued so as
% to invite the user to exploit the sectioning commands
% optional short argument.
%    \begin{macrocode}
\if@twoside
   \def\ps@headings{\let\@mkboth\markboth% autoretitolo
   \def\@oddfoot{\null\hfill {\scshape\lapagina} \hfill\null}
   \let\@evenfoot\@oddfoot
   %
   \def\@evenhead{\setbox\@intesta\hbox{\footnotesize\slshape
        \leftmark}%
   \ifdim\wd\@intesta>\textwidth \headWarn{\chapter}\fi%
   \underline{\makebox[\textwidth]{\footnotesize\slshape
        \strut\leftmark}}}%
   \def\@oddhead{\setbox\@intesta\hbox{\footnotesize\slshape
        \rightmark}%
   \ifdim\wd\@intesta>\textwidth \headWarn{\section}\fi%
   \underline{\makebox[\textwidth]{\footnotesize\slshape
        \strut\rightmark}}}%
   \def\chaptermark##1{%
       \markboth{\thechapter\ -- ##1}{\thechapter\ -- ##1}}
   \def\sectionmark##1{\markright{\ifnum\c@secnumdepth>\z@
   \thesection\ -- \fi ##1}}}
\else
   \def\ps@headings{\let\@mkboth\markboth
   \def\@oddfoot{\null\hfill {\scshape\lapagina} \hfill\null}
   \def\@evenfoot{}
   \def\@oddhead{\setbox\@intesta\hbox{\footnotesize\slshape
        \rightmark}%
   \ifdim\wd\@intesta>\textwidth \headWarn{\chapter}\fi%
   \underline{\makebox[\textwidth]{\footnotesize\slshape
        \strut\rightmark}}}%
   \def\chaptermark##1{\markright{\thechapter\ -- ##1}}}
\fi

\def\headWarn#1{\PackageWarning{toptesi}{%
*******************************************\MessageBreak
   THE HEADING IS TOO LONG\MessageBreak
   Use the optional argument of command \string#1\MessageBreak
   See the LaTeX Handbook (1994) on section C.4.1\MessageBreak
*******************************************\MessageBreak}}
%    \end{macrocode}
% It is also necessary to redefine the format of the unnumbered
% chapter entries in the table of contents so as to have page
% numbers in small caps.
%    \begin{macrocode}
\renewcommand\l@chapter[2]{%
  \ifnum \c@tocdepth >\m@ne
    \addpenalty{-\@highpenalty}%
    \vskip 1.0em \@plus\p@
    \setlength\@tempdima{1.5em}%
    \begingroup
      \parindent \z@ \rightskip \@pnumwidth
      \parfillskip -\@pnumwidth
      \leavevmode \bfseries
      \advance\leftskip\@tempdima
      \hskip -\leftskip
      #1\nobreak\hfil \nobreak
      \hb@xt@\@pnumwidth{\hss\ifPDFTeX\normalfont\fi\scshape{#2}}\par
      \penalty\@highpenalty
    \endgroup
  \fi}
%    \end{macrocode}
% The various tables of contents, or figures, or tables
% require some boolean variables to be defined; in facts,
% although the \sigla{ISO} regulations require that every technical
% report contains the list of figures and/or tables,
% in Italy theses rarely contain these lists; but is a
% thesis a technical report?
% We require also some other boolean variables to handle the
% difference between front matter and main matter; this
% differences are already defined in the \texttt{book}
% document class, but not in the \texttt{report} one.
%    \begin{macrocode}
\newif\iffigurespage
\newif\iftablespage
\newif\ifnumeriromani
\newif\iffrontmatter
%    \end{macrocode}
% The |\frontmatter| and |\mainmatter| commands are defined
% and at the beginning of the document the default situation
% of front matter is established. For the |\backmatter|
% command it suffices to turn out sectional numbering, and
% to allow opening chapters also on even numbered pages.
%    \begin{macrocode}
\def\frontmatter{\clearpage\ps@plain%
  \ifTOPnumerazioneromana\numeriromanitrue\fi
  \frontmattertrue\@openrightfalse\c@secnumdepth=-2\relax
}
    
\def\mainmatter{\if@twoside\@openrighttrue\fi
  \ifTOPnumerazioneromana\numeriromanifalse\fi
  \frontmatterfalse\c@secnumdepth=2
  \clearpage\ps@headings
  \ifTOPnumerazioneromana\pagenumbering{arabic}\fi
}
\newcommand\backmatter{\cleardoublepage[empty]%
\setcounter{secnumdepth}{-1}\if@twoside\@openrightfalse\fi}

\AtBeginDocument{%
  \ifTOPnumerazioneromana\pagenumbering{roman}\fi\frontmatter
}
%    \end{macrocode}
%
% The main matter is automatically established with the
% first |\chapter| command issued by the user; this means
% that every command that starts a section at the
% ``chapter'' level within the front matter must be
% executed without an explicit call to |\chapter|. This
% refers especially to the |\sommario|/|\summary| and
% |\ringraziamenti|/|\acknowledgements| sections. On the
% contrary even  an unnumbered chapter (for example a preface)
% turns on the main matter typesetting style.
%
% By default we set to false the boolean variables that
% control the typesetting of the list of figures and the list
% of tables.
%    \begin{macrocode}
\figurespagefalse
\tablespagefalse
%    \end{macrocode}
%
% Before going further on, we redefine the |\cleardoublepage|
% command so that it uses by default the plain page style
% for the blank page to be possibly inserted if chapters
% should start on an odd page: but with the |\blankpagestyle|
% command it can be set to any other style, for example the
% |empty| page style.
%    \begin{macrocode}
\let\ps@blank\ps@plain
\newcommand*\blankpagestyle[1]{%
   \expandafter\let\expandafter\ps@blank\csname ps@#1\endcsname}
\renewcommand\cleardoublepage[1][blank]{\clearpage
\unless\ifodd\value{page}
   \if@twoside
      \if@openright
         \null\thispagestyle{#1}\clearpage
   \fi\fi\fi}
%    \end{macrocode}
%
% We have to define the front matter sectioning names |\sommario|
% (alias |\summary|) and |\ringraziamenti| (alias
% |\acknowledgements|) so as to remain in the front matter.
%    \begin{macrocode}
\def\sommario{%
 \iffrontmatter\else\frontmattertrue\fi
 \chapter*{\summaryname}}
\let\summary\sommario
%
\def\ringraziamenti{%
 \iffrontmatter\else\frontmattertrue\fi
 \chapter*{\acknowledgename}}
\let\acknowledgements\ringraziamenti
%    \end{macrocode}
% The strings |\summaryname| and |\acknowledgename| are not
% defined for every language option to \texttt{babel}. Default
% definitions are given below, and the user must define new
% names for localising the package in a language different
% from Italian and English.
%
% We have to modify the |\chapter| and |\part| commands so
% that as the user first issues one of these commands the
% typesetting style is switched to that of the main matter.
% Actually it is not necessary to redefine all the  commands,
% but just those of the unstarred versions. In facts 
% |\chapter| commands to typeset the Summary and the
% Acknowledgements chapters, as defined above, are regularly
% typeset in the front matter style, opening on any page
% (even or odd) and numbered with small caps roman numerals.
%    \begin{macrocode}
\def\@chapter[#1]#2{\iffrontmatter\mainmatter\fi
                    \ifnum \c@secnumdepth >\m@ne
                         \refstepcounter{chapter}%
                         \typeout{\@chapapp\space\thechapter.}%
                         \addcontentsline{toc}{chapter}%
                                   {\protect\numberline{\thechapter}#1}%
                    \else
                      \addcontentsline{toc}{chapter}{#1}%
                    \fi
                    \chaptermark{#1}%
                    \addtocontents{lof}{\protect\addvspace{10\p@}}%
                    \addtocontents{lot}{\protect\addvspace{10\p@}}%
                    \if@twocolumn
                      \@topnewpage[\@makechapterhead{#2}]%
                    \else
                      \@makechapterhead{#2}%
                      \@afterheading
                    \fi}
%
\def\@part[#1]#2{\iffrontmatter\mainmatter\fi
    \ifnum \c@secnumdepth >-2\relax
      \refstepcounter{part}%
      \addcontentsline{toc}{part}{\thepart\hspace{1em}#1}%
    \else
      \addcontentsline{toc}{part}{#1}%
    \fi
    \markboth{}{}%
    {\centering
     \interlinepenalty \@M
     \normalfont
     \ifnum \c@secnumdepth >-2\relax
       \huge\bfseries \partname\nobreakspace\thepart
       \par
       \vskip 20\p@
     \fi
     \Huge \bfseries #2\par}%
    \@endpart}
%    \end{macrocode}
% At the same time we have to make sure that |\tableofcontents|,
% |\listoftables| and |\listoffigures| do not exit from the
% front matter style. We assume these commands are issued
% while in front matter, the default at the begin document step,
% so we have to avoid to use starred |\chapter| commands.
%    \begin{macrocode}
\renewcommand\tableofcontents{%
 \chapter*{\contentsname}%
 \@mkboth{\contentsname}{\contentsname}%
 \@starttoc{toc}%
 \clearpage
 \if@restonecol\twocolumn\fi
    }
\renewcommand\listoffigures{%
    \chapter*{\listfigurename}
    \@mkboth{\listfigurename}{\listfigurename}%
    \addcontentsline{toc}{chapter}{\listfigurename}
    \@starttoc{lof}%
    \clearpage
    \if@restonecol\twocolumn\fi
    }
\renewcommand\listoftables{%
    \chapter*{\listtablename}%
    \addcontentsline{toc}{chapter}{\listtablename}
    \@mkboth{\listtablename}{\listtablename}%
    \@starttoc{lot}%
    \clearpage
    \if@restonecol\twocolumn\fi
    }
%    \end{macrocode}
%
% We need to define |\indici| (alias |\allcontents|) that typesets
% the table of contents and, optionally, the lists of tables and/or
% figures while assuring that the front matter style is used for
% typesetting.
%    \begin{macrocode}
\def\indici{%
 \iffrontmatter\else\frontmattertrue\fi
        \tableofcontents
        \iftablespage
                {\addvspace{10pt}
                \let\saveaddvspace=\addvspace
                \def\addvspace##1{}
                \listoftables
                \let\addvspace=\saveaddvspace}
        \fi
        \iffigurespage
                {\addvspace{10pt}
                \let\saveaddvspace=\addvspace
                \def\addvspace##1{}
                \listoffigures
                \let\addvspace=\saveaddvspace}
        \fi
\ifbool{@twoside}{\cleardoublepage[empty]}{}}
\let\allcontents\indici
%    \end{macrocode}
%
% Command |\onecolumn| is not actually necessary; it simply
% overrides the possible misused option \texttt{twocolumn}
% in the opening document class statement; no thesis should
% be typeset in two columns.
%    \begin{macrocode}
\onecolumn
%    \end{macrocode}
%
% Here come some declarations for vertical justification
% and for avoiding  an hyphenated word at the bottom of a page
%    \begin{macrocode}
\if@twoside
    \unless\ifScuDo
        \flushbottom
    \else
        \raggedbottom
    \fi
\else
    \ifx\@xivpt\f@size
        \raggedbottom
    \else
        \flushbottom
    \fi
\fi
\brokenpenalty=10000
%    \end{macrocode}
%
% Here comes a questionable command and/or environment; good
% typesetting requires the baseline skip to be proportioned
% to the font size, generally it is some 10--20\%  larger
% than the font size. In some reasonable instances a larger
% or a smaller baseline skip might be required; the \LaTeX\ 
% kernel allows to use the |\linespread| command; in the
% previous versions of this bundle a command |\interlinea|
% and an environment \texttt{interlinea} were defined so as
% to allow setting the line spread factor. The experience has
% shown that some students tend to use this command so as
% to typeset a poor and thin thesis on more pages. Well, every
% instrument can be judiciously or maliciously used; this is
% one of those double sided instruments.
%    \begin{macrocode}
\def\interlinea#1{\linespread{#1}\selectfont}
\def\endinterlinea{\par}
%    \end{macrocode}
% But whatever might be the current line spread factor within
% figures and tables we reset this factor to the unit value;
% floating bodies do not belong to this or that section
% of text where a different spread factor might be reasonable.
%    \begin{macrocode}
\def\@floatboxreset{%
        \reset@font
        \linespread{1}%
        \normalsize
        \@setminipage
}
%    \end{macrocode}
% Since we are at it we define the floating bodies placing
% parameters; not only the values ``here'', ``page top''
% and ``page bottom'', that we set as the default ones
% leaving to the user to explicitly specify the
% ``page of floats'', but also the numerical and geometrical
% parameters that control the float placements.
% These geometrical parameters are critical and everybody
% has his/her own ideas of what are the best values for them.
% According to my experience the parameters set by this
% extension file work quite well but I would not suggest
% them for every kind of typewritten document.
%
% In particular the zero value for the text fraction appears
% strange, but students have the tendency to create large figures
% (more often than large tables) and these tend to clog the
% figures queue. A 100\% space for the top of page figures
% and a requirement of 0\% text allows large floats
% to exit the queue provided they do not exceed the
% text height.
%    \begin{macrocode}
\def\fps@figure{htb} \def\fps@table{htb}
%
\setcounter{topnumber}{2}
\def\topfraction{1}
\setcounter{bottomnumber}{1}
\def\bottomfraction{.5}
\setcounter{totalnumber}{2}
\def\textfraction{0}
\def\floatpagefraction{0}
\setcounter{dbltopnumber}{2}
\def\dbltopfraction{1}
\def\dblfloatpagefraction{0}

%    \end{macrocode}
% One command that the default definition does not satisfy
% me very much is the |\caption| command; of course it is
% possible to configure the caption styles by means of the
% \pack{caption} package and its handy set of option styles
% specified by means of the \textit{key$=$value} syntax;
% but if this package is not used, then it is convenient
% to redefine the internal |\@makecaption| macro that
% performs the job as explained hereafter. Since it does
% not cost much I add also the definition of |\captionof|.
%
% The issue is that I prefer a narrower justified caption
% rather than one where the last line is just a short word
% or the right segment of an hyphenated last word. If one
% sets the |\finalhyphendemerits| parameter to an extremely
% high value, one might succeed in avoiding a hyphenated
% penultimate  caption line, but this might lead to a very
% loose typesetting of the caption paragraph, especially
% if the horizontal box that contains it hardly exceeds the
% caption width. I initially set the caption width (a new
% length) to the overall text width diminished by 3em; then
% if the caption text, inclusive of the caption label and
% number, is shorter than the text width it is typeset as
% centred text; if it exceeds the text width, it is set as
% a justified paragraph whose line width equals the
% established caption width; but in any case the last
% line of the paragraph is measured and, if it is shorter
% than one third of the caption width, this width is
% shortened a little bit and the paragraph is set again
% with this shortened caption width; in order to be sure
% that one iteration is sufficient, the caption width
% shrinking must be computed according to the number of
% lines the paragraph occupies.
%
% In order to count the number of lines of the caption
% paragraph, it is necessary to recall that the first line
% occupies a vertical space that equals |\topskip| while
% the other lines occupy a vertical space equal to
% |\baselineskip|; the latter one is generally larger
% than |\topskip| therefore the integer division of the
% height of the vertical box divided by the |\baselineskip|
% is truncated to count one line less than the true value.
% In our case if the caption does not stay in one line,
% its text is typeset in a vertical box with a line spread
% of~0.95 so as to make the caption a little more compact
% than the regular text.
% Therefore the baseline skip is a little shorter than the
% regular one, but it should still exceed the default
% |\topskip|; when we first typeset the caption in a
% vertical box we strip off the last line and we must
% remember the presence of this line in our arithmetics.
%
% If the length of the stripped last line is larger that
% one third of the caption width, then the vertical box
% is recomposed by re\-stacking the individual lines,
% but if this last line is shorter than one third of the
% caption width, this width is recomputed in this way:
% let $N-1$ be the number of lines obtained with the
% integer division, i.e.\ one line less than those
% actually contained in the vertical box. Let $x$ be
% the initial caption width, and $x_2$ the last line
% width; let $y$ be the new caption width; then the total
% length of the caption of width $x$ is $Nx+x_2$ and this
% should be distributed over $N+1$ lines; if we obtained
% the new caption width $y$ by dividing the total length
% by $N+1$ we should be able to typeset the whole
% caption with all the lines of equal length. This does
% not actually take  place because the new lines do not
% necessarily contain the same amount of inter word space,
% some words might have been hyphenated in a different way,
% and so on.
%
% Moreover we do not want a caption that barely exceeds the
% length of |\captionwidth| to be retyped into a two
% line caption width that is about one half of the width of
% the other captions.
%
% Therefore we allow for some white space in the last
% line by computing the new caption width with the following formula
% \[ y=\frac{(N+0.5)x + x_2}{N+1} \]
% By the way, if the package \pack{caption} has been loaded
% in the preamble, this specific recalculation must not be 
% executed at all; it is necessary to delay this definition
% at |\AtBeginDocument| so we know for sure if that the package
% has been actually loaded. Of course if the \pack{caption} package
% has not been loaded the  |\captionsetup| macro is not
% defined and its use (for example, as a leftover of some
% experiments with the \pack{caption} package) would produce
% an error for this reason we define it as a gobbling macro
% so as to neutralise its presence.
%
% At this point it is very simple to add the |\captionof|
% command in order to use a caption also for figures that
% are set outside floating bodies; it might be useful when
% small figures or tables are inserted in unusual places,
% such as a table, a minipage, or similar non floating
% environments. The only thing that suffices to insert is
% the name of the corresponding not used float.
%    \begin{macrocode}
\AtBeginDocument{\@ifpackageloaded{caption}{}{%
\providecommand\captionsetup{}\renewcommand\captionsetup[2][]{}%
\providecommand\captionof[1]{\def\@captype{#1}\caption}
\let\originalcaption\caption
\newdimen\captionwidth
\newdimen\fullwidth 
\if@twocolumn\fullwidth=\columnwidth\else\fullwidth=\textwidth\fi
\belowcaptionskip=\abovecaptionskip
\long\def\@makecaption#1#2{%
   \begingroup
   \small \parskip\z@ \parindent\z@
   \finalhyphendemerits 100000\relax
   \linespread{0.95}\selectfont
   \vskip \abovecaptionskip
   \ifdim\hsize<\fullwidth \captionwidth=\hsize\else
   \captionwidth=\dimexpr\hsize-3em\fi
   \setbox0 \hbox{\ifblank{#1}{}{#1.\quad}#2}%
   \ifdim\wd0>\hsize
      \setbox1 \vbox{\hsize=\captionwidth
      \unhbox0\par\global\setbox2\lastbox}%
      \setbox2\hbox{\unhbox2}%
      \ifdim\wd2<0.333333\captionwidth
         \count255=\ht1 \advance\count255 \dp1
         \divide\count255\baselineskip
         \advance\count255\@ne
         \@tempdima=\wd2
         \advance\@tempdima \count255\captionwidth
         \advance\@tempdima 0.5\captionwidth
         \advance \count255\@ne
         \divide \@tempdima \count255
         \captionwidth=\@tempdima
         \setbox0 \vbox{\hsize\captionwidth
         #1.\quad#2}
      \else
         \setbox2\hbox to\captionwidth{\unhbox2 \hfill}%
         \setbox0\vbox{\unvbox1\box2}%
      \fi
   \fi
   \makebox[\hsize]{\box0}%
   \vskip \belowcaptionskip
   \endgroup
}}}

%    \end{macrocode}
%
% ^^A                      SDFIGURE
% Always delayed to the |\begin{document}| statement execution
% we  define a new (possibly floated) environment
% |SD| that creates a box where there is the main environment
% contents with its caption to a side, precisely to the side
% close to the outer  margin of the page. This implies that
% in one side printing  the caption is always on the right,
% while in two side printing the caption is on the right in
% right/odd pages and on the left in left/even pages.
%
% The environment opening statements requires the name of
% a floating object, even if this environment does not float
% anything, but the name is just used to describe its contents;
% it is possible therefore to specify \verb|\begin{SD}{figure}|
% or \verb|\begin{SD}{table}|, or whatever other environment
% name defined with the \pack{float} package functionality;
% this little repetition is for using the correct prefix to
% the caption text even when the environment is used outside
% a formal floating one. In the following paragraphs we speak
% about an ``image'' to mean the ``environment main contents''.
%
% This new special environment box gets actually  floated if
% explicitly inserted within a formal floating  environment;
% But this special environment box is usable even as 
% a standalone environment, or as an object in a fixed position,
% such as when it is set within a mini-page or another similar
% environment; in such situations it is not required to use
% |\captionof|; actually this command is simulated within this
% very \amb{SD} environment
%.
% The actual commands for inserting
% the picture and its caption within the |figure| environment
% have names starting with |SD|, namely |\SDcaption| and
% |\SDimage|. Its syntax is as follows:
%\begin{flushleft}\obeylines
%\Bambiente{SD}\marg{contents kind}\meta{*}\oarg{fraction}
%\cs{SDimage}\marg{image commands}
%\cs{SDcaption}\oarg{lof entry}\marg{caption}\marg{label}
%\Eambiente{SDfigure}
%\end{flushleft}
% The \meta{contents kind} equals one of the formal floating
% environments; this information is used to prefix the
% caption text with the proper name and number.
%
% The meaning of the optional asterisk is that its presence
% allows protrusion of the box into the external margin,
% while its absence forces the box width to equal the text
% width. The optional \meta{fraction} is used to specify the
% fraction of the current object width that is dedicated to the
% caption, the remaining space being dedicated to the image;
% actually both widths are reduced a little bit to allow a space
% between the caption and image boxes. But remember:  a small
% value of this \meta{fraction} implies a narrow caption.
%
% Since it is necessary to maintain in separate independent
% areas the picture and the caption, and we need to shuffle
% them when odd or even pages are involved, their handling
% is particular. The |\SDimage| (or |\SDtabular| as an alias
% of |\SDimage|) command syntax is very simple:
%\begin{flushleft}
%\cs{SDimage}\marg{image commands}
%\end{flushleft}
% where the \meta{image commands} may be anything suitable so be
% used within a a floating environment; a picture, a tabular,
% a drawing, a box, any command to place the object in the
% correct position within the dedicated space; any argument
% is substantially valid with the caution that this ``anything''
% remains local to that particular ``image''. In any case when
% the ``mage'' is a real picture  and has  to be resized  to a
% relative dimension, remember that both the picture and its
% caption are typeset within boxes narrower than the total text
% width or the total environment width; the best way to access
% the width is to use its |\hsize| value.
%
% The |\SDcaption| is very similar to that of |\caption| except
% for a small detail; its syntax is the following:
%\begin{flushleft}
%\cs{SDcaption}\oarg{lof entry}\marg{image}\marg{label}
%\end{flushleft}
% where the parameters have the following meanings.
%\begin{enumerate}
%\item The \oarg{lof entry} optional bracket delimited argument
%   contains the usual short caption entry for the list of
%   figures.
%\item The \marg{caption} mandatory brace delimited argument
%    contains the image caption.
%\item The \marg{label} mandatory brace delimited argument
%   contains the string that is usually passed to the |\label|
%   command; this string is mandatory in the sense that it
%   is the |\SDcaption| command that takes care of issuing
%   the |\label|\marg{label} command but, since this \meta{label}
%   is used to recover the page number (possibly after a second
%   compilation run) in order to establish the page number parity,
%   it is necessary that the |\label| command is issued at the
%   proper moment and that at the same time the \meta{label}
%   string is globally saved into a suitable macro so as to use
%   it at the right moment.
%\end{enumerate}
% Remember: the TOPtesi caption setting differs forme the
% standard one and from its redefinitions with other packages
% such as the |caption| one. The TOPtesi particular redefinition
% tries to adjust the caption with in such a way as to avoid an
% unusually short last line in multi line captions. Therefore
% it may be noticed that sometimes some captions are slightly
% less wide than expected. It is not a bug, it may be considered
% a feature.
%
% Here comes its documented code. First some dimension registers,
% counters and boxes must be defined.
%    \begin{macrocode}

\newdimen\SDcaptionwidth
\newdimen\SDfigurewidth
\newdimen\SDlinewidth
\newbox\SDfigbox \setbox\SDfigbox\hbox{}%
\newbox\SDcapbox \setbox\SDcapbox\hbox{}%
\newcounter{SDpage}%
\newcounter{NumberSD} \setcounter{NumberSD}{0}
\providecommand*\theNumberSD{\value{NumberSD}}
\newif\ifNumberFloat \NumberFloattrue
%    \end{macrocode}
%
% Then we redefine the |\caption| command in order to add
% the |\caption*| variety that prints just the caption
% without any numbering and any prefix; of course it does
% not increment the suitable float counter nor it adds any
% entry to the list of any float.
% We have to memorise  the original definition and add
% the |xparse| extended new definition. Notice that this
% new |\caption| macro accepts the same syntax as the
% original one when no asterisk is specified, while when
% the |\caption*| command is used only the long caption
% argument is used, and the list of floats entry, if
% present, is neglected. This new command may be directly
% used by the final user in those rare circumstances where
% just an explanation is required and the object being
% explained does not require to be cross referenced.
%^^A CONFLITTO CON \CAPTION* DI LONGTABLE? -- CONTROLLARE<<<< !
%    \begin{macrocode}
\NewDocumentCommand\captionSD{s O{} m}{% o: optionale fra graffe
  \IfBooleanTF{#1}%
    {%
     \@makecaption{}{#3}%
    }{%
      \ifempty{#2}%
        {%
          \originalcaption{#3}%
        }{%
          \originalcaption[#2]{#3}%
        }%
    }%
}
%    \end{macrocode}

% Then the environment |SDbox| is defined by means of the
% facilities of the |xparse| package; this allows the use of
% the ``non delimited'' asterisk argument.  It also requires other
% arguments, the first of which is mandatory and specifies the
% contents kind, be it a figure, or a table, or an algorithm,
% or anything else that has been introduced with the facilities
% of the |float| package.
% Among the opening the commands we
% deal with the (optional) asterisk dependent box width.
%    \begin{macrocode}
\DeclareDocumentEnvironment{SDbox}{ G{} s O{0.5} }{%------OPENING
\NumberFloattrue
\ifempty{#1}%
   {\NumberFloatfalse}%
   {\def\@captype{#1}}%
\IfBooleanTF{#2}%
  {\SDlinewidth=\dimexpr\textwidth+0.8\marginparwidth}%
  {\SDlinewidth=\textwidth}%
%    \end{macrocode}
% Then we fix our attention to the correct use of the optional third
% argument, whose descriptor  assigns it a default value of |0.5|:
%    \begin{macrocode}
\SDcaptionwidth=#3\SDlinewidth
\ifdim\SDcaptionwidth<0.3\SDlinewidth 
  \SDcaptionwidth=0.3\SDlinewidth
\fi
\ifdim\SDcaptionwidth>0.7\SDlinewidth
  \SDcaptionwidth=0.7\SDlinewidth
\fi
\SDfigurewidth=\dimexpr\SDlinewidth-\SDcaptionwidth\relax
\SDcaptionwidth=\dimexpr\SDcaptionwidth-\columnsep\relax
\SDfigurewidth=\dimexpr\SDfigurewidth-\columnsep\relax
%    \end{macrocode}
% We add some test to verify if the environment is used within
% the |figure| floating environment, or it is used ad a fixed
% object; we first est if we are within a floating environment;
% if not we define |\@captype| to contain the string `figure';
% if yes we have to test if we are really within the |figure|
% environment; if not, we locally define |\@captype|; if yes
% we relax. This means that we can insert the |SDfigure| even
% within a |table| environment as sometimes  a tabular and a
% drawing should float together (may be because a drawing with
% its own caption represents the tabular entries).
%
% And this is all for the opening commands; all these settings
% shall be used by |\SDimage| and |\SDcaption| described in the
% sequel; here we describe the environment closing commands,
% taking into account that the above commands have saved the
% \meta{label} string into the |\@tempSD| macro and have set
% their image and caption-text respectively into boxes
% |\SDfigbox| and |\SDcapbox|.
%
% The code tests if the saved \meta{label} string is empty;
% this should never take place, but it is better to perform
% such task' If the string is  empty, a fall back value equal
% to the page number of the page that is still open at
% processing time is used; may be a warning message would
% be helpful, but since this empty string should never be
% empty, it appears as a super luxury feature, that we can
% do without.
% But if the string ss not empty, we test if the saved
% |\label| data, possibly read from the |.aux| file on a second
% typesetting run,  gas the normal \LaTeX\ structure or the
% |hyperref| modified one; the former contains two braced
% arguments, while the latter contains five such arguments;
% in both cases the interesting page number is the second
% argument. The existence of a |ver@hyperref.sty| macro
% allows to know if the |hyperref| package has been invoked
% in the preamble, therefore upon testing this macro  we
% know if we should use |\@secondoffive| or |\@secondoftwo|
% to read the real page number of the current |SDfigure|
% environment. Before reading the actual value of the
% |.aux| file |\label| saved data, we must access the
% string that contains these data but we have to expand
% several times the saved macros until we have the real
% string in |\@temD|.
%    \begin{macrocode}
}{%---------------------------------------------------CLOSING
  {\ifcsdef{r@\@tempSD}%
      {% 
         \edef\@tempD{r@\@tempSD}%
         \edef\@tempD{\expandafter\csuse\expandafter{\@tempD}}%
         \edef\@tempD{\@tempD}%
         \ifcsdef{ver@hyperref.sty}%
           {%
             \edef\@tempD{\noexpand\@secondoffive\@tempD}%
           }%
           {%
             \edef\@tempD{\noexpand\@secondoftwo\@tempD}%
           }%
         \edef\@tempD{\@tempD}%
         \setcounter{SDpage}{\@tempD}%
      }%
      {% 
         \setcounter{SDpage}{\thepage}%
      }%
  }%
%    \end{macrocode}
% At this point we have all the necessary pieces of information
% and we can shuffle the caption and image boxes depending
% on one or two side typesetting, and on the parity of the
% correct page number.
%    \begin{macrocode}
\ifbool{@twoside}%
  {%
    \ifnumodd{\value{SDpage}}%
      {%
        \makebox[\textwidth][l]{\makebox[\SDlinewidth]{%
          \usebox{\SDfigbox}\hfill\usebox{\SDcapbox}}}%
      }{%
        \makebox[\textwidth][r]{\makebox[\SDlinewidth]{%
          \usebox{\SDcapbox}\hfill\usebox{\SDfigbox}}}%
      }%
  }{%
    \makebox[\textwidth][l]{\makebox[\SDlinewidth]{%
      \usebox{\SDfigbox}\hfill\usebox{\SDcapbox}}}%
  }
}
%    \end{macrocode}
%
% The |\SDcaption| code appears to be strangely complicated;
% but it actually performs several actions;  the first
% (optional) argument is intended to collect the ``short
% caption'' to be used, if specified, for the list of floats
% entry; if not specified ,as usual, the ``long caption''
% second argument is used for this entry. The (optional)
% brace delimited argument, on the contrary, is the label
% string that usually is passed as the argument of the
% |\label| command. If specified it is this |\SDcaption|
% macro duty to enter this command at the proper moment;
% the caption prefix is established by the value of the
% first (optional) argument of the |SDbox| environment,
% or by the floating environment where the |SDbox| one is
% nested. On the opposite if this macro |\SDcaption| argument
% is not specified, the caption is not numbered nor prefixed,
% and cannot be cross referenced; there are circumstances
% when this situation is very useful, but it is up to the
% user to decide when it is useful for his/her needs.
% interpretation.
%
% In this macro definition the boolean switch |NumberSD|
% has been already set by the opening statement commands
% of the |SDbox| environment; it is true only if the first
% environment (optional) argument has been specified or
% the environment is nested within a floating environment.
% Therefore if this boolean is false the unnumbered caption
% gets typeset, otherwise a second boolean is tested, the
% contents of the third (optional) argument of this macro
% |\SDcaption|. If it is specified, the numbered and prefixed
% caption gets typeset and this argument is used to label
% this caption; otherwise and unnumbered and not prefixed
% caption gets typeset with and internally generated label
% that allows to fetch the final page number so as to decide
% if the caption has to be typeset to the right or to the
% left side of the object depending on the page number parity.
%    \begin{macrocode}
\NewDocumentCommand\SDcaption{ O{} m  G{} }{%
  \notbool{NumberFloat}%
    {% 
      \stepcounter{NumberSD}%
      \edef\@tempSD{SD\the\c@NumberSD}%
      \begin{lrbox}{\SDcapbox}%
        \parbox[t]{\SDcaptionwidth}%
        {%
          \captionSD*{#2}\relax
          \expandafter\label\expandafter{\@tempSD}%
        }%
      \end{lrbox}%
    }%
    {% 
      \ifempty{#3}%
        {% 
          \NumberFloatfalse
          \stepcounter{NumberSD}%
          \edef\@tempSD{SD\the\c@NumberSD}%
          \begin{lrbox}{\SDcapbox}%
            \parbox[t]{\SDcaptionwidth}%
            {%
              \captionSD*{#2}\relax
              \expandafter\label\expandafter{\@tempSD}%
            }%
          \end{lrbox}%
        }{% 
          \begin{lrbox}{\SDcapbox}%
            \parbox[t]{\SDcaptionwidth}%
            {% 
               \captionSD[#1]{#2}\label{#3}\gdef\@tempSD{#3}%
            }%
          \end{lrbox}%
      }%
    }%
}
%    \end{macrocode}
% 
% We call the object to be shown within the |SDbox| with the
% name ``image'' wich can be anything that can be enclosed
% in a parbox or a minipage; it can be an actual picture
% imported with an |includegraphics| command, it can be a
% drawing or a diagram created with the |picture|, or |tikz|,
% or |pgfplots|, or |pstricks| commands; it can be a table,
% an algorithm, an example, a theorem; anything that can be
% typeset within a box. With this name in mind we create
% the |SDimage| command to insert the object into our |SDbox|.
% A |\SDtabular| command is an alias to |\SDimage|; the user
% may define other such aliases for inserting other objects
% that suit his/her preferences and the logical contents.
%
% Also the |\SDimage| command is not so simple; in order to
% have both caption and image boxes aligned at the top, we
% must be sure that such boxes have their reference points
% at their top left corner; due to the heterogeneity of such
% ``images'', we need to lower the image box to have the
% upper left corner raised with respect to the base line the
% amount of a capital `M' in the current font.
%    \begin{macrocode}
\newcommand\SDimage[1]{%
   \setbox0\hbox{M}\dimen0\ht0\relax
   \begin{lrbox}{\SDfigbox}\parbox[t]{\SDfigurewidth}{#1}\end{lrbox}%
   \dimen2\dimexpr\ht\SDfigbox+\dimen0\relax
   \setbox\SDfigbox\hbox{\raisebox{-\dimen2}{\box\SDfigbox}}%
}
\let\SDtabular\SDimage
%    \end{macrocode}
% We would not suggest to use a floating or a fixed |SDbox|
% environment for all figures, tables or what else; its
% use simply depends on the box contents; this environment
% may be useful when a normal setting would cause a lot of white
% space around the displayed material.
%
% We define the |\goodpagepbreak| command; we have to
% distinguish if the command is issued in vertical or in
% horizontal mode; in the former instance we assume it
% has been issued between two paragraphs and a
% potential page break is required before the second
% paragraph; while in the latter case we assume that
% a |\vadjust| element should be inserted so that a page
% break takes place after the current line has been typeset.
% The command accepts an optional argument that represents
% the maximum number of lines that should be present on
% the page in order to issue a |\newpage| command; in other
% words if the number of lines available on the page are
% less or equal to the optional number, then a |\newpage|
% command is issued, otherwise noting is done; the test
% |\ifhmode| is used in order to avoid any action if in math mode.
%    \begin{macrocode}
\newcommand\goodpagebreak[1][4]{\ifvmode
\unless\ifdim\dimexpr\pagegoal-\pagetotal>#1\baselineskip\newpage\fi
\else\ifhmode\vadjust{\newpage}\fi\fi}
%    \end{macrocode}
%
% The (deprecated) option \opz{chapterbib} requires a redefinition
% of the \texttt{thebibliogrpahy} environment in case a
% separate reference list is required for every chapter.
% The point is that for this task the reference key must
% contain also the chapter number; the rest is simply a
% redefinition of the environment that behaves differently
% according to the chosen option. In any case the bibliography
% goes to the table of contents as an unnumbered chapter or
% section.
%    \begin{macrocode}
\def\redef@bibitem{\def\@bibitem##1{\item\if@filesw
       \immediate\write\@auxout
       {\string\bibcite{##1}{\thechapter.\the\c@enumi}}\fi\ignorespaces}}
%
\def\thebibliography#1{%
 \ifchapterbibliography\section*{\bibname}\relax
    \if@twoside\markright{\bibname}\fi
    \addcontentsline{toc}{section}{\bibname}\relax
    \redef@bibitem
    \list{[\thechapter.\arabic{enumi}]}{%
    \settowidth\labelwidth{[\thechapter.#1]}\leftmargin\labelwidth
    \advance\leftmargin\labelsep\itemsep\z@ plus 1pt\parsep\z@
    \usecounter{enumi}}
 \else
    \chapter*{\bibname}\relax
    \@mkboth{\bibname}{\bibname}\relax
    \addcontentsline{toc}{chapter}{\bibname}\relax
    \list{[\arabic{enumi}]}{\settowidth\labelwidth{[#1]}%
    \leftmargin\labelwidth
    \advance\leftmargin\labelsep\itemsep\z@ plus 1pt\parsep\z@
    \usecounter{enumi}}
 \fi
 \def\newblock{\hskip .11em plus .33em minus -.07em}
 \sloppy
 \sfcode`\.=1000\relax}

\let\endthebibliography=\endlist
%    \end{macrocode}
%
% For what regards footnotes nothing is changed except
% resetting the line spread to one, in case the current
% value is different.
%    \begin {macrocode}
\long\def\@footnotetext#1{\insert\footins{\linespread{1}\footnotesize
    \interlinepenalty\interfootnotelinepenalty
    \splittopskip\footnotesep
    \splitmaxdepth \dp\strutbox \floatingpenalty \@MM
    \hsize\columnwidth \@parboxrestore
   \edef\@currentlabel{\csname p@footnote\endcsname\@thefnmark}%
   \@makefntext{\rule{\z@}{\footnotesep}\ignorespaces#1\strut}}}
%    \end{macrocode}
%
% These last heterogeneous definitions are partly important
% and partly residues of the good old times of MS-DOS~v.3
% when a Ctrl-Z character would be placed at the end of files.
% It's a long time that such DOS version is not being used,
% but some old time files might still be around.
%
% The cryptic code that redefines the comma in math mode
% establishes that this character is a normal math character,
% instead of a math punctuation mark. Actually the code
% that defines the mathematical active comma is a new
% addition that lets the comma perform correctly in its
% double function (decimal separator and punctuation mark).
% The only point where this code fails is when a list of
% numbers is typeset: when a numeric list must be typeset,
% such as, for example, $\forall i=0,\,1,\,2,\,3, n$, in the
% source code a space \emph{must} be inserted after every
% punctuating comma while no space follows a decimal comma:
% for example \verb*+$\forall i=0, 1, 2, 3,n$+.  The space
% before the |n| is not necessary (but it wouldn't hurt)
% because |n| is not recognised as a digit, therefore the
% ``intelligent'' comma inserts the necessary space by itself.
%
%    \begin{macrocode}
 \DeclareMathSymbol{\virgola}{\mathpunct}{letters}{"3B}
 \DeclareMathSymbol{\virgoladecimale}{\mathord}{letters}{"3B}
 \AtBeginDocument{\mathcode`\,=\string"8000}
 {\catcode `,=\active \gdef,{\futurelet\let@token\m@thcomma}}
\def\m@thcomma{\let\@tempB\virgola
\@tfor\@tempA:=0123456789\do{%
\expandafter\ifx\@tempA\let@token\let\@tempB\virgoladecimale 
\@break@tfor\fi}\@tempB}
%
\catcode`\^^Z=10
\topmargin 0pt
%    \end{macrocode}
%
% The TOPtesi bundle contains two new chapter like sections
% activated with the commands |\sommario| (alias |\summary|)
% and |\ringraziamenti| (alias |\acknowledgements|) respectively.
% The infix strings that start these sections depend on the
% used language.
%
% Because of this it is necessary to extend the list of infix
% string definitions provided by the |\captions|\meta{language}
% macros defined by the  \texttt{babel} or \texttt{polyglossia}
% language description files; for this purpose we define
% a macro for adding new items for these two new sectioning
% commands. This macro receives three arguments: the first
% is the babel language name, the second is the string for
% the summary name, and finally the third is the string for
% the acknowledgements name; everything is contained within
% a group and only the relevant sectioning macro is globally
% redefined. The token register `0' is normally for scratch
% usage, but the fact that its value is restored upon exiting
% the group provides the necessary protection against an
% involuntary reassignment to this register. At the same
% time if a specific language option was not specified, a
% warning message is issued, but compilation goes on any way
% without the sectioning string names. For being sure no
% other `undefined' error messages are issued, the |\summaryname|
% and |\acknowledgename| are let to |\empty|.
%    \begin{macrocode}
\providecommand*\summaryname{}
\providecommand*\acknowledgename{}
\newcommand*\ExtendCaptions[3]{{%
\@ifundefined{captions#1}{%
\PackageWarning{toptesi}{%
Language option #1 not specified\MessageBreak
Skipping any redefinition\MessageBreak}%
}{%
\expandafter\let\expandafter\@tempA\csname captions#1\endcsname
 \toks0=\expandafter{\@tempA%
    \def\summaryname{#2}%
    \def\acknowledgename{#3}}%
 \expandafter\xdef\csname captions#1\endcsname{\the\toks0}%
 }}}%
%    \end{macrocode}
% For Italian and English there are no problems; we provide
% immediately these extensions by means of the newly available macro:
%    \begin{macrocode}
\ExtendCaptions{italian}{Sommario}{Ringraziamenti}
\ExtendCaptions{english}{Summary}{Acknowledgements}
%    \end{macrocode}
% In facts the \texttt{babel} and \texttt{polyglossia}
% options for Italian and English have already been loaded
% by default; therefore both caption macros |\captionsitalian|
% and |\captionsenglish| are already defined and can be
% freely extended. For any other language the corresponding
% language option must be entered in the class opening
% statement, otherwise a warning is issued but compilation
% is not stopped. Therefore if, for example, the user wants
% to write the thesis in Spanish, the thesis main file shall
% start like this:
% \begin{verbatim}
% \documentclass[...,spanish]{toptesi}
% \ExtendCaptions{spanish}{Resumen}{Agradecimientos}
% ...
% \begin{document}
% \selectlanguahe{spanish}
% ...
% \end{verbatim}
% and the rest of the thesis will be typeset correctly in
% Spanish. Remember that Italian is the main language and
% nothing is necessary to set up the Italian defaults.
% If English is desired, then after |\begin{document}| it
% is necessary to specify the default language; for ease
% of use the following macros are defined so they can be
% used instead of the lengthy \texttt{babel} or
% \texttt{polyglossia} command; after the beginning of the
% document it is then possible to specify |\inglese| or
% |\english|, and the default language is changed to English.
% These shorthand commands can be alternated so as to switch
% from one language to the other; nevertheless remember that
% there are more correct ways to switch languages with the
% \texttt{babel} and \texttt{polyglossia} commands without
% changing the infix strings.
%    \begin{macrocode}
\def\italiano{\selectlanguage{italian}}%
\def\english{\selectlanguage{english}}%
\let\inglese\english
%    \end{macrocode}
% At the beginning of the document the following commands are
% executed; the general macro |\italiano| sets the summary
% and the acknowledgements names in Italian, as the main
% language; and the |\@chapapp| macro is redefined so as to
% agree with the default language. If a different default
% language is desired, we recall it again, it is necessary to
% do the following:
% \begin{itemize}[noitemsep]
% \item
% specify |\english| \emph{after} the |\begin{document}|
% statement, if English is supposed to be the main language, or
% \item
% specify the language name, other than Italian or English,
% among the class options; use the |\ExtendCaptions| macro
% for extending the list of sectioning commands infix strings
% as explained above; specify with |\selectlanguage| the new
% language as the default one \emph{after} the |\begin{document}|
% statement;
% \item
% if XeLaTeX\ or LuaLaTeX\ is being used for typesetting, it
% suffices to specify in the preamble the name of the other
% language to be used, by means of the |setotherlanguage|
% command, and to define the summary and acknowledgments names
% in the same way as with pdf\/\LaTeX; the same must be done
% at the beginning of the document to declare the new language
% as the default one.
% \end{itemize}
%    \begin{macrocode}
\AtBeginDocument{%
    \italiano
    \renewcommand\@chapapp{\chaptername}%
}
%    \end{macrocode}
%
% At last the subsidiary and independent packages
% \texttt{topcoman} and \texttt{topfront} are requested
% for input; the latter is loaded only when the 
% |documentclass| option |noTOPfront| has NOT been declared.
% This action, therefore has to be delayed till the ``begin
% document'' step, in order to be sure that other packages
% have possibly been loaded to typeset the title page, and/or
% other packages have been loaded to properly use the units
% of measure. This delayed loading implies that the
% commands defined in |topfront| and in |topcomand| cannot
% be used in the preamble. Notice that also the \textsl{scudo}
% option  implies that the title page is typeset with its
% special commands and environments, therefore it sets the
% suitable switches so as to avoid loading the |topfront|
% extension file. This applies also to the other modules particular to special kinds of thesis.
%    \begin{macrocode}
\AtBeginDocument{%
	\unless\ifT@Pfrontespizio\ifTOPfront
       \RequirePackage{topfront}
    \fi\fi
	\RequirePackage{topcoman}%
}
%    \end{macrocode}
%
% Last but not least, here comes the specification set for the
% \sigla{PDF/A}-1b format.
%
% First of all the patches to the macros |\not| and
% |\mapstochar| that produce problems with that format,
% because they have a declared width of 0\,pt; this
% is no problem for \texttt{pdflatex}, but it is a problem
% for the \sigla{PDF/A} format. Therefore these commands must be
% replaced by equivalent ones that do not use zero-width
% glyphs. For |\not|, another slash can be used, but in
% order to have it the right size in all math typesetting
% modes it is necessary to have a different command for
% every mode; this is achieved with the |\mathchoice|
% primitive as such:
%    \begin{macrocode}
\ifPDFTeX
\renewcommand*\not{\mathrel{\mathchoice%
	{\rlap{$\displaystyle\mkern2.5mu\mathnormal{/}$}}%
	{\rlap{$\textstyle\mkern2.5mu\mathnormal{/}$}}%
	{\rlap{$\scriptstyle\mkern2.5mu\mathnormal{/}$}}%
	{\rlap{$\scriptscriptstyle\mkern2.5mu\mathnormal{/}$}}%
}}
%    \end{macrocode}
% Actually the zero-width property of the slash must be
% simulated with a zero-width box but within this box it
% is necessary to specify the typesetting style of the math
% mode material.
%
% A similar trick is used to patch the |\mapstochar| command
% but no other glyph was found suitable for substituting
% the original one; therefore we had to make it up with the
% |picture| environment:
%    \begin{macrocode}
\renewcommand\mapstochar{\mathrel{\mathchoice
{\displaystyle\unitlength=0.0555\fontdimen6\font
   \picture(0,0)\put(1,1.5){\linethickness{0.72\unitlength}%
   \line(0,1){6}}\endpicture}
{\textstyle\unitlength=0.0555\fontdimen6\font
   \picture(0,0)\put(1,1.5){\linethickness{0.72\unitlength}%
   \line(0,1){6}}\endpicture}
{\scriptstyle\unitlength=0.0555\fontdimen6\font
   \picture(0,0)\put(1,0.55){\linethickness{0.61\unitlength}%
   \line(0,1){5}}\endpicture}
{\scriptscriptstyle\unitlength=0.0555\fontdimen6\font
   \picture(0,0)\put(1,0.2){\linethickness{0.56\unitlength}%
   \line(0,1){4}}\endpicture}
}}\fi
%    \end{macrocode}
% The above patches are introduced irrespective of using
% \texttt{pdflatex} for producing a \sigla{PDF/A} compliant file;
% one might need to produce a traditional \sigla{PDF} file, or even
% a DVI file, to be converted to the \sigla{PDF} format, in order to
% possibly further transform it into a \sigla{PDF/A} one by means
% of other useful packages, such as \pack{pdfpages}.
%
% \subsection{The \opz{classica} option}
% As mentioned above, the \opz{classica} option was
% devised in order to cope with theses in humanities and
% the specifications came from Paolo Ciacchi, a student
% of the University of Trieste, who was preparing a master
% thesis in classical Greek philology. The necessity
% of the large normal size derived from the necessity of
% having clear mark-up signs among the myriad signs
% philologists use, that sometimes I believe that the
% philological mark-up requires typesetting environments
% similar to the mathematical ones: in facts the material
% to be typeset does not form linear sequences, as in plain
% text, but bi-dimensional structures as in mathematics.
%
% The other requirements involve the title page and I agreed
% to implement them, since they are likely to be useful
% for other universities as well; the effort for localising
% this bundle makes this point less stringent compared to
% previous versions, but there are other layout fine points
% that cannot be solved with a simple substitution of infix
% strings.
%
% The singular and plural masculine and feminine strings
% for ``candidate'' are redefined in the |topfront| module.
% For other languages the configuration file is very handy.
%
% Since theses in humanities may end up to occupy several
% volumes\footnote{Classically ``tomo'' in Italian means ``volume'',
% although the latter word is valid also in Italian with 
% the same meaning but appears to be  neglected by the
% humanists; in English ``tome'' indicates a ``large book'';
% in Italian the meaning nuance of ``tomo'' is a little
% different because it is used to indicate the volumes into
% which a large printed work is divided; the humanists, as
% usual, know well their words and use them properly!} a
% mechanism is set up to create a different title page for
% each volume; we need a volume counter and a command to
% start each volume.
%
% The localisation allows to change the infix string that
% is going to be printed in the title page. But when several
% tomes are involved, instead of using the |\frontespizio|
% macro or |frontespizio| environment, use the |\tomo| macro. 
% There is no equivalent to the |frontespizio| environment.
% |\tomo| provides to stepping up the volume counter before
% actually printing the new tome title page.
%    \begin{macrocode}
\newcounter{tomo}
\newcommand*{\tomo}{\clearpage\stepcounter{tomo}\boolfalse{topTPTlogos}%
   \fr@ntespizio}
%    \end{macrocode}
% Folios as well are to be redefined and also the page styles
% require a redefinition so as to being able to use old style
% numbers. The macro |\lapagina| (it's not a case the this
% macro name is the direct translation of ``thepage'')
% contains the folio number; if it is in roman numerals
% nothing happens, but if the old style numbers are required
% the folio is surrounded by the proper macros so as to expand
% the folio macro before typesetting it in old style. The page
% style does not need an actual redefinition, because the
% original definition in file \texttt{toptesi.sty}
% already uses |\lapagina| as the typeset folio indicator.
%    \begin{macrocode}
\renewcommand*{\lapagina}{%
    \ifnumeriromani
        \thepage
    \else
        \if@ldstyle
            \expandafter\oldstylenums\expandafter{\thepage}%
        \else
            \thepage
        \fi
    \fi
}
%    \end{macrocode}
% But we actually have to redefine the page style for the new
% headings because the  \texttt{classica} option changes
% completely the left and the right headers depending on the
% option \opz{autoretitolo}; in this case the left header
% contains the candidate's name and a short version of the
% thesis title, while the right heading contains the chapter
% (short) title; if this option is not in force, headings
% appear as usual. The redefinition of the \texttt{headings}
% page style is made only if this option is in force. In case
% of two-side printing where the left head and the right
% heads are different, the left heading contains the author
% name and the short title name; if there are other authors
% the first author name is printed followed by ``et al.'';
% if the over all header string exceeds the text width, a
% message is printed so as to suggest the user to 
% use the optional |\titolo| argument, the one that is
% supposed to contain the short title. In the right heading
% there is the chapter title; again if the header turns out
% to be too wide, a message is issued to the user.
%
% More complicated things are done when typesetting on one
% side; in facts the author name (possibly followed by
% ``et al.'') and the short thesis title are typeset on the
% left of the only header while the chapter title is typeset
% on the right; in the unlikely situation where these two
% elements separated by at least 2em of white space do not
% exceed the text width, they are printed, but if they do,
% as it is likely to happen, then my suggestion would be not
% to use the \opz{autoretitolo} option; but if the user
% absolutely wants this layout, then the author's name,
% possibly followed by ``et al.'', and the short thesis title
% are set in a vertical box; the same happens for the
% chapter title in another box; these texts are set with
% ragged margins, and eventually these boxes are set one
% next to the other with an intermediate glob of infinitely
% stretchable glue, and finally set in the header box with
% a rule underneath the two of them. The result might be
% barely acceptable if both vertical boxes do not exceed
% two lines and no words have to be hyphenated, but in
% general I believe it is an ugly layout; the user is warned!
% Remember also that |\@@author| contains the author's name 
% without the ID number if any.
%    \begin{macrocode}
\if@utoretitolo
   \if@twoside
%
       \renewcommand*{\ps@headings}{\let\@mkboth\markboth%
       \def\@oddfoot{\null \hfill \textbf{\lapagina} \hfill \null}%
       \let\@evenfoot\@oddfoot
       \def\@evenhead{%
           \setbox\@intesta\hbox{\ifPDFTeX\latintext\fi
           \footnotesize\strut\textsc{%
           \@@author\ifcsvoid{@secondauthor}{\ et al.}: \@stitle}%
           }%
       \ifdim\wd\@intesta>\textwidth\headwrn{\titolo}\fi
       \underline{\makebox[\textwidth]{\box\@intesta}}}%
       \def\@oddhead{\ifPDFTeX\latintext\fi
           \setbox\@intesta\hbox{%
               \footnotesize\strut\textsl{\rightmark}}%
       \ifdim\wd\@intesta>\textwidth \headWarn{\section}\fi%
       \underline{\makebox[\textwidth]{\box\@intesta}}}%
       \def\chaptermark##1{\markright{\thechapter\ -- ##1}}%
       \def\sectionmark##1{}}%
   \else
       \renewcommand*{\ps@headings}{\let\@mkboth\markboth
           \def\@oddfoot{\null \hfill \textbf{\lapagina}\hfill \null}%
           \let\@evenfoot\empty\let\@evenhead\empty
           \def\@oddhead{{\setbox\z@\hbox{\ifPDFTeX\latintext\fi
           \footnotesize\textsc{%
           \@@author\ifcsvoid{@secondauthor}{}{\ et al.}: \@stitle}}%
           \setbox\tw@\hbox{\ifPDFTeX\latintext\fi
           \footnotesize\textsl{\rightmark}}%
           \dimen@=\dimexpr2em + \wd\z@ + \wd\tw@\relax
           \ifdim\dimen@<\textwidth \relax
           \else
             \setbox\z@\vbox{\hsize.48\textwidth\parindent\z@
             \raggedright
             \ifPDFTeX\latintext\fi\footnotesize\textsc{%
             \@@author\ifcsvoid{@secondauthor}{}{\ et al.}: \@stitle
             }}%
             \setbox\tw@\vbox{\hsize.48\textwidth\parindent\z@
             \raggedleft
             \ifPDFTeX\latintext\fi\footnotesize
                 \textsl{\rightmark}}%
             \fi
           \setbox\@intesta\vbox to\z@{%
                \vss\hbox to\textwidth{\strut\box\z@\hfill\box\tw@}}%
           \underline{\box\@intesta}}}%
           \def\chaptermark##1{\markright{\thechapter\ -- ##1}}}
   \fi
\fi
%    \end{macrocode}
%
% The footnote rule separator is also longer than the default
% one. Dealing with notes the footnote separator is also
% changed as well as the footnote skip. But the humanists
% like to have also unnumbered notes within numbered
% ones, as if they were two separate sets; of course
% specialised extension modules, such as, for example,
% the |eledmac| or |reledmac| packages, are available on
% CTAN, but as a poor man solution the command |\nota| was
% introduced that inserts a note with a symbol as a note
% mark; the symbol \emph{must} be a mathematical one as
% the dagger or the double dagger; by default it is the
% asterisk.  At the same time the default definition of
% the mathematical  asterisk is that of a binary operator;
% I have experienced that the note symbol gets a
% better positioning if it is used as an ordinary symbol
% through |\mathord|.
% This is why its math code is redefined.
%
% A final unusual request was to be able to put a blank
% unnumbered note, as a (rather wide) note separator.
% The command |\NoteWhiteLine| has to be used at the end
% of the text of the preceding note.
%    \begin{macrocode}
\renewcommand\footnoterule{%
  \kern-6\p@
  \hrule\@width.4\columnwidth
  \kern5.6\p@}
\setlength\footnotesep{12\p@}
\setlength{\skip\footins}{24\p@ \@plus 4\p@ \@minus 2\p@}
\newcommand*\nota[1][\mathord{*}]{%
    \xdef\@thefnmark{\ensuremath{\m@th#1}}\@footnotemark\@footnotetext
}
\newcommand*{\NoteWhiteLine}{\par\vspace*{-.3\baselineskip}}
%    \end{macrocode}
%
% The humanists asked me to create some other simple macros:
% one for skipping a whole page, without header and footer;
% another to compose a dedication page; a third one for
% typesetting a page with one or more witty sentences.
% The first request has a trivial solution, but at least
% |\paginavuota| (alias |\emptypage|) is much shorter to
% type in than its expansion.
%
% The environments \texttt{dedica} (alias \texttt{dedication})
% for the dedication, and \texttt{citazioni} (alias
% \texttt{wittysentences}) for the witty sentences are almost
% identical; both typeset their content with a reduced text
% width, half of the normal one; this column is typeset
% on the right of the page. The \texttt{dedica}/\texttt{dedication}
% environment is supposed to be used in the front matter,
% while the witty sentence environment may be used anywhere.
%
% The dedication is typeset in |\Large| font size and in
% italics; if the author wants a different size and/or a
% different shape s/he must specify it at the beginning
% of the dedication text.
% The \texttt{citazioni}/\texttt{wittysentences}
% environment typesets its material with the default font
% shape series and size, so the author has to specify any
% change s/he desires. These three commands, though, are
% available irrespective of the \texttt{classica} option,
% so that they can be used also for theses outside the
% field of humanities.
%
%    \begin{macrocode}
\newcommand*\paginavuota{\clearpage\thispagestyle{empty}\null\clearpage}
\let\emptypage\paginavuota
%
\newenvironment{dedica}{\clearpage
    \if@twoside
        \ifodd\c@page\else\thispagestyle{empty}\null\clearpage\fi
    \fi
    \thispagestyle{empty}%
    \list{}{\labelwidth\z@
    \leftmargin.5\textwidth
    \parindent\z@
    \raggedright\LARGE\itshape}\item[]
}{%
    \endlist\clearpage
}
\let\dedication\dedica \let\enddedication\enddedica
%
\newenvironment{citazioni}{%
    \clearpage\thispagestyle{empty}
    \list{}{\labelwidth\z@
    \leftmargin.5\textwidth
    \parindent\z@
    \raggedright}\item[]
}{%
    \endlist\clearpage
}
\let\wittysentences\citazioni \let\endwittysentences\endcitazioni
%    \end{macrocode}
% \iffalse
%</topsty>
% \fi
%
% \subsection{The \texttt{topfront.sty} code}\label{sec:topfront}
% This file is input by \texttt{toptesi}, version 6 and higher
% provided that no title page or thesis type options have been
% specified by the user; it can also be used as an independent
% extension module with (hopefully) any document class.
%
% It contains all the definition for the composition of just
% the title page od ``standard'' theses along the style
% requirements of \texttt{toptesi}, version 5.x or higher. It is not
% loaded by |toptesi| even when typesetting the title page
% for theses made according the requirements of the ScuDo
% doctoral school; with the \textsf{scudo} option to
% |toptesi| loading of |topfront| is inhibited and in its
% place another extension module, |toptesi-scudo|, is loaded.
% The same holds true with the other special thesis modules.
% This latter files are described in a subsequent sections.
%
% It |topfront| makes use of an optional configuration file
% where the user can define a lot of default information and
% all the infix language dependent strings that are peculiar
% to the title page. Among other things it can typeset
% also the copyright page on the verso of the title page if the
% |\retrofrontespizio| macro argument is not empty.
%
% This file specifies that it requires the \LaTeXe\ format
% and identifies itself. Since this module might be
% used to extend the performances of other classes, this
% file does not specify any input encoding, on the assumption
% that the calling class already provided this information.
% \iffalse
%<*topfrn>
% \fi
%
% In oder to use |topfront| as a stand-alone extension package
% with other classes it is necessary to verify if the \opz{classica}
% option  corresponds to a valid setting of |\ifclassica|;
% since this test is defined in the |toptesi| class,
% here we need to verify its existence and, in case, to set
% its value to |false|. This implies two points: we need a
% powerful macro package to test a ``switch'', and, when this
% |topfront| module is used as a stand-alone one, the settings
% of the \opz{classica} options are not available. This requires
% to define switches and  counters, besides options for this
% package, even if they may be specified to the |toptesi|
% class; of course when |topfront| is used as a standalone
% package, the class might not be |toptesi| therefore a complete
% set up even with the \opz{classica} option must be set again.
%    \begin{macrocode}
\@ifpackageloaded{etoolbox}{}{\RequirePackage{etoolbox}}
\@ifpackageloaded{xspace}{}{\RequirePackage{xspace}}
\@ifpackageloaded{xparse}{}{\RequirePackage{xparse}}
\ifcsundef{ifclassica}{\newif\ifclassica \classicafalse}{}
\ifcsundef{if@utoretitolo}{\newif\if@utoretitolo\@utoretitolofalse}{}
\ifcsundef{if@ldstyle}{\newif\if@ldstyle\@ldstylefalse}{}
\ifcsundef{if@xivpt}{\newif\if@xivpt \@xivptfalse}{}
\ifcsundef{if@binding}{\newif\if@binding\@bindingfalse}{}
\ifcsundef{ifScuDo}{\newif\ifScuDo \ScuDofalse}{}
\ifcsundef{ifmonografia}{\newif\ifmonografia\monografiafalse}{}
\ifcsundef{iftriennale}{\newif\iftriennale\triennalefalse}{}
\ifcsundef{ifmagistrale}{\newif\ifmagistrale\magistralefalse}{}
\ifcsundef{ifdottorale}{\newif\ifdottorale\dottoralefalse}{}
\ifcsundef{ifsecondaria}{\newif\ifsecondaria\secondariafalse}{}
\ifcsundef{ifplurale}{\newif\ifplurale\pluralefalse}{}
%\ifcsundef{iffrontspizio}{\newif\iffrontespizio\frontespiziofalse}{}
%\ifcsundef{}{\newif  false}{}

\ifcsundef{ifTOPfront}{\newif\ifTOPfront \TOPfronttrue
\newif\ifevenboxes \evenboxesfalse}{}
\ifcsundef{c@tomo}{\newcounter{tomo}}{}
\DeclareOption{evenboxes}{\evenboxestrue}
\DeclareOption{cucitura}{\@bindingtrue}
\DeclareOption{14pt}{\@xivpttrue}
\DeclareOption{classica}{\classicatrue}
\DeclareOption{autoretitolo}{\ifclassica\@utoretitolotrue\fi}
\DeclareOption{oldstyle}{\ifclassica\@ldstyletrue\fi}
\ProcessOptions*\relax

%    \end{macrocode}
%
% For the title page we need a special style, in order to
% put some information in the header and some other in the
% footer, without actually changing the page layout, except
% for horizontally centring the grid in the page.
% The headers, due to a specific request of Politecnico di
% Torino, is to have the university logo(s) in the header;
% other universities maintain their logo(s) in the lower part
% of the page as it was done all the time in the past.
% We need some device to switch position to the logos,
% without actually changing the page layout. Since the
% logo(s) are sort of large, the header must smash the
% header contents, so as to avoid any modification of the
% position and size of the other parts of the page.
% The |\TPT@logobox| box shall be defined later on and
% the |\logosede| command takes care of filling it up.
% Besides these devices, the |\frontespizio| command and
% the |frontespizio| environments (with or without asterisk)
% produce either layout depending on the state of the boolean
% |topTPTlogos|.
%    \begin{macrocode}
\def\headstrut{\vrule \@depth4\p@ \@height\z@ \@width\z@}
\def\ps@titlepage{\let\@mkboth\@gobbletwo
   \def\@oddfoot{\vbox to 0.05\paperheight{\vss
      \hbox to\hsize{\hfil{\Large{\@submitdate}}\hfil}}}%
   \let\@evenfoot\@oddfoot
   \def\@oddhead{%
      \vbox to\headheight{\vss
        \iftopTPTlogos
         \hbox to\textwidth{%
           \headstrut\hfil
           \raisebox{3\baselineskip}{\usebox\TPT@logobox}\hfil\null%
           }
         \fi\par
         \ifcsvoid{@ateneo}{}{%
         \hbox to\textwidth{\hss\LARGE\rule[-1ex]{0pt}{3.3ex}\MakeUppercase{\@ateneo}\hss}}
         \vss
      }%
   }%
   \let\@evenhead\@oddhead
   \def\chaptermark##1{}\def\sectionmark##1{}%
}
%    \end{macrocode}
%
% Similarly a different title page style for typesetting
% the logos in the lower half of the page is defined; since
% it is the only style usable with the \opz{classica} option, we
% call it the |classica| page style:
%    \begin{macrocode}
      \def\ps@classica{\let\@mkboth\markboth
      \def\@oddhead{\vbox{%
      \setbox256=\hbox{\LARGE\MakeUppercase{\@ateneo}}
      \unless\ifdim\wd256<\textwidth
         \setbox256=\hbox{\resizebox{\textwidth}{!}{\box256}}%
       \fi
      \hbox to \hsize{\hss\box256\hss}%
      \ifclassica
            \hbox to \hsize{\hfil\vrule\@width\z@
        \@height2ex\vrule\@height1.4\p@\@depth-\p@\@width50mm\hfil}%
      \fi
      }}%
      \def\@oddfoot{\vbox to \dimexpr\paperheight/20\relax{\vss
     \ifclassica
        \hbox to \hsize{\hfil\raisebox{-.3ex}[\z@][\z@]{%
        \vrule\@height-2.6\p@\@depth3\p@\@width
            \dimexpr\textwidth/3}\hfil}%
      \fi
      \hbox to\hsize{\hfill{\Large{\@submitdate}}\hfill}}%
      }%
      \let\@evenhead\@oddhead
      \let\@evenfoot\@oddfoot
      }%
%    \end{macrocode}
%
% The title page information depends on the type of ``thesis''
% that is being typeset. The following commands specify the
% kind of information that is going to be typeset. Some boolean
% variables are automatically set by the commands in order
% to change some formatting depending on the kind of thesis.
% For languages that distinguish feminine from masculine
% adjectives or qualifications, some automatic machinery is
% set up in order to format some infix strings in a way
% that copes with the singular or plural forms; in
% particular when there is a multitude of authors (maximum 
% three) of different gender, the adjectives or qualifications
% are set as masculine plural names, while when there is just
% one autho0.79549875r or the authors are of the same gender the
% adjectives or qualifications are set according to number
% and gender. All this is done by setting or resetting the
% truth value associated to  the boolean variable
% \texttt{femminile}. The boolean variable \texttt{dottorato}
% controls the Ph.D thesis format (except for the ScuDo
% school), while the other variable \texttt{triennale}
% controls the formatting of the bachelor's degree report.
% All other theses are treated as master theses,
% and in all cases the appropriate infix string is typeset
% in the title page:
%    \begin{macrocode}
\newif\iffemminile
\newif\ifdottorato \dottoratofalse
\newif\iftriennale \triennalefalse
%    \end{macrocode}
%
% The infix words for the |classica| option are redefined.
%    \begin{macrocode}
\ifTOPfront
\ifclassica
\def\Candidato{Laureando}
\def\Candidata{Laureanda}
\def\Candidati{Laureandi}
\def\Candidate{Laureande}
\fi\fi
%    \end{macrocode}
%
% The thesis title is specified by means of the following
% commands; |\monografia|, the name of the bachelors degree
% final report, sets also the corresponding boolean variables
% and redefines the command |\titolo| so as to avoid
% duplications and inconsistencies; of course something might
% still be inconsistent if the commands are given in the
% wrong order.
%
% |\titolo| accepts an optional argument, the ``short title'',
% more or less as the standard sectioning commands; this is
% due to the fact that with the class option \opz{autoretitolo}
% the thesis title is written together with the author's
% name in the even page headings; if the thesis full title
% is too long it produces overfull headlines with ugly
% results; a short title may solve the inconvenience.
% The |\sottotitolo| command is another way to maintain
% a short title; all the supplementary title information may be
% typeset in  the subtitle.
%    \begin{macrocode}
\def\monografia#1{\global\triennaletrue
      \global\dottoratofalse
      \global\def\titolo##1{\PackageWarning{topfront}%
      {Il titolo e' gia' stato impostato con
            il comando \string\monografia}}%
      \gdef\@titolo{#1}}

\let\@stitle\empty

\newcommand*{\titolo}[2][]{%
\ifbool{triennale}{%
\PackageWarning{topfront}{Il titolo deve essere impostato con
                               il comando \string\monografia}
}{%
\def\@tempA{#1}\ifdefempty{\@tempA}%
     {\gdef\@stitle{#2}}{\gdef\@stitle{#1}}%
     \gdef\@titolo{#2}%
     }%
}
\def\sottotitolo#1{\gdef\@subtitle{#1}}
%    \end{macrocode}
%
% The |\materia| or its alias |\Materia| are used to
% specify the subject of the thesis; as a silly example
% a set of commands that reflects this subtle difference
% might be the following:
% \begin{verbatim}
% \materia{Applied Tetratricotomy}
% \titolo{Tetratricotomy of blond hair}
% \sottotitolo{Accurate measurements of the four fourths
%   of tetratricotomised blond hair}
%\end{verbatim}
% and the title page, for example, will contain something like this:\\[\bigskipamount]
% \begin{minipage}{\textwidth}\centering
% {\Large Master Thesis\\ in\\ Applied Tetratricotomy\par}\bigskip
% {\huge Tetratricotomy of blond hair\par}\bigskip
% {\large Accurate measurements of the four fourths\\
%   of tetratricotomised blond hair}
% \end{minipage}\bigskip
%    \begin{macrocode}
\let\@materia\empty
\def\Materia#1{\def\@materia{#1}}\let\materia\Materia
%    \end{macrocode}
%
% Things get more complicated for doctoral theses; in general
% there is no supervisor; at most if a professor is assigned to supervise or control the PhD student's work this may be called
% in whatever mode but here we assume his name is input
% with the command |\tutore| even if ``tutor'' does not appear
% as the best choice; in any case in Italian ``tutore'' does
% not have the same meaning as the English ``tutor''.
%
% Most Doctoral Schools require to name the School's director
% or coordinator instead of the tutor. This is why this
% person's name can be introduced with |\direttore| or
% |\coordinatore|; the actual label printed over this person
% name is ``Direttore'' or ``Coordinatore''  but it can be
% changed with |\QualificaDirettore|.
%    \begin{macrocode}
\newif\ifDirettore \Direttorefalse
\def\tutore#1{\gdef\@tutore{#1}}
\def\direttore{\Direttoretrue\relatore}%
\def\coordinatore{\Direttorefalse\relatore}%
\def\QualificaDirettore#1{\gdef\@PhDdirector{#1}}%
%    \end{macrocode}
%
% For ``normal'' theses we may have from one to three
% supervisors and from one to three authors; not all
% universities accept a multitude of supervisors and/or
% authors of the same thesis, but some do; this is why
% this bundle accepts up to three names for each category.
% The |\second...| commands set the plural forms of the
% labels printed above the name lists. For the candidates
% there are different commands to input ladies or gentlemen
% names; according to the masculine (ending in `o')
% or feminine (ending in `a') commands, the appropriate
% truth values are assigned to the boolean variable
% \texttt{femminile} and the labels are set accordingly.
%
% Notice that in the case of bachelor degree final report
% no supervisor name is printed even if one or more supervisor
% names are specified. This must be kept in mind in order
% to avoid surprises in finding missing information in the
% title page. Further on, there are suggestions for
% getting around this feature.
%    \begin{macrocode}
\def\relatore#1{\gdef\@principaladviser{#1}}
\def\secondorelatore#1{\gdef\@secondadviser{#1}}
\def\terzorelatore#1{\gdef\@thirdadviser{#1}}
%    \end{macrocode}
%
% The next set of macros is used to typeset the ``date'' of
% the thesis defence or presentation or whatever is done
% for the final exam. The macro is sort of complicated
% because the input format for this ``date'' may vary from
% a single year, to a year range, to a month and year
% specification so that different actions must be taken;
% if the option \texttt{classica} is in force then the
% formatting of the ``date'' may be still different.
% This command is aliased with |\esamendidottorato|
% which literally means ``defence of the doctoral dissertation'';
% nevertheless both commands refer to a simple date
% in one of those formats.
%
% |\getseduta| splits the date into its two components,
% month and year; if the argument is a single string without
% intervening spaces, the first one is the string itself
% and the second is empty; this emptiness may be tested and,
% in case, the date formatting is modified accordingly.
% In particular if the string is a single spaceless one,
% this string is assigned to |\@submitdate|; otherwise a
% different treatment is made according to the fact that
% \texttt{classica} is in force; if  \texttt{classica} is not
% in force the total string, including spaces,  is assigned
% to |\@submitade|. If \texttt{classica} is in force,
% |\s@dutaclassica| is called with the whole string.
% On turn |\s@dutaclassica| verifies if the date should be
% typeset with old style numbers or with the usual lining
% figures; in the latter case the whole string is assigned
% to |\@submitdate|; in the former one the year part may
% be either a single year or a year range; this separation
% is tested by splitting the year part across one dash;
% if the dash is present the extremes of the year range are
% assigned to |\1| and |\2|, otherwise the year part is a
% single year.
% If a single year is given, this is simply typeset with
% old style numbers and the appropriate commands are assigned
% to the |\@submitdate| control sequence. If a year range is
% given, this year range is also typeset with old style numbers,
% and the dash is executed with a special en-dash surrounded
% with white space.
%
%    \begin{macrocode}
\def\sedutadilaurea#1{\getseduta#1 !}
\def\getseduta#1 #2!{%
    \def\@tempA{#2}%
    \ifx\@tempA\empty
        \def\@submitdate{#1}%
    \else
        \unless\ifclassica
            \def\@submitdate{#1 #2}%
        \else
            \s@dutaclassica#1 #2!%
        \fi
    \fi
}%
\def\s@dutaclassica#1 #2!{%
    \if@ldstyle
        \s@paranumeri#2-!%
        \ifx\2\empty
            \edef\@submitdate{\noexpand#1 \noexpand\oldstylenums{#2}}%
        \else
            \s@paranumeri#2!%
            \edef\@submitdate{\noexpand#1
            \noexpand\oldstylenums{\1} -- \noexpand\oldstylenums{\2}}%
        \fi
        \let\1\undefined
        \let\2\undefined
    \else
        \def\@submitdate{#1 #2}%
    \fi
}
\def\s@paranumeri#1-#2!{\def\1{#1}\def\2{#2}}%
\let\esamedidottorato\sedutadilaurea
%    \end{macrocode}
% Here |\annoaccademico| is defined to typesets the infix
% string corresponding to ``Anno accademico'' followed by
% the year range in old style numbers (irrespective of the
% \texttt{oldstyle} option); localisation commands are
% provided so as to set a different string, possibly
% through the configuration file. In order to set an
% adequate en-dash between the old style numbers a new
% definition is given that takes care to set the dash
% at a height above the base line that copes with the
% specific shape of the old style numbers.
% Notice that |\annoaccademico| and |\sedutadilaurea| do
% not do the same things, besides the fact that |\annoaccademico|
% works only with the |classica| option in force and a
% special format is chosen. Therefore the user has to use
% only one of these two commands, and if s/he uses
% |\annoaccademico|  while the |classica| option is not
% in force, s/he gets an warning and the ``date'' gets
% formatted in a special way so as to remember the user
% that |\annoaccademico| requires the |classica| option.
%    \begin{macrocode}
\newcommand*\NomeAnnoAccademico[1]{\gdef\AnnoAccademico{#1}}
\@ifundefined{AnnoAccademico}{\gdef\AnnoAccademico{Anno accademico}}{}
%
\def\annoaccademico#1{\ifclassica
  \def\@submitdate{{\large\textsc{\AnnoAccademico}} {\Large\s@tanno#1!}}
  \else
  \PackageWarning{toptesi}{\string\annoaccademico\space
            is usable only when the\MessageBreak
  			`classica' option is in force}%
  \def\@submitdate{\AnnoAccademico\ #1 --- Needs the `classica' option}
  \fi}
\def\s@tanno#1-#2!{\oldstylenums{#1}\EnDash\oldstylenums{#2}}

\def\EnDash{{\settowidth{\dimen@}{\large\scshape I}%
\setbox\tw@\hbox{2}\dimen\tw@.5\ht\tw@\advance\dimen\tw@-.5\dp\tw@
\dimen4\dimen\tw@\advance\dimen4by.0385ex\relax
\advance\dimen\tw@-.0385ex\relax
\makebox[1.5\dimen@]{%
    \vrule\@width\dimen@\@height\dimen4\@depth-\dimen\tw@}}}
%    \end{macrocode}
%
% The next macros are used to assign strings to some
% literal information to be typeset in the title page.
% |\ciclodidottorato| requires an uppercase roman numeral
% (in Italy), but it can accept anything that can precede
% the infix word ``cycle''.
% Macros |\corsodilaurea| and |\corsodidottorato| specify
% the degree course qualification; you would specify just
% ``Elettronica'', for example, and the module will write
% in the title page ``Corso di Laurea in Elettronica''.
% The infix part may be changed depending on the default
% language and the configuration file.
%
% |\scuoladidottorato| gets the name of the PhD School;
% |\ateneo| gets the generic name of the university;
% |\nomeateneo| gets the proper name of the university.
% In Italy Universities are generally named after the city they
% are in; in large cities where there are several
% universities, each one of them has a proper name.
% For example the generic name might be ``Universit\`a di
% Roma'' and the proper name might be ``La Sapienza''.
%
% |\facolta| may receive an optional argument that is the
% uppercase roman numeral specific of the faculty and
% a compulsory argument that corresponds to the type of
% faculty; for example, with |\facolta[II]{Ingegneria}|
% the package typesets in the title page ``II Facolt\`a di
% Ingegneria''; if the optional argument is not specified,
% no roman numeral is typeset; the infix string
% \verb*+Facolt\`a di + may be changed with the configuration
% file depending on the default language.
% It might be necessary to define another name in place of
% ``Facolt\`a di\textvisiblespace'', since with recent bills,
% the administrative structure of all Italian universities
% has been changed and the activities formally assigned
% to the Faculties may be now the responsibility of other
% structures that may have different names in different
% universities. If the internal command defined by
% |\FacoltaDi| is empty, no name is printed at all and the
% title page will not have any indication of a particular
% faculty or other educational structure.
% Therefore the Degree Course name, specified with
% |\corsodilaurea| should always be specified.
% The alias commands |\StrutturaDidattica| and |\struttura|
% are defined as equivalent commands to |\FacoltaDi| and
% |\facolta| respectively.
%
% Finally |\logosede| gets the name of the graphic file that
% contains the information relative to the university logo;
% it may receive also a comma separated list of logo file
% names, as it might be necessary when a thesis
% is developed in a multiple university environment.
% If such logo file is not available, the user should not
% specify this command; if the thesis is
% typeset on smaller paper size than A4 or letter,
% it would be much better to avoid inserting one or more logos
% in the title page; this is particularly important when using
% A5 paper size. Nevertheless this decision is left to
% the user and this package neither controls this fact
% nor outputs any warning.
% If the user uses this command to insert one or more
% university logo files but some file is not available,
% the usual \texttt{graphicx} package warning is issued but
% compilation may go on without the missing logo.
%
% The treatment of one or more logo files requires some extra
% commands and control sequences suitable to store temporary
% data or to specify style parameters: one is the name of
% a save box; another is the default spacing between the
% logos in the typeset page; the third is the height of the logos.
% The default spacing may be set with the help of the
% macro |\setlogodistance| -- notice that the default value
% is |3em|, and if a different distance is desired, it should
% not be much larger or much smaller than the default one;
% the default logo height is specified as an option
% command to the |\logosede| command, while the default
% size is given by the |\T@Tlogoheight| length. This length
% is specified in the main |toptesi| package in order
% to compute the typesetting grid, so that if the |topfront|
% module is used by itself with other classes, the existence
% of the |\T@Tlogoheight| control sequence is tested and if it is
% undefined, then and only then it is defined in this
% module and assigned a default value.
%
% The save box name is just for internal workings and does
% not require any customisation.
%
% The analysis and processing to the possible list of logo
% file names is done through the usual means of the delimited
% argument extraction of the single names from the list;
% the ``string'' of logos is then composed in a save box;
% as the list has been completely processed, the box is
% measured; if its width is shorter than the |\textwidth|
% it is typeset without further processing; if it is larger,
% on the opposite, the box gets scaled down so that its width
% equals the |\textwidth|.
%
% The |\tutoreaziendale| macro is the last title page
% addition; several students, who work on their thesis
% or final project in a company, want to have the company
% supervisor name printed in the title page; this does not
% preclude expressing the student's deepest thanks in the
% acknowledgements section, but it does not harm to name
% this person also in the title page.
%
% Eventually the |\retrofrontespizio| command, that by
% default is empty, allows to typeset a copyright page;
% the argument of this command is in  total responsibility
% to the user who must write it in the thesis main language;
% the user can specify from zero to several paragraphs,
% separated by the vertical spaces s/he thinks best; the
% argument by default is typeset at the bottom of the text
% block of the copyright page. The user can specify any
% pertinent space at the bottom of his/her argument, so as
% to set the text in the position s/he likes best.
% In order to handle the copyright page in the proper way
% we need to test if its definition is empty or blank so
% we need the powerful advanced macros of the package
% |etoolbox| that has already been loaded by this module or
% by the main |toptesi| one.
%    \begin{macrocode}
\def\ciclodidottorato#1{\gdef\@ciclo{#1 \@cyclename}}%
\def\corsodilaurea#1{\global\dottoratofalse\gdef\@corso{#1}}
\def\corsodidottorato#1{\global\dottoratotrue\global\triennalefalse
    \gdef\@corso{#1}}
\def\scuoladidottorato#1{\global\dottoratotrue\global\triennalefalse
    \gdef\@phdschool{#1}}
\def\ateneo#1{\gdef\@ateneo{#1}}
\def\nomeateneo#1{\gdef\@nomeat{\expandafter\uppercase{\expandafter #1}}}
\newcommand\facolta[2][]{\gdef\@facname{#2}\gdef\@facnumber{#1}}
\let\struttura\facolta

\newlength{\TPT@logospace}\TPT@logospace=3em\relax
\newsavebox{\TPT@logobox}
\newdimen\TPT@logoheight 
\newcommand*\setlogodistance[1]{\TPT@logospace=#1}
\providecommand*{\@logosede}{}

\ifcsundef{T@Tlogoheight}{%\
    \newlength\T@Tlogoheight
    \setlength\T@Tlogoheight{0.8\dimexpr(\paperwidth-\textwidth)/2}}{}
    
\newcommand\logosede[2][\T@Tlogoheight]{\def\@logosede{#2}\TPT@logoheight=#1\relax
\ifcsvoid{@logosede}{\sbox{\TPT@logobox}{}}{\begin{lrbox}{\TPT@logobox}%
\expandafter\fillup@TCP@logobox\@logosede,!}}

\def\fillup@TCP@logobox#1,#2!{%
	\ifblank{#1}{\end{lrbox}\ifdim\wd\TPT@logobox>\textwidth
	\sbox\TPT@logobox{\resizebox{\textwidth}{!}{\box\TPT@logobox}}\fi}%
	{\def\@logosede{#2}%
	\includegraphics[height=\TPT@logoheight]{#1}\hskip\TPT@logospace
   \expandafter\fillup@TCP@logobox\@logosede,!}}

\newcommand\printloghi{\unless\ifvoid\TPT@logobox\usebox{\TPT@logobox}\fi}

\def\tutoreaziendale#1{\gdef\@tutoreaziendale{#1}}
\newcommand\retrofrontespizio[1]{\long\gdef\@retrofrontespizio{#1}}
%    \end{macrocode}
%
% The following commands are user commands that modify the
% infix strings according to the language used and to the
% specifications of the actual university. All these
% commands can be put in the configuration file so as
% to specify what is desired as a default.
% If these commands are specifically used to redefine
% something before issuing the |\frontespizio| command,
% the command that actually typesets the title
% page, the new definitions override the configuration ones.
%
% \begin{description}[noitemsep]
% \def\Item[#1]{\item[\normalfont\texttt{\char92#1}]
% sets or changes the string}
% \Item[FacoltaDi] ``Facolt\`a di'' in, say, ``Faculty of''
% \Item[DottoratoIn] ``Dottorato in'' in, say, ``PhD in''
% \Item[CorsoDiLaureaIn] ``Corso di Laurea in'' in, say,
% ``Master of Science in''
% \Item[TesiDiLaurea] ``Tesi di Laurea'' in, say,
% ``Tesi di Laurea Magistrale''
% \Item[NomeMonografia] ``Monografia di Laurea'' in, say,
% ``Tesi di Laurea''
% \Item[NomeDissertazione] ``Dissertazione'' in, say,
% ``PhD dissertation''
% \Item[InName] ``in'' in, say, ``auf''
% \Item[CandidateName] ``Candidato'' in, say, ``Laureando''
% \Item[AdvisorName] ``Relatore'' in, say, ``Supervisors''
% \Item[CoAdvisorName] ``Correlatore'' in, say, ``Corapporteur''
% \Item[NomeTutoreAziendale] ``Supervisore aziendale'' in, say,
% ``XYZ Company Supervisor''
% \Item[TutorName] ``Tutore'' in, say, ``Supervisor''
% \Item[CycleName] ``ciclo'' in, say, ``cycle''
% \Item[NomePrimoTomo] ``Tomo primo'' in, say, ``First volume''
% \Item[NomeSecondoTomo] ``Tomo secondo'' in, say, ``Second volume''
% \Item[NomeTerzoTomo] ``Tomo terzo'' in, say, ``Third volume''
% \Item[NomeQuartoTomo] ``Tomo quarto'' in, say, ``Fourth volume''
% \end{description}
% In the above description the first string is generally
% the default one, while the second string is just an example
% of the corresponding string to be set in another language
% or to be changed in Italian. The last four commands clearly
% show the difficulty of localising language strings: it
% is necessary to localise the whole phrase, because of
% the position of the adjectives.
%    \begin{macrocode}
\newcommand\FacoltaDi[1]{\gdef\@faculty{#1}}
\let\StrutturaDidattica\FacoltaDi
\newcommand\DottoratoIn[1]{\gdef\@PhDname{#1}}
\newcommand\CorsoDiLaureaIn[1]{\gdef\@laureaname{#1}}
\newcommand\TesiDiLaurea[1]{\gdef\@TesiDiLaurea{#1}}
\newcommand\NomeMonografia[1]{\gdef\@monografia{#1}}
\newcommand\NomeDissertazione[1]{\gdef\@dissertazione{#1}}
\newcommand\InName[1]{\gdef\@InName{#1}}
\newcommand\CandidateName[1]{\gdef\@nomecandidato{#1}}
\newcommand\AdvisorName[1]{\gdef\Relatore{#1}\gdef\Relatori{#1}}
\newcommand\CoAdvisorName[1]{\gdef\Correlatore{#1}\gdef\Correlatori{#1}}
\newcommand\TutorName[1]{\gdef\Tutore{#1}}
\newcommand\NomeTutoreAziendale[1]{\gdef\@tutoreaziendalename{#1}}
\newcommand\CycleName[1]{\gdef\@cyclename{#1}}
\newcommand\NomePrimoTomo[1]{\gdef\PrimoTomo{#1}}
\newcommand\NomeSecondoTomo[1]{\gdef\SecondoTomo{#1}}
\newcommand\NomeTerzoTomo[1]{\gdef\TerzoTomo{#1}}
\newcommand\NomeQuartoTomo[1]{\gdef\QuartoTomo{#1}}
\providecommand\IDlabel{\\\quad matricola\ }
%    \end{macrocode}
%
% Now we can read the configuration file if it exists;
% in any case what is possibly defined or redefined in the
% configuration file must not be redefined in the following
% Lines and this is why everything is subject to the test
% |\@ifundefined|. Most default definitions are simply
% ``blank''; the others are in Italian. All of them, except
% the supervisor and candidate strings may be individually
% redefined in the configuration file or in the preamble.
% Those that cannot be redefined such as the four
% ``candidate'' strings may be actually redefined through
% the single |\CandidateName| that should be used in a
% language depended way and with the correct number and
% gender once for all.
% The four endings in the Italian strings allow to
% exercise the correct selection only for Italian;
% a specific test is made inside the |\frontespizio|
% command; because of this, the same machinery cannot
% be used, say, for French but maybe in the future this
% feature is resolved in a proper way. The same is true
% for the supervisor and the co-supervisor strings that
% may be changed once for all with |\AdvisorName| and
% |\CoAdvisorName|.
%    \begin{macrocode}
\IfFileExists{\jobname.cfg}{\input{\jobname.cfg}}%
    {\IfFileExists{toptesi.cfg}{%%
%% This is file `toptesi.cfg',
%% generated with the docstrip utility.
%%
%% The original source files were:
%%
%% toptesi.dtx  (with options: `topcfg')
%%   ______________________________________________________
%%   The TOPtesi bundle
%%   Copyright (C) 2015-2017 Claudio Beccari
%%   All rights reserved
%% 
%%   License information appended
%% 

%%
%%================================================================
%% This file is the only file of the TOPtesi bundle that the user
%% can modify without restrictions in order to customise the
%% contents of this configuration file to his/her needs. The user
%% can add or remove lines, comment or uncomment lines, change the
%% arguments to macros, add definitions and so on.
%% Use  this file by copying the relevant section to another file
%% to be named as the thesis main file and with extension .cfg.
%% This bundle will try to read "\jobname.cfg"; if this file does
%% not exist, the bundle files don't do anything. This implies that
%% the provided file toptesi.cfg is to be used as a model, not to
%% be used directly.
%%================================================================
%%
\ifmagistrale
    \ateneo{Politecnico di Torino}
    \facolta{}% nessun nome di default/ no default name for PoliTO
    \FacoltaDi{}% nessun prefisso per la facoltà/no default faculty label for PoliTO
    \CorsoDiLaureaIn{Corso di Laurea in}
    \TesiDiLaurea{Tesi di Laurea Magistrale}
    \InName{in}
    %%\retrofrontespizio{Questo testo è soggetto alla
    %% Creative Commons Licence}
\fi
\ifsecondaria

\fi
\ifScuDo
    \PhDschoolLogo{Logo-Scudo}
    \Ndissertation{Docoral Dissertation}
    \Ndoctoralprogram{Doctoral Program in\xspace}
    \NSupervisor{Supervisor}{Supervisors}
    \SupervisorNumber{1}
    \Nexaminationcommittee{Doctoral Examination Committee:}
    \Nlocation{Politecnico di Torino, Turin, Italy}
    \Disclaimer{\noindent I hereby declare that the
     contents and organisation of this dissertation
     constitute my own original work and does not
     compromise in any way the rights of third parties,
     including those relating to the security of personal
     data.}
     \CClicence{This thesis is licensed under a Creative Commons
     License, Attribution - Noncommercial- NoDerivative Works 4.0
     International: see \url{www.creativecommons.org}.
     The text may be reproduced for non-commercial purposes,
     provided that credit is given to the original author.}
\fi
%% 
%% Copyright 2005-2017 Claudio Beccari
%% 
%% Distributable under the LaTeX Project Public License,
%% version 1.3c or higher (your choice). The latest version of
%% this license is at: http://www.latex-project.org/lppl.txt
%% 
%% This work is "author-maintained"
%% 
%% This work consists of this file toptesi.dtx, a README file
%% the manifest.txt file, and the derived files:
%%     toptesi.cls,
%%     toptesi.sty,
%%     topfront.sty,
%%     topcommand.sty,
%%     toptesi-scudo.sty,
%%     toptesi.cfg,
%%     toptesi-monografia.sty,
%%     toptesi-sss.sty,
%%     toptesi-magistrale.sty,
%%     toptesi-dottorale.sty,
%% and the English documentation toptesi.pdf.
%% 
%% Furthermore the bundle contains the documentation source
%% file toptesi-it.tex and the derived file
%% toptesi-it.pdf.
%% 
%% The toptesi-example.tex source file is just an example
%% that shows how to use the various commands; by commenting
%% or uncommenting certain source lines it is possible to
%% typeset different kind of theses and their front pages.
%% 
%% The topfront-example.tex source file is an example of how
%% to produce just the title page with TOPtesi and the external
%% package frontespizio.
%% 
%% The toptesi-scudo-example.tex source file is an example
%% that shows the particular features available with TOPtesi,
%% when a doctoral thesis is produced for the Scuola di
%% Dottorato (ScuDo) of Politecnico di Torino.
%% 
%% The toptesi.cfg file is a sample of a local configuration
%% file that can be copied to another file and its copy freely
%% edited and customised.
%% 
%% The other toptesi-*.sty files are extension modules for
%% typesetting the specific thesis kinds described by the
%% suffix that replaces the asterisk.
%% 
%% By running pdflatex on toptesi.dtx the user gets the class,
%% sty and cfg files and the English documentation file in
%% PDF format.
%% 
%% The source file of the Italian documentation file
%% toptesi-it.pdf is another example of how to use TOPtesi.
%% This file toptesi-it.tex may be typeset with pdfLaTeX,
%% XeLaTeX, and LuaLateX; see the first comment lines of
%% the file for how-to information. In spite of being mainly
%% written in Italian, it contains several sections in English
%% for the benefit of foreign students attending the Doctoral
%% School of Turin.
%%
%% End of file `toptesi.cfg'.
}{}}
%
\@ifundefined{@cyclename}{\def\@cyclename{ciclo}}{}
\@ifundefined{@titolo}{\def\@titolo{}}{}

\providecommand\@author{}\providecommand\@@author{}
\NewDocumentCommand\candidato{m o}{\gdef\@@author{#1}%
\IfNoValueTF{#2}{\gdef\@author{#1}}%
               {\gdef\@author{#1\IDlabel#2}}
\femminilefalse\pluralefalse}
\NewDocumentCommand\candidata{m o}{\gdef\@@author{#1}%
\IfNoValueTF{#2}{\gdef\@author{#1}}%
               {\gdef\@author{#1\IDlabel#2}}
\femminiletrue\pluralefalse}
\providecommand\@secondauthor{}
\NewDocumentCommand\secondocandidato{m o}{%
\IfNoValueTF{#2}{\gdef\@secondauthor{#1}}%
                {\gdef\@secondauthor{#1\IDlabel#2}}%
\femminilefalse\pluraletrue}
\NewDocumentCommand\secondacandidata{m o}{%
\IfNoValueTF{#2}{\gdef\@secondauthor{#1}}%
                {\gdef\@secondauthor{#1\IDlabel#2}}%
\pluraletrue}
\providecommand\@thirdauthor{}
\NewDocumentCommand\terzocandidato{m o}{%
\IfNoValueTF{#2}{\gdef\@thirdauthor{#1}}%
                {\gdef\@thirdauthor{#1\IDlabel#2}}%
\femminilefalse\pluraletrue}
\NewDocumentCommand\terzacandidata{m o}{%
\IfNoValueTF{#2}{\gdef\@thirdauthor{#1}}%
                {\gdef\@thirdauthor{#1\IDlabel#2}}%
\pluraletrue}
\providecommand\IDlabel{matricola\ }

\@ifundefined{@principaladviser}{\def\@principaladviser{}}{}
\@ifundefined{@secondadviser}{\def\@secondadviser{}}{}
\@ifundefined{@thirdadviser}{\def\@thirdadviser{}}{}
\ifcsundef{@PhDdirector}{%
   \ifDirettore\def\@PhDdirector{Direttore del corso di dottorato}\else
    \def\@PhDdirector{Coordinatore del corso di dottorato}\fi}{}
\@ifundefined{@tutore}{\def\@tutore{}}{}
%
\@ifundefined{@nomerelatore}{\def\@nomerelatore{}}{}
\@ifundefined{@nomecandidato}{\def\@nomecandidato{}}{}
\@ifundefined{Candidato}{\def\Candidato{Candidato}}{}
\@ifundefined{Candidata}{\def\Candidata{Candidata}}{}
\@ifundefined{Candidati}{\def\Candidati{Candidati}}{}
\@ifundefined{Candidate}{\def\Candidate{Candidate}}{}
\@ifundefined{Relatore}{\def\Relatore{Relatore}}{}
\@ifundefined{Relatori}{\def\Relatori{Relatori}}{}
\@ifundefined{Correlatore}{\def\Correlatore{Correlatore}}{}
\@ifundefined{Correlatori}{\def\Correlatori{Correlatori}}{}
\@ifundefined{Tutore}{\def\Tutore{Tutore}}{}
\@ifundefined{@tutoreaziendale}{\def\@tutoreaziendale{}}{}
\@ifundefined{@tutoreaziendalename}%
        {\def\@tutoreaziendalename{Supervisore Aziendale}}{}
\@ifundefined{@retrofrontespizio}{\def\@retrofrontespizio{}}{}
\@ifundefined{@subtitle}{\def\@subtitle{}}{}
%
\@ifundefined{@corso}{\def\@corso{}}{}
\@ifundefined{@ciclo}{\def\@ciclo{}}{}
\@ifundefined{@ateneo}{\def\@ateneo{POLITECNICO DI TORINO}}{}
\@ifundefined{@nomeat}{\def\@nomeat{}}{}% Nome proprio dell'ateneo
\@ifundefined{@facolta}{\def\@facname{}}{}
\@ifundefined{@facnumber}{\def\@facnumber{}}{}
\@ifundefined{@faculty}{\def\@faculty{}}{}
%
\@ifundefined{PrimoTomo}{\def\PrimoTomo{Tomo primo}}{}
\@ifundefined{SecondoTomo}{\def\SecondoTomo{Tomo secondo}}{}
\@ifundefined{TerzoTomo}{\def\TerzoTomo{Tomo terzo}}{}
\@ifundefined{QuartoTomo}{\def\QuartoTomo{Tomo quarto}}{}
%    \end{macrocode}
%
% The infix word for the |classica| option are redefined.
%    \begin{macrocode}
\ifTOPfront
\ifclassica
\def\Candidato{Laureando}
\def\Candidata{Laureanda}
\def\Candidati{Laureandi}
\def\Candidate{Laureande}
\fi\fi
%    \end{macrocode}
% If the final exam date is not given the default value is
% the current month and the current year typeset in Italian;
% therefore the user is strongly requested to enter a date
% either with the |\sedutadilaurea| or the |\esamedidottorato|
% commands. The default value of the university logo command has already been defined ``empty''.
%    \begin{macrocode}
\@ifundefined{@submitdate}{\def\@submitdate{\ifcase\the\month\or%
  Gennaio\or Febbraio\or Marzo\or Aprile\or Maggio\or Giugno\or
  Luglio\or Agosto\or Settembre\or Ottobre\or Novembre\or Dicembre\fi
  \space \the\year}}{}
%
\@ifundefined{@TesiDiLaurea}{\def\@TesiDiLaurea{Tesi di Laurea}}{}
\@ifundefined{@phdschool}{\def\@phdschool{SCUOLA DI DOTTORATO}}{}
\@ifundefined{@PhDname}{\def\@PhDname{Dottorato in }}{}
\@ifundefined{@laureaname}{\def\@laureaname{Corso di Laurea in }}{}
\@ifundefined{@dissertazione}{\def\@dissertazione{Tesi di Dottorato}}{}
\@ifundefined{@monografia}{\def\@monografia{Monografia di Laurea}}{}
\@ifundefined{@InName}{\def\@InName{in}}{}
%    \end{macrocode}
%
% Finally we have the real macro |\frontespizio| and the
% corresponding environments, the real macros or environments
% that actually typeset the title page.
%
% I recommend to use the environments, a new feature of
% version 5.85. But the legacy command |\frontespizio| is
% still usable. The principle on which these environments
% work is that the |frontespizio| environment typesets the
% title page with logo(s) set in the page header, while the
% |frontspizio*| environment typesets the logos after the
% information on the title, the possible sub title and tome
% information, that is in the lower half of the title page.
% In order to achieve this result each environment sets the
% boolean variable |topTPTlogos| to either value |true|
% (for header logos) or |false| (otherwise).
% The key of the different typesetting style is this
% boolean-variable state.
%
% Now, since the internal |frontespizio| environment opening
% command is |\frontespizio| how is it possible to distinguish
% this opening statement from the homonymous user command?
% The solution is a little tricky, but, after all, very simple.
% The |\begin| command, with which an environment is started,
% before calling the opening statement defines the internal
% service macro |\@currenvir| to contain the environment name;
% this is used by the |\end| statement to control that it is
% closing the last opened environment.
%
% If the |\frontespizio| command is directly used, the
% |\@currenvir| macro does not contain the name ``frontespizio'';
% therefore if in the opening environment definition we check
% the contents of |\@currenvir| against the string
% |frontespizio| we can decide if the user resorted to
% the legacy command, or the environment was correctly
% opened by means of |\begin|.
% If the user resorted to the legacy command, the
% service macro |\fr@ontespizio| is called that typesets
% the title page according to the current status of the
% boolean |topTPTlogos|, a boolean that has a default
% value but the user can set at its will. Otherwise the
% |frontespizio| environment is regularly executed.
% Notice that the service macro |\fr@ntespizio| tests
% the state of the boolean |classica| and accordingly
% uses a different page style.
%    \begin{macrocode}
\newbool{topTPTlogos} \booltrue{topTPTlogos}

\newenvironment{frontespizio*}{\boolfalse{topTPTlogos}}{\fr@ntespizio}

\newenvironment{frontespizio}{%
    \ifdefstring{\@currenvir}{frontespizio}
           {\booltrue{topTPTlogos}}{\TPTmaybestar}
 }{%
     \fr@ntespizio
 }

%    \end{macrocode}
%
% With the new boolean |AteneoInHead| we can mark these
% situations where the university common name goes into the
% header; with the |classica| option the title page
% \emph{must} have this name in the header and if the user
% forgets to specify one, this module fakes it with a clear
% message that the university name has been forgotten, but
% at the same time this fake message fills up the header
% position. On the other hand, when the |classica| option
% has not been specified, the user can use either an empty
% university name or a specific name. So only when the
% |\@ateneo| macro remains empty the university name
% in the header remains really blank; this is good when the
% university name is part of the university logo (this is
% the case for Politecnico di Torino, and for many other
% universities). When no university name is set in the header
% some little attention in formatting the title page is necessary.
%
%    \begin{macrocode}  
\newbool{AteneoInHead}\boolfalse{AteneoInHead}
%    \end{macrocode}
%
% In order to use the command |\frontespizio*| as an isolated
% command instead of the starting command of the |frontespizio*|
% environment, we have to behave as with the |\frontespizio|
% command, but we must test for a possible asterisk following
% the command; for this reason we defined the |\TPTmaybestar|
% that% absorbs une following token: if the token is an
% asterisk, we set the appropriate settings for the previous
% behaviour of the isolated command, but if it is
% not an asterisk we must set it back into the list after
% finishing the execution of the |\fr@ntespizio| service macro,
% whose function is to set the title page information at the
% proper position, but must not contain any spurious material.
%    \begin{macrocode}
\newcommand\TPTmaybestar[1]{\def\@tempA{#1}%
\ifdefstring{\@tempA}{*}%
  {\boolfalse{topTPTlogos}\booltrue{AteneoInHead}\fr@ntespizio}
  {\booltrue{topTPTlogos}\fr@ntespizio\@tempA}
}
%    \end{macrocode}
%
% We start defining the complex macro |\fr@ntespizio|. We start
% with a group so that any settings performed by this command
% remain local; if the title page environments had been
% used, this group would be useless, but if the isolated
% commands are used, then this group protects the rest of the
% document from unusual local settings valid only for the
% title page.
%    \begin{macrocode}
\def\fr@ntespizio{%
\begingroup\par
%    \end{macrocode}
%
% We want also the title page to be set in the middle of the
% page irrespective of the binding correction; so we assign
% the average of the two side margins to both
% of them.
%    \begin{macrocode}
   \oddsidemargin=\dimexpr(\oddsidemargin+\evensidemargin)/2\relax
   \evensidemargin \oddsidemargin
%    \end{macrocode}
% The |\null| command inserts a void horizontal box into
% the vertical list; it is useful to act as a block against
% which the vertical glue pushes for setting the subsequent
% material. The normal font is chosen in case preceding
% commands did change the font characteristics.
%    \begin{macrocode}
\null
\setcounter{page}{1}%
\normalfont
%    \end{macrocode}
% Depending on the style of the title page a different
% |\pagestyle| is set with appropriate switches and settings.
% If with the classica style a university name is blank,
% it is set to an explicit string that the setting of  the
% university name has been forgotten.
%    \begin{macrocode}
\ifclassica
   \boolfalse{topTPTlogos}
   \thispagestyle{classica}
   \ifcsvoid{@ateneo}{\def\@ateneo{Manca il nome dell'ateneo}
   }{}
\else
   \thispagestyle{titlepage}
\fi
\ifcsvoid{@ateneo}{}{\booltrue{AteneoInHead}}
%    \end{macrocode}
%
% The generic university name should already be in the header
% either in the logo or in the header text; but in spite of
% this we test if the university generic name macro is void,
% if it contains something, then we typeset also the generic
% name; some candidates might obey to university regulations
% that require the name of the university be at the top, just
% under the logo. The switch |\ifcsvoid| is true if |@ateneo|
% is empty or blank, false otherwise; but even with page top
% logos, no test is made in order to give the possibility to
% repeat the university name. It is the user responsibility to
% set an empty value to the |\ateneo| macro so as to avoid
% repeating the university name possibly already present in
% the logo itself.
%    \begin{macrocode}
\ifcsvoid{@ateneo}{%
    \ifbool{topTPTlogos}
    {}{\booltrue{AteneoInHead}\def\@ateneo{Manca il nome dell'ateneo}}%
}{%
    \booltrue{AteneoInHead}%
}

   \ifbool{AteneoInHead}{}{%
      {{\centering\LARGE \@ateneo\par}}
   }
%    \end{macrocode}
%
% If it is non blank the first thing we set on the page is
% the university proper name and some vertical glue.
%    \begin{macrocode}
   \ifcsvoid{@nomeat}{}
      {\ifbool{topTPTlogos}{\vspace*{\dimexpr \headsep+2.5ex}}{\vspace*{-3ex}}%
      {\centering\@nomeat\par}\vfill}

%    \end{macrocode}
% Then the faculty name comes next; but for the doctoral
% school it uses the doctoral school name entered with
% |\scuoladidottorato|, otherwise it inserts the faculty
% ordinal number or prefix and name already entered with
% the optional and required arguments of |\facolta|.
%    \begin{macrocode}
      \begin{center}
      {\rmfamily\mdseries
      \ifdottorato
            \large \@phdschool\par\medskip
      \else
            \ifcsvoid{@faculty}{}{%
            	\LARGE\ifx\@facnumber\empty\else\@facnumber\space\fi
            	\@faculty\unskip\xspace\@facname\par\medskip
            }
      \fi
      }%
%    \end{macrocode}
% Further specification: it inserts the field of the PhD
% research or the degree course name; it inserts a line
% such as, for example, ``Philosophy Degree in Applied
% Tetratricotomy~--~XVI~cycle'' or ``Master of Science
% in Applied Tetratricotomy''.
%    \begin{macrocode}
      \ifcsvoid{@corso}{}{{\large
            \ifdottorato
                 \@PhDname\unskip\xspace
                      \@corso\ifx\@ciclo\empty\else~--~\@ciclo\fi
            \else
                  \@laureaname\unskip\xspace\@corso
            \fi
            \par}}
      \end{center}
%    \end{macrocode}
% It now centres the name of the report, be it ``Doctoral
% Dissertation'' or ``Master Thesis'' or whatever; in case
% the command |\materia| was used, it then centres the
% discipline which the thesis deals with.
%    \begin{macrocode}
      \vspace{\stretch{0.2}}
      \begin{center}
        \LARGE
        \ifdottorato
            \@dissertazione%
        \else
            \iftriennale
                  \@monografia%
            \else
                  \@TesiDiLaurea%
            \fi
        \fi
        \unless\ifx\empty\@materia
            \\\@InName\\\@materia
        \fi
      \end{center}
%    \end{macrocode}
%
% Next comes the real title entered with |\titolo| or
% |\monografia| and the possible subtitle.
%    \begin{macrocode}
      \vspace{\stretch{0.2}}
      \begin{center}
        {\huge\bfseries \baselineskip=0.95em plus 1pt
                \@titolo \par}
      \end{center}
%    \end{macrocode}
% with the possible subtitle:
%    \begin{macrocode}
      \unless\ifx\@subtitle\empty 
            \begin{center}%
              \large\textrm{\@subtitle}\par
            \end{center}%
      \fi
%    \end{macrocode}
% If the option \texttt{classica} is in force the thesis
% might be divided in several volumes; theses in humanities
% apparently are often oversized. In this case the |\tomo|
% command may be given at the beginning of every volume and the
% counter \texttt{tomo} is stepped up; the volume number
% is therefore printed in each title page; the infix string
% may be redefined as it was shown above.
%    \begin{macrocode}
      \ifclassica
            \ifnum\value{tomo}>\z@
                  \par\bigskip
                  \noindent\makebox[\textwidth]{%
                  \large\textbf{%
                  \ifcase\c@tomo%
                        \or \PrimoTomo%
                        \or \SecondoTomo%
                        \or \TerzoTomo%
                        \or \QuartoTomo%
                        \else
                            \PackageWarning{toptesi}{%
                            Counter tomo equals 
                               \the\c@tomo\MessageBreak
                            We never considered a thesis might get
                            divided in more than four volumes}%
                        \fi}}%
            \fi
            \vspace{1em}
      \fi
      \par
%    \end{macrocode}
% Going down in the title page, now comes the optional
% insertion of the university logo(s); ``optional'' in both
% meanings: one or more university logos are not generally
% required in a thesis, and in case it depends if the
% logo(s) have to be put in the header or here.
% This is a simple task since the |\logosede| already
% defined the contents to the box |\TPT@logobox|, and this
% was done either with an explicit command |\logosede|
% with its argument in the preamble, or with a specific
% line in the configuration file or within the |frontespizio|
% environment.
% If such box |\TPT@logobox| is void, the already defined
% |\printloghi| macro does not do anything.
%    \begin{macrocode}
     \unless\iftopTPTlogos
      {\vfill\centering \printloghi\par}\fi
%    \end{macrocode}
%
% The final task is to typeset the possible supervisors'
% names, the candidates' names and all the rest of the
% bureaucratical terms. We have to distinguish between
% a bachelor degree report that is not supposed to have
% a supervisor, from the doctoral dissertation where we
% do not indicate the supervisor, but the School Director,
% and the master thesis where there might be one or more
% supervisors; with the \texttt{classica} option in force
% no label is printed over the supervisor's name, unless
% there is a plurality of supervisors.
%    \begin{macrocode}
      \vfill
      \iftriennale
            \let\@nomerelatore\empty
      \else
            \ifdottorato
                  \edef\@nomerelatore{\@PhDdirector}%
            \else
                  \ifcsvoid{@principaladviser}{}{%
                        \def\@nomerelatore{\Relatore}}
                  \unless\ifclassica
                        \ifcsvoid{@secondadviser}{}{%
                              \def\@nomerelatore{\Relatori}}%
                  \fi
            \fi
      \fi
%    \end{macrocode}
% Similarly the label names for the exam candidates are
% chosen; in Italian such names are infix strings that are
% selected according the gender and the number; if these
% labels have to be set in a different language it is
% necessary to define one string that has to be selected
% by the user according to number and gender.
% The label for the PhD candidate is left empty.
%    \begin{macrocode}
      \ifdottorato
            \let\@nomecandidato\empty
      \else
        \iflanguage{italian}{%
            \iffemminile
                  \def\@nomecandidato{\Candidata}%
            \else
                  \def\@nomecandidato{\Candidato}%
            \fi
            \ifcsvoid{@secondauthor}{}{%
                  \iffemminile
                        \def\@nomecandidato{\Candidate}%
                  \else
                        \def\@nomecandidato{\Candidati}%
                  \fi}
         }{}%
     \fi
%    \end{macrocode}
%
% For the supervisor(s) and of the candidate(s) name(s)
% a different approach is used for each one of the three
% categories of theses. If a bachelor degree
% report is dealt with, the name of the \emph{single}
% candidate is centred and written in caps-and-small-caps.
%    \begin{macrocode}
      \iftriennale
            \begin{center}%
                  \large\mdseries\textsc{\@author}
            \end{center}%
%    \end{macrocode}
% For doctoral and master theses two virtual boxes
% (actually macros) are filled up so as to align the
% supervisor name(s) and, in a second virtual box,
% the candidate name(s). The label is set in the first
% line with proper number and gender; in the second line
% the first name, and in the subsequent lines,
% if there are any, the other names. These virtual boxes
% actually contain a tabular environment each; these
% environments shall be actually typeset when these virtual
% box macros are executed. If the \texttt{classica} option
% is in force no label is set over the principal advisor
% name, but a label is set over the co-advisor name(s).
% The type size is also a little different for the
% \texttt{classica} option. The |\protect| command is used
% to protect the names in case they contain accent macros
% that might be expanded at the wrong moment.
%
% The candidate name(s) are typeset in another nested
% tabular environment of two lines when the user typesets
% the entry in such a way:
%\begin{flushleft}
%\cs{candidato}\Marg{Mario Rossi}\Oarg{123456}\\[1ex]
%\end{flushleft}
% The optional argument contains the ID number of the candidate;
% if it is specified, it is typeset below the candidate name,
% with an indentation, and prefixed with  the \verb*|matricola |
% label; of corse if the optional ID number is not specified,
% nor are the enclosing braces, nothing is printed after or
% below the candidate's name.
% If the user wants it typeset on the same line and or
% with a different label s/he has to redefine the macro |\IDlabel|;
% its default definition is 
%\begin{verbatim*}
%\newcommand*\IDlabel{\\\quad matricola\ }
%\end{verbatim*}
% Example redefinitions are shown in the sample files.
% Notice that with the new definition of the candidates' setting
% macros it is guaranteed that the ID number, if specified, is
% typeset in the title page, but not in the headers if the
% |autoretitolo| option is in force.
%
% It is useful to recall that the ID number must be entered
% only if required by the University regulations; it must
% not be typeset in the title page just because it is possible
% to do it.
%
%    \begin{macrocode}
      \else
% For theses of any kind that expect the supervisor and co-supervisor names...
         \def\BoxRelatori{%
            \begin{tabular}[t]{l}%
            \hbox{\ifclassica\else\large\fi
                  \textbf{\protect\@nomerelatore}}\\[.6ex]
            \hbox{\large\textrm{\protect\@principaladviser}}%
            \ifx\@secondadviser\empty \else
               \ifclassica
                  \ifx\@thirdadviser\empty
                     \ifx\@secondadviser\empty\else
                           \\[1.5ex]\textbf{\Correlatore:}%
                     \fi
                  \else
                     \\[1.5ex]\textbf{\Correlatori:}%
                  \fi
               \fi
               \\[.6ex]\hbox{{\large\textrm{\protect\@secondadviser}}}%
            \fi
            \ifx\@thirdadviser\empty \else
               \\[.6ex] \hbox{{\large\textrm{\protect\@thirdadviser}}}%
            \fi
            \end{tabular}%
         }%
%    \end{macrocode}
% A similar approach is taken for the candidate name(s),
% although for code clarity I prefer to define two secondary
% macros in order to format the other candidates names
% and ID in a clearer way.
%    \begin{macrocode}
      \def\print@secondocandidato{\\\relax
                  \hbox{\large\tabular{@{}l@{}}\@secondauthor\endtabular}}
      \def\print@terzocandidato{\\\relax
                  \hbox{\large\tabular{@{}l@{}}\@thirdauthor\endtabular}}
      \def\BoxCandidati{%
         \begin{tabular}[t]{l}%
            \hbox{\unless\ifclassica\large\fi
               \textbf{\protect\@nomecandidato}}\\[.6ex]
               \hbox{\large\tabular{@{}l@{}}\@author\endtabular}%
               \ifcsvoid{@secondauthor}{}{\print@secondocandidato}%
               \ifcsvoid{@thirdauthor}{}{\print@terzocandidato}%
         \end{tabular}%
      }%
%    \end{macrocode}
% The real typesetting of these name lists takes place now;
% if the thesis is referred to the PhD school, one type of
% layout is used, otherwise the default master thesis layout
% is used; remember that the bachelor degree case has
% already taken place. For the doctoral dissertation the
% doctoral candidate name is typeset centred in one line
% by itself and everything else is set 3em below into a three
% column table extended to the |\hsize|, the first line
% containing the applicable labels and the second line
% containing the true names. The central column is used just
% for spacing, but it does not contain anything.
%    \begin{macrocode}
         \ifdottorato
            \begin{center}\large
               \textbf{\@author}\\[3em]
               {\normalsize
                 \begin {tabular*}{\hsize}{@{\extracolsep{\fill}}ccc}
                     \ifcsvoid{@tutore}{}{\textbf{\Tutore}}
                 &\relax&
                     \ifcsvoid{@principaladviser}{}{\textbf{\@nomerelatore}}
                 \\
                     \ifcsvoid{@tutore}{}{\@tutore}
                 &\relax&
                     \ifcsvoid{@principaladviser}{}{\@principaladviser}
                 \end{tabular*}
               }%
            \end{center}
         \else
%    \end{macrocode}
% For the master thesis the two virtual boxes are set one
% besides the other but skewed to the right or, respectively,
% to the left of every name so that there is enough space
% for the signature. With the \texttt{classica} option in force
% the two boxes are simply aligned.
%    \begin{macrocode}
           \unless\ifclassica
             \unless\ifevenboxes
             \begin{flushleft}%
                \BoxRelatori
             \end{flushleft}\par\vspace*{-1.5\baselineskip}
             \begin{flushright}%
                \BoxCandidati
             \end{flushright}%
             \else 
                \makebox[\textwidth]{\BoxRelatori\hfill\BoxCandidati}
             \fi
             \else
               \noindent
               \makebox[\textwidth]{%
               \BoxRelatori\hfill\BoxCandidati}\par
             \fi
          \fi
      \fi
%    \end{macrocode}
% The final item is the optional name of the company supervisor.
%    \begin{macrocode}
\ifcsvoid{@tutoreaziendale}{}{%
    \vfill\vfill
    {\centering \textbf{\@tutoreaziendalename}\\[.6ex]
    \@tutoreaziendale\par}}
%    \end{macrocode}
% All the material now is on the page; we put some more
% vertical glue and handle the copyright page; then we
% close the page sending it to the output file; the final
% |\endgroup| closes the |\begingroup| that was set at
% the beginning of this long macro.
% In order to handle the copyright page, we test if the
% internal definition of the copyright page text is empty;
% in this case no copyright page should be output, and
% a |\cleardoublepage| works fine in both one and two side
% printing; if the copyright page text is not empty, after
% shipping out the title page, we set the copyright page
% text flush bottom with the text block, and then we ship
% out also this copyright page.
%    \begin{macrocode}
      \par\clearpage
      \ifcsvoid{@retrofrontespizio}{}%
      {\null\vfill\thispagestyle{empty}\@retrofrontespizio\par\clearpage}%
\endgroup}
%    \end{macrocode}
% \iffalse
%</topfrn>
% \fi
%
% \subsection{Further settings for the ScuDo doctoral school}
%
% Doctoral theses to be developed at the \emph{Scuola di 
% Dottorato} (ScuDo) doctoral school of Politecnico di
% Torino require a particular set of other packages and
% commands to be available by default to the student
% writing his/her doctoral thesis. It requires also
% a title page with a layout completely different from
% those that can be typeset with the |topfront| extension
% file and, as far as I can tell, different from what
% is possible to create with the excellent |frontespizio|
% package.
%
% This task is reached by selecting  the thesis type
% by means of the |tipotesi=scudo| option to the |toptesi|
% class; this option sets on or off  the necessary switches
% and loads the |toptesi-scudo.sty| module in place of the
% |topfront| one. This specific module loads the extra
% packages, defines some special commands, provides the
% customising commands to modify the title page infix
% words and labels, and the commands to introduce the
% specific thesis data; eventually it specifies the title
% page layout and provides for its generation and to a
% copyright and disclaimer sentence on its verso page.
% 
% It produces also a colophon where some data concerning
% the thesis typesetting details are given. 
%
% Some if not all the data required for the title page may
% be entered by means of a configuration file |\jobname.cfg|
% that can be modelled on the |toptesi.cfg| file that is part of
% this bundle. The user should remember to copy the sample file in
% his/her thesis directory, change its name (not the extension)
% to match that of the thesis, modify the data according
% to his/her needs, and this extension file will read the
% configuration file at the proper moment and acquire its data.
%
% This |toptesi-scudo.sty| module is meant to load a certain
% number of packages in order to ease the user work when
% typesetting a dissertation on an engineering doctorate
% school as ScuDo. 
% 
% The chosen packages are supposed to be sufficient to
% typeset any technical thesis or dissertation; but this
% does not limit its use to the fields of engineering;
% anybody can load the packages
% s/he needs for the particular discipline s/he is
% working on. At the same time nobody forbids to load
% other packages, in addition, not in substitution, of
% the packages loaded by this extension file of the
% |toptesi| class. 
%
%\iffalse
%<*topscu>
%\fi
% We start by loading the math packages produced by the
% American Mathematical Society, differentiating what is
% necessary to load when using \pdfLaTeX\ or the OpenType
% aware programs \XeLaTeX\ or \LuaLaTeX. The extension file
% |iftex| that defines the switches \cs{ifPDFTeX},
% \cs{ifXeTeX}, and \cs{ifLuaTeX},
% has already been loaded. In particular AMS fonts should
% not be used when typesetting with \XeLaTeX\ or \LuaLaTeX,
% because these programs get them directly from the
% \sigla{UNICODE} math fonts, so that package |amssymb| has
% to be loaded only when \pdfLaTeX\ is used.
%    \begin{macrocode}
\ScuDotrue
\ifPDFTeX
   \RequirePackage{amsmath,amssymb,amsthm}
\else
   \RequirePackage{amsmath,amsthm}
   \RequirePackage{unicode-math}
\fi
%    \end{macrocode}
% The suitable macro for using the differential sign in the
% proper \sigla{ISO} compliant way (upright font) with the suitable
% spacing before and no space after, is the following:
%    \begin{macrocode}
\AtBeginDocument{%
\providecommand\diff{}%
\renewcommand{\diff}{\mathop{}\!\mathrm{d}}
\providecommand{\gei}{\ensuremath{{\mathop{\mathrm{j}}\nolimits}}}
\providecommand{\eu}{\ensuremath{{\mathop{\mathrm{e}}\nolimits}}}
\let\iu\gei}
%    \end{macrocode}
% Such a definition allows to typeset math expression containing
% integrals with a small space between the integrand and the
% integrating differential; on the contrary no space
% disturbs typesetting differential equations. The above
% definitions include also the \sigla{ISO} compliant ways of writing
% the Napier number `e', and the imaginary unit |\iu| that
% sets an operator-like roman `j' as it is customary in
% most engineering disciplines.
%
% Other packages are loaded next. Please take notice that
% |xparse| is very useful to define advanced commands and
% environments containing different kind of delimited
% arguments;  parameters entered as arguments to the
% opening statement of an environment are available also
% for the closing commands; read the documentation if you
% want to use this package facilities. The loading command is
% commented out, because as long as this module is used with the
% |toptesi| class, this package is already available. The
% commented line remains there to remember that this module
% has this dependence; should this module be used with another
% class, the user is reminded to expressly load it.
%
% Using the facilities of |setspace| may be tolerated only
% while typesetting a thesis draft. It is bad typography
% to use it in the final printing.
% 
% Packages |ifthen| and |calc|, in spite of being the
% standard traditional ones for accomplishing their tasks,
% are preferred by many users who are not aware that
% computations are now native of modern typesetting
% programs\footnote{All the three programs we are dealing
% with here; \LuaLaTeX\ can do also floating point computations
% thanks to the language Lua; all programs may benefit from the
% advances of the \LaTeX3 language, that has a special facility
% to perform floating point calculations.}; also advanced
% tests are natively available with such programs, but
% since this class loads the |etooolbox|, its powerful
% and robust commands are already available; therefore
% these packages are loaded only for the user comfort,
% although they are not really required.
%
% Packages |caption| and |subcaption| have been optional
% for several TOPtesi versions before version 6.x; here they
% are pre-loaded only for the \textsf{tipotesi=scudo} option.
%
% Package |tabularx| provides an alternative way to produce
% tables of specified width by means of a new column type
% |X|;  this column type behaves as |p|\marg{width}, where
% \meta{width} is computed by the package in order to expand
% the column(s) so as to fill the specified table
% width. It is very handy in a number of situations.
%
% For professional typesetting tables, package |booktabs|
% is highly recommended; we load also package |multirow|,
% although its use is deprecated by many professionals.
%
% The sample template file |toptesi-scudo.tex| shows a
% couple of tables typeset with and without using
% |multirow|; the results without using this package are
% marginally more elegant, but the user is free to specify
% every detail, without the support of automatic (and
% limited) typesetting macros.
% 
% Package |siunitx| is highly recommended; its management
% of fractional measures, of units of measure, and of table
% columns that contain numerical values aligned on the decimal
% separator is very useful.
%
% Package |float| is very handy to define new floating
% environments; unfortunately many users load it in
% order to have available the float positioning
% option \textsf{H}, which should never be used,
% especially while using this TOPtesi class, where
% provisions for large floats are already taken care of.
%    \begin{macrocode}
%\RequirePackage{xparse}  % Command and environment advanced definitions 
\RequirePackage{lscape}   % Supports landscape Layout
\RequirePackage{setspace} % Define line spacing in paragraph. Don't use!
\RequirePackage{calc}     % Calculation macros
\RequirePackage{ifthen}   % Conditional statements
\RequirePackage{caption}  % Configure captions
\RequirePackage{subcaption}% Defines env. subfigure and its caption
\RequirePackage{tabularx} % Tables with expandable columns
\RequirePackage{booktabs} % professional rules in tables
\RequirePackage{multirow} % It's more professional avoiding tables with
                      % cells spanning several rows
\RequirePackage{siunitx}  % advanced management of units of measure
\RequirePackage{float}    % Never ever use its code position H !
                      % Use only to define new floating objects
%    \end{macrocode}
%
% Package |nomencl| is loaded and a sample of nomenclature
% groups is defined; they are mostly examples of ``how to'';
% and in particular they are models to define other
% nomenclature groups. Notice that the string comparison
% tests are made through the testing robust commands
% provided by the |etoolbox| package.
%    \begin{macrocode}
\usepackage[intoc]{nomencl}% Generate nomenclatures or glossaries
\makenomenclature
\renewcommand{\nomgroup}[1]{%
\ifstrequal{#1}{A}{\item[\textbf{Roman Symbols}]}{%
\ifstrequal{#1}{G}{\item[\textbf{Greek Symbols}]}{%
\ifstrequal{#1}{Z}{\item[\textbf{Acronyms / Abbreviations}]}{%
\ifstrequal{#1}{R}{\item[\textbf{Superscripts}]}{%
\ifstrequal{#1}{S}{\item[\textbf{Subscripts}]}{%
\ifstrequal{#1}{X}{\item[\textbf{Other Symbols}]}{}}}}}}}

\renewcommand{\nompreamble}{\markboth{\nomname}{\nomname}}
%    \end{macrocode}
%
% It is important to note that the nomenclature
% raw file must be processed by means of program
% \prog{makeindex}  with special options. Assuming
% that your thesis main file is named |MyPhDthesis.tex|,
% after compiling with pdfLaTeX, you have to open a
% terminal, change the working directory into that
% where your main file is stored, and
% issue  and execute the line command
%{\small
%\begin{verbatim}
%makeindex -s nomencl.ist -t MyPhDthesis.nlg -o MyPhDthesis.nls MyPhDthesis.nlo
%\end{verbatim}
%\par}
% Afterwards you should have your processed nomenclature
% ready and up to date, so that by running again the
% typesetting program  your nomenclature appears in the
% back matter; a second typesetting run, as usual, updates
% the table of contents with the entry and page number of
% the Nomenclature section. It is important to do this
% action before printing the final version of the thesis;
% you do not need to update the nomenclature every time
% you edit one of your thesis source files.
%
% All this fuzz is bypassed in the sense that all typesetting
% programs can run safe OS commands, among which \prog{makeindex},
% in order to do all these actions by themselves if we
% procede by closing the auxiliary file that collect
% nomenclature entries, and run \prog{makeindex} with
% the proper options and file names directly through the
% ``magic command'' |\write18|. We then define a new command
% to process the nomenclature files and print them out.
%
% Again the test for an empty optional argument is made by
% means of a macro defined in |etoolbox|. When using LuaLaTeX
% it is necessary a special treatment, because the typesetting
% engine does not use the |\write18| functionality, but
% accesses the operating system commands by means
% of its own Lua facilities. The user is not obliged to
% know these details, but this extension file must be
% programmed accordingly.
%    \begin{macrocode}
\newcommand{\printnomencl}[1][]{%
\immediate\closeout\@nomenclaturefile
\ifLuaTeX
  \chardef\T@T@shellescape\directlua{tex.write(os.execute())}
  \def\exec@makeindex{%
     \directlua{os.execute("\luaescapestring
     {makeindex -s nomencl.ist -t \jobname.nlg -o \jobname.nls \jobname.nlo}")}}
\else
  \def\exec@makeindex{\immediate\write18{%
  makeindex -s nomencl.ist -t \jobname.nlg -o \jobname.nls \jobname.nlo}}
\fi
\exec@makeindex
\ifblank{#1}{\printnomenclature}{\printnomenclature[#1]}%
}
%    \end{macrocode}
% The result is a synchronous typesetting of the nomenclature
% in the same way as for the index.
%
% The bibliography is typeset through the facilities of
% package |biblatex| with suitable options; in particular,
% according to the tradition in scientific works, the
% references are labelled with bracketed numbers as well
% as citations.
%    \begin{macrocode}
\usepackage[autostyle]{csquotes} % necessary for biblatex
\usepackage[backend=biber, 
            style=numeric-comp,
            citestyle=numeric,
            sorting=nty,
            natbib]{biblatex} % LaTeX specific bibliography handler
\addbibresource{references.bib}% bibliographic data base(s}
%    \end{macrocode}
%
% Package |imakeidx| is loaded, but it is up to the user
% to specify command \texttt{\string\makeindex} with its
% options; in the sample template file |toptesi-scudo.tex|
% the index is made by configuring \cs{makeindex}
% by specifying only the \textsf{intoc} option, so as to
% have an entry for the index in the table of contents.
% The user who wants to prepare an index for his/her thesis
% is urged to read the |imakeidx| package documentation.
%    \begin{macrocode}
\usepackage{imakeidx}% to generate automatic indices
%    \end{macrocode}
%
% Eventually while using \pdfLaTeX we require the |indentfirst|
% package so as to indent the first paragraph after a
% sectioning command, according to a common usage in Europe
% and in Italy in particular. 
% 
%    \begin{macrocode}
\ifPDFTeX \usepackage{indentfirst}\fi
\raggedbottom

\AfterEndPreamble{\english}% Set English as default language

%    \end{macrocode}
%
% The |toptesi-scudo.sty| module contains also everything is
% needed to typeset the Ph.D. title page and the disclaimer
% page; remember that the \textsf{tipotesi=scudo} option to the
% class sets the default language to English, and inhibits
% loading the |topfront| extension file; therefore a
% complete new set of commands and a complete new title
% page environment are required to fill up the necessary
% information.
%
% We start with the default values for all
% pieces of information. Those macros that contain an empty
% expansion are obviously for mandatory data. Those that
% contain a text string insert default data, that can be
% possibly modified by suitable commands. Those that
% contain just macros insert into the title page the
% expansions of these macros.
%    \begin{macrocode}
\TOPfrontfalse

\providecommand*\@Ndissertation{Doctoral Dissertation}
\providecommand*\@Ndoctoralprogram{Doctoral Program in\xspace}
\providecommand*\@programname{}
\providecommand\@Cyclenumber{}
\providecommand*\@title{}
\providecommand*\@subtitle{\vspace{\stretch{1}}}
\providecommand*\@author{}
\providecommand*\@Nsupervisor{Supervisor}\providecommand*\@Nsupervisors{Supervisors}
\newcount\@supervisornumber
\providecommand*\@supervisorlist{}
\providecommand\@Nexaminationcommittee{Doctoral Examination Committee:}
\providecommand*\@Examinerlist{}
\providecommand*\@examlocation{Politecnico di Torino}
\providecommand*\@examinationdate{}
\providecommand*\@creativecommons{This thesis is licensed under a Creative Commons License, Attribution - Noncommercial- NoDerivative Works 4.0 International: see \url{www.creativecommons.org}.
The text may be reproduced for non-commercial purposes, provided that credit is given to the original author.}
\providecommand\@disclaimer{\noindent I hereby declare that
the contents and organisation of this dissertation constitute 
my own original work and does not compromise in any way the 
rights of third parties, including those relating to the 
security of personal data.}

%    \end{macrocode}
% There are the user macros to change or define the
% required data; of course the mandatory data must be
% present in the student's thesis source file; the other
% data may be changed, but don't need to be changed: it
% depends on the university specifications, if any are given.
%
% In any case if the ``examiner list'' is not specified, so
% that the default list remains empty, no label and no list
% is typeset; therefore leaving this list empty is an easy
% and logical method for completely omitting such a list.
%
%    \begin{macrocode}
\newcommand*\Ndissertation[1]{\gdef\@Ndissertation{#1}}
\newcommand*\Ndoctoralprogram[1]{\gdef\@Ndoctoralprogram{#1\xspace}}
\newcommand*\ProgramName[1]{\gdef\@programname{#1\xspace}}
\newcommand*\CycleNumber[1]{\gdef\@Cyclenumber{\ifblank{#1}{}{(#1 cycle)}}}
\renewcommand*\title[1]{\gdef\@title{#1}}
\newcommand*\subtitle[1]{\gdef\@subtitle{\ifblank{#1}{}{\medskip\par
   {\Large #1\par}}\vspace{\stretch{1}}}}
\renewcommand*\author[1]{\gdef\@author{#1}}
\newcommand*\NSupervisor[2]
   {\gdef\@Nsupervisor{#1}\gdef\@Nsupervisors{#2}}
\newcommand*\SupervisorList[1]{\gdef\@supervisorlist{#1}}
\newcommand\SupervisorNumber[1]{%
   \ifblank{#1}{\@supervisornumber=0}{\@supervisornumber=#1}}
\newcommand\Nexaminationcommittee[1]{\gdef\@Nexaminationcommittee{#1}}
\newcommand*\ExaminerList[1]{\gdef\@Examinerlist{#1}}
\newcommand*\Nlocation[1]{\gdef\@examlocation{#1}}
\newcommand*\ExaminationDate[1]{\gdef\@examinationdate{#1}}

\newlength{\TPT@logospace}\TPT@logospace=3em\relax
\newsavebox{\TPT@logobox} \setbox\TPT@logobox\hbox{}
\newdimen\TPT@logoheight 
\newcommand*\setlogodistance[1]{\TPT@logospace=#1}
\providecommand*{\@logosede}{}

\ifcsundef{T@Tlogoheight}{%\
    \newlength\T@Tlogoheight
    \setlength\T@Tlogoheight{0.8\dimexpr(\paperwidth-\textwidth)/2}}{}
    
\newcommand\PhDschoolLogo[2][\T@Tlogoheight]{\def\@logosede{#2}\TPT@logoheight=#1\relax
\ifcsvoid{@logosede}{\sbox{\TPT@logobox}{}}{\begin{lrbox}{\TPT@logobox}%
\expandafter\fillup@TCP@logobox\@logosede,!}}

\def\fillup@TCP@logobox#1,#2!{%
	\ifblank{#1}{\end{lrbox}\ifdim\wd\TPT@logobox>\textwidth
	\sbox\TPT@logobox{\resizebox{\textwidth}{!}{\box\TPT@logobox}}\fi}%
	{\def\@logosede{#2}%
	\includegraphics[height=\TPT@logoheight]{#1}\hskip\TPT@logospace
   \expandafter\fillup@TCP@logobox\@logosede,!}}

\newcommand\printloghi{\unless\ifvoid\TPT@logobox\usebox{\TPT@logobox}\fi}
%    \end{macrocode}
% Next the disclaimer and the signature macros are defined;
% the disclaimer text and the signature layout may be changed
% by these commands; but if the user does not use these
% macros, the name of the person who undersigns the disclaimer
% text is assumed to be the thesis author and the signature
% date is assumed to equal the examination date.
% 
% Remember: the disclaimer text is very important, in the
% sense that the student writing his/her Ph.D.\ thesis must
% be completely aware that by undesigning that statement
% he/she assumes any legal liability for any event that
% might betray other people rights. A kind of
% infringement, of course, is plagiarism. Therefore the 
% under signer must be completely aware that his/her signature
% is not just a simple bureaucratical action.
%    \begin{macrocode}
\newcommand\Disclaimer[1]{\gdef\@disclaimer{#1}}
\newcommand\CClicence[1]{\gdef\@creativecommons{#1}}
\providecommand\@signature{%
\begin{flushright}
\parbox{0.5\textwidth}{\centering
\dotfill\\
\@author\\
Turin, \@examinationdate
}%
\end{flushright}}
\newcommand\Signature[1]{\gdef\@signature{#1}}
%    \end{macrocode}
% The |ThesisTitlePage| environment is defined; in the
% opening commands the configuration file is read;
% this file may contain any kind of definition or
% redefinition by using the above commands with suitable
% arguments.
% The data possibly read from the configuration file now
% are available, or they may be entered in the body of the
% title page environment |ThesisTitlePage|. If some commands
% are repeated, the last ones, i.e.~those entered in the
% body of the environment, prevail.
%    \begin{macrocode}
\newenvironment{ThesisTitlePage}{% Apertura
\InputIfFileExists{\jobname.cfg}{\relax}{\relax}%
}{%Chiusura
\begin{titlepage}\parindent=0pt
{\centering
\raisebox{-1ex}[0pt][0pt]{\makebox[\textwidth]{\printloghi}}\par
\vspace{\baselineskip}
\@Ndissertation\\
\@Ndoctoralprogram \@programname\@Cyclenumber
\bigskip

{\Huge\bfseries \@title\par}
\@subtitle
{\Large\bfseries \@author\par}
\smallskip
\mbox{* * * * * *}
\vspace{\stretch{1}}

{\large\bfseries \ifcase\@supervisornumber
\relax\or \@Nsupervisor\else\@Nsupervisors\fi\par}
\@supervisorlist
\par}

\vspace{\stretch{1}}

\ifcsvoid{@Examinerlist}{\relax}{%
\noindent\begin{tabular}{@{}l}
\bfseries \@Nexaminationcommittee\\
\@Examinerlist
\end{tabular}}

\begin{center}
\raisebox{-3\baselineskip}[0pt][0pt]{%
\parbox{0.5\textwidth}{\centering
\@examlocation\\
\@examinationdate}}
\end{center}
\end{titlepage}
%
\newpage\thispagestyle{empty}
\vspace*{\stretch{1}}
\@creativecommons
\vspace{3\baselineskip}

\ifempty{\@keywords}{\relax}{Keywords: \@keywords\par
\vspace{2\baselineskip}}

\@disclaimer
\vspace{3\baselineskip}

\@signature}

%    \end{macrocode}
% The above code is almost everything is needed to enter
% the relevant data and to typeset the doctoral thesis
% and the signed disclaimer. But it is wise to enter
% some basic metadata for a reasonably well typeset \sigla{PDF}
% file. Notice that these metadata have nothing to do
% with the metadata required for \sigla{PDF/A} compliance; for
% the latter metadata it is necessary to load the |pdfx|
% package and enter the required metadata in a special
% way that is taken care by |pdfx|. Read the details in
% the Italian\discretionary{-}{}{-}English documentation
% |toptesi-it.pdf| that is part of the TOPtesi bundle.
%    \begin{macrocode}
\providecommand*\subject{}\providecommand*\@subject{}
\renewcommand\subject[1]{\gdef\@subject{#1}}
\providecommand*\keywords{}\providecommand*\@keywords{}
\renewcommand\keywords[1]{\gdef\@keywords{#1}}
\AtBeginDocument{% Set some metadata
  \hypersetup{%
    pdftitle   = {\@title},
    pdfauthor  = {\@author},
    pdfsubject = {\@subject},
    pdfkeywords= {\@keywords}
  }
}
%    \end{macrocode}
%
% The School required also a colophon where it is explained
% how the thesis has been typeset and who are the authors
% of the \LaTeX\ code.
%
% Of course this colophon must be typeset in the last page,
% possibly an even one if the thesis is typeset with the
% |twoside| option in force.
%    \begin{macrocode}
\AtEndDocument{%
\clearpage
\ifbool{@twoside}%
{\ifodd\value{page}\thispagestyle{empty}\newpage\fi}{\relax}%
\vspace*{\stretch{1}}%
\noindent
\begin{tabularx}{\textwidth}{Xp{0.45\textwidth}X}
&
This Ph.D.\ thesis has been typeset by means of the \TeX-system
facilities. The typesetting engine was
\ifPDFTeX
  \ifcsdef{pdfLaTeX}{\pdfLaTeX}{pdf\/\LaTeX}%
\else
   \ifLuaTeX
      \ifcsdef{LuaLaTeX}{\LuaLaTeX}{Lua\/\LaTeX}%
   \else
      \ifcsdef{Xe}{\Xe\LaTeX}{Xe\/\LaTeX}
   \fi
\fi.
The document class was \texttt{toptesi}, by Claudio Beccari, 
with option \texttt{tipotesi\discretionary{}{=}{=}scudo}.  
This class is available in every up-to-date and complete 
\TeX-system installation.
&
\end{tabularx}
\vspace*{\stretch{2}}%
}
%    \end{macrocode}

%\iffalse
%</topscu>
%\fi
%
% \subsection{A configuration file}
% The following code generates a sample configuration file
% that the user can change at will after changing also its
% name; it can be used as a template for generating a really
% personal configuration file. Remember: this template file
% is named |toptesi.cfg|, but in order to use it for a
% specific thesis, whose main file is named
% |JohnSmithMSthesis.tex|, the configuration file must be
% named |JohnSmithMSthesis.cfg|.
% \iffalse
%<*topcfg>
% \fi
%    \begin{macrocode}
%%
%%================================================================
%% This file is the only file of the TOPtesi bundle that the user
%% can modify without restrictions in order to customise the
%% contents of this configuration file to his/her needs. The user
%% can add or remove lines, comment or uncomment lines, change the
%% arguments to macros, add definitions and so on. 
%% Use  this file by copying the relevant section to another file
%% to be named as the thesis main file and with extension .cfg.
%% This bundle will try to read "\jobname.cfg"; if this file does
%% not exist, the bundle files don't do anything. This implies that
%% the provided file toptesi.cfg is to be used as a model, not to
%% be used directly.
%%================================================================
%%
\ifmagistrale
    \ateneo{Politecnico di Torino}
    \facolta{}% nessun nome di default/ no default name for PoliTO
    \FacoltaDi{}% nessun prefisso per la facoltà/no default faculty label for PoliTO
    \CorsoDiLaureaIn{Corso di Laurea in}
    \TesiDiLaurea{Tesi di Laurea Magistrale}
    \InName{in}
    %%\retrofrontespizio{Questo testo è soggetto alla 
    %% Creative Commons Licence}
\fi
\ifsecondaria

\fi
\ifScuDo
    \PhDschoolLogo{Logo-Scudo}
    \Ndissertation{Docoral Dissertation}
    \Ndoctoralprogram{Doctoral Program in\xspace}
    \NSupervisor{Supervisor}{Supervisors}
    \SupervisorNumber{1}
    \Nexaminationcommittee{Doctoral Examination Committee:}
    \Nlocation{Politecnico di Torino, Turin, Italy}
    \Disclaimer{\noindent I hereby declare that the
     contents and organisation of this dissertation
     constitute my own original work and does not 
     compromise in any way the rights of third parties, 
     including those relating to the security of personal 
     data.}
     \CClicence{This thesis is licensed under a Creative Commons
     License, Attribution - Noncommercial- NoDerivative Works 4.0 
     International: see \url{www.creativecommons.org}.
     The text may be reproduced for non-commercial purposes, 
     provided that credit is given to the original author.}
\fi
%    \end{macrocode}
% \iffalse
%</topcfg>
% \fi
%
% \subsection{The  \texttt{topcoman.sty} code}
% This file may be used as an independent extension package
% for the \texttt{report} document class, and possibly for
% other classes.
% \iffalse
%<*topcmn>
% \fi
%
% The new command |\DeclareSlantedCapitalGreekLetters|
% optionally sets the capital Greek letters in math mode
% with the glyphs taken from the math italic fonts, not
% from the math roman fonts, as it is by default; some
% authors prefer to use both symbols with different
% meanings, so this command lets them do so. This may be
% useful unless the \XeTeX\ or \LuaTeX\ typesetting engine
% is used; in facts the Unicode math defines specific
% commands for setting any Latin or Greek mathematical
% letter in any possible font shape and series.
% Actually with \XeLaTeX\ and \LuaLaTeX\ it is possible to
% invoke the |unicode-math| package wit the |style=\sigla{ISO}|
% option; in this way the \sigla{ISO} math typesetting become
% available; this includes the slanted Greek letters and
% many other facilities necessary to fulfil the \sigla{ISO}
% regulations for ``physical sciences and technology''.
%
% With \pdfLaTeX\ there are available several packages
% and font suitable for typesetting math according to
% the \sigla{ISO} regulations. The fonts Libertinus together
% with the math font LibertinusT1math are the best
% choice for this task; the |isomath| package can do
% an excellent work with other fonts; a ``poor man''
% solution with the \pdfLaTeX\ typesetting program is
% given by the |pm-isomath| package; this latter package
% is, in facts, a poor man approach to the problem, so
% it has several limitations, but it avoids completely
% the problem of loading several other math alphabets,
% therefore avoiding to exhaust the limited number of
% math alphabets \pdfLaTeX\ can handle.
%
% This |topcoman| module offers another ``poor man''
% solution only for using by default slanted capital
% Greek letters taken from the math italic font; the
% upright ones are still available within the |\mathrm|
% command argument.
% Attention: \cs{DeclareSlantedCapitalGreekLetters} is defined
% only when using \prog{pdflatex}; with the other two engines
% it is disabled and produces a warning.
%    \begin{macrocode}
\RequirePackage{iftex}
\ifPDFTeX
\newcommand*\DeclareSlantedCapitalGreekLetters{%
   \mathchardef\Gamma="7100
   \mathchardef\Delta="7101
   \mathchardef\Theta="7102
   \mathchardef\Lambda="7103
   \mathchardef\Xi="7104
   \mathchardef\Pi="7105
   \mathchardef\Sigma="7106
   \mathchardef\Upsilon="7107
   \mathchardef\Phi="7108
   \mathchardef\Psi="7109
   \mathchardef\Omega="710A
}\else
\newcommand*\DeclareSlantedCapitalGreekLetters{%
\PackageWarning{toptesi}{%
   When using LuaLaTeX or XeLaTeX specify\MessageBreak
   option math-style=ISO to package unicode-math\MessageBreak}
}
\fi
%    \end{macrocode}
% The |\ensuremath| command is defined in the \LaTeXe\ 
% kernel from a certain version on; should the user employ
% a really old \LaTeXe\ implementation, this definition
% supplies the missing one. Should the |babel| package
% not be loaded, then we provide the useful command
% |\textormath| provided by |babel|. We define the text
% version of the subscript and  ensure also that the
% |textcomp| package is loaded; of course if it is already
% loaded the |\RequirePackage| command performs the
% necessary tests and possibly does not load anything. 
%    \begin{macrocode}
\providecommand*\ensuremath[1]{\ifmmode#1\else$#1$\fi}%
\providecommand*\textormath{}
\renewcommand*\textormath{\ifmmode\expandafter\@secondoftwo\else
    \expandafter\@firstoftwo\fi}
\providecommand*\textsubscript{\raisebox{-0.5ex}}

\ifPDFTeX
\RequirePackage{textcomp}
\fi
%    \end{macrocode}
% The following commands may be already defined; should
% they be missing they are supplied here. Most of them
% are already defined in the Italian option to the
% \texttt{babel} language if the thesis is typeset with
% \textsf{pdflatex} that loads that package; these commands
% are not predefined if the thesis is typeset with \XeLaTeX\ 
% or \LuaLaTeX\ that do not load the \texttt{babel} package;
% but remember; this package may be used as a stand alone
% one, without the initial call by the \texttt{toptesi}
% document class, so that the language Italian might be
% undefined. These commands are also defined by the various
% \sigla{ISO} math compliant packages, in particular by |pm-isomath|;
% no problem: if they already defined they don't get redefined.
%    \begin{macrocode}
\providecommand{\ohm}{\textormath{\textohm}{\mathrm{\Omega}}}
\@ifpackageloaded{toptesi}{
\providecommand\ped{}\providecommand\ap{}
\renewcommand*{\ped}[1]{\textormath{\textsubscript{#1}}{_{\mathrm{#1}}}}
\renewcommand*{\ap}[1]{\textormath{\textsuperscript{#1}}{^{\mathrm{#1}}}}}{%
\providecommand*{\ped}[1]{\textormath{\textsubscript{#1}}{_{\mathrm{#1}}}}%
\providecommand*{\ap}[1]{\textormath{\textsuperscript{#1}}{^{\mathrm{#1}}}}}
\@ifpackageloaded{siunitx}{\def\unit#1{\si{\,#1}}}{%
\providecommand\unit{}%
\renewcommand\unit[1]{\ensuremath{{\mathrm{\,#1}}}}
}
\providecommand{\gei}{\ensuremath{{\mathop{\mathrm{j}}\nolimits}}}
\providecommand{\eu}{\ensuremath{{\mathop{\mathrm{e}}\nolimits}}}
\providecommand{\micro}{\textormath{\textmu}{%
  \ifPDFTeX
    \ifdefined\muup\muup\else\mbox{\textmu}\fi
  \else
    \mathup{\mu}%
  \fi
}}
\providecommand{\gradi}{\textormath{\textdegree}{^\circ}}
%    \end{macrocode}
%
% Before this version 6.xx.yy this package used to define
% the command |\listing| for listing program sources; those
% definitions got very obsolete in front of several packages
% already available for this task. For example the package
% |listings|, very elegant also for the various configurations
% it can use; un fortunately it works fine only for verbatim
% sources that contain only \textsc{ascii} characters; yes,
% it is possible the definition of special commands to typeset
% isolated characters not belonging to the \textsc{ascii}
% table, but it is an error prone process. Giving up the
% niceties of |listings|, the |fancyvrb| works very fine
% with \sigla{UNICODE} and UTF-8 encoded characters.
%
% For backwards compatibility we maintain the command
% |\listing| as a user interface to the |VerbatimInput|
% of package |fancyvrb| with an initial setting of a 
%|\small| font size.
%    \begin{macrocode}
\RequirePackage{fancyvrb}
\fvset{fontsize=\small}
\let\listing\VerbatimInput
%    \end{macrocode}
% Of course before using |\listing| the user can set any other
% special configuration s/he desiders among those usable with
% package |fancyvrb|; the default set up shown above is good
% for \LaTeX\ code input, as wel as for any language, whose
% source lines are shorter than 75--80 characters, depending
% on the normal font size and the document text width; the user
% can edit the source file as well as setting a smaller font
% size that |\small|. In certain circumstances the user
% should accept protruding lines as well as a compromise
% between editing the source file and selecting a smaller
% font size. This is not a bug of the package, nor a fault
% of the algorithm; verbatim typesetting is always problematic.
%
% Let us come to another specific typesetting problem: the
% double role the comma character plays in math mode: a
% decimal separator and a punctuation mark.
%
% Remember that the comma as a decimal separator is
% required for all languages except English. If you
% use this module outside TOPtesi, but as an extension
% of other classes or modules, it's up to you to define
% an ``intelligent comma'' macro or to load either the
% |icomma.sty| or the |ncccomma.sty| modules that define
% such a macro: the |icomma| package defines the comma
% as a mathematical active character that recognises if the
% subsequent token is a space token so as to insert a
% punctuation comma; |ncccomma| behaves more or less as
% the macro defined in TOPtesi and recognises if the
% following token is a digit so as to use a decimal comma.
%
% This particular package |topcoman| does not define any
% intelligent comma command; therefore it does not do
% anything for other packages; the main |toptesi.sty|
% packages already defines a form of intelligent comma,
% therefore |topcomand| does not have to add anything to
% |toptesi|. But if this package is used independently
% from |toptesi|, the user has two coices: either loads
% one of the two packages mentioned above, or, if s/he
% is typesetting in Italian, the |italian| module for
% |babel| already defines the intelligent comma, but its
% definition is disabled by default; the use can turn it
% on and off by using the commands |\IntelligentComma|
% and |\NoIntelligentComma|.
%
% The solution of the solutions would be using the |siunitx|
% package, whose command |\num| handles any fractional
% decimal number in the proper way adding also the digit
% triplets separation of digits when they exceed the number
% of four before and/or after the decimal separator. In
% many cases this might be a heavy burden to load a whole
% complex package just for this task, but  |siunitx|
% offers many other functionalities, so that it would be
% a good idea if the user loads that excellent package.
%
% The following commands are used to write the
% ``lower case'' roman numerals with the small-caps font;
% in order to avoid complications with missing fonts
% or with math environments, we make sure to typeset these
% numerals with script size capitals; this solution is
% not probably the best one but it works; it typesets
% these roman numerals with the current font; in TOPtesi
% roman numerals are used only for folios, but in order
% to comply with the |hyperref| module, I avoided using
% this new macro for folios; in other situations there
% are no problems with the choice of font shapes and
% series. We need a robust command in order to set the
% script math size
%    \begin{macrocode}
\DeclareRobustCommand*{\simulatedSC}[1]{%
	{\mbox{$\relax$}\fontsize{\sf@size}{\f@baselineskip}\selectfont#1}}%
%    \end{macrocode}
%
% A user, Antonio Mele, suggested and requested the
% possibility of having the figure and table names
% inserted automatically when the |\ref| command is
% issued. Actually this functionality is provided by
% package |hyperref|, but, in my opinion, it works fine
% in English, not in other languages. I do not think
% the solution reported here solves the problems that
% arise when a document is typeset in Italian. In any 
% case the definitions are here; they can be enabled and
% disabled; by default they are disabled.
%
% For single citations the solution works fine,
% but for range references it does not work. In Italian
% the name must be lower case while in other languages,
% specifically in English, the name has a capital initial.
% Since this feature might be handy in certain
% circumstances and annoying in other ones, this feature
% can be turned on and off at will with the enabling
% and disabling commands. As already said, by default the
% feature is disabled.
%    \begin{macrocode}
\def\ft@figure{\iflanguage{italian}{\MakeLowercase{\figurename}}%
            {\figurename}~}
\def\ft@table{\iflanguage{italian}{\MakeLowercase{\tablename}}%
            {\tablename}~}
%
\newcommand*\EnableFigTabNames{%
            \let\p@figure\ft@figure\let\p@table\ft@table}
\newcommand*\DisableFigTabNames{%
            \let\p@figure\empty\let\p@table\empty}
%
\DisableFigTabNames
%    \end{macrocode}
% \iffalse
%</topcmn>
% \fi
%
%\iffalse
%<*topmon>^^A               MONOGRAFIA -- TRIENNALE
%\fi
%\subsection{The \texttt{monografia} module code}
% We start by checking if certain file have ben loaded and in
% case they get loaded. We check if certain switches are defined,
% and in case such switches are defined.
%    \begin{macrocode}
\monografiatrue
\@ifpackageloaded{etoolbox}{}{\RequirePackage{etoolbox}}
\@ifpackageloaded{xparse}{}{\RequirePackage{xparse}}
\@ifpackageloaded{xspace}{}{\RequirePackage{xspace}}

\ifcsundef{ifclassica}{\newif\ifclassica \classicafalse}{}
\ifcsundef{if@autoretitolo}{\newif\if@utoretitolo \@utoretitolofalse}{}
\ifcsundef{if@ldstyle}{\newif\if@ldstyle \@ldstylefalse}{}
\ifcsundef{if@xivpt}{\newif\if@xivpt \@xivptfalse}{}
\ifcsundef{if@binding}{\newif\if@binding \@bindingfalse}{}

\ifcsundef{ifTOPfront}{\newif\ifTOPfront \TOPfronttrue}{}
\ifcsundef{ifevenboxes}{\newif\ifevenboxes \evenboxesfalse}{}
%    \end{macrocode}
% We define the applicable options.
%    \begin{macrocode}
\DeclareOption{evenboxes}{\evenboxestrue}
\DeclareOption{cucitura}{\@bindingtrue}
\DeclareOption{classica}{\classicatrue}
\DeclareOption{autoretitolo}{\ifclassica\@utoretitolotrue\fi}
\DeclareOption{oldstyle}{\ifclassica\@ldstyletrue\fi}
%\ProcessOptions*\relax
\ProcessOptions\relax

%    \end{macrocode}
% We define the titling commands...
%    \begin{macrocode}
\def\monografia#1{\gdef\@titolo{#1}}
\let\titolo\monografia
\let\@stitle\empty 
\let\@subtitle\empty
\def\sottotitolo#1{\gdef\@subtitle{#1}}
%    \end{macrocode}
% \dots and the other setting or resetting commands to be used
% within the |ThesiTitlePage| environment. Remember that all
% these commands may be used within the configuration file,
% if the user wants to use such file.
%    \begin{macrocode} 
\providecommand\@materia{}
\def\Materia#1{\gdef\@materia{#1}}
\let\materia\Materia
\providecommand\@nomerelatoresingolare{}
\providecommand\@nomerelatoreplurale{}
\newcommand\NomeRelatore[2]{%
   \gdef\@nomerelatoresingolare{#1}\gdef\@nomerelatoreplurale{#2}}
\NomeRelatore{Relatore}{Relatori}
\providecommand\@principaladviser{}
\def\relatore#1{\gdef\@principaladviser{#1}}
\providecommand\@secondadviser{}\def\secondorelatore#1{\gdef\@secondadviser{#1}}
\providecommand\@thirdadviser{}
\def\terzorelatore#1{\gdef\@thirdadviser{#1}}

\unless\ifcsname iffemminile\endcsname
   \newif\iffemminile\femminilefalse
\fi
\unless\ifcsname ifplurale\endcsname
   \newif\ifplurale\pluralefalse
\fi
\providecommand\@author{}\providecommand\@@author{}
\NewDocumentCommand\candidato{m o}{\gdef\@@author{#1}%
\IfNoValueTF{#2}{\gdef\@author{#1}}%
               {\gdef\@author{#1\IDlabel#2}}
\femminilefalse\pluralefalse}
\NewDocumentCommand\candidata{m o}{\gdef\@@author{#1}%
\IfNoValueTF{#2}{\gdef\@author{#1}}%
               {\gdef\@author{#1\IDlabel#2}}
\femminiletrue\pluralefalse}
\providecommand\@secondauthor{}
\NewDocumentCommand\secondocandidato{m o}{%
\IfNoValueTF{#2}{\gdef\@secondauthor{#1}}%
                {\gdef\@secondauthor{#1\IDlabel#2}}%
\femminilefalse\pluraletrue}
\NewDocumentCommand\secondacandidata{m o}{%
\IfNoValueTF{#2}{\gdef\@secondauthor{#1}}%
                {\gdef\@secondauthor{#1\IDlabel#2}}%
\pluraletrue}
\providecommand\@thirdauthor{}
\NewDocumentCommand\terzocandidato{m o}{%
\IfNoValueTF{#2}{\gdef\@thirdauthor{#1}}%
                {\gdef\@thirdauthor{#1\IDlabel#2}}%
\femminilefalse\pluraletrue}
\NewDocumentCommand\terzacandidata{m o}{%
\IfNoValueTF{#2}{\gdef\@thirdauthor{#1}}%
                {\gdef\@thirdauthor{#1\IDlabel#2}}%
\pluraletrue}

\providecommand\IDlabel{\\\quad matricola\ }

\edef\@submitdate{{\let\today\relax\unskip\today}}


\def\sedutadilaurea#1{\getseduta#1 !}
\def\getseduta#1 #2!{%
    \def\@tempA{#2}%
    \ifx\@tempA\empty
        \def\@submitdate{#1}%
    \else
        \unless\ifclassica
            \def\@submitdate{#1 #2}%
        \else
            \s@dutaclassica#1 #2!%
        \fi
    \fi
}%
\def\s@dutaclassica#1 #2!{%
    \if@ldstyle
        \s@paranumeri#2-!%
        \ifx\2\empty
            \edef\@submitdate{\noexpand#1 \noexpand\oldstylenums{#2}}%
        \else
            \s@paranumeri#2!%
            \edef\@submitdate{\noexpand#1
            \noexpand\oldstylenums{\1} -- \noexpand\oldstylenums{\2}}%
        \fi
        \let\1\undefined
        \let\2\undefined
    \else
        \def\@submitdate{#1 #2}%
    \fi
}
\def\s@paranumeri#1-#2!{\def\1{#1}\def\2{#2}}%

\newcommand*\NomeAnnoAccademico[1]{\gdef\AnnoAccademico{#1}}
\@ifundefined{AnnoAccademico}{\gdef\AnnoAccademico{Anno accademico}}{}

\newcommand\NomeCorsoDiStudi{Corso di laurea in\xspace}
\providecommand\@ateneo{}
\newcommand\ateneo[1]{\gdef\@ateneo{#1}}
\providecommand\@NomeAteneo{}
\newcommand\NomeAteneo[1]{\gdef\@NomeAteneo{#1}}
\let\nomeateneo\NomeAteneo
%    \end{macrocode}
% The following command accepts a first optional argument that
% should contain an ordinal number, in general an uppercase
% roman numeral, that optionally qualifies the ``academic
% structure'' in charge of the degree course. The user might,
% for example specify something as
% |\struttura[III]{Faculty of Engineering}|, if such kind of
% information is required.
%    \begin{macrocode}
\providecommand\@strutturadi{}
\newcommand\StrutturaDi[1]{\gdef\@strutturadi{#1\unskip\xspace}}
\newcommand\struttura[2][]{\gdef\@struttura{\ifempty{#1}{}{#1 }\@strutturadi #2}}
%    \end{macrocode}
% Generally bachelor degree courses are not developed within 
% international agreements such as Erasmus, but it may not be
% excluded; therefore several University names and several
% university logos might be necessary. The following is the
% general machinery for queuing several logos within a horizontal
% box to be output at the proper moment. Remember, though, that
% if the logos are output in the lower half of the page, the
% top of the title page should contain the university name
% list (if this list does not fit in one line, it is necessary
% to use some tricks described in the Italian documentation
% |toptesi-it.pdf|); but if the logos contain also the name
% of the university, and are printed at the top of the page,
% the university name list becomes unnecessary.
%    \begin{macrocode}
\ifcsundef{TPT@logospace}{\newlength{\TPT@logospace}\TPT@logospace=3em}{}
\ifcsundef{TPT@logobox}{\newsavebox{\TPT@logobox}}{}
\ifcsundef{TPT@logoheight}{\newdimen\TPT@logoheight}{}
\providecommand*\setlogodistance[1]{\TPT@logospace=#1}
\providecommand*{\@logosede}{}

\ifcsundef{T@Tlogoheight}{%\
    \newlength\T@Tlogoheight
    \setlength\T@Tlogoheight{0.8\dimexpr(\paperwidth-\textwidth)/2}}{}
    
\newcommand\logosede[2][\T@Tlogoheight]{\def\@logosede{#2}\TPT@logoheight=#1\relax
\ifcsvoid{@logosede}{\sbox{\TPT@logobox}{}}{\begin{lrbox}{\TPT@logobox}%
\expandafter\fillup@TCP@logobox\@logosede,!}}

\def\fillup@TCP@logobox#1,#2!{%
	\ifblank{#1}{\end{lrbox}\ifdim\wd\TPT@logobox>\textwidth
	\sbox\TPT@logobox{\resizebox{\textwidth}{!}{\box\TPT@logobox}}\fi}%
	{\def\@logosede{#2}%
	\includegraphics[height=\TPT@logoheight]{#1}\hskip\TPT@logospace
   \expandafter\fillup@TCP@logobox\@logosede,!}}

\providecommand\printloghi{}
\renewcommand\printloghi{%
\unless\ifvoid\TPT@logobox\usebox{\TPT@logobox}\fi}
%    \end{macrocode}
% The name of a company tutor may be required if the bachelor
% degree final work is developed within an institution or
% a company external to the university.
%    \begin{macrocode}
\providecommand\@tutoreaziendale{}
\def\tutoreaziendale#1{\gdef\@tutoreaziendale{#1}}
\providecommand\@tutoreaziendalename{Supervisore Aziendale}
\def\NomeTutoreAziendale#1{\gdef\@tutoreaziendalename{#1}}
%    \end{macrocode}
% The following command set up the information for a legal or
% copyright page; this information might contain just a simple
% statement that indicates a copyright notice or a licence that
% specifies the clauses for the intellectual property of the
% work. But it might also contain  a disclaimer where the only
% author takes his/her responsibility and liability for not
% infringing any applicable law on intellectual property;
% such disclaimer should be dated and  undersigned by the (only)
% author. Therefore within this |\retrofrontespizio| command
% two more commands are optionally used: |\luogo| to establish
% the location of the signature, and |\giorno| to establish a
% signature date different from the current date. If the |\giorno|
% is not specified, its value maintains a meaningless default
% string, and upon testing for such a string, no location,
% date and signature is typeset in the legal page; if the
% current date should be used, the user should specify
% |\giorno{}|, with a really empty argument.
%    \begin{macrocode}
\let\@luogo\empty\def\@giorno{*}
\def\luogo#1{\def\@luogo{#1}}%
\def\giorno#1{\def\@giorno{#1}}
\let\@retrofrontespizio\empty
\newcommand\retrofrontespizio[1]{%
\long\gdef\@retrofrontespizio{#1}}
%    \end{macrocode}
% We restart with the definition of other data setting commands.
%    \begin{macrocode} 
\providecommand\@corsodistudi{}
\newcommand\corsodistudi[1]{\gdef\@corsodistudi{#1}}
\let\corsodilaurea\corsodistudi
\providecommand\@monografia{Monografia}
\newcommand\NomeElaborato[1]{\gdef\@monografia{#1}}
\let\NomeMonografia\NomeElaborato
\def\@nomecandidato{\ifclassica Laureando\else Candidato\fi}
\def\@nomecandidata{\ifclassica Laureanda\else Candidata\fi}
\def\@nomecandidati{\ifclassica Laureandi\else Candidati\fi}
\def\@nomecandidate{\ifclassica Laureande\else Candidate\fi}
%    \end{macrocode}
% The label for the candidate list requires a list of comma
% separated names, to a maximum of four names; in order they
% should be the singular masculine, the plural masculine, the 
% singular feminine, and  the plural feminine names. No one is 
% mandatory, and the list may be completely empty if the label 
% should remain blank. The list is scanned and the names are
% assigned to some internal variables that will be used according
% the the gender and the number of candidates. If the list is void,
% no name is going to be used at all, and the candidates list is
% typeset as a centred one without any label.
%
% In this way, these settings are all valid:
%\begin{ttsintassi}
%\cs{TitoloListaCandidati}\Marg{}
%\cs{TitoloListaCandidati}\Marg{Estudiante,Estudiantes}
%\cs{TitoloListaCandidati}\Marg{ExamenKandidat,ExamenKandidaten\%
%\qquad,ExamenKandidatin,ExamenKandidatinen}
%\end{ttsintassi}
% This command, as it may be seen, may be used also to set the
% name list in languages different from Italian.
% It goes by itself that if the user typesets the thesis in a
% language where names are gender indifferent (as English,
% Spanish, and other ones) s/he has two choices: ether s/he
% uses the command with just two arguments, as shown above,
% and uses only the masculine commands |\candidato|,
% |\secondocandidato|, and |\terzocoandidato| even with names
% of female students, or s/he specifies the same names in the
% arguments~3 and~4, as in arguments ~1 and~2. Lacking this
% action, the corresponding titling names are blank.
%    \begin{macrocode}
\newcommand\TitoloListaCandidati[1]{\CandidateNames#1,,,,!}
\def\CandidateNames#1,#2,#3,#4,!{%
    \gdef\@nomecandidato{#1}% Masc/neutr-sing
    \gdef\@nomecandidati{#2}% Masc/neutr-plur
    \gdef\@nomecandidata{#3}% Fem-sing
    \gdef\@nomecandidate{#4}% Fem-plur
}
\providecommand\Relatore{Relatore}
\newcommand\AdvisorName[1]{\gdef\Relatore{#1}}
\providecommand\Correlatore{Correlatore}
\providecommand\Correlatori{Correlatori}
\newcommand\CoAdvisorName[2]{\gdef\Correlatore{#1}%
                             \gdef\Correlatori{#2}}
%    \end{macrocode}
% A new boolean is defined in order to have a suitable switch to set the logo(s) ``string'' at the top or in the second half of the page.
%    \begin{macrocode}
\ifcsundef{iftopTPTlogos}{\newbool{topTPTlogos} \booltrue{topTPTlogos}}{}
%    \end{macrodoe}
% We eventually define the whole construction of the title
% page with this new command defined by means of the |xparse|
% package facilities. This environment accepts an optional
% asterisk; its presence means th the logo(s) string should
% be typeset in the lower half of the title page; differently
% from the \LaTeX\ kernel environments that accept an optional
% asterisk that becomes part of the environment name, and must
% be repeated in the closing environment statement, with the
% |xparse| facilities the asterisk is just an argument to the
% argument list of the opening command: therefore the source
% file contains an opening statement of the form
% |\begin{environment}*|, not |\begin{environment*}|. The user
% should remember tis peculiar way of specifying certain optional
% arguments, in particular the optional ``star'' one.
% 
% Another important difference is that environments defined
% with the |xparse| facilities  may receive  optional and
% mandatory arguments of many kinds in the opening command,
% but differently form the standard \LaTeX\ kernel environments
% such arguments are available not only in the opening commands,
% but also in the closing ones. We actually don't use this
% property during the definition of the |ThesisTitlePage|
% environment, but we preferred to use the previously defined
% switch.
%    \begin{macrocode}
\NewDocumentEnvironment{ThesisTitlePage}{s}
{% OPENING
\IfBooleanTF{#1}{\boolfalse{topTPTlogos}}{\booltrue{topTPTlogos}}%
   \begin{titlepage}
   \oddsidemargin=\dimexpr(\oddsidemargin+\evensidemargin)/2\relax
   \evensidemargin \oddsidemargin
         \IfFileExists{\jobname.cfg}{\input{\jobname.cfg}%
        }{%C
          \PackageWarning{monografia}{No configuration file found}%
         }%C
}{% CLOSING
   \ifbool{topTPTlogos}{\noindent\raisebox{-0.5em}[\z@][\z@]{\makebox[\textwidth]{\printloghi}}\par\bigskip}{}
   \ifdefempty{\ateneo}{}{{\centering\huge \@ateneo\par}}
   \ifdefempty{\@NomeAteneo}{}{\medskip{\centering\large
       \expandafter\MakeUppercase\expandafter{\@NomeAteneo}\par}}
    \ifdefempty{\@strutturadi}{}{\vspace{\stretch{0.1}}\par{\centering \Large\@struttura\par}}
   \ifdefempty{\@corsodistudi}{}{\bigskip{\centering\large \NomeCorsoDiStudi\ \@corsodistudi\par}}
   \ifdefempty{\@materia}{}{\medskip{\centering\large\@materia\par}}
   \vspace{\stretch{0.25}}
   {\centering\large \@monografia\par}
   \vspace{\stretch{0.1}}
   {\centering\LARGE\bfseries \@titolo\par}
   \ifdefempty{\@subtitle}{\vspace{\stretch{0.3}}}{\bigskip
   {\centering\large\@subtitle\par}}
   \ifbool{topTPTlogos}{}{\vspace{3em}\par\centering\printloghi\par}
   \vspace{\stretch{0.25}}
   
\ifdefempty{\@principaladviser}{\def\BoxRelatori{}}{%
     \def\BoxRelatori{%
            \begin{tabular}[t]{l}%
            \hbox{\unless\ifclassica\large\fi
            \ifdefempty{\@secondadviser}%
            {\let\@nomerelatore\@nomerelatoresingolare}%
            {\let\@nomerelatore\@nomerelatoreplurale}%
                  \textbf{\@nomerelatore}}\\[.6ex]
            \hbox{\large\textrm{\protect\@principaladviser}}%
            \ifdefempty{\@secondadviser}{}{%
               \ifclassica
                  \ifdefempty{\@thirdadviser}{%
                     \ifdefempty{\@secondadviser}{}{%
                           \\[1.5ex]\textbf{\Correlatore:}%
                      }
                  }{%
                     \\[1.5ex]\textbf{\Correlatori:}%
                  }
               \fi
               \\[.6ex]
               \hbox{{\large\textrm{\protect\@secondadviser}}}%
            }
            \ifdefempty{\@thirdadviser}{}{%
               \\[.6ex] 
               \hbox{{\large\textrm{\protect\@thirdadviser}}}%
            }
            \end{tabular}%
    }%
}
%
\def\print@secondocandidato{\\\relax
            \hbox{\large\tabular{@{}l@{}}\@secondauthor\endtabular}}%
\def\print@terzocandidato{\\\relax
            \hbox{\large\tabular{@{}l@{}}\@thirdauthor\endtabular}}%
            
\def\BoxCandidati{%
     \begin{tabular}[t]{l}
        \hbox{\unless\ifclassica\large\fi
           \textbf{\ifplurale
              \iffemminile\@nomecandidate\else\@nomecandidati\fi
           \else
              \iffemminile\@nomecandidata\else\@nomecandidato\fi
           \fi}}\\[.6ex]
           \hbox{\large\tabular{@{}l@{}}\@author\endtabular}%
           \ifcsvoid{@secondauthor}{}{\print@secondocandidato}%
           \ifcsvoid{@thirdauthor}{}{\print@terzocandidato}%
     \end{tabular}%
}%
\unless\ifclassica
  \ifdefempty{\BoxRelatori}{%Bachelor degree without supervisors
    \noindent\makebox[\textwidth]{\tabular[t]{@{}c@{}}
    \unless\ifclassica\large\fi
       \textbf{\ifplurale
          \iffemminile\@nomecandidate\else\@nomecandidati\fi
       \else
          \iffemminile\@nomecandidata\else\@nomecandidato\fi
       \fi}\\[.6ex]
       \@author
       \ifcsvoid{@secondauthor}{}{\\ \@secondauthor}
       \ifcsvoid{@thirdauthor}{}{\\ \@thirdauthor}
    \endtabular}%
  }{%
    \unless\ifevenboxes
         \begin{flushleft}%
            \BoxRelatori
         \end{flushleft}\par
            \vspace*{-1.5\baselineskip}
         \begin{flushright}%
            \BoxCandidati
         \end{flushright}%
    \else 
         \noindent
         \makebox[\textwidth]{\BoxRelatori\hfill\BoxCandidati}%
    \fi}
\else
     \noindent
     \makebox[\textwidth]{\BoxRelatori\hfill\BoxCandidati}%
\fi
\par\vspace*{\stretch{0.5}}
\ifdefempty{\@tutoreaziendale}{\relax}{\begin{center}
\textbf{\@tutoreaziendalename}\\[0.6ex]
\@tutoreaziendale
\end{center}}
\noindent\raisebox{-\baselineskip}[0pt][0pt]{%
   \makebox[\textwidth]{\@submitdate}}%
\end{titlepage}\newpage
% Legal/Copyright page
   \ifdefempty{\@retrofrontespizio}{}{\thispagestyle{empty}%
   \vspace*{\stretch{1}}\par
   \@retrofrontespizio\par
   \ifdefstring{\@giorno}{*}{}{\medskip
   \noindent
   \begin{tabular*}{\textwidth}{@{\extracolsep{\fill}}lc@{}}
   \ifdefempty{\@luogo}{}{\@luogo, }%
   \ifdefempty{\@giorno}{\today}{\@giorno}  
            & \@@author\\[3ex]
            & \makebox[0.5\textwidth]{\dotfill}
   \end{tabular*}\par}
   \newpage}
}
%    \end{macrocode}
% This is the end of this module code.
%\iffalse
%</topmon>
%\fi
%
%\iffalse
%<*topmag>^^A               MAGISTRALE
%\fi
% \subsection{Code for \opz{magistrale}}
% Here we document the code for the \opz{magistrale} option.
% It is a little more complicated than the code for other
% thesis types, because the title page may be typeset in different
% modes, especially for theses in the humanities domain where
% they might be typeset in more than one volume (tome).
%
% We start by providing the necessary commands, without
% duplicating those provided by the calling package |toptesi|.
% Nevertheless this module may be used as a standalone one,
% i.e.\ a package loaded by a different class; may be not
% always and not with any class, but the idea is this one.
% In facts it this module does not receive the suitable
% settings from package |toptesi|,it is possible to call it
% with the suitable options as explained below.
%    \begin{macrocode}
\@ifpackageloaded{etoolbox}{}{\RequirePackage{etoolbox}}
\@ifpackageloaded{xspace}{}{\RequirePackage{xspace}}
\@ifpackageloaded{xparse}{}{\RequirePackage{xparse}}
\ifcsundef{ifclassica}{\newif\ifclassica \classicafalse}{}
\ifcsundef{if@utoretitolo}{\newif\if@utoretitolo\@utoretitolofalse}{}
\ifcsundef{if@ldstyle}{\newif\if@ldstyle\@ldstylefalse}{}
\ifcsundef{if@xivpt}{\newif\if@xivpt \@xivptfalse}{}
\ifcsundef{if@binding}{\newif\if@binding\@bindingfalse}{}
\ifcsundef{ifScuDo}{\newif\ifScuDo \ScuDofalse}{}
\ifcsundef{ifmonografia}{\newif\ifmonografia\monografiafalse}{}
\ifcsundef{iftriennale}{\newif\iftriennale\triennalefalse}{}
\ifcsundef{ifmagistrale}{\newif\ifmagistrale\magistralefalse}{}
\ifcsundef{ifdottorale}{\newif\ifdottorale\dottoralefalse}{}
\ifcsundef{ifsecondaria}{\newif\ifsecondaria\secondariafalse}{}
\ifcsundef{ifplurale}{\newif\ifplurale\pluralefalse}{}
\ifcsundef{iffemminile}{\newif\iffemminile\femminilefalse}{}

\magistraletrue

\ifcsundef{ifTOPfront}{\newif\ifTOPfront \TOPfrontfalse
\newif\ifevenboxes \evenboxesfalse}{}

%    \end{macrocode}
% We provide for the counter |tomo| and for a new definition
% the command |\tomo|; this new definition does not trigger
% the production of a title page as it does with the definition
% of package |toptesi.sty|; it accepts an optional argument,
% the tome (arabic) number (that shall be transformed into a
% capital roman one while typesetting the title page); if this
% number is not specified the counter value is stepped up
% with the |\refstepconter| macro that renders it accessible
% to the |\label| command; otherwise a value equal to the
% given number minus one is loaded and afterwars ``refstepped''
% up in the same way.
%
% Since \LaTeX\ counter settings are global, the value kept
% in the counter remains available anyhow; therefore it is
% not excluded that multiple tomes might be typeset in one 
% output \sigla{PDF} file to be split in separate files as described
% in section~\ref{sec:topmagistrale}; I did not test this
% functionality but I do not exclude it.^^A CONTROLLARE
%    \begin{macrocode}
\ifcsundef{c@tomo}{\newcounter{tomo}{\setcounter{tomo}{0}}}{}
\providecommand\tomo{}
\ifcsundef{tomo}{\renewcommand\tomo[1][]{%
  \ifstrequal{#1}{}{}{\setcounter{tomo}{\numexpr#1-1}}%
  \refstepcounter{tomo}}}{}
%    \end{macrocode}
%
% We define the options that set the same switches as those set
% by the class. If another class different from |toptesi| is used,
% it is possible that \opz{14pt} produces errors or does not
% produce the desired setting of the normal size to 14pt.
%    \begin{macrocode}
\DeclareOption{evenboxes}{\evenboxestrue}
\DeclareOption{cucitura}{\@bindingtrue}
\DeclareOption{14pt}{\@xivpttrue}
\DeclareOption{classica}{\classicatrue}
\DeclareOption{autoretitolo}{\ifclassica\@utoretitolotrue\fi}
\DeclareOption{oldstyle}{\ifclassica\@ldstyletrue\fi}
\ProcessOptions*\relax
%    \end{macrocode}
%
% We define two different page styles to be used normally
% or when the |classica| option is in force. Both redefine
% the text grid to be more or less centred on the physical
% page; since it is the first page, no binding correction
% is possibly taken care of, because the first page does
% not suffer from the curvature and thickness of the bound
% thesis. The headings and footers in both cases are already
% filled up with suitable title page data, therefore their
% space is used instead of wasted.
%    \begin{macrocode}
\def\headstrut{\vrule \@depth4\p@ \@height\z@ \@width\z@}
\def\ps@titlepage{\let\@mkboth\@gobbletwo
   \def\@oddfoot{\vbox to 0.05\paperheight{\vss
      \hbox to\hsize{\hfil{\Large{\@submitdate}}\hfil}}}%
   \let\@evenfoot\@oddfoot
   \def\@oddhead{%
      \vbox to\headheight{\vss\iftopTPTlogos
         \hbox to\textwidth{%
           \headstrut\hfil
           \raisebox{3\baselineskip}{\usebox\TPT@logobox}\hfil\null%
         }\fi
         \ifcsvoid{@ateneo}{}{\vskip\smallskipamount
         \hbox to\textwidth{\hss\LARGE\MakeUppercase{\@ateneo}\hss}}
         \vss
      }%
   }%
   \let\@evenhead\@oddhead
   \def\chaptermark##1{}\def\sectionmark##1{}%
}

\def\ps@classica{\let\@mkboth\markboth
  \def\@oddhead{\vbox{%
  \setbox256=\hbox{\LARGE\MakeUppercase{\@ateneo}}
  \unless\ifdim\wd256<\textwidth
     \setbox256=\hbox{\resizebox{\textwidth}{!}{\box256}}%
   \fi
  \hbox to \hsize{\hss\box256\hss}%
  \ifclassica
        \hbox to \hsize{\hfil\vrule\@width\z@
    \@height2ex\vrule\@height1.4\p@\@depth-\p@\@width50mm\hfil}%
  \fi
  }}%
  \def\@oddfoot{\vbox to \dimexpr\paperheight/20\relax{\vss
 \ifclassica
    \hbox to \hsize{\hfil\raisebox{-.3ex}[\z@][\z@]{%
    \vrule\@height-2.6\p@\@depth3\p@\@width
        \dimexpr\textwidth/3}\hfil}%
  \fi
  \hbox to\hsize{\hfill{\Large{\@submitdate}}\hfill}}%
  }%
  \let\@evenhead\@oddhead
  \let\@evenfoot\@oddfoot
}%
%    \end{macrocode}
%
% Some redefinitions to handle both the heading label of
% the authors block and the proper gender of such names.
%    \begin{macrocode}
\newif\iffemminile
%\newif\ifdottorato \dottoratofalse
%\newif\iftriennale \triennalefalse

\ifTOPfront\ifclassica
\def\Candidato{Laureando}
\def\Candidata{Laureanda}
\def\Candidati{Laureandi}
\def\Candidate{Laureande}
\fi\fi
%    \end{macrocode}
%
% We prepare the |titolo| command. Such command accepts
% a short title string no more, no less than the \LaTeX\ 
% sectioning commands. It is always advisable to use short
% titles; nevertheless when the |classica| |autoretitolo|
% option is in force, long titles don't fit into the page
% headings, therefore an alternative short title is desirable.
% Long titles may be also shortened by transferring part of
% their information to a subtitle; but this must be an
% author's decision, it cannot be automated. Therefore
% a |\sottotitolo| command is also provided.
%    \begin{macrocode}
\let\@stitle\empty % default short title string

\newcommand*{\titolo}[2][]{%
\ifstrequal{#1}{}%
     {\gdef\@stitle{#2}}{\gdef\@stitle{#1}}%
     \gdef\@titolo{#2}%
}%
\def\sottotitolo#1{\gdef\@subtitle{#1}}

%    \end{macrocode}
%
% The defining commands for both the preset fixed strings
% and the variable data strings are defined; somewhere
% these strings receive also the preset values; some of the
% following strings are used to modify such preset values,
% very useful to localise such strings for a particular language.
%    \begin{macrocode}
\let\@materia\empty
\def\Materia#1{\def\@materia{#1}}\let\materia\Materia
\def\relatore#1{\gdef\@principaladviser{#1}}
\def\secondorelatore#1{\gdef\@secondadviser{#1}}
\def\terzorelatore#1{\gdef\@thirdadviser{#1}}
\def\sedutadilaurea#1{\getseduta#1 !}
\def\getseduta#1 #2!{%
  \ifstrequal{#2}{}%
    {\def\@submitdate{#1}}%
    {\unless\ifclassica
      \def\@submitdate{#1 #2}%
     \else
      \s@dutaclassica#1 #2!%
     \fi}
}%
\def\s@dutaclassica#1 #2!{%
  \if@ldstyle
    \s@paranumeri#2-!%
    \ifx\2\empty
       \edef\@submitdate{\noexpand#1 \noexpand\oldstylenums{#2}}%
    \else
       \s@paranumeri#2!%
       \edef\@submitdate{\noexpand#1
       \noexpand\oldstylenums{\1} -- \noexpand\oldstylenums{\2}}%
    \fi
    \let\1\undefined
    \let\2\undefined
  \else
    \def\@submitdate{#1 #2}%
  \fi
}
\def\s@paranumeri#1-#2!{\def\1{#1}\def\2{#2}}%

\newcommand*\NomeAnnoAccademico[1]{\gdef\AnnoAccademico{#1}}
\@ifundefined{AnnoAccademico}{\gdef\AnnoAccademico{Anno accademico}}{}
\def\annoaccademico#1{%
\ifclassica
  \def\@submitdate{{\large\textsc{\AnnoAccademico}}%
    {\Large\s@tanno#1!}}
\else
  \PackageWarning{toptesi}{\string\annoaccademico\space
    is usable only when the\MessageBreak
   `classica' option is in force}%
  \def\@submitdate{\AnnoAccademico\ #1 --- 
      Needs the `classica' option}
\fi}
\def\s@tanno#1-#2!{\oldstylenums{#1}\EnDash\oldstylenums{#2}}

\def\EnDash{{\settowidth{\dimen@}{\large\scshape I}%
\setbox\tw@\hbox{2}\dimen\tw@.5\ht\tw@\advance\dimen\tw@-.5\dp\tw@
\dimen4\dimen\tw@\advance\dimen4by.0385ex\relax
\advance\dimen\tw@-.0385ex\relax
\makebox[1.5\dimen@]{%
    \vrule\@width\dimen@\@height\dimen4\@depth-\dimen\tw@}}}
\def\corsodilaurea#1{\gdef\@corso{#1}}
\def\ateneo#1{\gdef\@ateneo{#1}}
\def\nomeateneo#1{\gdef\@nomeat{\expandafter\uppercase{\expandafter #1}}}
\newcommand\facolta[2][]{\gdef\@facname{#2}\gdef\@facnumber{#1}}
\let\struttura\facolta

%    \end{macrocode}
%
% Here we define the necessary commands to fill up a comma
% separated list of logo graphic file names; we define
% everything up to the |\printlogos| command that provides
% to printing a list of images, duly separated by adequate
% space, all scaled to the same height, and, possibly, even
% the whole logo string duly scaled to fit in the title page
% width. The |\logosede| command, that can be used several
% times with different logo file names, accepts an optional
% argument consisting in a dimension; it is used to scale the
% height of the logo image to the desired value, so that the
% whole string of images does not have unusual differences
% logo heights. This dimension id preset to one seventh of
% the |\textwidth| and on the average it is well suited. With
% the standard |toptesi| class printed on A4 paper, such height
% equals 25\,mm; of course on smaller paper it is proportionally
% small; on American sized sheets it may be a little larger,
% but in general it is adequate to the page size.
%    \begin{macrocode}
\newlength{\TPT@logospace}\TPT@logospace=3em\relax
\newsavebox{\TPT@logobox}
\newdimen\TPT@logoheight
\newcommand*\setlogodistance[1]{\TPT@logospace=#1}
\providecommand*{\@logosede}{}

\ifcsundef{T@Tlogoheight}{%\
    \newlength\T@Tlogoheight
    \setlength\T@Tlogoheight{0.8\dimexpr(\paperwidth-\textwidth)/2}}{}
    
\newcommand\logosede[2][\T@Tlogoheight]{\def\@logosede{#2}\TPT@logoheight=#1\relax
\ifcsvoid{@logosede}{\sbox{\TPT@logobox}{}}{\begin{lrbox}{\TPT@logobox}%
\expandafter\fillup@TCP@logobox\@logosede,!}}

\def\fillup@TCP@logobox#1,#2!{%
\ifblank{#1}{\end{lrbox}\ifdim\wd\TPT@logobox>\textwidth
\sbox\TPT@logobox{\resizebox{\textwidth}{!}{\box\TPT@logobox}}\fi}%
{\def\@logosede{#2}%
\includegraphics[height=\TPT@logoheight]{#1}\hskip\TPT@logospace
   \expandafter\fillup@TCP@logobox\@logosede,!}}

\newcommand\printloghi{\unless\ifvoid\TPT@logobox\usebox{\TPT@logobox}\fi}
%    \end{macrocode}
%
% Still more commands to define the necessary string data.
% For the ID number label the same remarks made in 
% section~\ref{sec:topfront} hold true.
%    \begin{macrocode}
\def\tutoreaziendale#1{\gdef\@tutoreaziendale{#1}}
\newcommand\retrofrontespizio[1]{\long\gdef\@retrofrontespizio{#1}}
\newcommand\FacoltaDi[1]{\gdef\@faculty{#1}}
\let\StrutturaDidattica\FacoltaDi
\newcommand\CorsoDiLaureaIn[1]{\gdef\@laureaname{#1}}
\newcommand\TesiDiLaurea[1]{\gdef\@TesiDiLaurea{#1}}
\newcommand\InName[1]{\gdef\@InName{#1}}
\newcommand\CandidateName[1]{\gdef\@nomecandidato{#1}}
\newcommand\AdvisorName[1]{\gdef\Relatore{#1}\gdef\Relatori{#1}}
\newcommand\CoAdvisorName[1]{\gdef\Correlatore{#1}\gdef\Correlatori{#1}}
\newcommand\TutorName[1]{\gdef\Tutore{#1}}
\newcommand\NomeTutoreAziendale[1]{\gdef\@tutoreaziendalename{#1}}
\newcommand\NomePrimoTomo[1]{\gdef\PrimoTomo{#1}}
\newcommand\NomeSecondoTomo[1]{\gdef\SecondoTomo{#1}}
\newcommand\NomeTerzoTomo[1]{\gdef\TerzoTomo{#1}}
\newcommand\NomeQuartoTomo[1]{\gdef\QuartoTomo{#1}}
\newcommand\IDlabel{\\\quad matricola:\xspace}
\IfFileExists{\jobname.cfg}{\input{\jobname.cfg}}%
    {\IfFileExists{toptesi.cfg}{%%
%% This is file `toptesi.cfg',
%% generated with the docstrip utility.
%%
%% The original source files were:
%%
%% toptesi.dtx  (with options: `topcfg')
%%   ______________________________________________________
%%   The TOPtesi bundle
%%   Copyright (C) 2015-2017 Claudio Beccari
%%   All rights reserved
%% 
%%   License information appended
%% 

%%
%%================================================================
%% This file is the only file of the TOPtesi bundle that the user
%% can modify without restrictions in order to customise the
%% contents of this configuration file to his/her needs. The user
%% can add or remove lines, comment or uncomment lines, change the
%% arguments to macros, add definitions and so on.
%% Use  this file by copying the relevant section to another file
%% to be named as the thesis main file and with extension .cfg.
%% This bundle will try to read "\jobname.cfg"; if this file does
%% not exist, the bundle files don't do anything. This implies that
%% the provided file toptesi.cfg is to be used as a model, not to
%% be used directly.
%%================================================================
%%
\ifmagistrale
    \ateneo{Politecnico di Torino}
    \facolta{}% nessun nome di default/ no default name for PoliTO
    \FacoltaDi{}% nessun prefisso per la facoltà/no default faculty label for PoliTO
    \CorsoDiLaureaIn{Corso di Laurea in}
    \TesiDiLaurea{Tesi di Laurea Magistrale}
    \InName{in}
    %%\retrofrontespizio{Questo testo è soggetto alla
    %% Creative Commons Licence}
\fi
\ifsecondaria

\fi
\ifScuDo
    \PhDschoolLogo{Logo-Scudo}
    \Ndissertation{Docoral Dissertation}
    \Ndoctoralprogram{Doctoral Program in\xspace}
    \NSupervisor{Supervisor}{Supervisors}
    \SupervisorNumber{1}
    \Nexaminationcommittee{Doctoral Examination Committee:}
    \Nlocation{Politecnico di Torino, Turin, Italy}
    \Disclaimer{\noindent I hereby declare that the
     contents and organisation of this dissertation
     constitute my own original work and does not
     compromise in any way the rights of third parties,
     including those relating to the security of personal
     data.}
     \CClicence{This thesis is licensed under a Creative Commons
     License, Attribution - Noncommercial- NoDerivative Works 4.0
     International: see \url{www.creativecommons.org}.
     The text may be reproduced for non-commercial purposes,
     provided that credit is given to the original author.}
\fi
%% 
%% Copyright 2005-2017 Claudio Beccari
%% 
%% Distributable under the LaTeX Project Public License,
%% version 1.3c or higher (your choice). The latest version of
%% this license is at: http://www.latex-project.org/lppl.txt
%% 
%% This work is "author-maintained"
%% 
%% This work consists of this file toptesi.dtx, a README file
%% the manifest.txt file, and the derived files:
%%     toptesi.cls,
%%     toptesi.sty,
%%     topfront.sty,
%%     topcommand.sty,
%%     toptesi-scudo.sty,
%%     toptesi.cfg,
%%     toptesi-monografia.sty,
%%     toptesi-sss.sty,
%%     toptesi-magistrale.sty,
%%     toptesi-dottorale.sty,
%% and the English documentation toptesi.pdf.
%% 
%% Furthermore the bundle contains the documentation source
%% file toptesi-it.tex and the derived file
%% toptesi-it.pdf.
%% 
%% The toptesi-example.tex source file is just an example
%% that shows how to use the various commands; by commenting
%% or uncommenting certain source lines it is possible to
%% typeset different kind of theses and their front pages.
%% 
%% The topfront-example.tex source file is an example of how
%% to produce just the title page with TOPtesi and the external
%% package frontespizio.
%% 
%% The toptesi-scudo-example.tex source file is an example
%% that shows the particular features available with TOPtesi,
%% when a doctoral thesis is produced for the Scuola di
%% Dottorato (ScuDo) of Politecnico di Torino.
%% 
%% The toptesi.cfg file is a sample of a local configuration
%% file that can be copied to another file and its copy freely
%% edited and customised.
%% 
%% The other toptesi-*.sty files are extension modules for
%% typesetting the specific thesis kinds described by the
%% suffix that replaces the asterisk.
%% 
%% By running pdflatex on toptesi.dtx the user gets the class,
%% sty and cfg files and the English documentation file in
%% PDF format.
%% 
%% The source file of the Italian documentation file
%% toptesi-it.pdf is another example of how to use TOPtesi.
%% This file toptesi-it.tex may be typeset with pdfLaTeX,
%% XeLaTeX, and LuaLateX; see the first comment lines of
%% the file for how-to information. In spite of being mainly
%% written in Italian, it contains several sections in English
%% for the benefit of foreign students attending the Doctoral
%% School of Turin.
%%
%% End of file `toptesi.cfg'.
}{}}
%    \end{macrocode}
%
% Here we define the preset strings; most of the are empty
% strings, but some are real words; the setting commands
% listed before can easily be used to localise such string
% in different languages.
%    \begin{macrocode}
\@ifundefined{@titolo}{\def\@titolo{}}{}

\providecommand\@author{}\providecommand\@@author{}
\NewDocumentCommand\candidato{m o}{\gdef\@@author{#1}%
\IfNoValueTF{#2}{\gdef\@author{#1}}%
               {\gdef\@author{#1\IDlabel#2}}
\femminilefalse\pluralefalse}
\NewDocumentCommand\candidata{m o}{\gdef\@@author{#1}%
\IfNoValueTF{#2}{\gdef\@author{#1}}%
               {\gdef\@author{#1\IDlabel#2}}
\femminiletrue\pluralefalse}
\providecommand\@secondauthor{}
\NewDocumentCommand\secondocandidato{m o}{%
\IfNoValueTF{#2}{\gdef\@secondauthor{#1}}%
                {\gdef\@secondauthor{#1\IDlabel#2}}%
\femminilefalse\pluraletrue}
\NewDocumentCommand\secondacandidata{m o}{%
\IfNoValueTF{#2}{\gdef\@secondauthor{#1}}%
                {\gdef\@secondauthor{#1\IDlabel#2}}%
\pluraletrue}
\providecommand\@thirdauthor{}
\NewDocumentCommand\terzocandidato{m o}{%
\IfNoValueTF{#2}{\gdef\@thirdauthor{#1}}%
                {\gdef\@thirdauthor{#1\IDlabel#2}}%
\femminilefalse\pluraletrue}
\NewDocumentCommand\terzacandidata{m o}{%
\IfNoValueTF{#2}{\gdef\@thirdauthor{#1}}%
                {\gdef\@thirdauthor{#1\IDlabel#2}}%
\pluraletrue}
\providecommand\IDlabel{\\\quad matricola\ }
\@ifundefined{@principaladviser}{\def\@principaladviser{}}{}
\@ifundefined{@secondadviser}{\def\@secondadviser{}}{}
\@ifundefined{@thirdadviser}{\def\@thirdadviser{}}{}
\@ifundefined{@tutore}{\def\@tutore{}}{}
\@ifundefined{@nomerelatore}{\def\@nomerelatore{}}{}
\@ifundefined{@nomecandidato}{\def\@nomecandidato{}}{}
\@ifundefined{Candidato}{\def\Candidato{Candidato}}{}
\@ifundefined{Candidata}{\def\Candidata{Candidata}}{}
\@ifundefined{Candidati}{\def\Candidati{Candidati}}{}
\@ifundefined{Candidate}{\def\Candidate{Candidate}}{}
\@ifundefined{Relatore}{\def\Relatore{Relatore}}{}
\@ifundefined{Relatori}{\def\Relatori{Relatori}}{}
\@ifundefined{Correlatore}{\def\Correlatore{Correlatore}}{}
\@ifundefined{Correlatori}{\def\Correlatori{Correlatori}}{}
\@ifundefined{Tutore}{\def\Tutore{Tutore}}{}
\@ifundefined{@tutoreaziendale}{\def\@tutoreaziendale{}}{}
\@ifundefined{@tutoreaziendalename}%
        {\def\@tutoreaziendalename{Supervisore Aziendale}}{}
\@ifundefined{@retrofrontespizio}{\def\@retrofrontespizio{}}{}
\@ifundefined{@subtitle}{\def\@subtitle{}}{}
\@ifundefined{@corso}{\def\@corso{}}{}
\@ifundefined{@ateneo}{\def\@ateneo{POLITECNICO DI TORINO}}{}
\@ifundefined{@nomeat}{\def\@nomeat{}}{}% Nome proprio dell'ateneo
\@ifundefined{@facolta}{\def\@facname{}}{}
\@ifundefined{@facnumber}{\def\@facnumber{}}{}
\@ifundefined{@faculty}{\def\@faculty{}}{}
\@ifundefined{PrimoTomo}{\def\PrimoTomo{Tomo primo}}{}
\@ifundefined{SecondoTomo}{\def\SecondoTomo{Tomo secondo}}{}
\@ifundefined{TerzoTomo}{\def\TerzoTomo{Tomo terzo}}{}
\@ifundefined{QuartoTomo}{\def\QuartoTomo{Tomo quarto}}{}
\ifclassica
\def\Candidato{Laureando}
\def\Candidata{Laureanda}
\def\Candidati{Laureandi}
\def\Candidate{Laureande}
\fi
\@ifundefined{@submitdate}{\def\@submitdate{\ifcase\the\month\or%
  Gennaio\or Febbraio\or Marzo\or Aprile\or Maggio\or Giugno\or
  Luglio\or Agosto\or Settembre\or Ottobre\or Novembre\or Dicembre\fi
  \space \the\year}}{}
\@ifundefined{@TesiDiLaurea}{\def\@TesiDiLaurea{Tesi di Laurea}}{}
\@ifundefined{@laureaname}{\def\@laureaname{Corso di Laurea in }}{}
\@ifundefined{@InName}{\def\@InName{in}}{}
\newbool{topTPTlogos} \booltrue{topTPTlogos}
%    \end{macrocode}
%
% Here the real differences take place; the environments
% |frontespizio| and |frontespizio*| are still usable, but
% we discourage them. Even the commands |\frontespizio| and
% |\frontespizio*| are still usable but for even more reasons
% we discourage their use; they reman available for a sort
% of backwards compatibility.
%
% A new environment |ThesisTitlePage| is defined that actually
% resorts to the core command that stands behind the above
% discouraged environments and commands. This new environment
% uses the |xparse| commands to define it; therefore the
% asterisk that differentiates the four old environments and
% commands is not any more part of the environment name, but,
% if used, must be the first token after the environment
% opening statement, and should not be repeated with the
% closing statement; therefore
%\begin{verbatim}
%% Correct usage
%\begin{ThesisTitlePage}*
%...
%\end{ThesisTitlePage}
%\end{verbatim}
% is correct, while
%\begin{verbatim}
%% Wrong usage
%\begin{ThesisTitlePage*}
%...
%\end{ThesisTitlePage*}
%\end{verbatim}
% is wrong.
% This choice was made because its programming is simpler,
% but in view of the end user it is simpler to use: for
% example, if the user does not use the asterisk for typesetting
% the logos at the top of the page, and in a second time
% s/he decides that the title page should be typeset with
% the logos in the title page lower part, s/he has to add
% just one asterisk to the opening statement avoiding the
% possibility to forget its addition to the closing statement.
%    \begin{macrocode}
\newenvironment{frontespizio*}{\boolfalse{topTPTlogos}}{\fr@ntespizio}

\newenvironment{frontespizio}{%
    \ifdefstring{\@currenvir}{frontespizio}
           {\booltrue{topTPTlogos}}{\TPTmaybestar}
 }{%
     \fr@ntespizio
 }

\DeclareDocumentEnvironment{ThesisTitlePage}{s}{%
\IfBooleanTF{#1}{\boolfalse{topTPTlogos}}{\booltrue{topTPTlogos}}%
\InputIfFileExists{\jobname.cfg}{}{%
   \PackageInfo{toptesi-magistrale}{No configuration file found}}
}{\fr@ntespizio}
%    \end{macrocode}
% The following code is substantially identical to that of the
% |topfront| package; just the statements concerning the
% bachelor degree and the Ph.D.~degree theses have been deleted.
%    \begin{macrocode}
\newbool{AteneoInHead}\boolfalse{AteneoInHead}
\newcommand\TPTmaybestar[1]{\def\@tempA{#1}%
\ifdefstring{\@tempA}{*}%
  {\boolfalse{topTPTlogos}\booltrue{AteneoInHead}\fr@ntespizio}
  {\booltrue{topTPTlogos}\fr@ntespizio\@tempA}
}
\def\fr@ntespizio{%
\begingroup\par
   \oddsidemargin=\dimexpr(\oddsidemargin+\evensidemargin)/2\relax
   \evensidemargin \oddsidemargin
\null
\setcounter{page}{1}%
\normalfont
\ifclassica
   \boolfalse{topTPTlogos}
   \thispagestyle{classica}
   \ifcsvoid{@ateneo}{\def\@ateneo{Manca il nome dell'ateneo}
   }{}
\else
   \thispagestyle{titlepage}
\fi
\ifcsvoid{@ateneo}{}{\booltrue{AteneoInHead}}
\ifcsvoid{@ateneo}{%
    \ifbool{topTPTlogos}
    {}{\booltrue{AteneoInHead}\def\@ateneo{Manca il nome dell'ateneo}}%
}{%
    \booltrue{AteneoInHead}%
}

   \ifbool{AteneoInHead}{}{%
      {{\centering\LARGE \@ateneo\par}}
   }
   \ifcsvoid{@nomeat}{}
      {\ifbool{topTPTlogos}{\vspace*{\dimexpr \headsep+2.5ex}}%
      {\vspace*{-3ex}}%
  {\centering\@nomeat\par}\vfill}

  \begin{center}
  {\rmfamily\mdseries
  \ifdottorato
        \large \@phdschool\par\medskip
  \else
        \ifcsvoid{@faculty}{}{%
         \LARGE\ifx\@facnumber\empty\else\@facnumber\space\fi
         \@faculty\unskip\xspace\@facname\par\medskip
        }
  \fi
  }%
  \ifcsvoid{@corso}{}{{\large
        \ifdottorato
             \@PhDname\unskip\xspace
                  \@corso\ifx\@ciclo\empty\else~--~\@ciclo\fi
        \else
              \@laureaname\unskip\xspace\@corso
        \fi
        \par}}
  \end{center}
  \vspace{\stretch{0.2}}
  \begin{center}
    \LARGE\@TesiDiLaurea%
    \unless\ifx\empty\@materia
        \\\@InName\\\@materia
    \fi
  \end{center}
  \vspace{\stretch{0.2}}
  \begin{center}
    {\huge\bfseries \baselineskip=0.95em plus 1pt
            \@titolo \par}
  \end{center}
  \unless\ifx\@subtitle\empty
        \begin{center}%
          \large\textrm{\@subtitle}\par
        \end{center}%
  \fi
  \ifclassica
        \ifnum\value{tomo}>\z@
              \par\bigskip
              \noindent\makebox[\textwidth]{%
              \large\textbf{%
              \ifcase\c@tomo%
                    \or \PrimoTomo%
                    \or \SecondoTomo%
                    \or \TerzoTomo%
                    \or \QuartoTomo%
                    \else
                        \PackageWarning{toptesi}{%
                        Counter tomo equals \the\c@tomo\MessageBreak
                        We never considered a thesis might get
                        divided in more than four volumes}%
                    \fi}}%
        \fi
        \vspace{1em}
  \fi
  \par
 \unless\iftopTPTlogos
  {\vfill\centering \printloghi\par}\fi
  \vfill
  \iflanguage{italian}{%
    \iffemminile
          \def\@nomecandidato{\Candidata}%
    \else
          \def\@nomecandidato{\Candidato}%
    \fi
    \ifcsvoid{@secondauthor}{}{%
          \iffemminile
                \def\@nomecandidato{\Candidate}%
          \else
                \def\@nomecandidato{\Candidati}%
          \fi}
   }{}%
  \unless\iftriennale
     \def\BoxRelatori{%
     
        \begin{tabular}[t]{l}%
        \ifclassica
           \hbox{\textbf{\Relatore}}%
         \else
           \hbox{\large\bfseries
                 \ifcsvoid{@secondadviser}{\Relatore}{\Relatori}}
         \fi\\[.6ex]
        \hbox{\large\textrm{\protect\@principaladviser}}%
        \ifx\@secondadviser\empty \else
           \ifclassica
              \ifx\@thirdadviser\empty
                 \ifx\@secondadviser\empty\else
                       \\[1.5ex]\textbf{\Correlatore:}%
                 \fi
              \else
                 \\[1.5ex]\textbf{\Correlatori:}%
              \fi
           \fi
           \\[.6ex]\hbox{{\large\textrm{\protect\@secondadviser}}}%
            \fi
            \ifx\@thirdadviser\empty \else
               \\[.6ex]
               \hbox{{\large\textrm{\protect\@thirdadviser}}}%
            \fi
            \end{tabular}%
         }%
      \def\print@secondocandidato{\\\relax
                  \hbox{\large\tabular{@{}l@{}}%
                  \@secondauthor\endtabular}}
      \def\print@terzocandidato{\\\relax
                  \hbox{\large\tabular{@{}l@{}}%
                  \@thirdauthor\endtabular}}
      \def\BoxCandidati{%
         \begin{tabular}[t]{l}%
            \hbox{\unless\ifclassica\large\fi
               \textbf{\protect\@nomecandidato}}\\[.6ex]
               \hbox{\large\tabular{@{}l@{}}\@author\endtabular}%
               \ifcsvoid{@secondauthor}{}{\print@secondocandidato}%
               \ifcsvoid{@thirdauthor}{}{\print@terzocandidato}%
         \end{tabular}%
      }%
         \unless\ifdottorato
           \unless\ifclassica
             \unless\ifevenboxes
             \begin{flushleft}%
                \BoxRelatori
             \end{flushleft}\par\vspace*{-1.5\baselineskip}
             \begin{flushright}%
                \BoxCandidati
             \end{flushright}%
             \else
                \makebox[\textwidth]{\BoxRelatori\hfill\BoxCandidati}
             \fi
           \else
               \noindent
               \makebox[\textwidth]{%
               \BoxRelatori\hfill\BoxCandidati}\par
           \fi
         \fi
  \fi
   \ifcsvoid{@tutoreaziendale}{}{%
      \vfill\vfill
      {\centering \textbf{\@tutoreaziendalename}\\[.6ex]
         \@tutoreaziendale\par}%
      }
   \par\clearpage
%    \end{macrocode}
%
% It is useful to remind that the legal page in general is not
% compulsory, but it is becoming frequently requested by more
% and more universities. The user can fill up the
% |\retrofrontespizio| argument with as many paragraphs s/he
% is subject in accordance with his/her university requirements.
% In general there is a statement claiming the intellectual
% property if the thesis contents; very often it makes reference
% to the Creative Commons licence that is internationally well
% known; other international licences may me used. At the same
% time very often a statement is needed where the thesis author
% assume the necessary liability for any violation of other
% people intellectual property and/or copyright. Plagiarism and
% lack od attribution of other people rights are the most common
% illegal actions done by students writing their theses, and it
% is very important that the student responsibility is underlined
% by asking him/her to fulfil the legal obligations.
%    \begin{macrocode}
\ifcsvoid{@retrofrontespizio}{}%
      {\null\vfill\thispagestyle{empty}\@retrofrontespizio\par\clearpage}%
\endgroup}
%    \end{macrocode}
%\iffalse
%</topmag>
%\fi
%
%
%\iffalse
%<*topdoc>^^A               DOTTORALE
%\fi
% \subsection{The \texttt{toptesi-dottorale} module}
% What follows is mainly a replica of the |topfront| module
% where useless commands have been eliminated; ``useless''
% in view of creating a title page for a Ph.D.~thesis.
%
% First we load the necessary packages.
%    \begin{macrocode}
\@ifpackageloaded{etoolbox}{}{\RequirePackage{etoolbox}}
\@ifpackageloaded{xspace}{}{\RequirePackage{xspace}}
\@ifpackageloaded{xparse}{}{\RequirePackage{xparse}}
%    \end{macrocode}
% The we define, if they are not already defined some boolean
% switches related to the possible options. The above packages,
% and the following switches are necessary in case thei module
% is used by itself with a different class. The module accepts
% some options, that are declared just after these preliminary
% definitions.
%    \begin{macrocode}
\ifcsundef{ifclassica}{\newif\ifclassica \classicafalse}{}
\ifcsundef{if@utoretitolo}{\newif\if@utoretitolo\@utoretitolofalse}{}
\ifcsundef{if@ldstyle}{\newif\if@ldstyle\@ldstylefalse}{}
\ifcsundef{if@xivpt}{\newif\if@xivpt \@xivptfalse}{}
\ifcsundef{if@binding}{\newif\if@binding\@bindingfalse}{}
\ifcsundef{ifdottorale}{\newif\ifdottorale\dottoralefalse}{}
\ifcsundef{ifevenboxes}{\newif\ifevenboxes \evenboxesfalse}{}
%    \end{macrocode}
% Using this module implies that some switches used in the
% following commands should be set to false.
%    \begin{macrocode}
\ifcsundef{ifTOPfront}{\newif\ifTOPfront \TOPfrontfalse}{}
\ifcsundef{ifT@Pfrontespizio}{\newif\ifT@Pfrontespiziofalse}{}
\ifcsundef{c@tomo}{\newcounter{tomo}\setcounter{tomo}{0}}{}
\newif\iffemminile\newif\ifplurale
\newif\iftriennale\triennalefalse
\newif\ifdottorato \dottoratotrue

%    \end{macrocode}
% Here are the option declarations and their processing. Notice
% that the |14pt| option implies the the calling class loads
% some package that can set a normal size to 14pt or 14.4pt,
% otherwise this option may produce an error or a warning;
% in the best case it does not do anything. The |toptesi|
% class loads such a suitable package and allows to set any
% integer or fractional size; this package is |scrextend| and
% the user who wants to use a normal size larger than 12pt
% or smaller than 10pt, can resort to that package; possibly
% that is not the only package suited for such acrobatic
% settings, but the important point is to avoid ``acrobatics'';
% scientific books, reports, theses, and other such documents
% are supposed to be read by adults, not by children while
% normal sizes from 14pt on are more suited for elementary
% school pupils. On the opposite, normal sizes smaller than
% 10pt are difficult to read and should be avoided for
% documents to be read continuously; It is possible to use
% such small font sizes in foot notes, or in dictionaries,
% but both footnotes and dictionary entries do not generally
% occupy dozens of pages.
%    \begin{macrocode}
\DeclareOption{evenboxes}{\evenboxestrue}
\DeclareOption{cucitura}{\@bindingtrue}
\DeclareOption{14pt}{\@xivpttrue}
\DeclareOption{classica}{\classicatrue}
\DeclareOption{autoretitolo}{\ifclassica\@utoretitolotrue\fi}
\DeclareOption{oldstyle}{\ifclassica\@ldstyletrue\fi}
\ProcessOptions*\relax
%    \end{macrocode}
%
% We now define two specific page styles, one for ``normal''
% title pages, and one for theses typeset wit the |classica|
% option in force: they are |\ps@titlepge| and |\ps@classica|.
%    \begin{macrocode}
\def\headstrut{\vrule \@depth4\p@ \@height\z@ \@width\z@}
\def\ps@titlepage{\let\@mkboth\@gobbletwo
   \def\@oddfoot{\vbox to 0.05\paperheight{\vss
      \hbox to\hsize{\hfil{\Large{\@submitdate}}\hfil}}}%
   \let\@evenfoot\@oddfoot
   \def\@oddhead{%
      \vbox to\headheight{\vss\iftopTPTlogos
         \hbox to\textwidth{%
           \headstrut\hfil
           \raisebox{3\baselineskip}{\usebox\TPT@logobox}\hfil\null%
         }\fi
         \ifcsvoid{@ateneo}{}{\vskip\smallskipamount
         \hbox to\textwidth{\hss\LARGE\MakeUppercase{\@ateneo}\hss}}
         \vss
      }%
   }%
   \let\@evenhead\@oddhead
   \def\chaptermark##1{}\def\sectionmark##1{}%
}

  \def\ps@classica{\let\@mkboth\markboth
  \def\@oddhead{\vbox{%
  \setbox256=\hbox{\LARGE\MakeUppercase{\@ateneo}}
  \unless\ifdim\wd256<\textwidth
     \setbox256=\hbox{\resizebox{\textwidth}{!}{\box256}}%
   \fi
  \hbox to \hsize{\hss\box256\hss}%
  \ifclassica
        \hbox to \hsize{\hfil\vrule\@width\z@
    \@height2ex\vrule\@height1.4\p@\@depth-\p@\@width50mm\hfil}%
  \fi
  }}%
  \def\@oddfoot{\vbox to \dimexpr\paperheight/20\relax{\vss
 \ifclassica
    \hbox to \hsize{\hfil\raisebox{-.3ex}[\z@][\z@]{%
    \vrule\@height-2.6\p@\@depth3\p@\@width
        \dimexpr\textwidth/3}\hfil}%
  \fi
  \hbox to\hsize{\hfill{\Large{\@submitdate}}\hfill}}%
  }%
  \let\@evenhead\@oddhead
  \let\@evenfoot\@oddfoot
  }%
%    \end{macrocode}
%
% We provide also the variant settings for when the |classica|
% option is in force toghether with the |autoretitolo| one.
%    \begin{macrocode}
\if@utoretitolo
   \if@twoside
%
       \renewcommand*{\ps@headings}{\let\@mkboth\markboth%
       \def\@oddfoot{\null \hfill \textbf{\lapagina} \hfill \null}%
       \let\@evenfoot\@oddfoot
       \def\@evenhead{%
           \setbox\@intesta\hbox{\ifPDFTeX\latintext\fi
           \footnotesize\strut\textsc{%
           \@@author\ifcsvoid{@secondauthor}{\ et al.}: \@stitle}%
           }%
       \ifdim\wd\@intesta>\textwidth\headwrn{\titolo}\fi
       \underline{\makebox[\textwidth]{\box\@intesta}}}%
       \def\@oddhead{\ifPDFTeX\latintext\fi
           \setbox\@intesta\hbox{%
               \footnotesize\strut\textsl{\rightmark}}%
       \ifdim\wd\@intesta>\textwidth \headWarn{\section}\fi%
       \underline{\makebox[\textwidth]{\box\@intesta}}}%
       \def\chaptermark##1{\markright{\thechapter\ -- ##1}}%
       \def\sectionmark##1{}}%
   \else
       \renewcommand*{\ps@headings}{\let\@mkboth\markboth
           \def\@oddfoot{\null \hfill \textbf{\lapagina}\hfill \null}%
           \let\@evenfoot\empty\let\@evenhead\empty
           \def\@oddhead{{\setbox\z@\hbox{\ifPDFTeX\latintext\fi
           \footnotesize\textsc{%
           \@@author\ifcsvoid{@secondauthor}{}{\ et al.}: \@stitle}}%
           \setbox\tw@\hbox{\ifPDFTeX\latintext\fi
           \footnotesize\textsl{\rightmark}}%
           \dimen@=\dimexpr2em + \wd\z@ + \wd\tw@\relax
           \ifdim\dimen@<\textwidth \relax
           \else
             \setbox\z@\vbox{\hsize.48\textwidth\parindent\z@
             \raggedright
             \ifPDFTeX\latintext\fi\footnotesize\textsc{%
             \@@author\ifcsvoid{@secondauthor}{}{\ et al.}: \@stitle
             }}%
             \setbox\tw@\vbox{\hsize.48\textwidth\parindent\z@
             \raggedleft
             \ifPDFTeX\latintext\fi\footnotesize
                 \textsl{\rightmark}}%
             \fi
           \setbox\@intesta\vbox to\z@{%
                \vss\hbox to\textwidth{\strut\box\z@\hfill\box\tw@}}%
           \underline{\box\@intesta}}}%
           \def\chaptermark##1{\markright{\thechapter\ -- ##1}}}
   \fi
\fi
%    \end{macrocode}
% Now we define the title setting  commands for both
% title (|\titolo|) and subtitle (|\sottotitolo|) and
% all the other command for setting the fixed strings.
%    \begin{macrocode}
\let\@stitle\empty

\newcommand*{\titolo}[2][]{%
\def\@tempA{#1}\ifdefempty{\@tempA}%
 {\gdef\@stitle{#2}}{\gdef\@stitle{#1}}%
 \gdef\@titolo{#2}%
}%

\def\sottotitolo#1{\gdef\@subtitle{#1}}
\let\@materia\empty
\def\Materia#1{\def\@materia{#1}}\let\materia\Materia
\newif\ifDirettore \Direttorefalse
\def\tutore#1{\gdef\@tutore{#1}}
\def\direttore{\Direttoretrue\relatore}%
\def\coordinatore{\Direttorefalse\relatore}%
\def\QualificaDirettore#1{\gdef\@PhDdirector{#1}}%
\def\relatore#1{\gdef\@principaladviser{#1}}
\def\secondorelatore#1{\gdef\@secondadviser{#1}}
\def\terzorelatore#1{\gdef\@thirdadviser{#1}}
\def\esamedidottorato#1{\getseduta#1 !}
\def\getseduta#1 #2!{%
    \def\@tempA{#2}%
    \ifx\@tempA\empty
        \def\@submitdate{#1}%
    \else
        \unless\ifclassica
            \def\@submitdate{#1 #2}%
        \else
            \s@dutaclassica#1 #2!%
        \fi
    \fi
}%
\def\s@dutaclassica#1 #2!{%
  \if@ldstyle
    \s@paranumeri#2-!%
    \ifx\2\empty
      \edef\@submitdate{\noexpand#1 \noexpand\oldstylenums{#2}}%
    \else
      \s@paranumeri#2!%
      \edef\@submitdate{\noexpand#1
      \noexpand\oldstylenums{\1} -- \noexpand\oldstylenums{\2}}%
    \fi
      \let\1\undefined
      \let\2\undefined
  \else
    \def\@submitdate{#1 #2}%
  \fi
}
\def\s@paranumeri#1-#2!{\def\1{#1}\def\2{#2}}%

\newcommand*\NomeAnnoAccademico[1]{\gdef\AnnoAccademico{#1}}
\@ifundefined{AnnoAccademico}{\gdef\AnnoAccademico{Anno accademico}}{}
\def\annoaccademico#1{\ifclassica
  \def\@submitdate{{\large\textsc{\AnnoAccademico}} {\Large\s@tanno#1!}}
  \else
  \PackageWarning{toptesi}{\string\annoaccademico\space
            is usable only when the\MessageBreak
   `classica' option is in force}%
  \def\@submitdate{\AnnoAccademico\ #1 --- Needs the `classica' option}
  \fi}
\def\s@tanno#1-#2!{\oldstylenums{#1}\EnDash\oldstylenums{#2}}

\def\EnDash{{\settowidth{\dimen@}{\large\scshape I}%
\setbox\tw@\hbox{2}\dimen\tw@.5\ht\tw@\advance\dimen\tw@-.5\dp\tw@
\dimen4\dimen\tw@\advance\dimen4by.0385ex\relax
\advance\dimen\tw@-.0385ex\relax
\makebox[1.5\dimen@]{%
    \vrule\@width\dimen@\@height\dimen4\@depth-\dimen\tw@}}}
\def\ciclodidottorato#1{\gdef\@ciclo{#1 \@cyclename}}%
\def\corsodidottorato#1{\global\dottoratotrue
  \global\gdef\@corso{#1}}
\def\scuoladidottorato#1{\global\dottoratotrue
  \gdef\@phdschool{#1}}
\def\ateneo#1{\gdef\@ateneo{#1}}
\def\nomeateneo#1{\gdef\@nomeat{\expandafter\uppercase{\expandafter #1}}}
\newcommand\facolta[2][]{\gdef\@facname{#2}\gdef\@facnumber{#1}}
\let\struttura\facolta
%    \end{macrocode}
% Here we define how to handle one or more logos; even in
% PH.D.%schools there might be situations where the Ph.D.~program
% has been carried on in several universities; it might be
% required to set the logos where the research program
% leading to the Ph.D. has been developed. The user commands
% are |\logosede| and |\printloghi|; the latter does what
% its name suggests, i.e. it prints the string of logos;
% the former command accepts one logo file name, and queues
% it into a logo list; this command may be used multiple times;
% of course if there are several logos, their ``string''
% may become larger than the text width, and in this case
% |\printloghi| shrinks the whole string (maintaining the
% aspect ratio) so as to fit it within the measure. If the
% logos are too many and/or they are too wide, this
% shrinking may reduce them too much; there is no work
% around this problem except avoiding using too many and/or
% too wide logos.
%    \begin{macrocode}
\newlength{\TPT@logospace}\TPT@logospace=3em\relax
\newsavebox{\TPT@logobox}
\newdimen\TPT@logoheight
\newcommand*\setlogodistance[1]{\TPT@logospace=#1}
\providecommand*{\@logosede}{}

\ifcsundef{T@Tlogoheight}{%\
    \newlength\T@Tlogoheight
    \setlength\T@Tlogoheight{0.8\dimexpr(\paperwidth-\textwidth)/2}}{}
    
\newcommand\logosede[2][\T@Tlogoheight]{\def\@logosede{#2}\TPT@logoheight=#1\relax
\ifcsvoid{@logosede}{\sbox{\TPT@logobox}{}}{\begin{lrbox}{\TPT@logobox}%
\expandafter\fillup@TCP@logobox\@logosede,!}}

\def\fillup@TCP@logobox#1,#2!{%
\ifblank{#1}{\end{lrbox}\ifdim\wd\TPT@logobox>\textwidth
\sbox\TPT@logobox{\resizebox{\textwidth}{!}{\box\TPT@logobox}}\fi}%
{\def\@logosede{#2}%
\includegraphics[height=\TPT@logoheight]{#1}\hskip\TPT@logospace
   \expandafter\fillup@TCP@logobox\@logosede,!}}

\newcommand\printloghi{\unless\ifvoid\TPT@logobox\usebox{\TPT@logobox}\fi}
%    \end{macrocode}
%
% We keep defining macros to set fixed string or variable data.
%    \begin{macrocode}
\def\tutoreaziendale#1{\gdef\@tutoreaziendale{#1}}
\newcommand\retrofrontespizio[1]{\long\gdef\@retrofrontespizio{#1}}
\newcommand\FacoltaDi[1]{\gdef\@faculty{#1}}
\let\StrutturaDidattica\FacoltaDi
\newcommand\DottoratoIn[1]{\gdef\@PhDname{#1}}
\newcommand\CorsoDiLaureaIn[1]{\gdef\@laureaname{#1}}
\newcommand\TesiDiLaurea[1]{\gdef\@TesiDiLaurea{#1}}
\newcommand\NomeMonografia[1]{\gdef\@monografia{#1}}
\newcommand\NomeDissertazione[1]{\gdef\@dissertazione{#1}}
\newcommand\InName[1]{\gdef\@InName{#1}}
\newcommand\CandidateName[1]{\gdef\@nomecandidato{#1}}
\newcommand\AdvisorName[1]{\gdef\Relatore{#1}\gdef\Relatori{#1}}
\newcommand\CoAdvisorName[1]{\gdef\Correlatore{#1}\gdef\Correlatori{#1}}
\newcommand\TutorName[1]{\gdef\Tutore{#1}}
\newcommand\NomeTutoreAziendale[1]{\gdef\@tutoreaziendalename{#1}}
\newcommand\CycleName[1]{\gdef\@cyclename{#1}}
\newcommand\NomePrimoTomo[1]{\gdef\PrimoTomo{#1}}
\newcommand\NomeSecondoTomo[1]{\gdef\SecondoTomo{#1}}
\newcommand\NomeTerzoTomo[1]{\gdef\TerzoTomo{#1}}
\newcommand\NomeQuartoTomo[1]{\gdef\QuartoTomo{#1}}
\newcommand\IDlabel{\\matricola:\xspace}
\IfFileExists{\jobname.cfg}{\input{\jobname.cfg}}%
    {\IfFileExists{toptesi.cfg}{%%
%% This is file `toptesi.cfg',
%% generated with the docstrip utility.
%%
%% The original source files were:
%%
%% toptesi.dtx  (with options: `topcfg')
%%   ______________________________________________________
%%   The TOPtesi bundle
%%   Copyright (C) 2015-2017 Claudio Beccari
%%   All rights reserved
%% 
%%   License information appended
%% 

%%
%%================================================================
%% This file is the only file of the TOPtesi bundle that the user
%% can modify without restrictions in order to customise the
%% contents of this configuration file to his/her needs. The user
%% can add or remove lines, comment or uncomment lines, change the
%% arguments to macros, add definitions and so on.
%% Use  this file by copying the relevant section to another file
%% to be named as the thesis main file and with extension .cfg.
%% This bundle will try to read "\jobname.cfg"; if this file does
%% not exist, the bundle files don't do anything. This implies that
%% the provided file toptesi.cfg is to be used as a model, not to
%% be used directly.
%%================================================================
%%
\ifmagistrale
    \ateneo{Politecnico di Torino}
    \facolta{}% nessun nome di default/ no default name for PoliTO
    \FacoltaDi{}% nessun prefisso per la facoltà/no default faculty label for PoliTO
    \CorsoDiLaureaIn{Corso di Laurea in}
    \TesiDiLaurea{Tesi di Laurea Magistrale}
    \InName{in}
    %%\retrofrontespizio{Questo testo è soggetto alla
    %% Creative Commons Licence}
\fi
\ifsecondaria

\fi
\ifScuDo
    \PhDschoolLogo{Logo-Scudo}
    \Ndissertation{Docoral Dissertation}
    \Ndoctoralprogram{Doctoral Program in\xspace}
    \NSupervisor{Supervisor}{Supervisors}
    \SupervisorNumber{1}
    \Nexaminationcommittee{Doctoral Examination Committee:}
    \Nlocation{Politecnico di Torino, Turin, Italy}
    \Disclaimer{\noindent I hereby declare that the
     contents and organisation of this dissertation
     constitute my own original work and does not
     compromise in any way the rights of third parties,
     including those relating to the security of personal
     data.}
     \CClicence{This thesis is licensed under a Creative Commons
     License, Attribution - Noncommercial- NoDerivative Works 4.0
     International: see \url{www.creativecommons.org}.
     The text may be reproduced for non-commercial purposes,
     provided that credit is given to the original author.}
\fi
%% 
%% Copyright 2005-2017 Claudio Beccari
%% 
%% Distributable under the LaTeX Project Public License,
%% version 1.3c or higher (your choice). The latest version of
%% this license is at: http://www.latex-project.org/lppl.txt
%% 
%% This work is "author-maintained"
%% 
%% This work consists of this file toptesi.dtx, a README file
%% the manifest.txt file, and the derived files:
%%     toptesi.cls,
%%     toptesi.sty,
%%     topfront.sty,
%%     topcommand.sty,
%%     toptesi-scudo.sty,
%%     toptesi.cfg,
%%     toptesi-monografia.sty,
%%     toptesi-sss.sty,
%%     toptesi-magistrale.sty,
%%     toptesi-dottorale.sty,
%% and the English documentation toptesi.pdf.
%% 
%% Furthermore the bundle contains the documentation source
%% file toptesi-it.tex and the derived file
%% toptesi-it.pdf.
%% 
%% The toptesi-example.tex source file is just an example
%% that shows how to use the various commands; by commenting
%% or uncommenting certain source lines it is possible to
%% typeset different kind of theses and their front pages.
%% 
%% The topfront-example.tex source file is an example of how
%% to produce just the title page with TOPtesi and the external
%% package frontespizio.
%% 
%% The toptesi-scudo-example.tex source file is an example
%% that shows the particular features available with TOPtesi,
%% when a doctoral thesis is produced for the Scuola di
%% Dottorato (ScuDo) of Politecnico di Torino.
%% 
%% The toptesi.cfg file is a sample of a local configuration
%% file that can be copied to another file and its copy freely
%% edited and customised.
%% 
%% The other toptesi-*.sty files are extension modules for
%% typesetting the specific thesis kinds described by the
%% suffix that replaces the asterisk.
%% 
%% By running pdflatex on toptesi.dtx the user gets the class,
%% sty and cfg files and the English documentation file in
%% PDF format.
%% 
%% The source file of the Italian documentation file
%% toptesi-it.pdf is another example of how to use TOPtesi.
%% This file toptesi-it.tex may be typeset with pdfLaTeX,
%% XeLaTeX, and LuaLateX; see the first comment lines of
%% the file for how-to information. In spite of being mainly
%% written in Italian, it contains several sections in English
%% for the benefit of foreign students attending the Doctoral
%% School of Turin.
%%
%% End of file `toptesi.cfg'.
}{}}
\@ifundefined{@cyclename}{\def\@cyclename{ciclo}}{}
\@ifundefined{@titolo}{\def\@titolo{}}{}
\NewDocumentCommand\candidato{m o}{\gdef\@@author{#1}%
\IfNoValueTF{#2}{\gdef\@author{#1}}%
               {\gdef\@author{#1\IDlabel#2}}
\femminilefalse\pluralefalse}
\NewDocumentCommand\candidata{m o}{\gdef\@@author{#1}%
\IfNoValueTF{#2}{\gdef\@author{#1}}%
               {\gdef\@author{#1\IDlabel#2}}
\femminiletrue\pluralefalse}
\providecommand\@secondauthor{}
\NewDocumentCommand\secondocandidato{m o}{%
\IfNoValueTF{#2}{\gdef\@secondauthor{#1}}%
                {\gdef\@secondauthor{#1\IDlabel#2}}%
\femminilefalse\pluraletrue}
\NewDocumentCommand\secondacandidata{m o}{%
\IfNoValueTF{#2}{\gdef\@secondauthor{#1}}%
                {\gdef\@secondauthor{#1\IDlabel#2}}%
\pluraletrue}
\providecommand\@thirdauthor{}
\NewDocumentCommand\terzocandidato{m o}{%
\IfNoValueTF{#2}{\gdef\@thirdauthor{#1}}%
                {\gdef\@thirdauthor{#1\IDlabel#2}}%
\femminilefalse\pluraletrue}
\NewDocumentCommand\terzacandidata{m o}{%
\IfNoValueTF{#2}{\gdef\@thirdauthor{#1}}%
                {\gdef\@thirdauthor{#1\IDlabel#2}}%
\pluraletrue}

\@ifundefined{@principaladviser}{\def\@principaladviser{}}{}
\@ifundefined{@secondadviser}{\def\@secondadviser{}}{}
\@ifundefined{@thirdadviser}{\def\@thirdadviser{}}{}
\ifcsundef{@PhDdirector}{%
   \ifDirettore\def\@PhDdirector{Direttore del corso di dottorato}\else
    \def\@PhDdirector{Coordinatore del corso di dottorato}\fi}{}
\@ifundefined{@tutore}{\def\@tutore{}}{}
\@ifundefined{@nomerelatore}{\def\@nomerelatore{}}{}
\@ifundefined{@nomecandidato}{\def\@nomecandidato{}}{}
\@ifundefined{Candidato}{\def\Candidato{Candidato}}{}
\@ifundefined{Candidata}{\def\Candidata{Candidata}}{}
\@ifundefined{Candidati}{\def\Candidati{Candidati}}{}
\@ifundefined{Candidate}{\def\Candidate{Candidate}}{}
\@ifundefined{Relatore}{\def\Relatore{Relatore}}{}
\@ifundefined{Relatori}{\def\Relatori{Relatori}}{}
\@ifundefined{Correlatore}{\def\Correlatore{Correlatore}}{}
\@ifundefined{Correlatori}{\def\Correlatori{Correlatori}}{}
\@ifundefined{Tutore}{\def\Tutore{Tutore}}{}
\@ifundefined{@tutoreaziendale}{\def\@tutoreaziendale{}}{}
\@ifundefined{@tutoreaziendalename}%
        {\def\@tutoreaziendalename{Supervisore Aziendale}}{}
\@ifundefined{@retrofrontespizio}{\def\@retrofrontespizio{}}{}
\@ifundefined{@subtitle}{\def\@subtitle{}}{}
\@ifundefined{@corso}{\def\@corso{}}{}
\@ifundefined{@ciclo}{\def\@ciclo{}}{}
\@ifundefined{@ateneo}{\def\@ateneo{POLITECNICO DI TORINO}}{}
\@ifundefined{@nomeat}{\def\@nomeat{}}{}% Nome proprio dell'ateneo
\@ifundefined{@facolta}{\def\@facname{}}{}
\@ifundefined{@facnumber}{\def\@facnumber{}}{}
\@ifundefined{@faculty}{\def\@faculty{}}{}
\@ifundefined{PrimoTomo}{\def\PrimoTomo{Tomo primo}}{}
\@ifundefined{SecondoTomo}{\def\SecondoTomo{Tomo secondo}}{}
\@ifundefined{TerzoTomo}{\def\TerzoTomo{Tomo terzo}}{}
\@ifundefined{QuartoTomo}{\def\QuartoTomo{Tomo quarto}}{}
\@ifundefined{@submitdate}{\def\@submitdate{\ifcase\the\month\or%
  Gennaio\or Febbraio\or Marzo\or Aprile\or Maggio\or Giugno\or
  Luglio\or Agosto\or Settembre\or Ottobre\or Novembre\or Dicembre\fi
  \space \the\year}}{}
\@ifundefined{@TesiDiLaurea}{\def\@TesiDiLaurea{Tesi di Laurea}}{}
\@ifundefined{@phdschool}{\def\@phdschool{SCUOLA DI DOTTORATO}}{}
\@ifundefined{@PhDname}{\def\@PhDname{Dottorato in }}{}
\@ifundefined{@laureaname}{\def\@laureaname{Corso di Laurea in }}{}
\@ifundefined{@dissertazione}{\def\@dissertazione{Tesi di Dottorato}}{}
\@ifundefined{@monografia}{\def\@monografia{Monografia di Laurea}}{}
\@ifundefined{@InName}{\def\@InName{in}}{}
%    \end{macrocode}
%
% We now start setting the |ThesisTitlePage| environment;
% we use the |xparse| functionality to define it; the
% environment accepts an optional asterisk as its first
% and only argument, that as a single token may be entered
% without brackets as the very first token after the
% environment opening statement. If the user wants to use
% this optional argument,  s/he must enter
% |\begin{ThesisTitlePage}*| with the asterisk after the
% closing brace. The presence of the asterisk sets the
% logo position to the title page lower half, while its
% absence sets such position to the title page top.
%
% The opening commands provide also to reading a configuration
% file, if it exists in the same folder as the thesis main
% file; such configuration file must be named \meta{thesis
% main file name}|.cfg|. The environment body, besides the
% possible data already got from the configuration file
% accepts other setting commands for the title page necessary
% and optional data; if some commands are repeated with
% different data, the last ones prevail on the previous ones.
% The actual title page  typesetting is done by the closing
% commands through the |\fr@ntespizio| command.
%
% Among the setting commands there might be also the legal
% page |\retrofrontespizio| command; it closes the recto
% title page, and on the verso, if the command argument
% is void, nothing is printed except the blank verso page,
% but if it contains something the legal notice is printed
% at the verso page bottom.
%    \begin{macrocode}
\newbool{topTPTlogos}\newbool{AteneoInHead}

\DeclareDocumentEnvironment{ThesisTitlePage}{s}{%
  \IfBooleanTF{#1}{\boolfalse{topTPTlogos}\booltrue{AteneoInHead}}%
               {\booltrue{topTPTlogos}\boolfalse{AteneoInHead}}%
  \IfFileExists{\jobname.cfg}{\input{\jobname.cfg}}{%
    \PackageWarning{toptesi-dottorale}{%
       No configuration file found\MessageBreak}}
}{%
  \fr@ntespizio  
}
%    \end{macrocode}
% Here we define the actual typesetting of the title page.
% The code is substantially that of the generic module
% |topfront|where we deleted the commands and the settings
% related to thesis kinds that have nothing to do with
% a generic doctoral thesis. 
%    \begin{macrocode}
\def\fr@ntespizio{%
\begingroup\par
   \oddsidemargin=\dimexpr(\oddsidemargin+\evensidemargin)/2\relax
   \evensidemargin \oddsidemargin
\null
\setcounter{page}{1}%
\normalfont
\ifclassica
   \boolfalse{topTPTlogos}
   \thispagestyle{classica}
   \ifcsvoid{@ateneo}{\def\@ateneo{Manca il nome dell'ateneo}
   }{}
\else
   \thispagestyle{titlepage}
\fi
\ifcsvoid{@ateneo}{}{\booltrue{AteneoInHead}}
\ifcsvoid{@ateneo}{%
    \ifbool{topTPTlogos}
    {}{\booltrue{AteneoInHead}\def\@ateneo{Manca il nome dell'ateneo}}%
}{%
    \booltrue{AteneoInHead}%
}
   \ifbool{AteneoInHead}{}{%
      {{\centering\LARGE \@ateneo\par}}
   }
   \ifcsvoid{@nomeat}{}
      {\ifbool{topTPTlogos}{\vspace*{\dimexpr \headsep+2.5ex}}{\vspace*{-3ex}}%
      {\centering\@nomeat\par}\vfill}

      \begin{center}
      {\rmfamily\mdseries
      \ifdottorato
            \large \@phdschool\par\medskip
      \else
            \ifcsvoid{@faculty}{}{%
             \LARGE\ifx\@facnumber\empty\else\@facnumber\space\fi
             \@faculty\unskip\xspace\@facname\par\medskip
            }
      \fi
      }%
      \ifcsvoid{@corso}{}{{\large
            \ifdottorato
                 \@PhDname\unskip\xspace
                      \@corso\ifx\@ciclo\empty\else~--~\@ciclo\fi
            \else
                  \@laureaname\unskip\xspace\@corso
            \fi
            \par}}
      \end{center}
      \vspace{\stretch{0.2}}
      \begin{center}
        \LARGE\@dissertazione%
        \unless\ifx\empty\@materia
            \\\@InName\\\@materia
        \fi
      \end{center}
      \vspace{\stretch{0.2}}
      \begin{center}
        {\huge\bfseries \baselineskip=0.95em plus 1pt
                \@titolo \par}
      \end{center}
      \unless\ifx\@subtitle\empty
            \begin{center}%
              \large\textrm{\@subtitle}\par
            \end{center}%
      \fi
      \ifclassica
            \ifnum\value{tomo}>\z@
                  \par\bigskip
                  \noindent\makebox[\textwidth]{%
                  \large\textbf{%
                  \ifcase\c@tomo%
                        \or \PrimoTomo%
                        \or \SecondoTomo%
                        \or \TerzoTomo%
                        \or \QuartoTomo%
                        \else
                            \PackageWarning{toptesi}{%
                            Counter tomo equals \the\c@tomo\MessageBreak
                            We never considered a thesis might get
                            divided in more than four volumes}%
                        \fi}}%
            \fi
            \vspace{1em}
      \fi
      \par
     \unless\iftopTPTlogos
      {\vfill\centering \printloghi\par}\fi
      \vfill
      \iftriennale
            \let\@nomerelatore\empty
      \else
            \ifdottorato
                  \edef\@nomerelatore{\@PhDdirector}%
            \else
                  \ifcsvoid{@principaladviser}{}{%
                        \def\@nomerelatore{\Relatore}}
                  \unless\ifclassica
                        \ifcsvoid{@secondadviser}{}{%
                              \def\@nomerelatore{\Relatori}}%
                  \fi
            \fi
      \fi
      \ifdottorato
            \let\@nomecandidato\empty
      \else
        \iflanguage{italian}{%
            \iffemminile
                  \def\@nomecandidato{\Candidata}%
            \else
                  \def\@nomecandidato{\Candidato}%
            \fi
            \ifcsvoid{@secondauthor}{}{%
                  \iffemminile
                        \def\@nomecandidato{\Candidate}%
                  \else
                        \def\@nomecandidato{\Candidati}%
                  \fi}
         }{}%
     \fi
      \iftriennale
            \begin{center}%
                  \large\mdseries\textsc{\@author}
            \end{center}%
      \else
         \def\BoxRelatori{%
            \begin{tabular}[t]{l}%
            \hbox{\ifclassica\else\large\fi
                  \textbf{\protect\@nomerelatore}}\\[.6ex]
            \hbox{\large\textrm{\protect\@principaladviser}}%
            \ifx\@secondadviser\empty \else
               \ifclassica
                  \ifx\@thirdadviser\empty
                     \ifx\@secondadviser\empty\else
                           \\[1.5ex]\textbf{\Correlatore:}%
                     \fi
                  \else
                     \\[1.5ex]\textbf{\Correlatori:}%
                  \fi
               \fi
               \\[.6ex]\hbox{{\large\textrm{\protect\@secondadviser}}}%
            \fi
            \ifx\@thirdadviser\empty \else
               \\[.6ex] \hbox{{\large\textrm{\protect\@thirdadviser}}}%
            \fi
            \end{tabular}%
         }%
      \def\print@secondocandidato{\\\relax
                  \hbox{\large\tabular{@{}l@{}}\@secondauthor\endtabular}}
      \def\print@terzocandidato{\\\relax
                  \hbox{\large\tabular{@{}l@{}}\@thirdauthor\endtabular}}
      \def\BoxCandidati{%
         \begin{tabular}[t]{l}%
            \hbox{\unless\ifclassica\large\fi
               \textbf{\protect\@nomecandidato}}\\[.6ex]
               \hbox{\large\tabular{@{}l@{}}\@author\endtabular}%
               \ifcsvoid{@secondauthor}{}{\print@secondocandidato}%
               \ifcsvoid{@thirdauthor}{}{\print@terzocandidato}%
         \end{tabular}%
      }%
         \ifdottorato
            \begin{center}\large
               \textbf{\@author}\\[3em]
               {\normalsize
                 \begin {tabular*}{\hsize}{@{\extracolsep{\fill}}ccc}
                     \ifcsvoid{@tutore}{}{\textbf{\Tutore}}
                 &\relax&
                     \ifcsvoid{@principaladviser}{}{\textbf{\@nomerelatore}}
                 \\
                     \ifcsvoid{@tutore}{}{\@tutore}
                 &\relax&
                     \ifcsvoid{@principaladviser}{}{\@principaladviser}
                 \end{tabular*}
               }%
            \end{center}
         \else
           \unless\ifclassica
             \unless\ifevenboxes
             \begin{flushleft}%
                \BoxRelatori
             \end{flushleft}\par\vspace*{-1.5\baselineskip}
             \begin{flushright}%
                \BoxCandidati
             \end{flushright}%
             \else
                \makebox[\textwidth]{\BoxRelatori\hfill\BoxCandidati}
             \fi
             \else
               \noindent
               \makebox[\textwidth]{%
               \BoxRelatori\hfill\BoxCandidati}\par
             \fi
          \fi
      \fi
\ifcsvoid{@tutoreaziendale}{}{%
    \vfill\vfill
    {\centering \textbf{\@tutoreaziendalename}\\[.6ex]
    \@tutoreaziendale\par}}
      \par\clearpage
% Legal page
      \ifcsvoid{@retrofrontespizio}{}%
      {\null\vfill\thispagestyle{empty}\@retrofrontespizio\par\clearpage}%
\endgroup}
%    \end{macrocode}
%\iffalse
%</topdoc>
%\fi
%
% \subsection{The \texttt{toptesi-sss} module code}
% The module for typesetting high school final work
% title pages starts here. We avoid using a |.def| file and put
% all necessary commands in one file.
% 
% First we load the described packages using different loading
% schemes for \pdfLaTeX\ vs. \LuaLaTeX\ or \XeLaTeX.
%
%\iffalse
%<*topsss>^^A               SECONDARIA
%\fi
%    \begin{macrocode}
%% 
%%%% Math fonts and AMS extensions (fontspec is already loaded)
\secondariatrue
\ifPDFTeX
   \@ifpackageloaded{newtxmath}{}%
     {\@ifpackageloaded{newpxmath}{}%
       {%
         \RequirePackage{amsmath,amssymb,amsthm}%
       }%
     }%
\else
   \RequirePackage{amsmath,amsthm}
   \RequirePackage{unicode-math}
\fi
\RequirePackage{xcolor}   % Intelligent colors
\RequirePackage{xspace}   % Intelligent space
\RequirePackage{xparse}   % Command and environment advanced definitions
\RequirePackage{calc}     % Calculation macros
\RequirePackage{ifthen}   % Conditional statements
\RequirePackage{booktabs} % professional rules in tables
\RequirePackage{multirow} % It's more professional avoiding tables
                          % with cells spanning several rows
\RequirePackage[autostyle]{csquotes} % necessary for biblatex
\ifPDFTeX \RequirePackage{indentfirst}\fi
\raggedbottom
%    \end{macrocode}
%
% We initialise the internal macros to the default values; most
% of them are just empty strings.
%    \begin{macrocode}
\providecommand*\@SSSLogo{}
\providecommand*\@tiposcuola{}
\providecommand*\@indirizzomiur{}
\providecommand*\@opzionemiur{}
\providecommand*\@nomescuola{}
\providecommand*\@sedescuola{}
\providecommand*\@annoscolastico{}

\providecommand*\@Ntesina{Tesina di maturit\`a}
\providecommand*\@title{}
\providecommand*\@subtitle{}

\ifcsundef{ifemminile}{\newif\iffemminile\femminilefalse}{}
\providecommand*\@Ncandidato{\iffemminile Studentessa\else
                             Studente\fi}
\providecommand*\@author{}
\providecommand*\@Npresidente{Presidente di commissione}
\providecommand*\@presidente{}
\providecommand*\@numerocommissione{}

\newcommand*\SSSLogo[1]{\gdef\@SSSLogo{#1}}
\newcommand*\NomeTesina[1]{\gdef\@Ntesina{#1}}
\newcommand*\IndirizzoMiur[1]{\gdef\@indirizzomiur{#1}}
\newcommand*\OpzioneMiur[1]{\gdef\@opzionemiur{#1}}
\newcommand*\TipoScuola[1]{\gdef\@tiposcuola{#1}}
\newcommand*\SedeScuola[1]{\gdef\@sedescuola{#1}}
\newcommand*\NomeScuola[1]{\gdef\@nomescuola{#1}}
\newcommand*\AnnoScolastico[1]{\gdef\@annoscolastico{#1}}
\newcommand*\NomeCandidato[1]{\gdef\@Ncandidato{#1}}

\newcommand*\titolo[1]{\gdef\@title{#1}}
\newcommand*\sottotitolo[1]{\gdef\@subtitle{\ifblank{#1}{}%
   {\medskip\par{\Large #1\par}}%
   \vspace{\stretch{1}}}}
   
\NewDocumentCommand\studente{m o}{\gdef\@@author{#1}%
\IfNoValueTF{#2}{\gdef\@author{#1}}%
               {\gdef\@author{#1\IDlabel#2}}\femminilefalse}
\NewDocumentCommand\studentessa{m o}{\gdef\@@author{#1}%
\IfNoValueTF{#2}{\gdef\@author{#1}}%
               {\gdef\@author{#1\IDlabel#2}}\femminiletrue}
\newcommand*\IDlabel{\\\normalfont Classe\ }
\newcommand*\Presidente[1]{\gdef\@presidente{#1}}
\newcommand*\NumeroCommissione[1]{\gdef\@numerocommissione{#1}}
%    \end{macrocode}
% The following code certainly does not hurt, but it was
% necessary during the preliminary testing.
%    \begin{macrocode}
\newcommand*\ifBlank{\expandafter\ifblank\expandafter}
%    \end{macrocode}
% The environmnet |FrontespizioTesina| is being defined here.
% No optional arguments are requested, therefore it is not
% necessary to use the defining commands of package |xparse|.
% The environment starts by reading the configuration file
% |jobname.cfg| if it exists. Nothing is done if such file
% does not exist, not even an info message; after all the
% configuration file is optional.
%    \begin{macrocode}
\newenvironment{FrontespizioTesina}{%
\InputIfFileExists{\jobname.cfg}{\relax}{\relax}
}{%
%    \end{macrocode}
% After this preliminary test the |titlepage| environment is
% opened and the layout assembly starts. First the information
% on the school; if the school generic name has been omitted
% among the mandatory user commands, the string `Manca il tipo
% di scuola'  is printed in its place, so that the user is
% directly reminded of the mandatory nature of this piece of
% data. On the opposite if the Ministry of education data are
% missing, nothing is done. After these mandatory and optional
% data are printed a couple of horizontal lines are typed so
% as to mark the end of the title page heading.
%    \begin{macrocode}
\begin{titlepage}\parindent=0pt
\centering
\raisebox{0.025\textheight}[0pt][0pt]{\parbox[b]{\textwidth}{%
\centering
\ifBlank{\@tiposcuola}{Manca il tipo di scuola}{\@tiposcuola}%
\ifBlank{\@indirizzomiur}{\relax}{\space--\space\@indirizzomiur}%
\ifBlank{\@opzionemiur}{\relax}{\space--\space\@opzionemiur}\par
\vspace{-1ex}
\makebox[0.5\textwidth]{\rule{0.50\textwidth}{0.4\p@}}\par
\vspace{-1.5ex}
\makebox[0.25\textwidth]{\rule{0.25\textwidth}{0.8\p@}}\par
%    \end{macrocode}
% After this heading the proper name of the school is printed,
% possibly with its real address.
%    \begin{macrocode}
\medskip

\textsc{\@nomescuola}}}\par
\null\hfill\ifBlank{\@sedescuola}{\relax}{%
\raisebox{0.5\baselineskip}[0pt][0pt]{%
\parbox{0.4\textwidth}{\raggedleft\@sedescuola}}}\par
\vspace{\stretch{1}}
%    \end{macrocode}
%
% next the name of the document and its title and subtitle
% are set: the title  in |\Large| boldface font, while the subtitle 
% in normal size.
%    \begin{macrocode}
{\large \@Ntesina\par}
\vspace{\stretch{0.5}}

{\Large\textbf{\@title}\par}
{\vspace{0.5\baselineskip}

\normalsize\@subtitle\par}% uno \stretch{1} è già stato fatto
%    \end{macrocode}
% If a file name is specified for the graphic file that
% contains the school logo, such image is inserted in the
% title page.
%    \begin{macrocode}
\ifBlank{\@SSSLogo}%
  {\vspace{\stretch{1}}}%
  {\includegraphics[height=30mm]{\@SSSLogo}}\par\vspace{\stretch{1}}
%    \end{macrocode}
% The examining committee chairperson name and the Ministry
% of education committee ID number are typeset within a
% |minipage|. Another |minipage| contain the student's name.
% Pay attention: the names of the student and of the president
% are typeset in bold face small caps if the default fonts have
% this typeface; if the chosen fonts do not have this typeface
% the actual used typeface is bold roman.
%    \begin{macrocode}
\begin{minipage}[t]{0.45\textwidth}\flushleft
\ifBlank{\@presidente}{}{\@Npresidente\\
{\bfseries\scshape\@presidente}\par
\vspace{\baselineskip}}

\ifBlank{\@numerocommissione}{}{Commissione numero\\
\textbf{\@numerocommissione}}
\end{minipage}
\hfill
\begin{minipage}[t]{0.45\textwidth}\flushright
\@Ncandidato\\
{\bfseries\scshape\@author}
\end{minipage}

\vspace{\stretch{1}}

%    \end{macrocode}
% Eventually the school year is printed with an initial string
% `Anno scolastico'. This terminates the title page; the
% environment is closed and the page ejected.
%    \begin{macrocode}
\raisebox{-0.05\textheight}[0pt][0pt]{%
\parbox[t]{\textwidth}{\centering
\makebox[0.25\textwidth]{\hrulefill}\\[\medskipamount]
Anno scolastico \@annoscolastico}}
\end{titlepage}
\clearpage}%% 
%% 
\endinput
%    \end{macrocode}
% This is the end of this module code.
%\iffalse
%</topsss>
%\fi
%
% \Finale
% \endinput
%
\endinput
%%%%%% Recent progress
%%% (version 6.1.8 -- 2017-10-15) Extended version 5.92.xx in order 
%    to modularise the whole bundle; this requires the key=value options 
%    in order to smoothly handle the various new options, especially those 
%    that offer the user the new environment ThesisTitlePage 
%    (FrontespizioTesi for high school tesina); a new environment SDbox 
%    to produce figures and tables and any other possible floating body, 
%    defined through package float, with their captions on a side, 
%    possibly protruding into the external margin; in any case the SD 
%    box can be used by itself without using the corresponding floating 
%    environment. Modules available for different thesis types include 
%    high school tesina, bachelor, master, doctoral theses, the doctoral 
%    thesis title page for the Doctorate School of Politecnico di Torino; 
%    custom and frontespizio key-value options are available for a totally 
%    hand made title page, and a title page typeset by means of the 
%    external package frontespizio by Enrico Gregorio; backwards 
%    compatibility with the previous version is almost always maintained 
%    for both title pages; even the ID number of the candidates is 
%    maintained, although with a slightly different modern interface, 
%    since the xparse package facilities are used to define some of the 
%    new commands and environments. The a4paper paper size is 
%    maintained as the default one, although alla paper sizes available
%    to the report/book/article class can be used (this required a loto
%    of thought, because the xkeyval package documentations require
%    thoughtful interpretation.



